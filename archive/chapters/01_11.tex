\chapter[Split in the Congress and Revolutionary Terrorism]{The Split in the Congress and the Rise of Revolutionary Terrorism}


The Indian National Congress split in December 1907. Almost at the name time revolutionary terrorism made its appearance in Bengal. The two events were not unconnected.

\begin{center}*\end{center}



By 1907, the Moderate nationalists had exhausted their historical role. Their achievements, as we have seen in the previous chapter, we immense, considering the low level of political consciousness and the immense difficulties they had to face when they began. Their failures too were numerous. They lacked faith in the common people, did no work among them and consequently failed to acquire any roots among them. Even their propaganda did not reach them. Nor did they organize any all- India campaigns and when, during 1905-07, such an all-India campaign did come up in the form of the Swadeshi and Boycott Movement, they were not its leader \& (though the Bengal Moderates did play an active role in their own province). Their politics were based on the belief that they would be able to persuade the rulers to introduce economic and political reforms but their practical achievements in this respect were meagre. Instead of respecting them for their moderation, the British treated them with contempt, sneered at their politics and met popular agitations with repression.

Their basic failure, however, was that of not keeping pace with events. They could not see that their own achievements had made their Politics obsolete. They failed to meet the demands of the new stage of the national movement) Visible proof f this was their failure to attract the younger generation.

\begin{center}*\end{center}



The British had been suspicious of the National Congress from its inception. But they had not been overtly hostile, in the first few years of its existence because they believed its activities would remain academic and confined to a handful of intellectuals. However, as soon as it became apparent that the Congress would not remain so narrowly confined, and that it was becoming a focus of Indian nationalism, the officials turned openly critical of the Congress, the nationalist leaders and the Press.

They now began to brand the nationalists as ‘disloyal babus’ ‘seditious Brahmins,’ and ‘violent villains.’ The Congress was described as ‘a factory of sedition’ and Congressmen as ‘disappointed candidates for office and discontented lawyers who represent no one but themselves.’ In 1888, Dufferin, the Viceroy, attacked the National Congress in a public speech and ridiculed it as representing only the elite ‘a microscopic minority.” George Hamilton, Secretary of State for India, accused the Congress leaders of possessing ‘seditious and double sided character.’

This hostility did not abate when the Moderates, who then controlled the Congress, began to distance themselves from the rising militant nationalist tendencies of certain sections of the Congress which became apparent when the government unleashed a repressive policy against the Indian Press in 1897. Instead the British appeared even more eager to attack and finish the Congress. Why was this so? First, because however moderate and loyal in their political perception the Moderates were, they were still nationalists and propagators of anti-colonialist politics and ideas. As Curzon, the Viceroy, put it in 1905: ‘Gokhale either does not see where he is going, or if he does see it, then he is dishonest or his pretensions. You Cannot awaken and appeal to the spirit of nationality in India and at the same time, profess loyal acceptance of British rule.’ Or, as George Hamilton, the Secretary of State, had complained to Dadabhai Naoroji an 1900: ‘You announce yourself as a sincere supporter of British rule; you vehemently denounce the condition, and consequences which are it inseparable from the maintenance of that rule.”

Second, the British policy-makers felt that the Moderate-led Congress could be easily finished because it was weak and without a popular base. Curzon, in particular, supported by George Hamilton, pursued this policy. He declared in 1900: ‘The Congress is tottering to its fall, and one of my greatest ambitions while in India is to assist it to a peaceful demise’. In 1903, he wrote to the Madras Governor: ‘My policy, ever since I came to India, has been to reduce the Congress to impotence.’ In 1904, he had insulted the Congress by refusing to meet its delegation headed by its President.

This policy was changed once the powerful Swadeshi, and Boycott Movement began and the militant nationalist trend became strong. An alternative policy of weakening the nationalist movement was now to be followed. Instead of sneering at the Moderates, the policy was to be that of ‘rallying’ them as John Morley, the new Secretary of State for India, put it in 1907. The new policy, known as the policy of the carrot and the stick, was to be a three pronged one. It may be described as a policy of repression-conciliation-suppression. The Extremists, as we shall refer to the militant nationalists from now on, were to be repressed, though mildly in the first stage, the purpose being to frighten the Moderates. The Moderates were then to be placated through some concessions and promises and hints were to be given that further concessions would be forthcoming if they disassociated themselves from the Extremists. The entire objective of the new policy was to isolate the Extremists. Once the Moderates fell into the trap, the Extremists could be suppressed through the use of the full might of the state. The Moderates, in turn, could then be ignored. Unfortunately for the national movement, neither the Moderates nor the Extremists were able to understand the official strategy and consequently suffered a number of reverses.

\begin{center}*\end{center}



The Government of India, headed by Lord Minto as Viceroy and John Morley as the Secretary of State, offered a bait of fresh reforms in the Legislative Councils and in the beginning of 1906 began discussing them with the Moderate leadership of the Congress. The Moderates agreed to cooperate with the Government and discuss reforms even while a vigorous popular movement, which the Government was trying to suppress, was going on in the country. The result was a total split in the nationalist ranks.

Before we take up this split at some length, it is of some interest to note that the British were to follow this tactic of dividing the Moderates from the militants in later years also — for example in 1924, vis-a-vis Swarajists, in 1936, vis-a-vis Nehru and the leftists, and so on. The difference was that in the later years the national leadership had learnt a lesson from the events of 1907-1909, and refused to rise to the bait, remaining united despite deep differences.

\begin{center}*\end{center}



There was a great deal of public debate and disagreement among Moderates and Extremists in the years 1905-1907, even when they were working together against the partitioning of Bengal. The Extremists wanted to extend the Swadeshi and the Boycott Movement from Bengal to the rest of the country. They also wanted to gradually extend the boycott from foreign goods to every form of association or cooperation with the colonial Government. The Moderates wanted to confine the boycott part of the movement to Bengal and were totally opposed to its extension to the Government.

Matters nearly came to a head at the Calcutta Congress in 1906 over the question of its Presidentship. A split was avoided by choosing Dadabhai Naoroji, who was respected by all the nationalists as a great patriot. Four compromise resolutions on the Swadeshi, Boycott, National Education, and Self-Government demands were passed. Throughout 1907 the two sides fought over differing interpretations of the four resolutions. By the end of 1907, they were looking upon each other as the min political enemy. The Extremists were convinced that the battle for freedom had begun as the people had been roused. They felt it was time for the big push and in their view the Moderates were a big drag on the movement. Most of them, led by Aurobindo Ghose, felt that the time had come to part company with the Moderates, push them out of the leadership of the Congress, and split the organization if the Moderates could not be deposed.

Most of the Moderates, led by Pherozeshah Mehta, were no less determined on a split. To remain with the Extremists was, they felt, to enter dangerous waters. They were afraid that the Congress organization built carefully over the last twenty years, would be shattered. The Government was bound to suppress any large-scale antiimPerIat1st movement; why invite premature repression? As Gokhale put it in 1907, ‘You (the Extremists) do not realize the enormous reserve of power behind the Government, if the Congress were to do anything such as you suggest, the Government would have no difficulty in throttling it in five minutes.’ Minto and Morley were holding up hopes of brighter prospects. Many Moderates thought that their dream of Indians sharing political and administrative power was going to come true. Any hasty action by the Congress under Extremist pressure could annoy the Liberals in power in Britain. Why not get rid of the Extremists while there was still time?

As H.A. Wadya, representing Pherozeshah Mehta’s thinking, wrote in an article in which, after referring to ‘he Extremists as ‘the worst enemies of our cause,’ said: ‘The union of these men with the Congress is the union of a diseased limb to a healthy body, and the only remedy is surgical severance, if the Congress is to be saved from death by blood poisoning.’ Both sides had it wrong — from the nationalist point of view as well as their own factional point of view. The Moderates did not see that the colonial state was negotiating with them not because of their inherent political strength but because of the fear of the Extremists. The Extremists did not see that the Moderates were their natural outer defence line (in terms of civil liberties and so on) and that they did not possess the required strength to face the colonial state’s juggernaut. Neither saw that in a vast country like India ruled by a powerful imperialist nation only a broad- based united movement had any chance of success. It wasn’t as though the whole leadership was blind to the danger. The main public leaders of the two wings, Tilak (of the Extremists) and Gokhale (of the Moderates) were mature politicians who had a clear grasp of the dangers of disunity in the nationalist ranks. Tilak did not want the united national front to break. He saw clearly that a powerful movement could not be built up at that stage nor political demands successfully pressed on the rulers without the unity of different political trends. His tactics were to organize massive support for his political line and, thus, force a favourable compromise on the Moderates. But having roused his followers in Maharashtra arid pushed on by the more extreme elements of Bengal. Tilak found that he could not afford to dismount from the tiger he found himself riding. When it came to the crunch, he had to go with the more extreme leaders like Aurobindo Ghose.

Gokhale, too, saw the dangers of a split in the nationalist ranks and tried to avoid it. Already, in October 1907, he had written to a friend: ‘If a split does come it means a disaster, for the Bureaucracy will then put down both sections without much difficulty.’ But he did not have the personality to stand upto a wilful autocrat like Pherozeshah Mehta. He, too, knuckled under pressure of his own extremists.

The Congress session was held on 26 December, 1907 at Surat, on the banks of the river Tapti. The Extremists were excited by the rumours that the Moderates wanted to scuttle the four Calcutta resolutions. The Moderates were deeply hurt by the ridicule and venom poured on them in mass meetings held at Surat on the previous three days. The delegates, thus, met in an atmosphere surcharged with excitement and anger.

The Extremists wanted a guarantee that the four resolutions would be passed. To force the Moderates to do so they decided to object to the duly elected President for the year, Rash Behari Ghose. Both sides came to the session prepared for a confrontation. In no time, the 1600 delegates were shouting, coming to blows and hurling chairs at each other. En the meantime, some unknown person hurled a shoe at the dais which hit Pherozeshah Mehta and Surendranath Banerjea. The police came and cleared the hall. The Congress session was over. The only victorious party was the rulers. Minto immediately wrote to Morley that the ‘Congress collapse’ at Surat was ‘a great triumph for us.”

Tilak had seen the coming danger and made last minute efforts to avoid it. But he was helpless before his followers. Lajpat Rai, a participant in the events from the Extremist side, wrote later: ‘Instead of leading his party, he (Tilak) allowed himself to be led by some of its wild spirits. Twice on my request, at Surat, he agreed to waive his opposition to the election of Dr. Rash Behari Ghose and leave the matter of the four Calcutta resolutions to the Subjects Committee, but the moment I left him he found himself helpless before the volume of opinion that surrounded him.”

The suddenness of the Surat fiasco took Tilak by surprise. He had not bargained for it because, as Aurobindo Ghose wrote later, Tilak viewed the split as a ‘catastrophe.’ He valued the Congress ‘as a great national fact and for its unrealized possibilities.”He now tried to undo the damage. He sent a virtual letter of regret to his opponents, accepted Rash Behari Ghose as the President of the Congress and offered his cooperation in working fm Congress unity. But Pherozeshah Mehta and his colleagues would not relent. They thought they were on a sure wicket. The Government immediately launched a massive attack on the Extremists. Extremist newspapers were suppressed. Tilak, their main leader, was sent to Mandalay jail for six years. Aurobindo Ghose, their ideologue, was involved in a revolutionary Conspiracy case and immediately after being judged innocent gave up politics and escaped to Pondicherry to take up religion.

B.C. Pal temporarily retired from politics and Lajpat Rai, who had been a helpless onlooker at Surat, left for Britain in 1908 to come back in 1909 and then to go off to the United States for an extended stay. The Extremists were not able to organize an effective alternative party or to sustain the movement. The Government had won, at least for the moment.’

The Moderates were indulging their own foolish beliefs. They gave up all the radical measures adopted at the Benaras and Calcutta sessions of the Congress, spurned all overtures for unity from the Extremists and excluded them from the party. They thought they were going to rebuild, to quote Pherozeshah Mehta, a ‘resuscitated, renovated, reincarnated Congress.’ But the spirit had gone out of the Congress and all efforts to restore it failed. They had lost the respect and support of the political Indians, especially the youth, and were reduced to a small coterie. Most of the Moderate leaders withdrew into their shells; only Gokhale plodded on, with the aid of a small band of co-workers from the Servants of India Society. And the vast majority of politically conscious Indians extended their support, however passive, to Lokamanya Tilak and the militant nationalists.

After 1908 the national movement as a whole declined. In 1909, Aurobindo Ghose noted the change: ‘When I went to jail the whole country was alive with the cry of Bande Mataram, alive with the hope of a nation, the hope of millions of men who had newly risen out of degradation. When I came out of jail I listened for that cry, but there was instead a silence. A hush had fallen on the country.” But while the upsurge was gone, the arouse nationalist sentiments did not disappear. The people waited for the next phase. In 1914, Tilak was released and he picked up the threads of the movement.

\begin{center}*\end{center}



The Moderates and the country as a whole were disappointed by the ‘constitutional’ reforms of 1909. The Indian Councils Act of 1909 increased the number of elected members in the imperial Legislative Council and the provincial legislative councils. Most of the elected members were still elected indirectly. An Indian was to be appointed a member of the Governor-General’s Executive Council. Of the sixty-eight members of the Imperial Legislative Council, thirty-six were officials and five were nominated non-officials. Out of twenty- seven elected members, six were elected by big landlords and two by British capitalists. The Act permitted members to introduce resoluti9r s; it also increased their power to ask questions. Voting on separate budget items was allowed. But the reformed councils still enjoyed no real power and remained mere advisory bodies. They also did not introduce an element of democracy or self- government. The undemocratic, foreign and exploitative character of British rule remained unchanged. Morley openly declared in Parliament: ‘If it could be said that this CHAPTER of reforms led directly or necessarily up to the establishment of a Parliamentary system in India, I, for one, would have nothing at all to do with it.’

The real purpose of the Morley-Minto Reforms was to divide the nationalist ranks and to check the growing unity among Indians by encouraging the growth of Muslim communalism. To achieve the latter objective, the Reforms introduced the system of separate electorates under which Muslims could only vote for Muslim candidates in constituencies specially reserved for them. This was done to encourage the notion that the political, economic and cultural interests of Hindus and Muslims were separate and not common. The institution of separate electorates was one of the poisonous trees which was to yield a bitter harvest in later years.

\begin{center}*\end{center}



The end of 1907 brought another political trend to the fore. The impatient young men of Bengal took to the path of individual heroism arid revolutionary terrorism (a term we use without any pejorative meaning and for want of a different term). This was primarily because they could find no other way of expressing their patriotism It is necessary at this point to reiterate the fact that, while the youth of Bengal might have been incensed at the official arrogance and repression and the ‘mendicancy’ of the Congress Moderates, they were also led to ‘the politics of the bomb’ by the Extremists’ failure to give a positive lead to the people. The Extremists had made a sharp and on the whole correct and effective critique of the Moderates. They had rightly emphasized the role of the masses and the need to go beyond propaganda and agitation. They had advocated persistent opposition to the Government and put forward a militant programme of passive resistance and boycott of foreign cloth, foreigners’ courts, education and so on. They had demanded self- sacrifice from the youth. They had talked and written about direct action.

But they had failed to find forms through which all these ideas could find practical expression. They could neither create a viable organization to lead the movement nor could they really define the movement in a way that differed from that of the Moderates. They were more mi1itant their critique of British rule was couched in stronger language, they were willing to make greater sacrifices and undergo greater suffering, but they did not know how to go beyond more vigorous agitation. They were not able to put before people new forms of political struggle or mass movements. Consequently, they too had come to a political dead end by the end of 1907. Perhaps that is one reason why they expended so much of their energy in criticizing the Moderates and capturing the Congress. Unsurprisingly, the Extremists’ waffling failed to impress the youth who decided to take recourse to physical force. The Yugantar, a newspaper echoing this feeling of disaffection, wrote in April 1906, after the police assault on the peaceful Barisal Conference: ‘The thirty crores of people inhabiting India must raise their sixty crores of hands to stop this curse of oppression. Force must be stopped by force.’

But the question was what form would this movement based on force take. Organizing a popular mass uprising would necessarily be an uphill and prolonged task. Many thought of trying to subvert the loyalty of the army, but they knew it would not be easy. However, these two objectives were kept as long-term goals and, for the present, revolutionary youth decided to copy the methods of the fish nationalists and Russian nihilists and populists. That is to say, they decided to organize the assassination of unpopular British officials. Such assassinations would strike terror into the hearts of the rulers, amuse the patriotic instincts of the people, inspire them and remove the fear of authority from their minds. Each assassination, and if the assassins were caught, the consequent trial of the revolutionaries involved, would act as ‘propaganda by deed’’ All that this form of struggle needed was numbers of young people ready to sacrifice their lives.

Inevitably, it appealed to the idealism of the youth; it aroused their latent sense of heroism. A steadily increasing number of young men turned to this form of political struggle.

Here again the Extremist leadership let the young people down, While it praised their sense of self-sacrifice and courage, it failed to provide a positive outlet for their revolutionary energies and to educate them on the political difference between a evolution based on the activity of the masses and a revolutionary feeling based on individual action, however heroic. It also failed to oppose the notion that to be a revolutionary meant to be a believer in violent action. In fact, Aurobindo Ghose encouraged this notion. Perhaps the actions of the Extremist leadership were constrained by the feeling that it was not proper to politically criticize the heroic youth who were being condemned and hunted by the authorities. But this failure to politically and ideologically oppose the young revolutionaries proved a grievous error, for it enabled the individualistic and terrorist conception of revolution to take root in Bengal.

In 1904, V.D. Sarvarkar organized Abhinav Bharat as a secret society of revolutionaries. After 1905 several newspapers openly (and a few leaders secretly) began to advocate revolutionary terrorism. In 1907, an unsuccessful attempt was made on the life of the Lieutenant-Governor of Bengal. In April 1908, Prafulla Chaki and Khudiram Bose threw a bomb at a carriage which they believed was occupied by Kingsford, the unpopular judge at Muzzafarpur. Unfortunately, they killed two English ladies instead. Prafulla Chaki shot himself dead while Khudiram Bose was tried and hanged. Thousands wept at his death and he and Chaki entered the ranks of popular nationalist heroes about whom folk songs were composed and sung all over the country

The era of revolutionary terrorism had begun. Very soon secret societies of revolutionaries came up all over the country, the most famous and long lasting being Anushilan Samity and Jugantar. Their activities took two forms---the assassination of oppressive officials and informers and traitors from their own ranks and dacoities to raise funds for purchase of arms etc. The latter came to be popularly known as Swadeshi dacoities! Two of the most spectacular revolutionary terrorist actions of the period were the unsuccessful attempt under the leadership of Rash Behari Bose and Sachin Sanyal to kill the Viceroy, Lord Hardinge who was wounded by the bomb thrown at him while he was riding an elephant in a state procession — and the assassination of Curzon-Wylie in London by Madan Lal Dhingra. In all 186 revolutionaries were killed or convicted between the years 1908­ 1918. The revolutionary terrorists also established centres abroad. The more famous of them were Shyamji Krishnavarma,

V.D. Savarkar and Har Dayal in London and Madame Cama and Ajit Singh in Europe.

Revolutionary terrorism gradually petered out. Lacking a mass base, despite remarkable heroism, the individual revolutionaries, organized in small secret groups, could not withstand suppression by the still strong colonial state. But despite their ‘small numbers and eventual failure, they made a valuable contribution to the growth of nationalism in India. As a historian has put it, ‘they gave us back the pride of our manhood.’
