\chapter[Struggle for Gurdwara Reform and Temple Entry]{The Struggle for Gurdwara Reform and Temple Entry}



The rising tide of nationalism and democracy inevitably began to overflow from the political to the religious and social fields affecting the downtrodden castes and classes. And many nationalists began to apply the newly discovered technique of non-violent Satyagraha and mobilization of public opinion to issues which affected the internal structure of Indian society. Quite often this struggle to reform Indian social and religious institutions and practices led the reformers to clash with the colonial authorities. Thus, the struggle to reform Indian society tended to merge with the anti-imperialist struggle. This was in part the result of the fact that as the national movement advanced, the social base of colonialism was narrowed and the colonial authorities began to seek the support of the socially, culturally and economically reactionary sections of Indian society. This aspect of the national movement is well illustrated by the Akali Movement in Punjab and the Temple Entry Movement in Kerala.

\begin{center}*\end{center}



The Akali Movement developed on a purely religious issue but ended up as a powerful episode of India’s freedom struggle. From 1920 to 1925 more than 30,000 men and women underwent imprisonment, nearly 400 died and over 2,000 were wounded. The movement arose with the objective of freeing the Gurdwaras (Sikh temples) from the control of ignorant and corrupt mahants (priests). The Gurdwaras had been heavily endowed with revenue-free land and money by Maharaja Ranjit Singh, Sikh chieftains and other devout Sikhs during the 18th and 19th centuries. These shrines came to be managed during the 18th century by Udasi Sikh mahants who escaped the wrath of Mughal authorities because they did not wear their hair long. (Many ignorant people therefore believe that these mahants were Hindus. This is, of course, not true at all). In time corruption spread among these mahants and they began to treat the offerings and other income of the Gurdwaras as their personal income. Many of them began to live a life of luxury and dissipation. Apart from the mahants, after the British annexation of Punjab in 1849, some control over the Gurdwaras was exercised by Government- nominated managers and custodians, who often collaborated with mahants.

The Government gave full support to the mahants. It used them and the managers to preach loyalism to the Sikhs and to keep them away from the rising nationalist movement. The Sikh reformers and nationalists, on the other hand, wanted a thorough reformation of the Gurdwaras by taking them out of the control of the mahants and agents of the colonial regime. The nationalists were especially horrified by two incidents - - when the priests of the Golden Temple at Amritsar issued a Hukamnama (directive from the Gums or the holy seats of the Sikh authority) against the Ghadarites, declaring them renegades, and then honoured General Dyer, the butcher of Jallianwala massacre, with a saropa (robe of honour) and declared him to be a Sikh.

A popular agitation for the reform of Gurdwaras developed rapidly during 1920 when the reformers organized groups of volunteers known as jathas to compel the mahants and the Government-appointed managers to hand over control of the Gurdwaras to the local devotees. The reformers won easy victories in the beginning with tens of Gurdwaras being liberated in the course of the year. Symbolic of this early success was the case of the Golden Temple and the Akal Takht. The reformers demanded that this foremost seat of Sikh faith should be placed in the hands of a representative body of the Sikhs,’ and organized a series of public meetings in support of their demand. The Government did n want to antagonize the reformers at this stage and decided to stem the rising tide of discontent on such an emotional religious issue by appeasing the popular sentiment. It, therefore, permitted the Government-appointed manager to resign and, thus, let the control of the Temple pass effectively into the reformers’ hands.

To control and manage the Golden Temple, the Akal Takht and other Gurdwaras, a representative assembly of nearly 10,000 reformers met in November 1920 and elected a committee of 175 to be known as the Shiromani Gurdwara Prabhandak Committee (SGPC). At the same time, the need was felt for a central body which would organize the struggle on a more systematic basis. The Shiromani Akali Dal was established in December for this purpose. It was to be the chief organizer of the Akali jathos whose backbone was provided by Jat peasantry while their leadership was in the hands of the nationalist intellectuals. Under the influence of the contemporary Non-Cooperation Movement — and many of the leaders were common to both the movements — the Akali Dal and the SGPC accepted complete non-violence as their creed.

\begin{center}*\end{center}



The Akali movement faced its first baptism by blood at Nankana, the birth place of Guru Nanak, in February 1921. The mahant of the Gurdwara there, Narain Das, was not willing to peacefully surrender his control to the Akalis. He gathered a force of nearly 500 mercenaries and armed them with guns, swords, lathis and other lethal weapons to meet the challenge of the peaceful Akali volunteers. On 20 February, an Akali jatha entered the Gurdwara to pray. Immediately, the mahant‘s men opened fire on them and attacked them with other weapons. Nearly 100 Akalis were killed and a large number of jathas under Kartar Singh Jhabbar’s command marched into the Gurdwara and took complete control despite dire warnings by the Deputy Commissioner. The mahant had already been arrested. The Government policy was still of vacillation. On the one hand, it did not want to earn the ire of the Sikhs, and, on the other, it did not want to lose control over the Gurdwaras.

The Nankana tragedy was a landmark in the Akali struggle. As Kartar Singh Jhabbar, the liberator of the Nankana Gurdwara put it, ‘the happening had awakened the Sikhs from their slumber and the march towards Swaraj had been quickened.’ The tragedy aroused the conscience of the entire country. Mahatma Gandhi, Maulana Shaukat Ali, Lala Lajpat Rai and other national leaders visited Nankana to show their solidarity. The Government now changed its policy. Seeing the emerging integration of the Akali movement with the national movement, it decided to follow a two-pronged policy. To win over or neutralize the Moderates and those concerned purely with religious reforms, it promised and started working for legislation which would satisfy them. It decided to suppress the extremist or the anti- imperialist section of the Akalis in the name of maintaining law and order.

The Akalis, too, changed their policy. Heartened by the support of nationalist forces in the country, they extended the scope of their movement to completely root out Government interference in their religious places. They began to see their movement as an integral part of the national struggle. Consequently, within the SGPC, too, the non-cooperator nationalist section took control. In May 1921, the SGPC passed a resolution in favour of non-cooperation, for the boycott of foreign goods and liquor, and for the substitution of panchayats for the British courts of law. The Akali leaders, arrested for the breaking of law, also refused to defend themselves, denying the jurisdiction of foreign-imposed courts.

A major victory was won by the Akalis in the Keys Affair’ in October 1921. The Government made an effort to keep possession of the keys of the Toshakhana of the Golden Temple. The Akalis immediately reacted, and organized massive protest meetings; tens of Akali jathas reached Amritsar immediately. The SGPC advised Sikhs to join the hartal on the day of the arrival of the Prince of Wales in India. The Government retaliated by arresting the prominent, militant nationalist leaders of the SGPC like Baba Kharak Singh and Master Tara Singh. But, instead of dying down, the movement began to spread to the remotest rural areas and the army. The Non-Cooperation Movement was at its height in the rest of the country. The Government once again decided not to confront Sikhs on a religious issue. It released all those arrested in the ‘Keys Affair’ and surrendered the keys of the Toshakhana to Baba Kharak Singh, head of the SGPC. Mahatma Gandhi immediately sent a telegram to the Baba: ‘First battle for India’s freedom won. Congratulations.’

\begin{center}*\end{center}



The culmination of the movement to liberate the Gurdwaras came with the heroic non-violent struggle around Guru-Ka-Bagh Gurdwara which shook the whole of India. Smarting under its defeat in the ‘Keys Affair,’ the Punjab bureaucracy was looking for an opportunity to teach the Akalis a lesson and to recover its lost prestige. It was further emboldened by the fact that the Non- Cooperation Movement had been withdrawn in February 1922. It began to look for a pretext.

The pretext was provided by events at a little known village, Ghokewala, about 20 kilometres from Amritsar. The mahant of the Gurdwara Guru-Ka-Bagh had handed over the Gurdwara to the SGPC in August 1921, but claimed personal possession of the attached land. When the Akalis cut a dry kikkar tree on the land for use in the community kitchen, he complained to the police ‘of the theft of his property from his land.’ The officials seized this opportunity to provoke the Akalis. On 9 August 1922, five Akalis were arrested and put on trial. The Akali Dal reacted immediately to the new challenge. Akali jathas began to arrive and cut trees from the disputed Land. The Government started arresting all of them on charges of theft and rioting. By 28 August more than 4,000 Akalis had been arrested.

The authorities once again changed their tactics. Instead of arresting the Akali volunteers they began to beat them mercilessly with lathis. But the Akalis stood their ground and would not yield till felled to the ground with broken bones and lacerated bodies. C.F. Andrews described the official action as inhuman, brutal, foul, cowardly and incredible to an Englishman and a moral defeat of England. The entire country was outraged. National leaders and journalists converged on Guru-Ka-Bagh. Massive protest meetings were organized all over Punjab. A massive Akali gathering at Amritsar on io September was attended by Swami Shraddhaflafld, Hakim Ajmal Khan and others. The Congress Working Committee appointed a committee to investigate the conduct of the police.

Once again the Government had to climb down. As a face saving device, it persuaded a retired Government servant to lease the disputed land from the mahant and then allow the Akalis to cut the trees. It also released all the arrested Akali volunteers. With the Gurdwaras under the control of the SGPC, the militant Akalis looked for some other opportunity of confronting the Government since they felt that the larger Gurdwara -— the country was not yet liberated. In September 1923, the SGPC took up the cause of the Maharia of Nabba who had been forced by the Government to abdicate. This led to the famous morcha at Jaito in Nabha. But the Akalis could not achieve much success on the issue since it neither involved religion nor was there much support in the rest of the country. In the meanwhile, the Government had succeeded in winning over the moderate Akalis with the promise of legislation which was passed in July 1925 and which handed over control over all the Punjab Gurdwaras to an elected body of Sikhs which also came to be called the SGPC.

Apart from its own achievement, the Akali Movement made a massive contribution to the political development of Punjab. It awakened the Punjab peasantry. As Mohinder Singh, the historian of the Akali Movement, has pointed out: ‘It was only during the Akali movement that the pro-British feudal leadership of the Sikhs was replaced by educated middle-class nationalists and the rural and urban classes united on a common platform during the two-pronged Akali struggle.’ This movement was also a model of a movement on a religious issue which was utterly non-communal. To further quote Mohinder Singh: ‘It was this idea of Liberation of the country from a foreign Government that united all sections of the Sikh community and brought the Hindus, the Muslims and the Sikhs of the province into the fold of the Akali movement.’3 The Akali Movement also awakened the people of the princely states of Punjab to political consciousness and political activity. There were also certain weaknesses with long-term consequences. The movement encouraged a certain religiosity which would be later utilized by communalism.

The Akali Movement soon divided into three streams because it represented three distinct political streams, which had no reasons to remain united as a distinct Akali party once Gurdwara reform had taken place. One of the movement’s streams consisted of moderate, pro- Government men who were pulled into the movement because of its religious appeal and popular pressure. These men went back to loyalist politics and became a part of the Unionist Party. Another stream consisted of nationalist persons who joined the mainstream nationalist movement, becoming a part of the Gandhian or leftist Kirti-Kisan and Communist Wings. The third stream, which kept the title of Akali, although it was not the sole heir of the Akali Movement, used to the full the prestige of the movement among the rural masses, and became the political organ of Sikh communalism, mixing religion and politics and inculcating the ideology of political separation from Hindus and Muslims. In pre-1947 politics the Akali Dal constantly vacillated between nationalist and loyalist politics.

\begin{center}*\end{center}



Till 1917, the National Congress had refused to take up social reform issues lest the growing political unity of the Indian people got disrupted. 11 reversed this position in 1917 when it passed a resolution urging upon the people ‘the necessity, justice and righteousness of removing all disabilities imposed by custom upon the depressed classes.’ At this stage, Lokamanya Tilak also denounced untouchability and asked for its removal. But they did not take any concrete steps in the direction. Among the national leaders, it was Gandhi who gave top priority to the removal of untouchability and declared that this was no less important than the political struggle for freedom.

In 1923, the Congress decided to take active steps towards the eradication of untouchability. The basic strategy it adopted was to educate and mobilize opinion among caste Hindus on the question. The nationalist challenge in this respect came to be symbolized by two famous struggles in Kerala.

The problem was particularly acute in Kerala where the depressed classes or avarnas (those without caste, later known as Harijans) were subjected to degrading and de-humanising social disabilities. For example, they suffered not only from untouchability but also theendal or distance pollution — the Ezhavas and Putayas could not approach the higher castes nearer than 16 feet and 72 feet respectively. Struggle against these disabilities was being waged since the end of 19th century by several reformers and intellectuals such as Sri Narayan Guru,

N. Kumaran Asan and T.K. Madhavan.

Immediately after the Kakinada session, the Kerala Provincial Congress Committee (KPCC) took up the eradication of untouchability as an urgent issue While carrying on a massive propaganda campaign against untouchability and for the educational and social upliftment of the Harijans, it was decided to launch an immediate movement to open Hindu temples and all public roads to the avarnas or Harijans. This, it was felt, would give a decisive blow to the notion of untouchability since it was basically religious in character and the avarnas’ exclusion from the temples was symbolic of their degradation and oppression.

A beginning was made in Vaikom, a village in Travancore. There was a major temple there whose four walls were surrounded by temple roads which could not be used by avarnas like Ezhavas and Pulayas. The KPCC decided to use the recently acquired weapon of Satyagraha to fight wnouchability and to make a beginning at Vaikom by defyrng the unapproachability rule by leading a procession of savarnas (caste Hindus) and avarnas on the temple roads on 30 March 1924.

The news of the Satyagraha aroused immediate enthusiasm among political and social workers and led to an intense campaign to arouse the conscience of savarnas and mobilize their active support. Many savarna organizations such as the Nair Service Society, Na Samajam and Kerala Hindu Sabha supported the Satyagraha. Yogakshema Sabha, the leading organization of the Namboodins (highest Brahmins by caste), passed a resolution fuvouring the opening of temples to avarnas. The temple authorities and the Travancore Government put up barricades on the roads leading to the temple and the District Magistrate served prohibitory orders on the leaders of the Satyagraha. On 30th March, the Satyagrahis, led by K.P. Kesava Menon, marched from the Satyagraha camp towards the temple. They, as well as the succeeding batches of Satyagrahis, consisting of both savarnas and avarnas, were arrested and sentenced to imprisonment.

The Vaikom Satyagraha created enthusiasm all over the country and volunteers began to arrive from different parts of 1ndia An Akali jatha arrived from Punjab. E.V. Ramaswami Naicker (popularly known as Periyar later) led ajatha from Madurai and undeiwent imprisonment. On the other hand, the orthodox and reactionaiy section of caste Hindus met at Vaikom and decided to boycott all pro-Satyagraha Congressmen and not to employ them as teachers or lawyers or to vote for them.

On the death of the Maharaja in August 1924, the Maharani, as Regent, released all the Satyagrahis. As a positive response to this gesture, it was decided to organize a jatha (a group of volunteers) of caste Hindus to present a memorial to the Maharani asking for the opening of the temple roads to all. Batches of caste Hindus from all over Kerala converged on Vaikom. On 31 October, a jatha of nearly one hundred caste Hindus started their march on foot to Trivandrum. It was given warm receptions at nearly 200 villages and towns on the way. By the time it reached Trivandrum, it consisted of over 1,000 persons. The Maharani, however, refused to accept their demand and the Satyagraha was continued.

In early March 1925, Gandhi began his tour of Kerala and met the Maharani and other officials. A compromise was arrived at. The roads around the temple were opened to avarnas but those in the Sankethan of the temple remained closed to them. In his Kerala tour, Gandhi did not visit a single temple because avarnas were kept out of them.

\begin{center}*\end{center}



The struggle against untouchability and for the social and economic uplift of the depressed classes continued all over India after 1924 as a part of the Gandhian constructive programme. Once again the struggle was most Intense m Kerala. Prodded by K Kelappan, the KPCC took up the question of temple entry in 1931 during the period when the Civil Disobedience Movement was suspended. A vast campaign of public meetings was organized throughout Malabar. The KPCC decided to make a beginning by organizing a temple entry Satyagraha at Guruvayur on 1st November 1931.

A jatha of sixteen volunteers, led by the poet Subramanian Tirumambu, whn became famous as the ‘Singing Sword of Kerala,’ began a march from Cannanore in the north to Guruvayur on 21 October. The volunteers ranged from the lowliest of Harijans to the highest caste Namboodiris. The march stirred the entire country and aroused anti-caste sentiments. The 1st of November was enthusiastically observed as All-Kerala Temple Entry Day with a programme of prayers, processions, meetings, receptions and fund collections. It was also observed in cities like Madras, Bombay, Calcutta, Delhi and Colombo (Sri Lanka). The popular response was tremendous. Many all-India leaders visited Malabar. Money and volunteers poured in from everywhere. The youth were specially attracted and were in the forefront of the struggle. The anti-untouchability movement gained great popularity. Many religious devotees transferred the offerings they would have made to the temple to the Satyagraha camp, feeling that the camp was even more sacred than the temple.

The temple authorities also made arrangements. They put up barbed wire all around the temple and organized gangs of watchmen to keep the Satyagrahis out and to threaten them with beating.

On 1 November, sixteen white khadi-clad volunteers marched to the eastern gate of the temple where their way was barred by a posse of policemen headed by the Superintendent of Police. Very soon, the temple servants and local reactionaries began to use physical force against the peaceful and non-violent Satyagrahis while the police stood by. For example, P Krishna Pillai and A.K. Gopalan, who were to emerge later as major leaders of the Communist movement in Kerala, were mercilessly beaten. The Satyagraha continued even after the Civil Disobedience Movement was resumed in January 1932 and all Congress Committees were declared unlawful and most of the Congressmen leading the Satyagraha were imprisoned.

The Satyagraha entered a new phase on 21 September 1932 when K. Kelappan went on a fast unto death before the temple until it was opened to the depressed classes. The entire country was again stirred to its depths. Once again meetings and processions engulfed Kerala and many other parts of the country. Caste Hindus from Kerala as well as rest of India made appeals to the Zamonn of Calicut, custodian of the temple, to throw open the temples to all Hindus; but without any success. Gandhiji made repeated appeals to Kelappan to break his fast, at least temporarily, with an assurance that he would himself, if necessary, undertake the task of getting the temple opened. Finally, Kelappan broke his fast on October 2, 1932. The Satyagraha was also suspended. But the temple entry campaign was carried on ever more vigorously.

A jatha led by A.K. Gopalan toured whole of Kerala on foot, carrying on propaganda and addressing massive meetings everywhere. Before it was disbanded the jatha had covered nearly 1,000 miles and addressed over 500 meetings.

Even though the Guruvayur temple was not opened immediately, the Satyagraha was a great success in broader terms. As A.K. Gopalan has recorded in his autobiography, ‘although the Guruvayur temple was still closed to Harijans, I saw that the movement had created an impetus for social change throughout the country. It led to a transformation everywhere.’

The popular campaign against untouchability and for temple entry continued in the succeeding years. In November 1936, the Maharaja of Travancore issued a proclamation throwing open all Government-controlled temples to all Hindus irrespective of caste. Madras followed suit in 1938 when its Ministry was headed by C. Rajagopalachari. Other provinces under Congress rule also took similar steps.

The temple entry campaign used all the techniques developed by the Indian people in the course of the nationalist struggle. Its organizers succeeded in building the broadest possible unity, imparting mass education, and mobilizing the people on a very wide scale on the question of untouchability. Of course, the problem of caste inequality, oppression and degradation was very deep-seated and complex, and temple entry alone could not solve it. But Satyagrahas like those of Vaikom and Guruvayur and the movements around them did make a massive contribution in this respect. As E.M.S. Namboodiripad was to write years later: ‘Guruvayur Temple Satyagraha was an event that thrilled thousands of young men like me and gave inspiration to a vast majority of the people to fight for their legitimate rights with self-respect. . . It was the very same youth who gave this bold lead, who subsequently became founder- leaders of the worker-peasant organizations that were free from the malice of religious or communal considerations.”

The main weakness of the temple entry movement and the Gandhian or nationalist approach in fighting caste oppression was that even while amusing the people against untouchability they lacked a strategy for ending the caste system itself. The strength of the national movement in this respect was to find expression in the Constitution of independent India which abolished caste inequality, outlawed untouchability and guaranteed social equality to all citizens irrespective of their caste. Its weakness has found expression in the growth of casteism and the continuous existence in practice of oppression and discrimination against the lower castes in post-1947 India.
