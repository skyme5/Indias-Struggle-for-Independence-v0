\chapter[The Indian Working Class and National Movement]{The Indian Working Class and the National Movement}



The modem worker makes his appearance in India in the second half of the 19th century with the slow beginnings of modem industry and the growth of utilities like the railways and the post and the telegraph network The process of the disparate groups of workers in various parts of country emerging as an organized, self-conscious, all India class is inextricably linked with the growth of the Indian national movement and the process of the Indian ‘nation-in-the-making’ because the notion of the Indian working class could not exist before the notion of the Indian ‘people’ had begun to take root.

\begin{center}*\end{center}



Before the Indian nationalist intelligentsia began to associate itself with working class agitations towards the end of the 19th century, there were several agitations, including strikes by workers in the textile mills of Bombay, Calcutta, Ahmedabad, Surat, Madras, Coimbatore, Wardha, and so on, in the railways and in the plantations. However, they were mostly sporadic, spontaneous and unorganized revolts based on immediate economic grievances, and had hardly any wider political implications.

There were also some early attempts at organized efforts to improve the condition of the workers. These efforts were made as early as the 1870s by philanthropists. In 1878, Sorabjee Shapoorji Bengalee tried unsuccessfully to introduce a Bill in the Bombay Legislative Council to limit the working hours for labour. In Bengal, Sasipada Banerjea, a Brahmo Social reformer, set up a Workingmen’s Club in 1870 and brought out a monthly journal called Bharat Sramjeebi (Indian Labour), with the primary idea of educating the workers. In Bombay, Narayan Meghajee Lokhanday brought out an Anglo-Marathi weekly called DinaBandhu (Friend of the Poor) in 1880, and started the Bombay Mill and Millhands’ Association in 1890. Lokhanday held meetings of workers and in one instance sent a memorial signed by 5,500 mill workers, to the Bombay Factory Commission, putting forward some minimum workers’ demands. All these efforts were admittedly of a philanthropic nature and did not represent the beginnings of an organized working class movement. Moreover, these philanthropists did not belong to the mainstream of the contemporary national movement.

The mainstream nationalist movement in fact was as yet, by and large, indifferent to the question of labour. The early nationalists in the beginning paid relatively little attention to the question of workers despite the truly wretched conditions under which they existed at that time. Also, they had a strikingly, though perhaps understandably, differential attitude towards the workers employed in Europeans enterprises and those employed in Indian enterprises.

One major reason for the relatively lukewarm attitude of the early that, at this time, when the anti-imperialist movement was in its very infancy, the nationalists did not wish to, in any way, weaken the common struggle against British rule — the primary task to be achieved in a colonial situation — by creating any divisions within the ranks of the Indian people. Dadabhai Naoroji, in the very second session of the Indian National Congress (1886), made it clear that the Congress ‘must confine itself to questions in which the entire nation has a direct participation, and it must leave the adjustment of social reforms and other class questions to class Congresses.” Later, with the national movement gaining in strength, and the emergence within the nationalist ranks of ideological trends with less inhibitions towards labour and increasingly with an actively pro­ labour orientation, efforts were made to organize labour and secure for it a better bargaining position vis-a -vis the more powerful classes in the common anti-imperialist front. While still endeavouring to maintain an anti-imperialist united front, unity was no longer sought at the unilateral cost of the worker and the oppressed but was to be secured through sacrifices or concessions from all classes including the powerful propertied class.

At this stage, however, the nationalists were unwilling to take up the question of labour versus the indigenous employer. Most of the nationalist newspapers, in fact, denied the need for any Government legislation to regulate working conditions and actively opposed the Factories Act of 1881 and 1891. Similarly, strikes in Indian textiles mills were generally not supported. Apart from the desire not to create any divisions in the fledgling anti-imperialist movement, there were other reasons for the nationalist stance. The nationalists correctly saw the Government initiative on labour legislation as dictated by British manufacturing interests which, when faced with growing Indian competition and a shrinking market in India, lobbied for factor legislation in India which would, for example, by reducing the working hours for labour, reduce the competitive edge enjoyed by Indian industry. Further, the early nationalists saw rapid industrialisation as the panacea for the problems of Indian poverty and degradation and were unwilling to countenance any measure which would impede this process. Labour legislation which would adversely affect the infant industry in India, they said, was like killing the goose that laid the golden eggs. But there was also the nationalist newspaper, Mahratta, then under the influence of the radical thinker, G.S. Agarkar, which even at this stage supported the workers’ cause and asked the mill owners to make concessions to them. This trend was, however, still a very minor one.

The scenario completely altered when the question was of Indian labour employed in British-owned enterprises. Here the nationalists had no hesitation in giving full support to the workers. This was partially because the employer and the employed, in the words of P. Ananda Charlu, the Congress president in 1891, were not ‘part and parcel of the same nation.’

The Indian National Congress and the nationalist newspapers began a campaign against the manner in which the tea plantation workers in Assam were reduced to virtual slavery, with European planters being given powers, through legislation to arrest, punish and prevent the running away of labour. An appeal was made to national honour and dignity to protest against this unbridled exploitation by foreign capitalists aided by the colonial state.

It was not fortuitous, then, that perhaps the first organized strike by any section of the working class should occur ma British-owned and managed railway. This was the signallers’ strike in May 1899 in the Great Indian Peninsular (GIP) Railway and the demands related to wages, hours of work and other conditions of service. Almost all nationalist newspapers came out fully in support of the strike, with Tilak’s newspapers Mahratta and Kesari campaigning for it for months. Public meetings and fund collections in aid of the strikers were organized in Bombay and Bengal by prominent nationalists like Pherozeshah Mehta,

D.E. Wacha and Surendranath Tagore. The fact that the exploiter in these cases was foreign was enough to take agitation against it a national issue and an integral part of national movement. At the turn of the century, with the growth of the working class, there emerged a new tendency among the nationalist intelligentsia. B.C. Pal and G. Subramania Iyer, for example, began to talk of the need for legislation to protect the workers, the weaker section, against the powerful capitalists. In 1903, G. Subramania Iyer urged that workers should combine and organize themselves into unions to fight for their rights and the public must give every help to the workers in achieving this task.

The Swadeshi upsurge of 1903-8 was a distinct landmark in the history of the labour movement. An official survey pinpointed the rise of the ‘professional agitator’ and the ‘power of organization’ of labour into industrial strikes as the two distinct features of this period.4 The number of strikes rose sharply and many Swadeshi leaders enthusiastically threw themselves into the tasks of organizing stable trade unions, strikes, legal aid, and fund collection drives. Public meetings in support of striking workers were addressed by national leaders like B.C. Pal, C.R. Das and Liaqat Hussain. Four prominent names among the Swadeshi leaders who dedicated themselves labour struggles were Aswinicoomar Banerjea, Prabhat Kumar Roy Chowdhuri, Premtosh Bose and Apurba Kumar Ghose were active in a large number of strikes but their greatest success, both in setting up workers’ organizations and in terms of popular support, was among workers in the Government Press, Railways and the jute industry — significantly all areas in which either foreign capital or the colonial state held sway.

Frequent processions in support of the strikers were taken out in the Streets of Calcutta. People fed the processionists on the way. Large numbers including women and even police constables made contributions of money, rice, potatoes, and green vegetables. The first tentative attempts to form all-India unions were also made at this timer but these were unsuccessful. The differential attitude towards workers employed in European enterprises and those in Indian ones, however, persisted throughout this period.

Perhaps the most important feature of the labour movement during the Swadeshi days was the shift from agitations and struggles on purely economic questions to the involvement of the worker with the wider political issues of the day. The labour movement had graduated from relatively unorganized and spontaneous strikes on economic issues to organized strikes on economic issues with the support of the nationalists and then on to working class involvement in wider political movements. The national upsurge on 16 October 1905, the day the partition of Bengal came into effect, included a spurt of working class strikes and hartals in Bengal. Workers in several jute mills and jute press factories, railway coolies and carters, all struck work. Workers numbering 12,000 in the Bum Company shipyard in Howrah struck work on being refused leave to attend the Federation Hall meeting called by the Calcutta Swadeshi leaders. Workers also went on strike when the management objected to their singing Bande Mataram or tying rakhis on each others’ wrists as a symbol of unity.

In Tuticorin, in Tamil Nadu, Subramania Siva campaigned for a strike in February-March 1908 in a foreign-owned cotton mill saying that strikes for higher wages would lead to the demise of foreign mills. When Siva and the famous Swadeshi leader Chidambaram Pillai were arrested, there were widespread strikes and riots in Tuticorin and Tirunelveli. In Rawalpindi, in Punjab, the arsenal and railway engineering workers went on strike as part of the 1907 upsurge in the Punjab which had led to the deportation of Lajpat Rai and Ajit Singh. Perhaps the biggest political demonstration by the working class in this period occurred during Tilak’s trial and subsequent conviction as has already been discussed earlier.

The Swadeshi period was also to see the faint beginnings of a socialist tinge among some of the radical nationalist leaders who were exposed to the contemporary Marxist and social democratic forces in Europe. The example of the working class movement in Russia as a mechanism of effective political protest began to be urged for emulation in India.

With the decline in the nationalist mass upsurge after 1908, the labour movement also suffered an eclipse. It was only with the coming of the next nationalist upsurge in the immediate post World-War I years that the working class movement was to regain its élan, though now on a qualitatively higher plane.

\begin{center}*\end{center}



Beginning with the Home Rule Leagues in 1915 and continuing through the Rowlatt Satyagraha in 1919, the national movement once again reached a crescendo in the Non- Cooperation and Khilafat Movement in 1920-22. It was in this context that there occurred a resurgence of working class activity in the years from 1919 to 1922. The working class now created its own national level organisation to defend its class rights. It was in this period that the working class also got involved in the mainstream of nationalist politics to a significant extent.

The most important development was the formation of the All India Trade Union Congress (AITUC) in 1920 Lokamanya Tilak, who had developed a close association with Bombay work., was one of the moving spirits in the formation of the AITUC, which had Lala Lajpat Rai, the famous Extremist leader from Punjab, as its first president and Dewan Chaman Lal, who was to become a major name in the Indian labour movement, as its General Secretary. In his presidential address to the first AITUC, Lala Lajpat Rai emphasized that, ‘...Indian labour should lose no time to organize itself on a national scale... the greatest need in this Country is to organize, agitate, and educate. We must organize our workers, make them class conscious... ‘ While aware that ‘for some time to come’ the workers will need all the help and guidance and cooperation they can get from such among the intellectuals as are prepared to espouse their cause, he maintained that ‘eventually labour shall find its leaders from among its own ranks.’

The manifesto issued to the workers by the AITUC urged them not only to organize themselves but also to intervene in nationalist politics: ‘Workers of India! . . . Your nation’s leaders ask for Swaraj, you must not let them, leave you out of the reckoning. Political freedom to you is of no worth without economic freedom. You cannot therefore afford to neglect the movement for national freedom. You are part and parcel of that movement. You will neglect it only at the peril of your liberty.”

Lajpat Rai was among the first in India to link capitalism with imperialism and emphasize the crucial of the working class in fighting this combination. He said on 7 November, 1920: ‘India... has... been bled by the forces of organized capital and is today lying prostrate at its feet. Militarism and Imperialism are the twin-children of capitalism; they are one in three and three in one. Their shadow, their fruit and their bark all are poisonous. It is only lately that an antidote has been discovered and that antidote is organized labour.’

Reflecting the emerging change in nationalist attitudes towards labour employed in Indian enterprise, Lajpat Rai said. ‘We are often told that in order successfully to compete with Manchester and Japan, capital in India should be allowed a high rate of profit and cheap labour is a necessity for that purpose . . . We are not prepared to admit the validity of this plea... An appeal to patriotism must affect the rich and the poor alike, in fact, the rich more than the poor . . . Surely . . . the way to develop Indian industries... is to be... (not) at the expense of labour alone... The Indian capitalist must meet labour half way and must come to an understanding with it on the basis of sharing the profits in a reasonable arid just proportion... If, however, Indian capital wants to ignore the needs of labour and can think only of its huge profits, it should expect no response from labour and no sympathy from the general public.’

Similarly second-session-of the AITUC, Dewan Chaman Lal while moving a resolution in favour of Swaraj pointed out that it was to be a Swaraj, not for the capitalists but for the workers.

Apart from Lajpat Rai, several of the leading nationalists of the time became closely associated with the AITUC. C.R. Das presided over its third and fourth sessions, and among the other prominent names were th of C.F. Andrews, J.M. Sengupta, Subhas Bose, Jawaharlal Nehru, and Satyamurti. The Indian National Congress at its Gaya session in 1922 welcomed the formation of the AITUC and formed a committee consisting of prominent Congressmen to assist its work.

C.R. Das, in his presidential address to the Gaya Congress, said that the Congress must ‘take labour and the peasantry in hand... and organize them both from the point of view of their own special interests and also from the point of view of the higher ideal which demands satisfaction of their special interests and the devotion of such interests to the cause of Swaraj.’ If this was not done, he warned, organization of workers arid peasants would come up ‘dissociated from the cause of Swaraj’ and pursuing ‘class struggles and the war of special interest.’

The workings responded to the changed political atmosphere in a magnificent manner. In 1920, there were 125 unions with a total membership of 250,000, and large proportion of these had been formed during 1919-20. The workers’ participation in the major national political events was also very significant. In April 1919, following the repression in Punjab and Gandhiji’s arrest, the working class in Ahmedabad and other parts of Gujarat resorted to strikes, agitations and demonstrations. In Ahmedabad, Government buildings were set on fire, trains derailed, and telegraph wires snapped. Suppression led to at least twenty-eight people being killed and 123 wounded. Waves of working class protest rocked Bombay and Calcutta. Railway workers’ agitations for economic demands and against racial discrimination also coincided with the general anti- colonial mass struggle. Between 1919 and 1921, on several occasions railway workers struck in support of the Rowlatt agitation and the Non-Cooperation and Khilafat Movement. The call for an All-India general strike given by the North Western Railway workers in April l919 got after enthusiastic response in the northern region. Lajpat Jagga has shown that for railwaymen in large parts of the country Gandhiji came to symbolize resistance to colonial rule and exploitation, just as the Indian Railways symbolized the British Empire, ‘the political and commercial will of the Raj.”

In November 1921, at the time of the visit of the Prince of Wales, the workers responded to the Congress call of a boycott by a countrywide general strike. In Bombay, the textile factories were closed and about 1,40,000 workers were on the streets participating in the rioting and attacks on Europeans and Parsis who had gone to welcome the Prince of Wales. The spirit and the urges that moved the workers in these eventful years, the relationship seen between the nationalist upsurge and the workers’ own aspirations, s best expressed in the words of Arjun Atmaram Alwe, an illiterate worker in a Bombay textile mill, who was later to become a major figure in the working class movement: ‘While our struggle . . . was going on in this manner, the drum of political agitation was being beaten in the country. The Congress started a great agitation demanding rights for India to conduct her own administration. At that time we workers understood the meaning of this demand for Swaraj to be only this, that our indebtedness would disappear, the oppression of the moneylender would stop, our wages would increase, and the oppression of the owner on the worker, the kicks and blows with which they belabour us, would stop by legislation, and that as a result of it, the persecution of us workers would come to an end. These and other thoughts came into the minds of us workers, and a good many workers from among us, and I myself, enlisted ourselves as volunteers in the Non-Cooperation movement.”

Any discussion of these years would remain incomplete without mentioning the founding in 1918 by Gandhiji of the Ahmedabad Textile Labour Association (TLA) which, with 14,000 workers on its rolls, was perhaps the largest single trade union of the time. Too often and too casually had Gandhiji’s experiment based on the principle of trusteeship (the capitalist being the trustee of the workers’ interest) and arbitration been dismissed as class collaborationist and against the interests of the workers.

Apart from the fact that the TLA secured one of the highest hikes in wages (27 1t2 per cent) during a dispute in 1918, Gandhiji’s conception of trusteeship also had a radical potential which is usually missed. As Acharya J.B. Kripalani, one of Gandhiji’s staunchest followers, explained: ‘The Trustee by the very term used means that he is not the owner. The owner is one whose interest he is called upon to protect,’ i.e., the worker. Gandhiji himself told the textile workers of Ahmedabad ‘that they were the real masters of the mills and if the trustee, the mill owner, did not act in the interest of the real owners, then the workers should offer Satyagraha to assert their rights.” Gandhiji’s philosophy for labour, with its emphasis on arbitration and trusteeship, also reflected the needs of the anti- imperialist movement which could ill-afford an all-out class war between the constituent classes of the emerging nation.

After 1922, there was again a lull in the working class movement, and a reversion to purely economic struggles, that is, to corporatism. The next wave of working class activity came towards the end of the 1920s, this time spurred by the emergence of a powerful and clearly defined Left Bloc in the national movement.

\begin{center}*\end{center}



It was in the second half of the l920s that a consolidation of various Left ideological trends occurred and began to have a significant impact on the national movement. Various Communist groups in different parts of India had by early 1927 organized themselves into the Workers’ and Peasants’ Parties (WPP), under the leadership of people like S.A. Dange, Muzaffar Ahmed, P.C. Joshi and Sohan Singh Josh. The WPPs, functioning as a left-wing within the Congress, rapidly gained in strength within the Congress organization at the provincial and the all- India levels.

Also, by working within a broad Left from under the WPPs, Communist influence in the trade union movement, marginal till early 1927, had become very strong indeed, by the end of 1928. In Bombay, following the historic six-month-long general strike by the textile workers (April-September 1928), the Communist- led Gimi Kamgar Union (KU) acquired a pre-eminent position. Its membership rose from 324 to 54,000 by the end of 1928. Communist influence also spread to workers in the railways, jute mills, municipalities, paper mills etc., in Bengal and Bombay and in the Burma Oil Company in Madras. In the AITUC too, by the time of the 1928 Jharia session, the broad Left including the Communists had acquired a dominating position. This resulted in the corporatist trend led by people like N.M. Joshi splitting away from the AITUC at the subsequent session presided over by Jawaharlal Nehru. By the end of 1928, the Government was anxiously reporting that ‘there was hardly a single public utility service or industry which had not been affected in whole or in part, by the wave of communism which swept the country.”

The workers under Communist and radical nationalist influence participated in a large number of strikes and demonstrations all over the country between 1922 and 1929. The AITUC in November 1927 took a decision to boycott the Simon Commission and many workers participated in the massive Simon boycott demonstrations. There were also numerous workers’ meetings organized on May Day, Lenin Day, the anniversary of the Russian Revolution, and so on.

The Government, nervous the growing militancy and political involvement of the working class, and especially at the coming together or the nationalist and the Left trends, launched a-two-pronged attack on the labour movement. On the one hands it enacted repressive laws like the Public Safety Act and Trade Disputes Acts and arrested in one swoop virtually the entire radical leadership of the labour movement and launched the famous Meerut Conspiracy Case against them. On the other hand, it attempted, not without some success, to wean away through concessions (for example the appointment of the Royal Commission on Labour in 1929) a substantial section of the labour movement and commit it to the constitutionalist and corporatist mould.

The labour movement suffered a major setback partially due to this Government offensive and partially due to a shift in Stance of the Communist-led wing of the movement. We shall look at this aspect in more detail later on; suffice it to say that from about the end of 1928, the Communists reversed their policy of aligning themselves with and working within the mainstream of the national movement. This led to the isolation of the Communists from the national movement and greatly reduced their hold over even the working class. The membership of the GKU fell from 54,000 in December 1928 to about 800 by the end of 1929. Similarly, the Communists got isolated within the AITUC and were thrown out in the split of 1931.

A CPI document of 1930 clearly brings out the impact of this dissociation from the Civil Disobedience Movement on the workers of Bombay:’ . . . we actually withdrew from the struggle (civil disobedience) and left the field entirely to the Congress. We limited our role to that of a small group. The result was . . . that in the minds of workers there grew an opinion that we are doing nothing and that the Congress is the only organization which is carrying on the fight against imperialism and therefore the workers began to follow the lead of the Congress.”

Nevertheless, workers participated in the Civil Disobedience Movement all over the country. The textile workers of Sholapur, dock labourers of Karachi, transport and mill owners of Calcutta, and the mill workers of Madras heroically clashed with the Government during the movement. In Sholapur, between the 7th and the 16th of May, the textile workers went on a rampage after the police fired to stop an anti-British procession. Government offices, law courts, police stations and railway stations were attacked and rebels virtually took over the city administration for some days. The national flag was hoisted over the town. The Government had to declare martial law to crush the insurgents. Several workers were hanged or sentenced to long-terms of imprisonment.

In Bombay, where the Congress slogan during civil disobedience was that the ‘workers and peasants are the hands and the feet of the Congress,’ about 20,000 workers mostly from the GIP Railway struck work on 4 February 1930. The day Gandhiji breached the salt law, 6 April, a novel form of Satyagraha was launched by the workers of GIP Railwaymen’s Union. Batches of workers went to the suburban stations of North Bombay and prostrated themselves on the tracks with red flags posted in front of them. The police had to open fire to clear the tracks. On 6 July, Gandhi Day was declared by the Congress Working Committee to protest against large scale arrests, and about 50,0O0 people took part in the hartal that day with workers from forty-nine factories downing their tools.

\begin{center}*\end{center}



There was a dip in the working class movement between 1931 and 1936. Neither did the workers take an active part in the Civil Disobedience Movement of 1932-34. The next wave of working class activity came with provincial autonomy and the formation of popular ministries during 1937-l939.

The Communists had, in the meantime, abandoned their suicidal sectarian policies and since 1934 re-enacted the mainstream of nationalist politics. They also rejoined the AITUC in 1935. Left influence in nationalist politics and the trade union movement once again began to grow rapidly. The Communists, the Congress Socialists and the Left nationalists led by Jawaharlal Nehru and Subhas Bose now formed a powerful Left consolidation within the Congress and other mass organizations.

When the campaign for the 1937 elections began, the AITUC, barring a few centres, gave its support to the Congress candidates. The Congress election manifesto declared that the Congress would take steps for the settlement of labour disputes and take effective measures for securing the rig1ts to form unions and go on strike. During the tenure of the Congress Provincial Governments the trade union movement showed a phenomenal rise. Between 1937 and 1939 the number of trade unions increased from 271 to 362 and the total membership of these unions increased from 261,047 to 399,159. The number of strikes also increased considerably.

One of the principal factors which gave a fillip to the trade union movement in this period was the increased civil liberties under the Congress Governments and the pro-labour attitude of many of the Congress ministries. It is significant that a peculiar feature of the strikes in this period was that a majority of them ended successfully, with full or partial victory for the workers.’

World War II began on 3 September 1939 and the working class of Bombay was amongst the first in the world to hold an anti-war strike on 2 October, 1939. About 90,000 workers participated in the strike. There were several strikes on economic issues all over the country despite the severe repression let loose by a government keen to prevent any disruption of the war effort.

However, with the Nazi attack on the Soviet Union in 1941, the Communists argued that the character of the War had changed from an imperialist war to a people’s war. It was now the duty of the working class to support the Allied powers to defeat Fascism which threatened the socialist fatherland. Because of this shift in policy, the Communist party dissociated itself from the Quit India Movement launched by Gandhiji in August 1942. They also successfully followed a policy of industrial peace with employers so that production and war-effort would not be hampered.

The Quit India Movement, however, did not leave the working class untouched, despite the Communist indifference or opposition to it immediately after the arrest of Gandhiji and other leaders on 9 August 1942, following the Quit India Resolution, there were strikes and hartals all over the country, lasting for about a week, by workers in Delhi, Lucknow, Kanpur, Bombay, Nagpur, Ahmedabad, Jamshedpur, Madras, Indore and Bangalore. The Tata Steel Plant was closed for thirteen days with the strikers’ slogan being that they would not resume work till a national government was formed. In Ahmedabad, the textile strike lasted for about three-and-a-half months with the mill owners in their nationalist euphoria actually cooperating! The participation of workers was, however, low in pockets of Communist influence though in many areas the Communist rank and file, actively joined the call of Quit India despite the party line.

\begin{center}*\end{center}



There was a tremendous resurgence in working class activity between 1945-47. The workers in large numbers participated in the post-war political upsurge. They were part of the numerous meetings and demonstrations organized in towns and cities (especially in Calcutta) on the issue of the INA trials. Towards the end of 1945, the Bombay and Calcutta dock workers refused to load ships going to Indonesia with supplies for troops meant to suppress the national liberation struggles of South-East Asia.

Perhaps the most spectacular action by the workers in this period was the strike and hartal by the Bombay workers in solidarity with the mutiny of the naval ratings in 1946. On 22 February, two to three hundred thousand workers downed their tools, responding to a call given by the Communist Party and supported by the Socialists. Peaceful meetings and demonstrations developed into violent clashes as the police intervened. Barricades were set up on the streets which were the scene of pitched battles with the police and the army. Two army battalions were needed to restore order in the city; nearly 250 agitators laid down their lives. The last years of colonial rule also saw a remarkably sharp increase in strikes on economic issues all over the country — the all-India strike of the Post and Telegraph Department employees being the most well known among them. The pent-up economic grievances during the War, coupled with the problems due to post-war demobilization and the continuation of high prices, scarcity of food and other essentials, and a drop in real wages, all combined to drive the working class to the limits of its tolerance. Also, the mood in anticipation of freedom was pregnant with expectation. Independence was seen by all sections of the Indian people as signalling an end to their miseries. The workers were no exception. They too were now struggling for what they hoped freedom would bring them as a matter of right.
