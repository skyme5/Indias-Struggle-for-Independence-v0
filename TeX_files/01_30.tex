
\chapter{The Development of a Nationalist Foreign Policy}

In the course of their own anti-imperialist struggle, the Indian people evolved a policy of opposition to imperialism as also the expression and establishment of solidarity with anti- imperialist movements in other parts of the world. From the beginning, the Indian nationalists opposed the British policy of interfering in the internal affairs of other countries and the use of the Indian army and India's resources to promote, extend and defend British imperialism in Africa and Asia.

\begin{center}*\end{center}

\paragraph*{}

The broad basis for the nationalist foreign policy was laid in the initial years of the national movement, which coincided with a particularly active phase of British imperial expansionism. From 1878 onwards, the Government of India undertook a number of large-scale military expeditions outside India's frontiers and its armed forces were used in some of the wars waged by the British Government in Asia and Africa. These wars and expeditions were a major source of the rapid and massive increase in India's military expenditure. The early Indian national leaders condemned India's involvement in each of these wars and expeditions because of the financial burden of the Indian people, and on grounds of political morality, and also on the basis that these involved not Indian interests and purposes but British imperialist schemes of territorial and commercial expansion. They invariably demanded that the British Government should hear their entire cost. They also argued that India's interests would be best secured by a policy of peace. The Second Afghan War was waged in 1878--80. Voicing the Indian opinion, Surendranath Banerjea publicly branded the war as an act of sheer aggression and `as one of the most unrighteous wars that have blackened the pages of history.'' The Indians demanded that since the unjust war was waged in pursuance of Imperial aims and policies, Britain should meet the entire cost of the war. The 

Amrita Bazar Patrika of 19 March 1880 wrote in its usual vein of irony: `Nothing throws an Englishman into a passion as when his pocket is touched and nothing pleases him more than when he can serve his own interests at the expense of others.' 

In 1882, the Government of India participated in the expedition sent by England to Egypt to put down the nationalist revolt led by Colonel Arabi. Condemning the `aggressive' and `immoral' British policy in Egypt, the Indian nationalists said that the war in Egypt was being waged to protect the interests of British capitalists, merchants and bond-holders. 

At the end of 1885, the Government of India attacked and annexed Burma. With one voice the Indian nationalists condemned the war upon the Burmese people as being immoral, unwarranted, unjust, arbitrary and an act of uncalled for aggression. The motive force behind the policy was once again seen to be the promotion of British commercial interests in Burma and its northern neighbor, China. The nationalists opposed the annexation of Burma and praised the guerrilla fight put up by the Burmese people in the succeeding years. In 1903, Lord Curzon launched an attack upon Tibet. The nationalist attitude was best summarized by R.C. Dutt's denunciation of the `needless, cruel, and useless war in Tibet,' once again motivated by commercial greed and territorial aggrandizement. 

Above all, it was the expansionist, `forward' policy followed by the Government during the 1890s on India's north-western frontier that aroused the Indians' ire. Claiming to safeguard India against Russian designs, the Government of India got involved, year after year, in costly expeditions leading to the deployment of over 60,000 troops against rebellious tribesmen which led to the annexation of more and more new territory and, at the same time, to the continuous draining of the Indian treasury The Indians claimed, on the one hand, that Anglo-Russian rivalry was the result of the clash of interests of the two imperialisms in Europe and Asia, and, on the other hand, that Russian aggression was a bogey, `a monstrous bugbear,' raised to justify' imperialist expansion. The nationalists justified the resistance put up by the frontier tribes in defending their independence. Refusing to accept the official propaganda that the Government's armed actions were provided by the lawlessness and blood-thirstiness of the frontier tribesmen, they condemned the Government for its savage measures in putting down the tribal uprisings. They were quite caustic about the claim of the British Prime Minister, Lord Salisbury that the frontier wars were `but the surf that marks the edge and the advance of the wave of civilization.' `Philanthropy, it is said,' quipped Tilak's Mahratta on 17 October 1897, `is the last resort of the scoundrel and the statesman It is the straw at which they will catch when reason is exhausted and sophistry is exposed.' The Indian leaders argued that the expansionist policy of the Government of India's frontiers, a product of Britain's world-wide imperialist policy, was the most important cause of the maintenance of a large standing army, the increase in Indian military expenditure, the deplorable financial position of the Government, and the consequent increase of taxation in India after 1815. The Indians advocated, instead, a policy of peace, the demand for which was made by C. Sankaran Nair, the Congress President. in 1897 in words that have a remarkably modern and familiar ring: `Our true policy is a peaceful policy ... With such capacity for internal development as our country possesses, with such crying need to carry out the reforms absolutely necessary for our well-being, we want a period of prolonged peace.'3 Three other major themes in the area of nationalist foreign policy emerged during the period 1880--1914. One was that of sympathy and support for people fighting for their independence and liberation. Thus, sentiments of solidarity with the people of Ireland, Russia, Turkey, Burma, Afghanistan, Egypt and Sudan, Ethiopia and other people of Africa were vigorously expressed and popularized through the Press. Foreign intervention in China during the I Ho-Tuan (Boxer) Uprising was vigorously opposed and the despatch of Indian troops to China condemned. The second theme was that of Asia-consciousness. It was during their opposition to the Burma war in 1885 that consciousness of an Asian identity emerged, perhaps for the first time. Some of the nationalist newspapers bemoaned the disappearance of an independent, fellow Asian country. The rise of modern Japan as an industrial power after 1868 was hailed by Indians as proof of the fact that a backward Asian country could develop itself within Western control. .But despite their admiration for Japan, the nationalist newspapers criticized it for attacking China in 1895 and for participating in the international suppression of the I Ho- Tuan uprising. The imperialist effort to partition China was condemned because its success would lead to the disappearance of a major independent Asian power. The defeat of Czarist Russia by Japan further exploded the myth of European solidarity and led to the resurgence of a pan-Asian feeling. 

Indians also began to understand and expound the economic rationale, including the role of foreign capital exports, behind the resurgence of imperialism in the last quarter of the 19th century. Thus, commenting on the reasons behind the attack upon Burma, the Mahratta of 15 November 1885, edited at the time of Tilak and Agarkar, wrote: The truth was `that England with its superfluous human energy and overflowing capital cannot but adhere to the principle of political conduct --- might is right --- for centuries to come in order to find food for her superfluous population and markets for her manufacturers.' Similarly, the Hindu of 23 September 1889 remarked: `Where foreign capital has been sunk in a country, the administration of that country becomes at once the concern of the bond-holders.

\begin{center}*\end{center}

\paragraph*{}

World War I broke out in June 1914. The Indian nationalist leaders, including Lokamanya Tilak, decided to support the war effort of the Government. Sentiments of loyalty to the empire and of the desire to defend it were loudly and widely expressed. But, as Jawaharlal Nehru has pointed out in his Autobiography: `There was little sympathy with the British in spite of loud professions of loyalty. Moderate and Extremist alike learnt with satisfaction of German victories. There was no love for Germany, of course, only the desire to see our own rulers humbled.'4 The hope was that a grateful Britain would repay India's loyalty with economic and political concessions enabling India to take a long step towards self-government, that Britain would apply to India the principles of democracy for which she and the Allies were claiming to be fighting the War. 

After the War, the nationalists further developed their foreign policy in the direction of opposition to political and economic imperialism and Cooperation of all nations in the cause of world peace. As part of this Policy, at its Delhi session in 1919, the Congress demanded India's representation at the Peace Conference through its elected representatives. 

Indians also continued to voice their sympathy for the freedom fight of other countries. The Irish and Egyptian people and the Government of Turkey were extended active support. At its Calcutta session in 1920, the Congress asked the people not to join the army to fight in West Asia. In May 1921, Gandhiji declared that the Indian people would oppose any attack on Afghanistan. The Congress branded the Mandate system of the League of Nations as a cover for imperialist greed. In 1921, the Congress congratulated the Burmese people on their struggle for freedom. Burma was at that time a part of India, but the Congress announced that free India favoured Burma's independence from India. Gandhiji wrote in this context in 1922: `1 have never been able to take pride in the fact that Burma has been made part of British India. It never was and never should be. The Burmese have a civilization of their own.' In 1924, the Congress asked the Indian settlers in Burma to demand no separate rights at the cost of the Burmese people. 

In 1925, the Northern March of the Chinese Nationalist army began under Sun Yat-Sen's leadership and the foreign powers got ready to intervene. The Congress immediately expressed a strong bond of sympathy with the Chinese people in their struggle for national unity and against the common enemy arid protested against the dispatch of Indian troops to China. In 1925, Gandhiji described the use of Indian soldiers to shoot the innocent Chinese students as a `humiliating and degrading spectacle.' `It demonstrates also most forcibly that India is being kept under subjection, not merely for the exploitation of India herself, but that it enables Great Britain to exploit the great and ancient Chinese nation.' 

In January 1927, S. Srinavasa Iyengar moved an adjournment motion in the Central Legislative Assembly to protest against Indian troops being used to suppress the Chinese people. The strong Indian feelings on the question were repeatedly expressed by the Congress during 1927 (including it Its Madras session). The Madras Congress advised Indians not to go to China to fight or work against the Chinese people who were fellow fighters in the struggle against imperialism. It also asked for the withdrawal of Indian troops from Mesopotamia and Iran and all other foreign counties. In 1928, the Congress assured the people of Egypt, Syria, Palestine, Iraq, and Afghanistan of its full support in their national liberation struggles. 

Sentimerns of the solidarity of the Indian people with the colonial people and the awareness of India's role as the gendarme of British imperialism the world over were summed up by Dr. MA. Ansari in his presidential address at the Congress session of 1927: `The history of this philanthropic burglary on the part of Europe is written in blood and suffering from Congo to Canton... Once India is free the whole edifice (of imperialism) will collapse as she is the key-stone of the arch of Imperialism.'

\begin{center}*\end{center}

\paragraph*{}

In 1926--27, Jawaharlal Nehru travelled to Europe and came into contact with left-wing European political workers and thinkers. This had an abiding impact on his political development, including in the field of foreign affairs. This was, of course, not the first time that major Indian political leaders had made an effort to establish links with, and get the support of, the anti- imperialist sections of British and European public opinion. Dadabhai Naoroji was a close friend of the socialist H.M. Hyndman. He attended the Hague session of the International Socialist Congress in August 1904 and after describing imperialism as a species of barbarism declared, that the Indian people had lost all faith in British political parties' and parliament and looked for cooperation only to the British working class. Lajpat Rai also established close relations with American socialists during his stay in the US from 1914--18. In 1917, he opposed US participation in the World War because of the War's imperialistic character. Gandhiji also developed close relations with outstanding European figures such as Tolstoy and Romain Rolland. 

The highlight of Jawaharlal's European visit was his participation as a representative of the Congress in the International Congress against colonial Oppression and Imperialism held in Brussels in February 1927. The basic objective of the Conference was to bring together the colonial people of Africa, Asia and Latin America struggling against imperialism and the working people of the capitalist countries fighting against capitalism. Nehru was elected one of the honorary presidents of the Conference along with Albert Einstein, Romain Rolland, Madame Sun Yat-Sen and George Lansbury. In his speeches and statements at the Conferences, Nehru emphasized the close connection between colonialism and capitalism and the deep commitment of Indian nationalism to internationalism and to anti-colonial struggles the world over. A major point of departure from previous Indian approaches was his understanding of the significance of US imperialism as a result of his discussions with Latin American delegates. In this confidential report on the Conference to the Congress Working Committee, he wrote: `Most of us, specially from Asia, were wholly ignorant of the problems of South America, and of how the rising imperialism of the United States, with its tremendous resources and its immunity from outside attack, is gradually taking a stranglehold of Central and South America. But we are not likely to remain ignorant much longer for the great problem of the near future will be American imperialism, even more than British imperialism.' 

The Brussels Conference decided to found the League Against Imperialism and for National Independence. Nehru was elected to the Executive Council of the League. The Congress also affiliated to the League as an associated member. At its Calcutta session, the Congress declared that the Indian struggle was a part of the worldwide struggle against imperialism. It also decided to open a Foreign Department to develop contacts with other peoples and movements fighting against imperialism. Nor was this understanding confmed to Nehru and other leftists. Gandhiji, for example, wrote to Nehru in September 1933: `We must recognize that our nationalism must not be inconsistent with progressive internationalism.. . I can, therefore, go the whole length with you and say that ``we should range ourselves with the progressive forces of the world.'

\begin{center}*\end{center}

\paragraph*{}

A very active phase of nationalist foreign policy began in 1936. From then onwards, there was hardly an important event in the world to which the Congress and its leaders did not react. Fascism had already triumphed in Italy, Germany and Japan and was raising its ugly head in other parts of the capitalist world. The Congress condemned it as the most extreme form of imperialism and racialism. It fully recognized that the future of India was closely interlinked with the coming struggle between Fascism and the forces of freedom, socialism and democracy. It extended full support to the people of Ethiopia, Spain, China and Czechoslovakia in their struggle against fascist aggression. 

The nationalist approach to world problems was clearly enunciated by Jawaharlal Nehru, the chief Congress spokesperson on world affairs, in his presidential address to the Lucknow Congress in 1936. Nehru analysed the world situation in detail and focused on the Indian struggle in the context of the coming world struggle against Fascism. `Our struggle was but part of a far wider struggle for freedom, and the forces that moved us were moving people all over the world into action... Capitalism, in its difficulties, took to fascism ... what its imperialist counterpart had long been in the subject colonial countries. Fascism and imperialism thus stood out as the two faces of the now decaying capitalism.' And again: `Thus we see the world divided up into two vast groups today --- the imperialist and fascist on one side, the socialist and nationalist on the other. Inevitably we take our stand with the progressive forces of the world which are ranged against fascism and imperialism.''Nehru went back to these themes again and again in the later years. `The frontiers of our struggle lie not only in our own country but in Spain and China also,' he wrote in January l939. 

Gandhiji, too, gave expression to strong anti-fascist feelings. He condemned Hitler for the genocide of the Jews and for `propounding a new religion of exclusive and militant nationalism in the name of which any inhumanity becomes an act of humanity.' `If there ever could be a justifiable war in the name of and for humanity,' he wrote, `a war against Germany, to prevent the wanton persecution of a whole race, would be completely justified.'' 

When Ethiopia was attacked by fascist Italy in early 1936, the Congress declared the Ethiopian people's struggle to be part of all exploited people's struggle for freedom. The Congress declared 9 May to be Ethiopia Day on which demonstrations and meetings were held all over India expressing sympathy and solidarity with the Ethiopians. On his way back from Europe, Jawaharlal refused to meet Mussolini, despite his repeated invitations, lest the meeting was used for fascist propaganda. 

The Congress expressed strong support for Spanish Republicans engaged in a life and death struggle with fascist Franco in the Spanish Civil War. In his presidential address to the Faizpur Congress in December 1936, Nehru emphasized that the struggle going on in Spain was not merely between Republicans and Franco or even Fascism and democracy but between forces of progress and reaction throughout the world. `In Spain today,' he declared, `our battles are being fought and we watch this struggle not merely with the sympathy of friendly outsiders, but with the painful anxiety of those who are themselves involved in it.'' In June 1938, he visited Spain accompanied by Krishna Menon, visited the battlefront and spent five days in Barcelona which was under constant bombardment, on 13 October 1938, Gandhiji sent a message to Juan Negrin. Prime Minister of Spain: `My whole heart goes out to you in sympathy. May true freedom be the outcome of your agony.'' 

In late 1938, Hitler began his diplomatic and political aggression against Czechoslovakia leading to its betrayal by Britain and France at Munich. The Congress Working Committee, meeting in Nehru's absence, passed a resolution viewing `with grave concern the unabashed attempt that is being made by Germany to deprive Czechoslovakia of its independence or to reduce it to impotence,' and sending its `profound sympathy to the brave people of Czechoslovakia.'' Gandhiji wrote in the Harijan: `Let the Czechs know that the Working Committee wrung itself with pain while their doom was being decided.' Speaking for himself, Gandhiji wrote that the plight of the Czechs `moved me to the point of physical and mental distress.''6 Nehru, then in Europe, refused to visit Germany as a state guest and went to Prague instead. He was angry with the British Government for encouraging Germany. In a letter to the Manchester Guardian he wrote: `Recent developments in Czechoslovakia and the way the 

British Government, directly and through its mediators, had baulked and threatened the Czech Government at every turn has produced a feeling of nausea in me.'' He was disgusted with the Munich Agreement and in an article in the National Herald of 5 October 1938, he described it as `the rape of Czechoslovakia by Germany with England and France holding her forcibly down!'' His interpretation of this betrayal of Czechoslovakia was that Britain and France wanted to isolate the Soviet Union and maintain Fascism in Europe as a counterpoise to it. At Tripuri, in early 1939, the Congress passed a resolution dissociating itself `entirely from the British foreign policy, which has consistently aided the fascist Powers and helped the destruction of the democratic countries.' 

In 1937, Japan launched an attack on China. The Congress passed a resolution condemning Japan and calling upon the Indian people to boycott Japanese goods as a mark of their sympathy with the Chinese people. At its Haripura session in early 1938, the Congress reiterated this call while condemning `the aggression of a brutal imperialism in China and horrors and frightfulness that have accompanied it.' It warned that the invasion of China was `fraught with the gravest consequences for the future of the world peace and of freedom in Asia.' As an expression of its solidarity with the Chinese people, 12 June was celebrated throughout India as China Day. The Congress also sent a medical mission, headed by Dr. M. Atal, to work with the Chinese armed forces. One of its members, Dr. Kotnis, was to lay down his life working with the Eighth Route Army under Mao Ze-Dong's command. The complexity, the humanist approach, and anti- imperialist content of the Indian nationalist foreign policy were brought out in the approach to the problem of Palestine. While Arabs were fighting against British imperialism in Palestine, many of the Jews, hunted and killed in Nazi Germany and discriminated against and oppressed all over Europe. were trying to carve out under Zionist leadership a homeland in Palestine with British support. Indians sympathized with the persecuted Jews, victims of Nazi genocide, but they criticized their efforts to deprive the Arabs of their due. They supported the Arabs and urged the Jews to reach an agreement with the Arabs directly. The Congress observed 27 September 1936 as Palestine Day. In October 1937, the Congress protested against the reign of terror in Palestine and the proposal to partition it and assured the Arabs of the solidarity of the Indian people. In September 1938, it again condemned the partition decision, urged the British to `leave the Jews and Arabs to amicably settle the issues between them,' and appealed to the Jews `not to take shelter behind British imperialism.' Gandhiji reiterated all these views in December 1938 in an important editorial in the Harijan on the plight of the Jews in Europe. `My sympathies are all with the Jews,' he wrote. But it would also be `wrong and inhuman to impose the Jews on the Arabs. . - It would be crime against humanity to reduce the proud Arabs.' Appealing to the Jews to reason with the Arabs and `discard the help of the British bayonet,' he pointed out that `as it is, they are co-sharers with the British in despoiling a people who have done no wrong to them.'' Nehru gave expression to similar views on the Palestinian question from 1936 to 1939. 

A major aspect of the nationalists' world outlook, especially of the youth, was the admiration and immense goodwill for the Soviet Union. Nearly all the major Indian political leaders of the time --- for example, Lokamanya Tilak, Lajpat Rai, Bipin Chandra Pal --- had reacted favourably to the Russian Revolution during 1917--18, seeing in it the success of an oppressed people. During the 1920s, the rising socialist and communist groups and young intellectuals were attracted by the Soviet Union, its egalitarianism, socialist idealism, anti-imperialism, and the Five Year Plan and were full of admiration for the socialist homeland. In November 1927, Jawaharlal and Motilal visited the Soviet Union. On his return, Jawaharlal wrote a series of articles for the Hindu which were also published in book form. His reaction was very positive and idealistic and was reflected in the lines he put on the title page of the book: `Bliss was it in that dawn to be alive, but to be young was very heaven.' In 1928 and after Nehru 9 repeatedly praised the Soviet Union `as the greatest opponent of imperialism,' this admiration for the Soviet Union was to deepen as he came more and more under the influence of Marxism. At Lucknow, in 1936, he said that though he was pained and disagreed with much that was happening in the Soviet Union, he looked upon `that great and fascinating unfolding of a new order and a new civilization as the most promising feature of our dismal age.' In fact, `if the future is full of hope it is largely because of Soviet Russia.' The mass trials and purges of Stalin's opponents in the 1930s repelled him, but he still retained his faith in the Soviet regime, especially as, in his view, it `stood as the one real effective bulwark against fascism in Europe and Asia.' 

Other Congress leaders, for example, C.R. Das and Gandhiji were also friendly to the Soviet Union but were put off by what they believed to be the Communist emphasis on the role of violence. This was: particularly true of Gandhiji. But he, too, gradually began to change his appraisal. In a discussion with students of Gujarat Vidyapith in late 1928, Gandhiji, on one hand, praised the Bolshevik ideal of the abolition of the `institution of private property' and, on the other, condemned the Bolsheviks for accomplishing it through violence. While predicting the downfall of the Bolshevik regime, he said: `If it continued to base itself on violence, there is no questioning the fact that the Bolshevik ideal has behind it the purest sacrifice of countless men and women who have given up their all for its sake, and an ideal that is sanctified by the sacrifices of such master spirits as Lenin cannot go in vain; the noble example of their renunciation will be emblazoned for ever and quicken and purify the ideal as time passes.' 

Goodwill, admiration and support for the Soviet Union were to acquire major proportions during the I 930s as the Communist Party, the Congress Socialist Party, the kisan sabhas, and trade unions developed and in their propaganda and agitation cited the Soviet Union as an example of what workers' and peasants' power could achieve.

\begin{center}*\end{center}

\paragraph*{}

War clouds had begun to gather again around the world since the late 1920s. The Congress had declared at its Madras session in 1927 that India could not be a party to an imperialist war and in no case should India be made to join a war without the consent of its people. This declaration was to become one of the foundations of nationalist foreign policy in the later years and was repeated time and again. The rise of Fascism and the threat it posed to peace, democracy and socialism and to the independence of nations transformed the situation to a certain extent. As pointed out earlier, the Indian national leadership was firmly opposed to Fascism and the fascist drive towards war and conquest. At the same time, it was afraid that Britain would go to war, when it did, not in defence of peace and democracy but to protect its imperialist interests. Indian could not support an imperialist war. Moreover, imperialism itself was a major cause of war. Imperialism must disappear if the fascist threat was to be successfully met; and lasting peace could be established only if the domination and exploitation of one nation by another was ended. The character of the war in which Britain participated would be determined by its attitude towards India's freedom. For enslaved India could not fight for the freedom of others. India could, and would, actively support an anti-fascist war provided its independence was immediately recognized. On the other hand, the Congress repeatedly declared, during 1936--39, it would resist every effort to use Indian men, money and resources in a war to, serve British imperialism. Summing up the nationalist position, Nehru wrote on 18 April 1939: `For us in India our path is clear. It is one of complete opposition to the fascists; it is also one of opposition to imperialism. We are not going to line up under Chamberlainism; we are not going to throw our resources in defence of empire. But we would gladly offer those very resources for the defence of democracy, the democracy of a free India lined up with other free countries.' This position was reiterated by the Congress Working Committee meeting in the second week of August 1939, virtually on the eve of war. Because of this commitment to non-violence, Gandhiji had a basic difference with this approach. But he agreed to go along. The Congress position was to be sorely tested in the coming three years.
