\chapter{Peasant Movements and Uprisings After 1857}
\begin{multicols}{2}

It is worth taking a look at the effects of colonial exploitation of the Indian peasants. Colonial economic policies, the new land revenue system, the colonial administrative and judicial systems, and the ruin of handicraft leading to the over-crowding of land, transformed the agrarian structure and impoverished the peasantry. In the vast zamindari areas, the peasants were left to the tender mercies of the zamindars who rack-rented them and compelled them to pay the illegal dues and perform beggar. In Ryotwari areas, the Government itself levied heavy land revenue. This forced the peasants to borrow money from the moneylenders. Gradually, over large areas, the actual cultivators were reduced to the status of tenants-at-will, share-croppers and landless laborers, while their lands, crops and cattle passed into the hands of landlords, trader-moneylenders and rich peasants.

When the peasants could take it no longer, they resisted against the oppression and exploitation; and, they found whether their target was the indigenous exploiter or the colonial administration, that their real enemy, after the barriers were down, was the colonial state.

One form of elemental protest, especially when individuals and small groups found that collective action was not possible though their social condition was becoming intolerable, was to take to crime. Many dispossessed peasants took to robbery, dacoity and what has been called social banditry, preferring these to starvation and social degradation.

\begin{center}*\end{center}

\paragraph*{}
The most militant and widespread of the peasant movements was the Indigo Revolt of 1859--60. The indigo planters, nearly all Europeans, compelled the tenants to grow indigo which they processed in factories set up in rural (mofussil) areas. From the beginning, indigo was grown under an extremely oppressive system which involved great loss to the cultivators. The planters forced the peasants to take a meager amount as advance and enter into fraudulent contracts. The price paid for the indigo plants was far below the market price. The comment of the Lieutenant Governor of Bengal, J.B. Grant, was that `the root of the whole question is the struggle to make the raiyats grow indigo plant, without paying them the price of it.' The peasant was forced to grow indigo on the best land he had whether or not he wanted to devote his land and labour to more paying crops like rice. At the time of delivery, he was cheated even of the due low price. He also had to pay regular bribes to the planter's officials. He was forced to accept an advance. Often he was not in a position to repay it, but even if he could he was not allowed to do so. The advance was used by the planters to compel him to go on cultivating indigo.

Since the enforcement of forced and fraudulent contracts through the courts was a difficult and prolonged process, the planters resorted to a reign of terror to coerce the peasants. Kidnapping, illegal confinement in factory go-downs, flogging, attacks on women and children, carrying off cattle, looting, burning and demolition of houses and destruction of crops and fruit trees were some of the methods used by the planters. They hired or maintained bands of lathyals (armed retainers) for the purpose.

In practice, the planters were also above the law. With a few exceptions, the magistrates, mostly European, favored the planters with whom they dined and hunted regularly. Those few who tried to be fair were soon transferred. Twenty-nine planters and a solitary Indian zamindar were appointed as Honorary Magistrates in 1857, which gave birth to the popular saying `je rakhak se bhakak' (Our protector is also our devourer).

The discontent of indigo growers in Bengal boiled over in the autumn of 1859 when their case seemed to get Government support. Misreading an official letter and exceeding his authority, Hem Chandra Kar, Deputy Magistrate of Kalaroa, published on 1859-08-17, a proclamation to policemen that `in case of disputes relating to Indigo Ryots, they (ryots) shall retain possession of their own lands, and shall sow on them what crops they please, and the Police will be careful that no Indigo Planter nor anyone else be able to interface in the matter.

The news of Kar's proclamation spread all over Bengal, and peasant felt that the time for overthrowing the hated system had come. Initially, the peasants made an attempt to get redressal through peaceful means. They sent numerous petitions to the authorities and organized peaceful demonstrations. Their anger exploded in 1859-09-00, when they asserted their right not to grow indigo under duress and resisted the physical pressure of the planters and their lathiyals backed by the police and the courts.

The beginning was made by the ryots of Govindpur village in Nadia district when, under the leadership of Digambar Biswas and Bishnu Biswas, ex-employees of a planter, they gave up indigo cultivation. And when, on 13 September, the planter sent a band of 100 lathyals to attack their village, they organized a counter force armed with lathis and spears and fought back.

The peasant disturbances and indigo strikes spread rapidly to other areas. The peasants refused to take advances and enter into contracts, pledged not to sow indigo, and defended themselves from the planters' attacks with whatever weapons came to hand --- spears, slings, lathis, bows and arrows, bricks, bhel-fruit, and earthen-pots (thrown by women).

The indigo strikes and disturbances flared up again in the spring of 1860 and encompassed all the indigo districts of Bengal. Factory after factory was attacked by hundreds of peasants and village after village bravely defended itself. In many cases, the efforts of the police to intervene and arrest peasant leaders were met with an attack on policemen and police posts.

The planters then attacked with another weapon, their zamindari powers. They threatened the rebellious ryots with eviction or enhancement of rent. The ryots replied by going on a rent strike. They refused to pay the enhanced rents; and they physically resisted attempts to evict them. They also gradually learnt to use the legal machinery to enforce their rights. They joined together and raised funds to fight court cases filed against them, and they initiated legal action on their own against the planters. They also used the weapon of social boycott to force a planter's servants to leave him.

Ultimately, the planters could not withstand the united resistance of the ryots, and they gradually began to close their factories. The cultivation of indigo was virtually wiped out from the districts of Bengal by the end of 1860.

A major reason for the success of the Indigo Revolt was the tremendous initiative, cooperation, organization and discipline of the ryots. Another was the complete unity among Hindu and Muslim peasants. Leadership for the movement was provided by the more well-off ryots and in some cases by petty zamindars, moneylenders and ex-employees of the planters.

A significant feature of the Indigo Revolt was the role of the intelligentsia of Bengal which organized a powerful campaign in support of the rebellious peasantry. It carried on newspaper campaigns, organized mass meetings, prepared memoranda on peasants' grievances and supported them in their legal battles. Outstanding in this respect was the role of Harish Chandra Mukherji, editor of the Hindoo Patriot. He published regular reports from his correspondents in the rural areas on planters' oppression, officials' partisanship and peasant resistance. He himself wrote with passion, anger and deep knowledge of the problem which, he raised to a high political plane. Revealing an insight into the historical and political significance of the Indigo Revolt, he wrote in May 1860: Bengal might well be proud of its peasantry. . Wanting power, wealth, political knowledge and even leadership, the peasantry of Bengal have brought about a revolution inferior in magnitude and importance to none that has happened in the social history of any other country ... With the Government against them, the law against them, the tribunals against them, the Press against them, they have achieved a success of which the benefits will reach all orders and the most distant generations of our countrymen.'

Din Bandhu Mitra's play, Neel Darpan, was to gain great fame for vividly portraying the oppression by the planters.

The intelligentsia's role in the Indigo Revolt was to have an abiding impact on the emerging nationalist intellectuals. In their very political childhood they had given support to a popular peasant movement against the foreign planters. This was to establish a tradition with long run implications for the national movement.

Missionaries were another group which extended active support to the indigo ryots in their struggle.

The Government's response to the Revolt was rather restrained and not as harsh as in the case of civil rebellions and tribal uprisings. It had just undergone the harrowing experience of the Santhal hool uprising and the Revolt of 1857. It was also able to see, in time, the changed temper of the peasantry and was influenced by the support extended to the Revolt by the intelligentsia and the missionaries. It appointed a commission to inquire into the problem of indigo cultivation. Evidence brought before the Indigo Commission and its final report exposed the coercion and corruption underlying the entire system of indigo cultivation. The result was the mitigation of the worst abuses of the system. The Government issued a notification in November 1860 that ryots could not be compelled to sow indigo and that it would ensure that all disputes were settled by legal means. But the planters were already closing down the factories they felt that they could not make their enterprises pay without the use of force and fraud.

\begin{center}*\end{center}

\paragraph*{}
Large parts of East Bengal were engulfed by agrarian unrest during the 1870s and early 1880s. The unrest was caused by the efforts of the zamindars to enhance rent beyond legal limits and to prevent the tenants from acquiring occupancy rights under Act X of 1859. This they tried to achieve through illegal coercive methods such as forced eviction and seizure of crops and cattle as well as by dragging the tenants into costly litigation in the courts.

The peasants were no longer in a mood to tolerate such oppression. In 1873-05-00, an agrarian league or combination was formed in Yusufshahi Parganah in Pabna district to resist the demands of the zamindars. The league organized mass meetings of peasants. Large crowds of peasants would gather and march through villages frightening the zamindars and appealing to other peasants to join them. The league organized a rent-strike --- the ryots were to refuse to pay the enhanced rents --- and challenged the zamindars in the courts. Funds were raised from the ryots to meet the costs. The struggle gradually spread throughout Pabna and then to the other districts of East Bengal. Everywhere agrarian leagues were organized, rents were withheld and zamindars fought in the courts. The main form of struggle was that of legal resistance. There was very little violence --- it only occurred when the zamindars tried to compel the ryots to submit to their terms by force. There were only a few cases of looting of the houses of the zamindars. A few attacks on police stations took place and the peasants also resisted attempts to execute court decrees. But such cases were rather rare. Hardly any zamindar or zamindar`s agent was killed or seriously injured. In the course of the movement, the ryots developed a strong awareness of the law and their legal rights and the ability to combine and form associations for peaceful agitation.

Though peasant discontent smoldered till 1885, many of the disputes were settled partially under official pressure and persuasion and partially out of the zamindar`s fear that the united peasantry would drag them into prolonged and costly litigation. Many peasants were able to acquire occupancy rights and resist enhanced rents.

The Government rose to the defense of the zamindars wherever violence took place. Peasants were then arrested on a large sale. But it assumed a position of neutrality as far as legal battles or peaceful agitations were concerned. The Government also promised to undertake legislation to protect the tenants from the worst aspects of zamindari oppression, a promise it fulfilled however imperfectly in 1885 when the Bengal Tenancy Act was passed.

What persuaded the zamindars and the colonial regime to reconcile themselves to the movement was the fact that its aims were limited to the redressal of the immediate grievances of the peasants and the enforcement of the existing legal rights and norms. It was not aimed at the zamindari system. It also did not have at any stage an anti-colonial political edge. The agrarian leagues kept within the bounds of law, used the legal machinery to fight the zamindars, and raised no anti-British demands. The leaders often argued that they were against zamindars and not the British. In fact, the leaders raised the slogan that the peasants want `to be the ryots of Her Majesty the Queen and of Her only.' For this reason, official action was based on the enforcement of the Indian Penal Code and it did not take the form of armed repression as in the case of the Santhal and Munda uprisings. Once again the Bengal peasants showed complete Hindu-Muslim solidarity, even though the majority of the ryots were Muslim and the majority of zamindars Hindu. There was also no effort to create peasant solidarity on the grounds of religion or caste.

In this case, too, a number of young Indian intellectuals supported the peasants' cause. These included Bankim Chandra Chatterjea and R.C. Dutt. Later, in the early 1880s, during the discussion of the Bengal Tenancy Bill, the Indian Association, led by Surendranath Banerjee, Anand Mohan Bose and Dwarkanath Ganguli, campaigned for the rights of tenants, helped form ryot' unions, and organized huge meetings of upto 20,000 peasants in the districts in support of the Rent Bill. The Indian Association and many of the nationalist newspapers went further than the Bill. They asked for permanent fixation of the tenant's rent. They warned that since the Bill would confer occupancy rights even on non-cultivators, it would lead to the growth of middlemen --- the jotedars --- who would be as oppressive as the zamindars so far as the actual cultivators were concerned. They, therefore, demanded that the right of occupancy should go with actual cultivation of the soil, that is, in most cases to the under ryots and the tenants-at-will.

\begin{center}*\end{center}

\paragraph*{}
A major agrarian outbreak occurred in the Poona and Ahmednagar districts of Maharashtra in 1875. Here, as part of the Ryotwari system, land revenue was settled directly with the peasant who was also recognized as the owner of his land. Like the peasants in other Ryotwari areas, the Deccan peasant also found it difficult to pay land revenue without getting into the clutches of the moneylender and increasingly losing his land. This led to growing tension between the peasants and the moneylenders most of whom were outsiders --- Marwaris or Gujaratis.

Three other developments occurred at this time. During the early 1860s, the American Civil War had led to a rise in cotton exports which had pushed up prices. The end of the Civil War in 1864 brought about an acute depression in cotton exports and a crash in prices. The ground slipped from under the peasants' feet. Simultaneously, in 1867, the Government raised land revenue by nearly 50 per cent. The situation was worsened by a succession of bad harvests.

To pay the land revenue under these conditions, the peasants had to go to the moneylender who took the opportunity to further tighten his grip on the peasant and his land. The peasant began to turn against the perceived cause of his misery, the moneylender. Only a spark was needed to kindle the fire.

A spontaneous protest movement began in 1874-12-00, in Kardab village in Sirur taluq. When the peasants of the village failed to convince the local moneylender, Kalooram, that he should not act on a court decree and pull down a peasant's house, they organized a complete social boycott of the `outsider' moneylenders to compel them to accept their demands a peaceful manner. They refused to buy from their shops. No peasant would cultivate their fields. The bullotedars (village servants) --- barbers, washermen, carpenters, ironsmiths, shoemakers and others would not serve them. No domestic servant would work in their houses and when the socially isolated moneylenders decided to run away to the taluq headquarters, nobody would agree to drive their carts. The peasants also imposed social sanctions against those peasants and bullotedars who would not join the boycott of moneylenders. This social boycott spread rapidly to the villages of Poona, Ahmednagar, Sholapur and Satara districts.

The social boycott was soon transformed into agrarian riots when it did not prove very effective. On 1875-05-12, peasants gathered in Supa, in Bhimthari taluq, on the bazar day and began a systematic attack on the moneylenders' houses and shops. They seized and publicly burnt debt bonds and deeds --- signed under pressure, in ignorance, or through fraud --- decrees, and other documents dealing with their debts. Within days the disturbances spread to other villages of the Poona and Ahmednagar districts. There was very little violence in this settling of accounts. Once the moneylenders' instruments of oppression --- debt bonds --- were surrendered, no need for further violence was felt. In most places, the `riots' were demonstrations of popular feeling and of the peasants' newly acquired unity and strength. Though moneylenders' houses and shops were looted and burnt in Supa, this did not occur in other places.

The Government acted with speed and soon succeeded in repressing the movement. The active phase of the movement lasted about three weeks, though stray incidents occurred for another month or two. As in the case of the Pabna Revolt, the Deccan disturbances had very limited objectives. There was once again an absence of anti-colonial consciousness. It was, therefore, possible for the colonial regime to extend them a certain protection against the moneylenders through the Deccan Agriculturists' Relief Act of 1879. Once again, the modern nationalist intelligentsia of Maharashtra supported the peasants' cause. Already, in 1873­74, the Poona Sarvajanik Sabha, led by Justice Ranade, had organized a successful campaign among the peasants, as well as at Poona and Bombay against the land revenue settlement of 1867. Under its impact, a large number of peasants had refused to pay the enhanced revenue. This agitation had generated a mentality of resistance among the peasants which contributed to the rise of peasant protest in 1875. The Sabha as well as many of the nationalist newspapers also supported the D.A. R. Bill.

Peasant resistance also developed in other parts of the country. Mappila outbreaks were endemic in Malabar. Vasudev Balwant Phadke, an educated clerk, raised a Ramosi peasant force of about 50 in Maharashtra during 1879, and organized social banditry on a significant scale. The Kuka Revolt in Punjab was led by Baba Ram Singh and had elements of a messianic movement. It was crushed when 49 of the rebels were blown up by a cannon in 1872. High land revenue assessment led to a series of peasant riots in the plains of Assam during 1893--94. Scores were killed in brutal firings and bayonet charges.

\begin{center}*\end{center}

\paragraph*{}
There was a certain shift in the nature of peasant movements after 1857. Princes, chiefs and landlords having been crushed or co-opted, peasants emerged as the main force in agrarian movements. They now fought directly for their own demands, centered almost wholly on economic issues, and against their immediate enemies, foreign planters and indigenous zamindaris and moneylenders. Their struggles were directed towards specific and limited objectives and redressal of particular grievances. They did not make colonialism their target. Nor was their objective the ending of the system of their subordination and exploitation. They did not aim at turning the world upside down.

The territorial reach of these movements was also limited. They were confined to particular localities with no mutual communication or linkages. They also lacked continuity of struggle or long-term organization. Once the specific objectives of a movement were achieved, its organization, as also peasant solidarity built around it, dissolved and disappeared. Thus, the Indigo strike, the Pabna agrarian leagues and the social-boycott movement of the Deccan ryots left behind no successors. Consequently, at no stage did these movements threaten British supremacy or even undermine it.

Peasant protest after 1857 often represented an instinctive and spontaneous response of the peasantry to its social condition. It was the result of excessive and unbearable oppression, undue and unusual deprivation and exploitation, and a threat to the peasant's existing, established position. The peasant often rebelled only when he felt that it was not possible to carry on in the existing manner.

He was also moved by strong notions of legitimacy, of what was justifiable and what was not. That is why he did not fight for land ownership or against landlordism but against eviction and undue enhancement of rent. He did not object to paying interest on the sums he had borrowed; he hit back against fraud and chicanery by the moneylender and when the latter went against tradition in depriving him of his land. He did not deny the state's right to collect a tax on land but objected when the level of taxation overstepped all traditional bounds. He did not object to the foreign planter becoming his zamindar but resisted the planter when he took away his freedom to decide what crops to grow and refused to pay him a proper price for his crop.

The peasant also developed a strong awareness of his legal rights and asserted them in and outside the courts. And if an effort was made to deprive him of his legal rights by extra-legal means or by manipulation of the law and law courts, he countered with extra-legal means of his own. Quite often, he believed that the legally-constituted authority approved his actions or at least supported his claims and cause. In all the three movements discussed here, he acted in the name of this authority, the sarkar.

In these movements, the Indian peasants showed great courage and a spirit of sacrifice, remarkable organizational abilities, and a solidarity that cut across religious and caste lines. They were also able to wring considerable concessions from the colonial state. The latter, too, not being directly challenged, was willing to compromise and mitigate the harshness of the agrarian system though within the broad limits of the colonial economic and political structure. In this respect, the colonial regime's treatment of the post-1857 peasant rebels was qualitatively different from its treatment of the participants in the civil rebellions, the Revolt of 1857 and the tribal uprisings which directly challenged colonial political power. A major weakness of the 19th century peasant movements was the lack of an adequate understanding of colonialism --- of colonial economic structure and the colonial state --- and of the social framework of the movements themselves. Nor did the 19th century peasants possess a new ideology and a new social, economic and political programme based on an analysis of the newly constituted colonial society. Their struggles, however militant, occurred within the framework of the old societal order. They lacked a positive conception of an alternative society --- a conception which would unite the people in a common struggle on a wide regional and all-India plane and help develop long-term political movements. An all-India leadership capable of evolving a strategy of struggle that would unify and mobilize peasants and other sections of society for nation-wide political activity could be formed only on the basis of such a new conception, such a fresh vision of society. In the absence of such a flew ideology, programme, leadership and strategy of struggle, it was not to difficult for the colonial state, on the one hand, to reach a Conciliation and calm down the rebellious peasants by the grant of some concessions arid on the other hand, to suppress them with the full use of its force. This weakness was, of course, not a blemish on the character of the peasantry which was perhaps incapable of grasping on its own the new and complex phenomenon of colonialism. That needed the efforts of a modem intelligentsia which was itself just coming into existence.

Most of these weaknesses were overcome in the 20th century when peasant discontent was merged with the general anti-imperialist discontent and their political activity became a part of the wider anti-imperialist movement. And, of course, the peasants' participation in the larger national movement not only strengthened the fight against the foreigner it also, simultaneously, enabled them to organize powerful struggles around their class demands and to create modem peasant organization.
\end{multicols}