
\chapter{The Home Rule Movement and it's Fallout}

The romantic adventure of the Ghadar revolutionaries was the dramatic response of Indians living abroad to the First World War. We now turn to the less charged, but more effective, Indian response --- the Home Rule League Movement, led by Lokamanya Tilak and Annie Besant.

\begin{center}*\end{center}

\paragraph*{}

On 16 June 1914, Bal Gangadhar Tilak was released after serving a prison sentence of six years, most of which he had spent in Mandalay in Burma. He returned to India very different to the one he had been banished from. Aurobindo Ghose, the firebrand of the Swadeshi days, had taken sanyas in Pondicherry, and Lala Lajpat Rai was away in the United States of America. The Indian National Congress had yet to recover from the combined effects of the split at Surat in 1907, the heavy government repression of the activists of the Swadeshi Movement, and the disillusionment of the Moderates with the constitutional reforms of 1909.

Tilak initially concentrated all his attention on seeking readmission, for himself and other Extremists, into the Indian National Congress. He was obviously convinced that the sanction of this body, that had come to symbolize the Indian national movement, was a necessary pre-condition for the success of any political action. To conciliate the Moderates and convince them of his bonafides, as well as to stave off any possible government repression, he publicly declared: I may state once for all that we are trying in India, as the Irish Home-rulers have been doing in Ireland, for a reform of the system of administration and not for the overthrow of Government; aid I have no hesitation in saying that the acts of violence which had been committed in the different Parts of India are not only repugnant to me, but have, in my opinion, only unfortunately retarded to a great extent, the pace of our political progress.'' He further assured the Government of his loyalty to the Crown and urged all Indians to assist the British Government in its hour of crisis.

Many of the Moderate leaders of the Congress were also unhappy with the choice they had made in 1907 at Surat, and also with the fact that the Congress had lapsed into almost total inactivity. They were, therefore, quite sympathetic to Tilak's overtures. Further, they were under considerable pressure from Mrs. Annie Besant, who had just joined the Indian National Congress and was keen to arouse nationalist political activity, to admit the Extremists.

Annie Besant, already sixty-six in 1914, had begun her political career in England as a proponent of Free Thought, Radicalism, Fabianism and Theosophy, and had come to India in 1893 to work for the Theosophical Society. Since 1907, she had been spreading the message of Theosophy from her headquarters in Adyar, a suburb of Madras, and had gained a large number of followers among the educated members of many communities that had experienced no cultural revival of their own. In 1914, she decided to enlarge the sphere of her activities to include the building of a movement for Home Rule on the lines of the Irish Home Rule League. For this, she realized it was necessary both to get the sanction of the Congress, as well as the active cooperation of the Extremists. She devoted her energies, therefore, to persuading the Moderate leaders to open the doors of the Congress to' Tilak and his fellow- Extremists. But the annual Congress session in December 1914 was to prove a disappointment --- Pherozeshah Mehta and his Bombay Moderate group succeeded, by winning over Gokhale and the Bengal Moderates, in keeping out the Extremists. Tilak and Besant thereupon decided to revive political activity on their own, while maintaining their pressure on the Congress to re-admit the Extremist group.

\begin{center}*\end{center}

\paragraph*{}

In early 1915, Annie Besant launched a campaign through her two papers, New India and Commonweal, and organized public meetings and conferences to demand that India be granted self-government on the lines of the White colonies after the War. From April 1915, her tone became more peremptory and her stance more aggressive.

Meanwhile, Lokamanya began his political activities, but, not yet saving gained admittance into the Congress, was careful that he did not in any way alarm the Moderates or appear to be by-passing the Congress. This is clear from the fact that at the meeting of his followers convened at Poona in May 1915, it was decided that their initial phase of action would be to set up an agency `to enlighten the villagers regarding the objects and work of the Congress.'2 The local associations that were set up in many Maharashtra towns in August and September of that year also concentrated more on emphasizing the need for unity in the Congress than on the stepping up of political activity. While sometimes resorting to threats to pressurize the more conservative among the Moderates, Tilak still hoped to persuade the majority to accept him because of his reasonableness and caution.

His efforts and those of Annie Besant were soon to meet with success, and at the annual session of the Congress in December 1915 it was decided that the Extremists be allowed to rejoin the Congress. The opposition from the Bombay group had been greatly weakened by the death of Pherozeshah Mehta. But Annie Besant did not succeed in getting the Congress and the Muslim League to support her decision to set up Home Rule Leagues. She did manage, however, to persuade the Congress to commit itself to a programme of educative propaganda and to a revival of the local level Congress committees. Knowing that the Congress, as constituted at the time, was unlikely to implement this, she had inserted a condition by which, if the Congress did not start this activity by September 1916, she would be free to set up her own League.

Tilak, not bound by any such commitment, and having gained the right of readmission, now took the lead and set up the Home Rule League at the Bombay Provincial Conference held at Belgaum in April 1916. Annie Besant's impatient followers, unhappy with her decision to wait till September, secured her permission to start Home Rule groups. Jamnadas Dwarkadas, Shankerlal Banker and Indulal Yagnik set up a Bombay paper Young India and launched an All India Propaganda Fund to publish pamphlets in regional languages and in English. In September 1916, as there were no signs of any Congress activity, Annie Besant announced the formation of her Home Rule League, with George Arundale, her Theosophical follower, as the Organizing Secretary. The' two Leagues avoided any friction by demarcating their area of activity; Tilak's League was to work in Maharashtra, (excluding Bombay city), Karnataka, the Central Provinces and Berar, and Annie Besant's League was given charge of the rest of India. The reason the two Leagues did not merge was because, in Annie Besant's words, `some of his followers disliked me and some of mine disliked him. We, however, had no quarrel with each other.''

Tilak promoted the Home Rule campaign with a tour of Maharashtra and through his lectures clarified and popularized the demand for Home Rule. `India was like a son who had grown up and attained maturity it was right now that the trustee or the father should give him what was his due. The people of India must get this effected. They have a right to do so.' He also linked up the question of Swaraj with the demand for the formation of linguistic states and education in the vernacular. `Form one separate state each for Marathi, Telugu and Kanarese provinces ... The principle that education should be given through the vernaculars is self- evident and clear. Do the English educate their people through the French language? Do Germans do it through English or the Turks through French?' At the Bombay Provincial Conference in 1915, he told V.B. Alur who got up to support his condolence resolution on Gokhale's death: `Speak in Kannada to establish the right of Kannada language.' It is clear that the Lokamanya had no trace of regional or linguistic Marathi chauvinism.

His stand on the question of non-Brahmin representation and on the issue of untouchability demonstrated that he was no casteist either. When the non-Brahmins in Maharashtra sent a separate memorandum to the Government dissociating themselves from the demands of the advanced classes, Tilak urged those who opposed this to be patient: `If we can prove to the non-Brahmins, by example, that we are wholly on their side in their demands from the Government, I am sure that in times to come their agitation, now based on social inequality, will merge into our struggle.' To the non-Brahmins, he explained that the real difference was not between Brahmin and non-Brahmin, but between the educated and the non-educated. Brahmins were ahead of others in jobs because they were more educated, and the Government, in spite of its sympathy for non-Brahmins and hostility towards Brahmins, was forced to look to the needs of the administration and give jobs to Brahmins. At a conference for the removal of untouchability, Tilak declared: `If a God were to tolerate untouchability, I would not recognize him as God at all.'

Nor can we discern in his speeches of this period any trace of religious appeal; the demand for Home Rule was made on a wholly secular basis. The British were aliens not because they belonged to another religion but because they did not act in the Indian interest. `He who does what is beneficial to the people of this country, be he a Muhammedan or an Englishman, is not alien. `Alienness' has to do with interests. Alienness is certainly not concerned with white or black skin ... or re1igion.'

Tilak's League furthered its propaganda efforts by publishing six Marathi and two English pamphlets, of which 47,000 copies were sold. Pamphlets were brought out in Gujarati and Kannada as well. The League was organized into six branches, one each in Central Maharashtra, Bombay city, Karnataka, and Central Provinces, and two in Berar.

As soon as the movement for Home Rule began to gather steam, the Government hit back, and it chose a particularly auspicious day for the blow. The 23rd of July, 1916, was Tilak's sixtieth birthday, and, according to custom, it was the occasion for a big celebration. A purse of Rs. one lakh was presented to him. The same day the Government offered him their own present: a notice asking him to show cause why he should not be bound over for good behavior for a period of one year and demanding securities of Rs. 60,000. For Tilak, this was the best gift he could have wanted for his birthday. `The Lord is with us,' he said, `Home Rule will now spread like wildfire.'9 Repression was sure to fan the fire of revolt.

Tilak was defended by a team of lawyers led by Mohammed Au Jinnah. He lost the case in the Magistrate's Court but was exonerated by the High Court in November. The victory was hailed all over the country. Gandhiji's Young India summed up the popular feeling: `Thus, a great victory has been won for the cause of Home Rule which has, thus, been freed from the chains that were sought to be put upon it.''° Tilak immediately pushed home the advantage by proclaiming in his public speeches that Home Rule now had the sanction of the Government and he and his colleagues intensified their propaganda campaign for Home Rule. By April 1917 Tilak had enlisted 14,000 members.

\begin{center}*\end{center}

\paragraph*{}

Meanwhile, Annie Besant had gone ahead with the formal founding of her League in September 1916. The organization of her League was much looser than that of Tilak's, and three members could form a branch while in the case of Tilak's League each of the six branches had a clearly defined area and activities. Two hundred branches of Besant's League were established, some consisting of a town and others of groups of villages. And though a formal Executive Council of seven members was elected for three years by thirty-four `founding branches,' most of the work was carried on by Annie Besant and her lieutenants --- Arundale, C.P. Ramaswamy Aiyar, and B.P. Wadia --- from her headquarters at Adyar. Nor was there any organized method for passing on instructions --- these were conveyed through individual members and through Arundale's column on Home Rule in New India. The membership of Annie Besant's League increased at a rate slower than that of Tilak's. By March 1917, her League had 7,000 members. Besides her existing Theosophical followers, many others including Jawaharlal Nehru in Allahabad and B. Chakravarti and J. Banerjea in Calcutta joined the Home Rule League. However, the strength of the League could not be judged from the number of branches because, while many were extremely active, others remained adjuncts of the Theosophical societies. In Madras city, for example, though the number of branches was very large, many were inactive, while the branch in Bombay city, the four branches in the U.P. towns, and many village branches in Gujarat were very active.

The main thrust of the activity was directed towards building up an agitation around the demand for Home Rule. This was to be achieved by promoting political education and discussion. Arundale, through New India, advised members to promote political discussions, establish libraries containing material on national politics, organize classes for students on politics, print and circulate pamphlets, collect funds, organize social work, take part in local government activities, arrange political meetings and lectures, present arguments to friends in favour of Home Rule and urge them to join the movement. At least some of these activities were carried on by many of the branches, and especially the task of promotion of political discussion and debate. Some idea of the immensity of the propaganda effort that was launched can be gauged from the fact that by the time Annie Besant's League was formally founded in September 1916, the Propaganda Fund started earlier in the year had already sold 300,000 copies of twenty-six English pamphlets which focused mainly on the system of government existing in India and the arguments for self-government. After the founding of the League, these pamphlets were published again and, in addition, new ones in Indian languages were brought out. Most branches were also very active in holding public meetings and lectures. Further, they would always respond when a nation-wide call was given for protest on any specific issue. For example, when Annie Besant was externed from the Central Provinces and Berar in November 1916, most of the branches, at Arundale's instance, held meetings and sent resolutions of protest to the Viceroy and the Secretary of State. Tilak's externment from Punjab and Delhi in February 1917 elicited a similar response.

Many Moderate Congressmen, who were dissatisfied with the inactivity into which the Congress had lapsed, joined the Home Rule agitation. Members of Gokhale's Servants of India Society, though not permitted to become members of the League, were encouraged to add their weight to the demand for Home Rule by undertaking lecture tours and publishing pamphlets. Many other Moderate nationalists joined the Home Rule Leaguers in U.P. in touring the surrounding towns and villages in preparation for the Lucknow session of the Congress in December 1916. Their meetings were usually organized in the local Bar libraries, and attended by students, professionals, businessmen and, if it was a market day, by agriculturists. Speaking in Hindi, they contrasted India's current poverty with her glorious past, and also explained the main features of European independence movements. The participation of Moderates was hardly Surprising, since the Home Rule Leagues were after all only implementing the programme of political propaganda and education that they had been advocating for so long.

\begin{center}*\end{center}

\paragraph*{}

The Lucknow session of the Congress in December 1916 presented the Home Rule Leaguers with the long-awaited opportunity of demonstrating their strength. Tilak's Home Rule League established a tradition that was to become an essential part of later Congress annual sessions --- a special train, known variously as the `Congress Special' and the `Home Rule Special,' was organized to carry delegates from Western India to Lucknow. Arundale asked every member of the League to get himself elected as a delegate to the Lucknow session --- the idea being quite simply to flood the Congress with Home Rule Leaguers. Tilak and his men were welcomed back into the Congress by the Moderate president, Ambika Charan Mazumdar: `After nearly 10 years of painful separation and wanderings through the wilderness of misunderstandings and the mazes of unpleasant controversies ... both the wings of the Indian Nationalist party have come to realize the fact that united they stand, but divided they fall, and brothers have at last met brothers ... '

The Lucknow Congress was significant also for the famous Congress League Pact, popularly know as the Lucknow Pact. Both Tilak and Annie Besant had played a leading role in bringing about this agreement between the Congress and the League, much against the wishes of many important leaders, including Madan Mohan Malaviya. Answering the criticism that the Pact had acceded too much to the Muslim League, Lokamanya Tilak said: `It has been said, gentlemen, by some that we Hindus have yielded too much to our Mohammedan brethren. I am sure I represent the sense of the Hindu community all over India when I say that we could not have yielded too much. I would not care if the rights of self-government are granted to the Mohammedan community only. I would not care if they are granted to the Rajputs. I would not care if they are granted to the lower and the lowest classes of the Hindu population provided the British Government consider them more fit than the educated classes of India for exercising those rights. I would not care if those rights are granted to any section of the Indian community ... When we have to fight against a third party --- it is a very important thing that we stand on this platform united, united in race, united in religion, united as regards all different shades of political creed.''

Faced with such a stand by one who was considered the most orthodox of Hindus and the greatest scholar of the ancient religious texts, the opposition stood little chance of success, and faded away. And though the acceptance of the principle of separate electorates for Muslims was certainly a most controversial decision, it cannot be denied that the Pact was motivated by a sincere desire to allay minority fears about majority domination.

The Lucknow Congress also demanded a further dose of constitutional reforms as a step towards self-government. Though this did not go as far as the Home Rule Leaguers wished, they accepted it in the interests of Congress unity. Another very significant proposal made by Tilak --- that the Congress should appoint a small and cohesive Working Committee that would carry on the day to day affairs of the Congress and be responsible for implementing the resolutions passed at the annual sessions, a proposal by which he hoped to transform the Congress from a deliberative body into one capable of leading a sustained movement --- was unfortunately quashed by Moderate opposition. Four years later, in 1920, when Mahatma Gandhi prepared a reformed `constitution for the Congress, this was one of the major changes considered necessary if the Congress was to lead a sustained movement.

After the end of the Congress session, a joint meeting of the two Home Rule Leagues was held in the same pandal, and was attended by over 1,000 delegates. The Congress League Pact was hailed and the gathering was addressed by both Annie Besant and Tilak. On their return journeys, both the leaders made triumphant tours through various parts of North, Central and Eastern India.

The increasing popularity of the Home Rule Movement soon attracted the Government's wrath. The Government of Madras was the most harsh and first came out with an order banning students from attending political meetings. This order was universally condemned and Tilak commented: `The Government is fully aware that the wave of patriotism strikes the students most, and if at all a nation is to prosper, it is through an energetic new generation.''

\begin{center}*\end{center}

\paragraph*{}

The turning point in the movement came with the decision of the Government of Madras in June 1917 to place Mrs. Besant and her associates, B.P. Wadia and George Arundale, under arrest. Their internment became the occasion for nation-wide protest. In a dramatic gesture, Sir S. Subramania Aiyar renounced his knighthood. Those who had stayed away, including many Moderate leaders like Madan Mohan Malaviya, Surendranath Banerjee and M.A. Jinnah now enlisted as members of the Home Rule Leagues to record their solidarity with the internees and their condemnation of the Government's action. At a meeting of the AICC on 28 July, 1917, Tilak advocated the use of the weapon of passive resistance or civil disobedience if the Government refused to release the internees. The proposal for adopting passive resistance was sent for comment to all the Provincial Congress Committees, and while Berar and Madras were willing to adopt it immediately, most of the others were in favour of waiting for more time before taking a decision. At Gandhiji's instance, Shankerlal Banker and Jamnadas Dwarkadas collected signatures of one thousand men willing to defy the internment orders and march to Besant's place of detention. They also began to collect signatures of a million Peasants and workers on a petition for Home Rule. They made regular visits to Gujarat towns and villages and helped found branches of the League. In short, repression only served to harden the attitude of the agitators and strengthen their resolve to resist the Government. Montague, writing in his Diary, commented: `... Shiva cut his wife into fifty-two pieces only to discover that he had fifty-two wives. This is really what happens to the Government of India when it interns Mrs. Besant.'

The Government in Britain decided to effect a change in policy and adopt a conciliatory posture. The new Secretary of State, Montague, made a historic declaration in the House of Commons, On 20 August, 1917 in which he stated: `The policy of His Majesty's Government ... is that of the increasing association of Indians in every branch of the administration and the gradual development of self-governing institutions, with a view to the progressive realization of responsible government in India as an integral part of the British Empire.'' This statement was in marked contrast to that of Lord Morley who, while introducing the Constitutional Reforms in 1909, had stated categorically that these reforms were in no way intended to lead to self-government. The importance of Montague's Declaration was that after this the demand for Home Rule or self- government could no longer be treated as seditious.

This did not, however, mean that the British Government was about to grant self-government. The accompanying clause in the statement which clarified that the nature and the timing of the advance towards responsible government would be decided by the Government alone gave it enough leeway to prevent any real transfer of power to Indian hands for a long enough time.

In keeping with the conciliatory stance of the Montague Declaration, Annie Besant was released in September 1917. Annie Besant was at the height of her popularity and, at Tilak's suggestion, was elected President at the annual session of the Congress in December 1917.

\begin{center}*\end{center}

\paragraph*{}

During 1918, however, various factors combined to diffuse the energies that had concentrated in the agitation for Home Rule. The movement, instead of going forward after its great advance in 1917, gradually dissolved. For one, the Moderates who had joined the movement after Besant's arrest were pacified by the promise of reforms and by Besant's release. They were also put off by the talk of civil disobedience and did n attend the Congress from September 1918 onwards. The publication of the scheme of Government reforms in July 1918 further divided the nationalist ranks. Some wanted to accept it outright and others to reject it outright, while many felt that, though inadequate, they should be given a trial. Annie Besant herself indulged in a lot of vacillation on this question as well as on the question of passive resistance. At times she would disavow passive resistance, and at other times, under pressure from her younger followers, would advocate it. Similarly, she initially, along with Tilak, considered the reforms unworthy of Britain to offer and India to accept, but later argued in favour of acceptance. Tilak was more consistent in his approach, but given Besant's vacillations, and the change in the Moderate stance, there was little that he could do to sustain the movement on his own. Also, towards the end of the year, he decided to go to England to pursue the libel case that he had filed against Valentine Chirol, the author of Indian Unrest, and was away for many critical months. With Annie Besant unable to give a firm lead, and Tilak away in England, the movement was left leaderless.

The tremendous achievement of the Home Rule Movement and its legacy was that it created a generation of ardent nationalists who formed the backbone of the national movement in the coming years when, under the leadership of the Mahatma, it entered its truly mass phase. The Home Rule Leagues also created organizational links between town and country which were to prove invaluable in later years. And further, by popularizing the idea of Home Rule or self-government, and making it a commonplace thing, it generated a widespread pro- nationalist atmosphere in the country

By the end of the First World War, in 1918, the new generation of nationalists aroused to political awareness and impatient with the pace of change, were looking for a means of expressing themselves through effective political action. The leaders of the Home Rule League, who themselves were responsible for bringing them to this point, were unable to show the way forward. The stage was thus set for the entry of Mohandas Karamchand Gandhi, a man who had already made a name for himself with his leadership of the struggle of Indians in South Africa and by leading the struggles of Indian peasants and workers in Champaran, Ahmedabad and Kheda. And in March 1919, when he gave a call for a Satyagraha to protest against the obnoxious `Rowlatt' Act, he was the rallying point for almost all those who had been awakened to politics by the Home Rule Movement.
