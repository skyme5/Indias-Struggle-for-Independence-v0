\chapter{The Crisis at Tripuri to the Cripps Mission}
\begin{multicols}{2}

The Congress victory in the 1937 election and the consequent formation of popular ministries changed the balance of power within the country vis-a-vis the colonial authorities. The growth of left-wing parties and ideas led to a growing militancy within the nationalist ranks. The stage seemed to be set for another resurgence of the nationalist movement. Just at this time, the Congress had to undergo a crisis at the top --- an occurrence which plagued the Congress every few years. Subhas Bose had been a unanimous choice as the President of the Congress in 1938. In 1939, he decided to stand again --- this time as the spokesperson of militant politics and radical groups. Putting forward his candidature on 21 January 1939, Bose said that he represented the `new ideas, ideologies, problems and programmes' that had emerged with `the progressive sharpening of the anti-imperialist struggle in India.' The presidential elections, he said, should be fought among different candidates `on the basis of definite problems and programmes.'' On 24 January, Sardar Patel, Rajendra Prasad, 

J.B. Kripalani and four other members of the Congress Working Committee issued a counter statement, declaring that the talk of ideologies, programmes and policies was irrelevant in the elections of a Congress president since these were evolved by the various Congress bodies such as the AICC and the Working Committee, and that the position of the Congress President was like that of a constitutional head who represented and symbolized the unity and solidarity of the nation. With the blessings of Gandhiji, these and other leaders put up Pattabhi Sitaramayya as a candidate for the post. Subhas Bose was elected on 29 January by 1580 votes against 1377. Gandhiji declared that Sitaramayya's defeat was `more mine than his.' 

But the election of Bose resolved nothing, it only brought the brewing crisis to a head at the Tripuri session of the Congress. There were two major reasons for the crisis. One was the line of propaganda adopted by Bose against Sardar Patel and the majority of the top Congress leadership whom he branded as rightists. He openly accused them of working for a compromise with the Government on the question of federation, of having even drawn up a list of prospective central' ministers and therefore of not wanting a leftist as the president of the Congress `who may be a thorn in the way of a compromise and may put obstacles in the path of negotiations.' He had, therefore, appealed to Congressmen to vote for a leftist and `a genuine anti-federationist.'3 In the second part of his autobiography, Subhas put forward his thinking of the period even more crudely: `As Congress President, the writer did his best to stiffen the opposition of the Congress Party to any compromise with Britain and this caused annoyance in Gandhian circles who were then looking forward to an understanding with the British Government.' `The Gandhiists', he wrote, `did not want to be disturbed in their ministerial and parliamentary work' and `were at that time opposed to any national struggle.' 

The Congress leaders, labelled as compromisers, resented such charges and branded them as a slander. They pointed out in a statement: `Subhas Babu has mentioned his opposition to the federation. This is shared by all the members of the Working Committee. It is the Congress policy.' After Subhas's election, they felt that they could not work with a president who had publicly cast aspersions on their nationalist bonafides. Earlier, Gandhiji had issued a statement on 31 January saying: `I rejoice in this defeat' because `Subhas Babu, instead of being President on the sufferance of those whom he calls rightists, is now President elected in a contested election. This enables him to choose a homogeneous cabinet and enforce his programme without let or hindrance.' 

Jawaharlal Nehru did not resign along with the twelve other Working Committee members. He did not like the idea of confronting Bose publicly. But he did not agree with Bose either. Before the elections, he had said that in the election no principles or programmes were at stake. He had been unhappy with Bose's aspersions on his colleagues. Nor did he agree that the fight was between the Left and the Right. His letter to Subhas on 4 February 1939 would bear a long quotation: `I do not know who you consider a leftist and who a rightist. The way these words were used by you in your statements during the presidential contest seemed to imply that Gandhiji and those who are considered as his group in the Working Committee are the rightist leaders. Their opponents, whoever they might be, are the leftists. That seems to me an entirely wrong description. It seems to me that many of the so-called leftists are more right than the so-called rightists. Strong language and a capacity to criticize and attack the old Congress leadership is not a test of leftism in politics ... I think the use of the words left and right has been generally wholly wrong and confusing. If, instead of these words% we talked about policies it would be far better. What policies do you stand for? Anti-federation, well and good. I think that the great majority of the members of the Working Committee stand for that and it is not fair to hint at their weakness in this respect.' 

However, more importantly, basic differences of policy and tactics were involved in the underlying Bose-Gandhian debate. They were partially based on differing perceptions of the political reality, and differing assessments of the strength and weakness of the Congress and the preparedness of the masses for struggle. Differing styles regarding how to build up a mass movement were also involved. 

Subhas Bose believed that the Congress was strong enough to bunch an immediate struggle d that the masses were ready for such struggle. He was convinced, as he wrote later, `that the country was internally more ripe for a revolution than ever before and that the coming international crisis would give India an opportunity for achieving her emancipation, which is rare in human history.' He, therefore, argued in his presidential address at Tripuri for a programme of immediately giving the British Government a six-months ultimatum to grant the national demand for independence and of launching a mass civil disobedience movement if it failed to do so.' 

Gandhiji's perceptions were very different. He, too, believed that another round for mass struggle was necessary to win freedom, for Indians were facing `an impossible situation.' Already, in the middle of July 1938, he had written: `The darkness that seems to have enveloped me will disappear, and that, whether with another battle more brilliant than the Dandi March or without, India will come to her own.''° But, he believed, the time was not yet ripe for an ultimatum because neither the Congress nor the masses were yet ready for struggle. Indians should first `put our own house in order.' Making his position clear in an interview on 5 May 1939, Gandhiji declared: `He (Subhas Bose) holds that we possess enough resources for a fight. I am totally opposed to his views. Today we possess no resources for a fight ... There is no limit to communal strife ... We do not have the same hold among the peasants of Bihar as we used to ... If today I am asked to start the ``Dandi March,'' I have not the courage to do so. How can we do anything without the workers and peasants? The country belongs only to them. I am not equipped to issue an ultimatum to the Government. The country would only be exposed to ridicule.'' 

Gandhiji's views were above all based on his assessment of the Congress organization. He was convinced that corruption and indiscipline had vitiated its capacity to fight. As we have seen earlier, during 1938 and early 1939, he repeatedly and publicly raised the issues of mutual rivalries and bickerings among Congressmen, bogus membership and impersonation at party elections, efforts to capture Congress Committees, and the general decline of authority in the Congress. 

The internal strife reached its climax at the Tripuri session of the Congress, held from 8 to 12 March 1939. Bose had completely misjudged his support and the meaning of his majority in the presidential election. Congressmen had voted for him for diverse reasons, and above all because he stood for militant politics, and not because they wanted to have him as the supreme leader of the national movement. They were not willing to reject Gandhiji's leadership or that of other older leaders who decided to bring this home to Subhas. Govind Ballabh Pant moved a resolution at Tripuri expressing lull confidence in the old Working Committee, reiterating full faith in Gandhiji's leadership of the movement and the Congress policies of the previous twenty years, and asking Subhas to nominate his Working Committee `in accordance with the wishes of Gandhiji.' The resolution was passed by a big majority, but Gandhiji did not approve of the resolution and refused to impose a Working Committee on Subhas. He asked him to nominate a Committee of his own choice. 

Subhas Bose refused to take up the challenge. He had placed himself in an impossible situation. He knew that he could not lead the organization on his own, but he was also not willing to accept the leadership of the majority. To place the best construction on his policy, he wanted Gandhiji to be the leader of the coming struggle but he wanted Gandhiji to follow the strategy and tactics laid down by him and the left-wing parties and groups. Gandhiji, on the other hand, would either lead the Congress on the basis of his own strategy and style of politics or surrender the position of the leader. As he wrote to Bose: `if your prognosis is right, I am a back number and played out as the generalissimo of Satyagraha.'' In other words, as Rajendra Prasad later wrote in his Autobiography, Gandhiji and the older leaders would not accept a situation where the strategy and tactics were not theirs but the responsibility for implementing them would be theirs.' 

Bose could see no other way out but to resign from the presidentship. Nehru tried to mediate between the two sides and persuade Bose not to resign, while asking Gandhiji and the older leaders to be more accommodative. But Bose would not resign from his position. On the one hand, he insisted that the Working Committee should be representative of the new radical trends and groups which had elected him, on the other, he would not nominate his own Working Committee. He preferred to press his resignation. This led to the election of Rajendra Prasad in his place. The Congress had weathered another storm. 

Bose could also not get the support of the Congress Socialists and the Communists at Tripuri or after for they were not willing to divide the national movement and felt that its unity must be preserved at all costs. Explaining its position, the CPI declared after Tripuri that the interests of the anti-imperialist struggle demanded not the exclusive leadership of one wing but a united leadership under the guidance of Gandhiji.'' P.C. Joshi, General Secretary of the CPI, wrote in April 1939 that the greatest class struggle today is our national struggle,' that the Congress was the main organ of this struggle, and that the preservation of its unity was a primary task.' 

Subsequently, in May, Subhas Bose and his followers formed the Forward Bloc as a new party within the Congress. And when he gave a call for an All-India protest on 9 July against an AICC resolution, the Working Committee took disciplinary action against him, removing him from the presidentship of the Bengal Provincial Congress Committee and debarring him from holding any Congress office for three years. 

World War II broke Out On 1 September 1939 when Nazi Germany invaded Poland. Earlier Germany had occupied Austria in March 1938 and Czechoslovakia in 1939. Britain and France, which had been following a policy of appeasement towards Hitler, were now forced to go to Poland's aid and declare war on Germany. This they did on 3 September 1939. The Government of India immediately declared India to be at war with Germany without consulting the Congress or the elected members of the central legislature. 

The Congress, as we have seen earlier, was in full sympathy with the victims of fascist aggression, and its immediate reaction was to go to the aid of the anti-fascist forces. Gandhiji's reaction was highly emotional. He told the Viceroy that the very thought of the possible destruction of the House of Parliament and Westminster Abbey produced a strong emotional reaction in him and that, fully sympathizing with the Allied Cause, he was for full and unquestioning cooperation with Britain. But a question most of the Congress leaders asked was --- how was it possible for an enslaved nation to aid others in their fight for freedom? The official Congress stand was adopted at a meeting of the Congress W8rking Committee held at Wardha from 10 to 14 September to which, in keeping with the nationalist tradition of accommodating diversity of opinion, Subhas Bose, Acharya Narendra Dev, and Jayaprakash Narayan ware also invited. Sharp differences emerged in this meeting. Gandhiji was for taking a sympathetic view of the Allies. He believed that there was a clear difference between the democratic states of Western Europe and the totalitarian Nazi state headed by Hitler. The Socialists and Subhas Bose argued that the War was an imperialist one since both sides were fighting for gaining or defending colonial territories. Therefore, the question of supporting either of the two sides did not arise. Instead the Congress should take advantage of the situation to wrest freedom by immediately starting a civil disobedience movement. 

Jawaharlal Nehru had a stand of his own. He had been for several years warning the world against the dangers of Nazi aggression, and he made a sharp distinction between democracy and Fascism. He believed that justice was on the side of Britain, France and Poland. But he was also convinced that Britain and France were imperialist countries and that the War was the result of the inner contradictions of capitalism' maturing since the end of World War I. He, therefore, argued that India should neither join the War till she herself gained freedom nor take advantage of Britain's difficulties by starting an immediate struggle. Gandhiji found that his position was not supported by even his close followers such as Sardar Patel and Rajendra Prasad. Consequently, he decided to support Nehru's position which was then adopted by the Working Committee. Its resolution, while unequivocally condemning the Nazi attack on Poland as well as Nazism and Fascism, declared that India could not be party to a war which was ostensibly being fought for democratic freedom while that freedom was being denied to her, If Britain was fighting for democracy and freedom, she should prove this in India. In particular, she should declare how her war aims would be implemented in India at the end of the War, Indians would then gladly join other democratic nations in the war effort to starting a mass struggle, but it warned that the decision could not be delayed for long. As Nehru put it, the Congress leadership wanted `to give every chance to the Viceroy and the British Government.' 

The British Government's response was entirely negative. Linlithgow, the Viceroy, in his well considered statement of 17 October 1939 harped on the differences among Indians, tried to use the Muslim League and the Princes against the Congress, and refused to define Britain's war aims beyond stating that Britain was resisting aggression. As an immediate measure, he offered to set up a consultative committee whose advice might be sought by the Government whether it felt it necessary to do so. For the future, the promise was that at the end of the War the British Government would enter into consultations with representatives of several communities, parties, and interests in India and with the Indian princes' as to how the Act of 1935 might be modified. In a private communication to Zetland, the Secretary of State, Linlithgow was to remark a few months later: `I am not too keen to start talking about a period after which British rule will have ceased in India. I suspect that that day is very remote and I feel the least we say about it in all probability the better.'' On 18 October, Zetland spoke in the House of Lords and stressed differences among Indians, especially among Hindus and Muslims. He branded the Congress as a purely Hindu organization.' It, thus, became clear that the British Government had no intention of loosening their hold on India during or after the War and that it was willing, if necessary, to treat the Congress as an enemy. 

The reaction of the Indian people and the national leadership was sharp. The angriest reaction came from Gandhiji who had been advocating more or less unconditional support to Britain. Pointing out that the British Government was continuing to pursue `the old policy of divide and rule,' he said: `The Indian declaration (of the Viceroy) shows clearly that there is to be no democracy for India if Britain can prevent it ... The Congress asked for bread and it has got a stone.' Referring to the question of minorities and special interests such as those of the princes, foreign capitalists, zamindars, etc., Gandhiji remarked: `The Congress will safeguard the rights of every minority so long as they do not advance claims inconsistent with India's independence.' But, he added, `independent India will not tolerate any interests in conflict with the true interests of the masses.' 

The Working Committee, meeting on 23 October, rejected the Viceregal statement as a reiteration of the old imperialist policy, decided not to support the War, and called upon the Congress ministries to resign as a protest. This they did as disciplined soldiers of the national movement. But the Congress leadership still stayed its hand and was reluctant to give a call for an immediate and a massive anti-imperialist struggle. In fact, the Working Committee resolution of 23 October warned Congressmen against any hasty action. 

While there was agreement among Congressmen on the question of attitude to the War and the resignation of the ministries, sharp differences developed over the question of the immediate starting of a mass satyagraha. Gandhiji and the dominant leadership advanced three broad reasons for not initiating an immediate movement. First, they felt that since the cause of the Allies --- Britain and France --- was just, they should not be embarrassed in the prosecution of the War. Second, the lack of Hindu- Muslim unity was a big barrier to a struggle. In the existing atmosphere any civil disobedience movement could easily degenerate into communal rioting or even civil war. Above all, they felt that there did not exist in the country an atmosphere for an immediate struggle. Neither the masses were ready nor was the Congress organizationally in a position to launch a struggle. The Congress organization was weak and had been corrupted during 1938--39. There was indiscipline and lack of cohesion within the Congress ranks. Under these circumstances, a mass movement would not be able to withstand severe repressive measures by the Government. It was, therefore, necessary to carry on intense political work among the people, to prepare them for struggle, to tone up the Congress organization and purge it of weaknesses, to negotiate with authorities till all the possibilities of a negotiated settlement were exhausted and the Government was clearly seen by all to be in the wrong. The time for launching a struggle would come when the people were strong and ready for struggle, the Congress organization had been put on a sound footing, and the Government took such aggressive action that the people felt the absolute necessity of going into mass action. This view was summed up in the resolution placed by the Working Committee before the Ramgarh Session of the Congress in March 1940. The resolution, after reiterating the Congress position on the War and asserting that `nothing short of complete independence can be accepted by the people,' declared that the Congress would resort to civil disobedience `as soon as the Congress organization is considered fit enough for the purpose, or in case circumstances so shape themselves as to precipitate a crisis.'' 

An alternative to the position of the dominant leadership came from a coalition of various left-wing groups: Subhas Bose and his Forward Bloc, the Congress Socialist Party, the Communist Party, the Royists, etc. The Left characterized the War as an imperialist war and asserted that the war-crisis provided the opportunity to achieve freedom through an all-out struggle against British imperialism. It was convinced that the masses were fully ready for action and were only waiting for a call from the leadership. They accepted that hurdles like the communal problem and weaknesses in the Congress organization existed; but they were convinced that these would be easily and automatically swept away once a mass struggle was begun. Organizational strength, they said, was not to be built up prior to a struggle but in the course of the struggle. Making a sharp critique of the Congress leadership's policy of `wait and see,' the Left accused the leadership of being afraid of the masses, of having lost zest for struggle, and consequently of trying to bargain and compromise with imperialism for securing petty concessions. They urged the Congress leadership to adopt immediate measures to launch a mass struggle. While agreeing on the need for an immediate struggle, the Left was internally divided both in its understanding of political forces and on the Course of political action in case the dominant leadership of the Congress did not accept the line of immediate struggle. Subhas Bose wanted the Left to split the Congress if it did not launch a struggle, to organize a parallel Congress and to start a struggle on its own. He was convinced that the masses and the overwhelming majority of Congress would support the Left-ted parallel Congress and join the movement it would launch. The CSP and CPI differed from this view. They were convinced that Bose was grossly overestimating the influence of the Left and no struggle could be launched without the leadership of Gandhiji and the Congress. Therefore an attempt should be made not to split the Congress and thus disrupt the national united fronts but persuade and pressurize its leadership to launch a struggle. 

Jawaharlal Nehru's was an ambivalent position. On the one hand, he could clearly see the imperialistic character of the Allied countries, on the other, he would do nothing that might lead to the triumph of Hitler and the Nazis in Europe. His entire personality and political thinking led to the line of an early commencement of civil disobedience, but he would do nothing that would imperil the anti-Nazi struggle in Europe and the Chinese people's struggle against Japanese aggression. In the end, however, the dilemma was resolved by Nehru going along with Gandhiji and the majority of the Congress leadership. 

But politics could not go on this placid note for too long. The patience of both the Congress leadership and the masses was getting exhausted. The Government refused to budge and took up the position that no constitutional advance could be made till the Congress came to an agreement with the Muslim communalists. It kept issuing ordinance after ordinance taking away the freedom of speech and the Press and the right to organize associations. Nationalist workers, especially those belonging to the left-wing, were harassed, arrested and imprisoned all over the country. The Government was getting ready to crush the Congress if it took any steps towards a mass struggle. 

In this situation, the Indians felt that the time had come to show the British that their patience was not the result of weakness, As Nehru put it in an article entitled `The Parting of the Ways,' the British rulers believed that `in this world of force, of bombing aeroplanes, tanks, and armed men how weak we are! Why trouble about us? But perhaps, even in this world of armed conflict, there is such a thing as the spirit of man, and the spirit of a nation, which is neither ignoble nor weak, and which may not be ignored, save at peril.' Near the end of 1940, the Congress once again asked Gandhiji to take command. Gandhiji now began to take steps which would lead to a mass struggle within his broad strategic perspective. He decided to initiate a limited Satyagraha on an individual basis by a few selected individuals in every locality. The demand of a Satyagrahi would be for the freedom of speech to preach against participation in the War. The Satyagrahi would publicly declare: `It is wrong to help the British war-effort with men or money. The only worthy effort is to resist all war with non-violent resistance.' The Satyagrahi would beforehand inform the district magistrate of the time and place where he or she was going to make the anti-war speech. The carefully chosen Satyagrahis --- Vinoba Bhave was to be the first Satyagrahi on 17 October 1940 and Jawaharlal Nehru the second --- were surrounded by huge crowds when they appeared on the platform, and the authorities could often arrest them only after they had made their speeches. And if the Government did not arrest a Satyagrahi, he or she would not only repeat the performance but move into the villages and start a trek towards Delhi, thus participating in a movement that came to be known as the `Delhi Chalo' (onwards to Delhi) movement. 

The aims of the Individual Satyagraha conducted as S. Gopal has put it, `at a low temperature and in very small doses' were explained as follows by Gandhiji in a letter to the Viceroy: `The Congress is as much opposed to victory for Nazism as any Britisher can be. But their objective cannot be carried to the extent of their participation in the war. And since you and the Secretary of State for India have declared that the whole of India is voluntarily helping the war effort, it becomes necessary to make clear that the vast majority of the people of India are not interested in it. They make no distinction between Nazism and the double autocracy that rules India.' Thus, the Individual Satyagraha had a dual purpose --- while giving expression to the Indian people's strong political feeling, it gave the British Government further opportunity to peacefully accept the Indian demands. Gandhiji and the Congress were, because of their anti-Nazi feelings, still reluctant to take advantage of' the British predicament and embarrass her war effort by a mass upheaval in India. More importantly, Gandhiji was beginning to prepare the people for the coming struggle. The Congress organization was being put back in shape; opportunist elements were being discovered and pushed out of the organization; and above all the people were being politically aroused, educated and mobilized. By 15 May 1941, more than 25,000 Satyagrahis had been convicted for offering individual civil disobedience. Many more --- lower level political workers - --- had been left free by the Government. 

Two major changes in British politics occurred during 1941. Nazi Germany had already occupied Poland, Belgium, Holland, Norway and France as well as most of Eastern Europe. It attacked the Soviet Union on 22 June 1941. In the East, Japan launched a surprise attack on the American fleet at Pearl Harbour on 7 December. It quickly overran the Philippines, Indo- China, Indonesia, Malaysia and Burma. It occupied Rangoon in March 1942. War was brought to India's doorstep. Winston Churchill, now the British Prime Minister, told the King that Burma, Ceylon, Calcutta and Madras might fall into enemy hands. 

The Indian leaders, released from prisons in early December, were worried about the safety and defence of India. They also had immense concern for the Soviet Union and China. Many felt that Hitler's attack on the Soviet Union had changed the character of the War. Gandhiji had earlier denounced the Japanese slogan of `Asia for Asiatics' and asked the people of India to boycott Japanese products. Anxious to defend Indian territory and to go to the aid of the Allies, the Congress Working Committee overrode the objections of Gandhiji and Nehru and passed a resolution at the end of December offering to fully cooperate in the defence of India and the Allies if Britain agreed to give full independence after the War arid the substance of power immediately. It was at this time that Gandhiji designated Jawaharlal as his chosen successor. Speaking before the AICC on January 1941, he said: `Somebody suggested that Pandit Jawaharlal and I were estranged. It will require much more than differences of opinion to estrange us. We have had differences from the moment we became co-workers, and yet I have said for some years and say now that not Rajaji (C. Rajagopalachari) but Jawaharlal will be my successor. He says that he does not understand my language, and that he speaks a language foreign to me. This may or may not be true. But language is no bar to union of hearts. And I know that when I am gone he will speak my language.' 

As the war situation worsened, President Roosevelt of the USA and President Chiang Kai-Shek of China as also the Labour Party leaders of Britain put pressure on Churchill to seek the active cooperation of Indians in the War. To secure this cooperation the British Government sent to India in March 1942 a mission headed by a Cabinet minister Stafford Cripps, a left- wing Labourite who had earlier actively supported the Indian national movement. Even though Cripps announced that the aim of British policy in India was `the earliest possible realization of self- government in India,' the Draft Declaration he brought with him was disappointing. The Declaration promised India Dominion Status and a constitution-making body after the War whose members would be elected by the provincial assemblies and nominated by the rulers in case of the princely states. The Pakistan demand was accommodated by the provision that any province which was not prepared to accept the new constitution would have the right to sign a separate agreement with Britain regarding its future status. For the present the British would continue to exercise sole control over the defence of the country. Amery, the Secretary of State, described the Declaration as in essence a conservative, reactionary and limited offer. Nehru, a friend of Cripps, was to write later: When I read those proposals for the first time I was profoundly depressed.' 

Negotiations between Cripps and the Congress leaders broke down. The Congress objected to the provision for Dominion Status rather than full independence, the representation of the princely states in the constituent assembly not by the people of the states but by the nominees of the rulers, and above all by the provision for the partition of India. The British Government also refused to accept the demand for the immediate transfer of effective power to the Indians and for a real share in the responsibility for the defence of India. An important reason for the failure of the negotiations was the incapacity of Cripps to bargain and negotiate. He had been told not to go beyond the Draft Declaration. Moreover, Churchill, the Secretary of State, Amery, the Viceroy, Linlithgow, and the Commander-in-Chief, Wavell, did not want Cripps to succeed and constantly opposed and sabotaged his efforts to accommodate Indian opinion. Stafford Cripps returned home in the middle of April leaving behind a frustrated and embittered Indian people. Though they still sympathized with the anti-fascist, especially the people of China and the Soviet people, they felt that the existing situation in the country had become intolerable. The time had come, they felt, for a final assault on imperialism.
\end{multicols}{2}
