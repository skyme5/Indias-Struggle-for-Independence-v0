\chapter[The Revolutionary Terrorist]{Bhagat Singh, Surya Sen and the Revolutionary Terrorists}
\begin{multicols}{2}

The revolutionary terrorists were severely suppressed during World War I, with most of the leaders in jail or absconding. Consequently, in order to create a more harmonious atmosphere for the Montague-Chelmsford reforms, the Government released most of them under a general amnesty in early 1920. Soon after, the National Congress launched the Non Cooperation Movement and on the urging of Gandhiji, C.R. Das and other Leaders most of the revolutionary terrorists either joined the movement or suspended their own activities in order to give the Gandhian mass movement a chance.

But the sudden suspension of the Non-Cooperation Movement shattered the high hopes raised earlier. Many young people began to question the very basic strategy of the national leadership and its emphasis on non violence and began to look for alternatives. They were not attracted by the parliamentary politics of the Swarajists or the patient and undramatic constructive work of the no-changers.. Many were drawn to the idea that violent methods alone would free India. Revolutionary terrorism again became attractive. It is not accidental that nearly all the major new leaders of the revolutionary terrorist politics, for example, Jogesh Chandra Chatterjea, Surya Sen, Jatin Das, Chandrashekhar Azad, Bhagat Singh, Sukhdev, Shiv Varma, Bhagwati Charan Vohra and Jaidev Kapur, had been enthusiastic participants in the non-violent Non-Cooperation Movement. Gradually two separate strands of revolutionary terrorism developed --- one in Punjab, U.P. and Bihar and the other in Bengal. Both the strands came under the influence of several new social forces. One was the upsurge of working class trade unionism after the War. They could see the revolutionary potential of the new class and desired to harness it to the nationalist revolution. The second major influence was that of the Russian Revolution and the success of the young Socialist State in consolidating itself. The youthful revolutionaries were keen to learn from and take the help of the young Soviet State and its ruling Bolshevik Party. The third influence was that of the newly sprouting Communist groups with their emphasis on Marxism, Socialism and the proletariat.

\begin{center}*\end{center}

\paragraph*{}

The revolutionaries in northern India were the first to emerge out of the mood of frustration and reorganize under the leadership of the old veterans, Ramprasad Bismil, Jogesh Chatterjea and Sachindranath Sanyal whose Bandi Jiwan served as a textbook to the revolutionary movement. They met in Kanpur in October 1924 and founded the Hindustan Republican Association (or Army) to organize armed revolution to overthrow colonial rule and establish in its place a Federal Republic of the United States of India whose basic principle would be adult franchise.

Before armed struggle could be waged, propaganda had to be organized on a large scale, men had to be recruited and trained and arms had to be procured. All these required money. The most important `action' of the HRA was the Kakori Robbery. On 9 August 1925, ten men held up the 8-Down train at Kakori, an obscure village near Lucknow, and looted its official railway cash. The Government reaction was quick and hard. It arrested a large number of young men and tried them in the Kakori Conspiracy Case. Ashfaqulla Khan, Ramprasad Bismil, Ràshan Singh and Rajendra Lahiri were hanged, four others were sent to the Andamans for life and seventeen others were sentenced to long terms of imprisonment. Chandrashekhar Azad remained at large.

The Kakori case was a major setback to the revolutionaries of northern India but it was not a fatal blow. Younger men such as Bejoy Kumar Sinha, Shiv Varma and Jaidev Kapur in U.P.,- Bhagat Singh, Bhagwati Charan Vohra and Sukhdev in Punjab set out to reorganize the HRA under the overall leadership of Chandrashekhar Azad. Simultaneously, they were being influenced by socialist ideas. Finally, nearly all the major young revolutionaries of northern India met at Ferozeshah Kotla Ground at Delhi on 9 and 10 September 1928, created a new collective leadership, adopted socialism as their official goal and changed the name of the party to the 1-lindustan Socialist Republican Association (Army).

\begin{center}*\end{center}

\paragraph*{}

Even though, as we shall see, the HSR.A and its leadership was rapidly moving away from individual heroic action and assassination and towards mass politics, Lala Lajpat Rai's death, as the result of a brutal lathi-charge when he was leading an anti-Simon Commission demonstration at Lahore on 30 October her 1928, led them once again to take to individual assassination. The death of this great Punjabi leader, popularly known as Sher-e-Punjab, was seen by the romantic youthful leadership of the HSRA as a direct challenge. And so, on 17 December 1928, Bhagat Singh, Azad and Rajguru assassinated, at Lahore, Saunders, a police official involved in the lathi charge of Lab Lajpat Rai. In a poster, put up by the HSRA after the assassination, the assassination was justified as follows: `The murder of a leader respected by millions of people at the unworthy hands of an ordinary police official ... was an insult to the nation. it was the bounden duty of young men of India to efface it ... We regret to have had to kill a person but he was part and parcel of that inhuman and unjust order which has to be destroyed.'

The HSRA leadership now decided to let the people know about its changed objectives and the need for a revolution by the masses. Bhagat Singh and B.K. Dutt were asked to throw a bomb in the Central Legislative Assembly on 8 April 1929 against the passage of the Public Safety Bill and the Trade Disputes Bill which would reduce the civil liberties of citizens in general and workers in particular. The aim was not to kill, for the bombs were relatively harmless, but, as the leaflet they threw into the Assembly hail proclaimed, `to make the deaf hear'. The objective was to get arrested and to use the trial court as a forum for propaganda so that people would become familiar with their movement and ideology.

Bhagat Singh and B.K. Dutt were tried in the Assembly Bomb Case. Later, Bhagat Singh, Sukhdev, Rajguru and tens of other revolutionaries were tried in a series of famous conspiracy cases. Their fearless and defiant attitude in the courts --- every day they entered the court-room shouting slogans `Inquilab Zindabad,' `Down, Down with Imperialism,' `Long Live the Proletariat' and singing songs such as `Sarfaroshi ki tamanna ab hamare dil mei hai' (our heart is filled with the desire for martyrdom) and `Mera rang de basanti chola' (dye my clothes in saffron colour (the colour of courage and sacrifice) --- was reported in newspapers; unsurprisingly this won them the support and sympathy of people all over the country including those who had complete faith in non-violence. Bhagat Singh became a household name in the land. And many persons, all over the country, wept and refused to eat food, attend schools, or carry on their daily work, when they heard of his hanging in March 1931.

The country was also stirred by the prolonged hunger strike the revolutionary under-trials undertook as a protest against the horrible conditions in jails. They demanded that they be treated not as criminals but as political prisoners. The entire nation rallied behind the hunger- strikers. On 13 September, the 64th day of the epic fast, Jatin Das, a frail young man with an iron will, died. Thousands came to pay him homage at every station passed by the train carrying his body from Lahore to Calcutta. At Calcutta, a two-mile-long procession of more than six lakh people carried his coffin to the cremation ground.

A large number of revolutionaries were convicted in the Lahore Conspiracy Case and other similar cases and sentenced to long terms of imprisonment; many of them were sent to the Andamans. Bhagat Singh, Sukhdev and Rajguru were sentenced to be hanged. The sentence was carried out on 23 March 1931.

\begin{center}*\end{center}

\paragraph*{}

In Bengal, too, the revolutionary terrorists started reorganizing and developing their underground activities. At the same time, many of them continued to work in the Congress organization. This enabled them to gain access to the vast Congress masses; on the other hand, they provided the Congress with an organizational base in small towns and the countryside. They cooperated with C.R. Das in his Swarajist work. After his death the Congress leadership in Bengal got divided into two wings, one led by Subhas Chandra Bose and the other by J.M. Sengupta, the Yugantar group joined forces with the first and Anushilan with the second.

Among the several `actions' of the reorganized groups was the attempt to assassinate Charles Tegart, the hated Police Commissioner of Calcutta, by Gopinath Saha in January 1924. By an error, another Englishman named Day was killed. The Government came down on the people with a heavy hand. A large number of people, suspected of being terrorists, or their supporters, were arrested under a newly promulgated ordinance. These included Subhas Chandra Bose and many other Congressmen. Saha was hanged despite massive popular protest. The revolutionary activity suffered a severe setback.

Another reason for stagnation in revolutionary terrorist activity lay in the incessant factional and personal quarrels within the terrorist groups, especially where Yugantar and Anushilan rivalry was concerned. But very soon younger revolutionaries began to organize themselves in new groups, developing fraternal relations with the active elements of both the Anushilan and Yugantar parties. Among the new Revolt Groups,' the most active and famous was the Chittagong group led by Surya Sen.

Surya Sen had actively participated in the Non-Cooperation Movement and had become a teacher in a national school in Chittagong, which led to his being popularly known as Masterda. Arrested and imprisoned for two years, from 1926 to 1928, for revolutionary activity, he continued to work in the Congress. He and his group were closely associated with the Congress work in Chittagong. In 1929, Surya Sen was the Secretary and five of his associates were members of the Chittagong District Congress Committee.

Surya Sen, a brilliant and inspiring organizer, was an unpretentious, soft-spoken and transparently sincere person. Possessed of immense personal courage, he was deeply humane in his approach. He was fond of saying: `Humanism is a special virtue of a revolutionary.' He was also very fond of poetry, being a great admirer of Rabindranath Tagore and Kazi Nazrul Islam. Surya Sen soon gathered around himself a large band of revolutionary youth including Anant Singh, Ganesh Ghosh and Lokenath Baul. They decided to organize a rebellion, on however small a scale, to demonstrate that it was possible to challenge the armed might of the British Empire in India. Their action plan was to include occupation of the two main armouries in Chittagong and the seizing of their arms with which a large band of revolutionaries could be formed into an armed detachment; the destruction of the telephone and telegraph systems of the city; and the dislocation of the railway communication system between Chittagong and the rest of Bengal. The action was carefully planned and was put into execution at 10 o'clock on the night of 18 April 1930. A group of six revolutionaries, led by Ganesh Ghosh, captured the Police Armoury, shouting slogans such as Inquilab Zindabad, Down with Imperialism and Gandhiji`s Raj has been established. Another group of ten, led by Lokenath Paul, took over the Auxiliary Force Armoury along with its Lewis guns and 303 army rifles. Unfortunately they could not locate the ammunition. This was to prove a disastrous setback to the revolutionaries' plans. The revolutionaries also succeeded in dislocating telephone and telegraph communications and disrupting movement by train. In all, sixty- five were involved in the raid, which was undertaken in the name of the Indian Republican Army, Chittagong Branch.

All the revolutionary groups gathered outside the Police Armoury where Surya Sen, dressed in immaculate white khadi dhoti and a long coat and stiffly ironed Gandhi cap, took a military salute, hoisted the National Flag among shouts of Bande Mataram and Inquilab Zindabad, and proclaimed a Provisional Revolutionary Government.

It was not possible for the band of revolutionaries to put up a fight in the town against the army which was expected. They, therefore, left Chittagong town before dawn and marched towards the Chittagong hill ranges, looking for a safe place. It was on the Jalalabad hill that several thousand troops surrounded them on the afternoon of 22 April. After a fierce fight n which over eighty British troops and twelve revolutionaries died, Surya Sen decided to disperse into the neighbouring villages; there they formed into small groups and conducted raids on Government, personnel and property. Despite several repressive measures and combing operations by the authorities, the villagers, most of them Muslims, gave food and shelter to the revolutionary outlaws and enabled them to survive for three years. Surya Sen was finally arrested on 16 February 1933, tried and hanged on 12 January 1934. Many of his co-fighters were caught and sentenced to long terms of imprisonment.

The Chittagong Armoury Raid had an immense impact on the people of Bengal. As an official publication remarked, it `fired the imagination of revolutionary-minded youth' and `recruits poured into the various terrorist groups in a steady stream.' The year 1930 witnessed a major revival of revolutionary activity, and its momentum carried over to 1931 and 1932. There were numerous instances of death-defying heroism. In Midnapore district alone, three British magistrates were assassinated. Attempts were made on the lives of two Governors; two Inspectors- General of Police were killed. During this three-year period, twenty-two officials and twenty non-officials were killed.

The official reaction to the Armoury Raid and the revival of revolutionary terrorist activity was initially one of panic and, then of brutal reprisals. The Government armed itself with twenty repressive Acts and let loose the police on all nationalists. In Chittagong, it burnt several villages, imposed punitive fine on many others, and in general established a reign of terror. In 1933, it arrested and sentenced Jawaharlal Nehru to a two-year term in jail for sedition. He had in a speech in Calcutta condemned imperialism, praised the heroism of revolutionary youth (even while criticizing the policy of terrorism as futile and out-of-date) and condemned police repression.

A remarkable aspect of this new phase of the terrorist movement in Bengal was the large-scale participation of young women Under Surya Sen's leadership, they provided shelter, acted as messengers and custodians of arms, and fought, guns in hand. Pritilata Waddedar died while conducting a raid, while Kalpana Dutt (now Joshi) was arrested and tried along with Surya Sen and given a life sentence. In December 1931, two school girls of Comilla, Santi Ghosh and Suniti Chowdhury, shot dead the District Magistrate. In February 1932, Bina Das fired point blank at the Governor while receiving her degree at the Convocation.

Compared to the old revolutionary terrorists, as also Bhagat Singh and his comrades, the Chittagong rebels made an important advance. Instead of an individual's act of heroism or the assassination of an individual, theirs was a group action aimed at the organs of the colonial state. But the objective still was to set an example before the youth, and to demoralize the bureaucracy. As Kalpana Joshi(Dutt) has put it, the plan- was that when, after the Chittagong rebellion, `the Government would bring in troops to take back Chittagong they (the terrorists) would die fighting --- thus creating a legend and setting an example before their countrymen to emulate.' Or as Surya Sen told Ananda Gupta: `A dedicated band of youth must show the path of organized armed struggle in place of individual terrorism. Most of us will have to die in the process but our sacrifice for such a noble cause will not go in vain.' The Bengal revolutionaries of the l920s and 1930s had shed some of their earlier Hindu religiosity --- they no longer took religious oaths and vows. Some of the groups also no longer excluded Muslims --- the Chittagong IRA cadre included many Muslims like Sattar, Mir Ahmad, Fakir Ahmad Mian, Tunu Mian and got massive support from Muslim villagers around Chittagong. But they still retained elements of social conservatism, nor did they evolve broader socio-economic goals. In particular, those revolutionary terrorists, who worked in the Swaraj party, failed to support the cause of Muslim peasantry against the zamindars.

\begin{center}*\end{center}

\paragraph*{}

A real breakthrough in terms of revolutionary ideology and the goals of revolution and the forms of revolutionary struggle was made by Bhagat Singh and his comrades. Rethinking had, of course, started on both counts in the HRA itself. Its manifesto had declared in 1925 that it stood for `abolition of all systems which make the exploitation of man by man possible.'4 Its founding council, in its meeting in October 1924, had decided `to preach social revolutionary and communistic principles.' Its main organ, The Revolutionary, had proposed the nationalization of the railways and other means of transport and large-scale industries such as steel and ship building. The HRA had also decided `to start labour and peasant organizations' and to work for `an organized and armed revolution.'

In a message from the death-cell, Ramprasad Bismil had appealed to the youth to give up `the desire to keep revolvers and pistols', `not to work in revolutionary conspiracies,' and to work in `the open movement.' He had asked the people to establish Hindu-Muslim unity and unite all political groups under the leadership of the Congress. He had also affirmed his faith in communism and the principle that `every human being has equal rights over products of nature.'

Bhagat Singh, born in 1907 and a nephew of the famous revolutionary Ajit Singh, was a giant of an intellectual. A voracious reader, he was one of the most well-read of political leaders of the time. He had devoured books in the Dwarkadas Library at Lahore on socialism, the Soviet Union and revolutionary movements, especially those of Russia, Ireland and Italy. At Lahore, he organized several study circles with the help of Sukhdev and others and carried on intensive political discussions. When the HSRA office was shifted to Agra, he immediately set up a library and urged members to read and discuss socialism and other revolutionary ideas. His shirt pockets always bulged with books which he constantly offered to lend his comrades. After his arrest he transformed the jail into a veritable university. Emphasizing the role of ideas in the making of revolution, he declared before the Lahore High Court: `The sword of revolution is sharpened on the whetting-stone of ideas.' This atmosphere of wide reading and deep thinking pervaded the ranks of the HSRA leadership. Sukhdev, Bhagwati Charan Vohra, Shiv Varma, Bejoy Sinha, Yashpal, all were intellectuals of a high order. Nor would even Chandrashekar- Azad, who knew little English, accept any idea till it was frilly explained to him. He followed every major turn in the field of ideas through discussion. The draft of the famous statement of revolutionary position, The Philosophy of the Bomb, was written by Bhagwati Charan Vohra at the instance of Azad and after a full discussion with him.

Bhagat Singh had already, before his arrest in 1929, abandoned his belief in terrorism and individual heroic action. He had turned to Marxism and had come to believe that popular broad-based mass movements alone could lead to a successful revolution; in other words revolution could only be achieved `by the masses for the masses.' That is why Bhagat Singh helped establish the Punjab Naujawan Bharat Sabha in 1926 (becoming its founding Secretary), as the open wing of the revolutionaries. The Sabha was to carry out open political work among the youth, peasants and workers. It was to open branches in the villages. Under its auspices, Bhagat Singh used to deliver political lectures with the help of magic lantern slides. Bhagat Singh and Sukhdev also organized the Lahore Students Union for open, legal work among the students.

Bhagat Singh and his comrades also gave expression to their understanding that revolution meant the development and organization of a mass movement of the exploited and suppressed sections of society by the revolutionary intelligentsia in the course of their statements from 1929 to 1931 in the courts as well as outside. Just before his execution, Bhagat Singh declared that `the real revolutionary armies are in the villages and in factories.' Moreover, in his behest to young political workers, written on 2 February 1931, he declared: `Apparently, I have acted like a terrorist. But I am not a terrorist ... Let me announce with all the strength at my command, that I am not a terrorist and I never was, except perhaps in the beginning of my revolutionary career. And I am convinced that we cannot gain anything through those methods.'

Then why did Bhagat Singh and his comrades still take recourse to individual heroic action? One reason was the very rapidity of the changes in their thinking. The past formed a part of their present, for these young men had to traverse decades within a few years. Moreover, effective acquisition of a new ideology is not an event; it is not like a religious conversion: it is always a prolonged historical process. Second, they were faced with a classic dilemma: From where would come the cadres, the hundreds of full-time young political workers, who would fan out among the masses? How were they to be recruited? Patient intellectual and political work appealed to be too slow and too akin to the Congress style of politics which the revolutionaries wanted to transcend. The answer appeared to be to appeal to the youth through `propaganda by deed,' to recent the initial cadres of a mass revolutionary party through heroic dramatic action and the consequent militant propaganda before the courts. In the last stage, during 1930 and 1931, they were mainly fighting to keep the glory of the sacrifice of their comrades' wider sentence shining as before. As Bhagat Singh put it, he had to ask the youth to abandon revolutionary terrorism without tarnishing the sense of heroic sacrifice by appearing to have reconsidered his politics under the penalty of death.'' Life was bound to teach, sooner or later, correct politics; the sense of sacrifice once lost would not be easy to regain.

Bhagat Singh and his comrades also made a major advance in broadening the scope and definition of revolution. Revolution was no longer equated with mere militancy or violence. Its first objective was national liberation --- the overthrow of imperialism. But it must go beyond and work for a new socialist social order, it must bend exploitation of man by man.' The Philosophy of the Bomb, written by Bhagwati Charan Vohra. Chandrasekhar Azad and Yashpal, defined revolution as independence, social, political and economic' aimed at establishing `a new order of society in which political and economic exploitation will be an impossibility'.' In the Assembly Bomb Case, Bhagat Singh told the cowl. ``Revolution,'' does not necessarily involve sanguinary strife, nor is there any place in it for individual vendetta. It is not the cult of the bomb and the pistol. By ``Revolution'' we mean that the present order of things, which is based on manifest injustice, must change.'' In a letter from jail, he wrote: `The peasants have to liberate themselves not only from foreign yoke bum also from the yoke of landlords and capitalists.'' In his last message of 3 March 1931, he declared that the struggle in India would continue so long as `a handful of exploiters go on exploiting the labour of common people for their own ends. It matters little whether these exploiters are purely British capitalism, or British and Indians in alliance, or even purely Indians.' ``(Bhagat Singh defined socialism in a scientific manner --- it must mean abolition of capitalism and class domination. He fully accepted Marxism and the class approach to society.1 In fact, he saw himself above all as a Precursor and not maker of the `revolution, as a propagator of the ideas of socialism ad communism as a humble initiator of the socialist movement in India.'

Bhagat Singh was a great innovator in two areas of politics. Being fully and consciously secular, he understood, more clearly than many of his contemporaries, the danger that communalism posed to the nation and the national movement. He often told his audience that communalism was as big an enemy as colonialism.

In April 1928, at the conference of youth where Naujawan Bharat Sabha was reorganized, Bhagat Singh and his comrades openly opposed the suggestion that youth belonging to religious- communal organizations should be permitted to become members of the Sabha. Religion was one's private concern and communalism was an enemy to be fought, argued Bhagat Singh.'' Earlier in 1927, condemning communal killings as barbaric, he had pointed out that communal killers did not kill a person because he was guilty of any particular act but simply because that person happened to be a Hindu, Muslim or Sikh. But, wrote Bhagat Singh, a new group of youth was coming forward who did not recognize any differences based on religion and saw a person first as a human being and then as an Indian.

Bhagat Singh revered Lajpat Rai as a leader. But he would not spare even Lajpat Rai, when, during the last years of his life, Lajpat Rai turned to communal politics. He then launched a political-ideological campaign against him. Because Lajpat Rai was a respected leader, he would not publicly use harsh words of criticism against him. And so he printed as a pamphlet Robert Browning's famous poem, `The Lost Leader,' in which Browning criticizes Wordsworth for turning against liberty. The poem begins with the line `Just for a handful of silver he left us.' A few more of the poem's lines were: `We shall march prospering, --- not thro' his presence; Songs may inspirit us, --- not from his lyre,' and `Blot out his name, then, record one lost soul more.' There was not one word of criticism of Lajpat Rai. Only, on the front cover, he printed Lajpat Rai's photograph!

Significantly, two of the six rules of the Naujawan Bharat Sabha, drafted by Bhagat Singh, were: `To have nothing to do with communal bodies or other parties which disseminate communal ideas' and `to create the spirit of general toleration among the public considering religion as a matter of personal belief of man and to act upon the same fully.'

Bhagat Singh also saw the importance of freeing the people from the mental bondage of religion and superstition. A few weeks before his death, he wrote the article `Why I am an Atheist' in which he subjected religion and religious philosophy to a scathing critique. He traced his own path to atheism, how he first gave up belief `in the mythology and doctrines of Sikhism or any other religion,' and in the end lost faith in the existence of God. To be a revolutionary, he said, one required immense moral strength, but one also required `criticism and independent thinking.' In the struggle for self-emancipation, humanity had to struggle against `the narrow conception of religion' as also against the belief in God. `Any man who stands for progress,' he wrote, `has to criticise, disbelieve and challenge every item of the old faith. Item by item he has to reason out every nook and corner of the prevailing faith.' Proclaiming his own belief in atheism and materialism, he asserted that he was `trying to stand like a man with an erect head to the last; even on the gallows.'

\begin{center}*\end{center}

\paragraph*{}

Government action gradually decimated the revolutionary terrorist ranks. With the death of Chandrashekhar Azad in a shooting encounter a public park at Allahabad in February 1931, the revolutionary terrorist movement virtually came to an end in Punjab, U.P. and Bihar. Surya Sen's martyrdom marked an end to the prolonged saga of revolutionary terrorism in Bengal' A process of rethinking in jails and in the Andamans began large number of the revolutionaries turned to Marxism and the idea of a socialist revolution by the masses. They joined the Communist Party, the Revolutionary Socialist Party, and other Left parties. Many others joined the Gandhian wing of the Congress. The politics of the revolutionary terrorists had severe limitations --- above all theirs was not the politics of a mass movement; they failed to politically activate the masses or move them into political actions; they could not even establish contact with the masses. All the same, they made an abiding contribution to the national freedom movement. Their deep patriotism, courage and determination, and sense of sacrifice stirred the Indian people. They helped spread nationalist consciousness in the land; and in northern India the spread of socialist consciousness owed a lot to them.
\end{multicols}