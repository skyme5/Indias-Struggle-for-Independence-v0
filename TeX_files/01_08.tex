\chapter{The Fight to Secure Press Freedom}
\begin{multicols}{2}

Almost from the beginning of the 19th century, politically conscious Indians had been attracted to modem civil rights, especially the freedom of the Press. As early as 1824, Raja Rammohan Roy had protested against a regulation restricting the freedom of the Press. In a memorandum to the Supreme Court, he had said that every good ruler `will be anxious to afford every individual the readiest means of bringing to his notice whatever may require his interference. To secure this important object, the unrestricted liberty of publication is the only effectual means that can be employed.'

In the period from 1870 to 1918, the national movement had not yet resorted to mass agitation through thousands of small and large maidan meetings, nor did political work consist of the active mobilization of people in mass struggles. The main political task still was that of politicization, political propaganda and education and formation and propagation of nationalist ideology. The Press was the chief instrument for carrying out this task, that is, for arousing, training, mobilizing and consolidating nationalist public opinion.

Even the work of the National Congress was accomplished during these years largely through the Press. The Congress had no organization of its own for carrying on political work. Its resolutions and proceedings had to be propagated through newspapers. Interestingly, nearly one-third of the founding fathers of the Congress in 1885 were journalists.

Powerful newspapers emerged during these years under distinguished and fearless journalists. These were the Hindu and Swadesamitran under the editorship of G. Subramaniya Iyer, Kesari and Mahratta under B.G. Tilak, Bengalee under Surendranath Banerjee, Amrita Bazar Patrika under Sisir Kumar Ghosh and Motilal Ghosh, Sudharak under G.K. Gokhale, Indian Mirror under N.N. Sen, Voice of India under Dadabhai Naoroji, Hindustani and Advocate under G.P. Varma and Tribune and Akhbar-i-Am in Punjab, Indu Prakash, Dnyan Prakash, Kal and Gujarati in Bombay, and Som Prakash, Banganivasi, and Sadharani in Bengal. In fact, there hardly existed a major political leader in India who did not possess a newspaper or was not writing for one in some capacity or the other.

The influence of the Press extended far beyond its literate subscribers. Nor was it confined to cities and large towns. A newspaper would reach remote villages and would then be read by a reader to tens of others. Gradually library movements sprung up all over the country. A local `library' would be organized around a single newspaper. A table, a bench or two or a charpoy would constitute the capital equipment. Every piece of news or editorial comment would be read or heard and thoroughly discussed. The newspaper not only became the political educator; reading or discussing it became a form of political participation.

Newspapers were not in those days business enterprises, nor were the editors and journalists professionals. Newspapers were published as a national or public service. They were often financed as objects of philanthropy. To be a journalist was often to be a political worker and an agitator at considerable self-sacrifice. It was, of course, not very expensive to start a newspaper, though the editor had usually to live at a semi starvation level or earn his livelihood through a supplementary source. The Amrita Bazar Patrika was started in 1868 with printing equipment purchased for Rs. 32. Similarly, Surendranath Banerjee purchased the goodwill of the Bengalee in 1879 for Rs. 10 and the press for another Rs. 1600.

Nearly all the major political controversies of the day were conducted through the Press. It also played the institutional role of opposition to the Government. Almost every act and every policy of the Government was subjected to sharp criticism, in many cases with great care and vast learning backing it up. `Oppose, oppose, oppose' was the motto of the Indian Press. Regarding the role of the nationalist Press, Lord Dufferin, the Viceroy, wrote as early as March 1886: `Day after day, hundreds of Sharp-witted babus pour forth their indignation against their English Oppressors in very pungent and effective diatribe.' And again in May: `In this way there can be no doubt there is generated in the minds of those who read these papers ... a sincere conviction that we are all enemies of mankind in general and of India in particular.`

To arouse political consciousness, to inculcate nationalism, to expose colonial rule, to `preach disloyalty' was no easy task, for there had existed since 1870 Section 124A of the Indian Penal Code according to Which `whoever attempts to excite feelings of disaffection to the Government established by law in British India' was to be punished with transportation for life or for any term or with imprisonment upto three years. This clause was, moreover, later supplemented with even more strident measures.

Indian journalists adopted several clever strategems and evolved a distinctive style of writing to remain outside the reach of the law. Since Section 124A excluded writings of persons whose loyalty to the Government was undoubted, they invariably prefaced their vitriolic writing with effusive sentiments of loyalty to the Government and the Queen. Another strategem was to publish anti-imperialist extracts from London-based socialist and Irish newspapers or letters from radical British citizens knowing that the Indian Government could not discriminate against Indians by taking action against them without touching the offending Britishers. Sometimes the extract from the British newspaper would be taken without quotation marks and acknowledgement of the source, thus teasing the British-Indian bureaucracy into contemplating or taking action which would have to be given up once the real source of the comment became known. For example, a sympathetic treatment of the Russian terrorist activities against Tsarism would be published in such a way that the reader would immediately draw a parallel between the Indian Government and the Revolutionary Terrorists of Bengal and Maharashtra. The officials would later discover that it was an extract from the Times, London, or some such other British newspaper.

Often the radical expose would take the form of advice and warning to the Government as if from a well-wisher, as if the writer's main purpose was to save the authorities from their own follies! B.G. Tilak and Motilal Ghosh were experts at this form of writing. Some of the more daring writers took recourse to irony, sarcasm, banter, mock-seriousness and burlesque.

In all cases, nationalist journalists, especially of Indian language newspapers, had a difficult task to perform, for they had to combine simplicity with subtlety --- simplicity was needed to educate a semi-literate public, subtlety to convey the true meaning without falling foul of the law. They performed the task brilliantly, often creatively developing the languages in which they were willing, including, surprisingly enough, the English language.

The national movement from the beginning zealously defended the freedom of the Press whenever the Government attacked it or tried to curtail it. In fact, the struggle for the freedom of the Press became an integral part of the struggle for freedom.

\begin{center}*\end{center}

\paragraph*{}

Indian newspapers began to find their feet in the 1870s. They became highly critical of Lord Lytton's administration, especially regarding its inhuman approach towards the victims of the famine of 1876--77. As a result the Government decided to make a sudden strike at the Indian language newspapers, since they reached beyond the middle class readership. The Vernacular Press Act of 1878, directed only against Indian language newspapers, was conceived in great secrecy and passed at a single sitting of the Imperial Legislative Council. The Act provided for the confiscation of the printing press, paper and other materials of a newspaper if the Government believed that it was publishing seditious materials and had flouted an official warning.

Indian nationalist opinion firmly opposed the Act. The first great demonstration on an issue of public importance was organized in Calcutta on this question when a large meeting was held in the Town Hall. Various public bodies and the Press also campaigned against the Act. Consequently, it was repealed in 1881 by Lord Ripon.

The manner in which the Indian newspapers cleverly fought such measures was brought out by a very amusing and dramatic incident. The Act was in particular aimed at the Amrita Bazar Patrika which came out at the time in both Bengali and English.

The objective was to take summary action against it. But when the officials woke up the morning after the Act was passed, they discovered to their dismay that the Patrika had foxed them; overnight, the editors had converted it into an English newspaper!

\begin{center}*\end{center}

\paragraph*{}

Another remarkable journalistic coup occurred in 1905. Delivering the Convocation Address at Calcutta University, Lord Curzon, the Viceroy said that `the highest ideal of truth is to a large extent a Western conception. Undoubtedly, truth took a high place in the moral codes of the West before it had been similarly honored in the East.' The insinuation was that the British had taught this high Conception of truth to Indians.

Next day, the Amrita Bazar Patrika came out with this speech on the front page along with a box reproducing an extract from Curzon's book the Problems of the East in which he had taken credit for lying while a visit to Korea. He had written that he had told the President of the Korean Foreign Office that he was forty when he was actually thirty-nine because he had been told that in the East respect went with age. He has ascribed his youthful appearance to the salubrious climate of Korea! Curzon had also recorded his reply to the President's question whether he was a near relation of Queen Victoria as follows: ```No,'' I replied, ``I am not.'' But observing the look of disgust that passed over his countenance, I was fain to add, ``I am, however, as yet an unmarried man,'' with which unscrupulous suggestion I completely regained the old gentleman's favour.'

The whole of Bengal had a hearty laugh at the discomfiture of the strait-laced Viceroy, who had not hesitated to insult an entire people and who was fond of delivering homilies to Indians. The Weekly Times of London also enjoyed the episode. Lord Curzon's `admiration for truth,' it wrote, `was perhaps acquired later on in life, under his wife's management. It is pre-eminently a Yankee quality.' (Curzon's wife was an American heiress).

\begin{center}*\end{center}

\paragraph*{}

Surendranath Banerjee, one of the founding fathers of the Indian national movement, was the first Indian to go to jail in performance of his duty as a journalist. A dispute concerning a family idol, a saligram, had come up before Justice Norris of the Calcutta High Court. To decide the age of the idol, Norris ordered it to be brought to the Court and pronounced that it could not be a hundred years old. This action deeply hurt the sentiments of the Bengali Hindus. Banerjea wrote an angry editorial in the Bengalee of 2 April 1883. Comparing Norris with the notorious Jeffreys and Seroggs (British judges in the 17th century, notorious for infamous conduct as judges), he said that Norris had done enough `to show how unworthy he is of his high office.' Banerjea suggested that `some public steps should be en to put a quietus to the wild eccentricities of this young and raw Dispenser of Justice'.

Immediately, the High Court hauled him up for contempt of court before a bench of five judges, four of them Europeans. With the Indian judge, Romesh Chandra Mitra, dissenting, the bench convicted and sentenced him to two months imprisonment. Popular reaction was immediate and angry. There was a spontaneous hartal in the Indian part of Calcutta. Students demonstrated outside the courts smashing windows and pelting the police with stones. One of the rowdy young men was Asutosh Mukherjea who later gained fame as a distinguished Vice Chancellor of Calcutta University. Demonstrations were held all over Calcutta and in many other towns of Bengal as also in Lahore, Amritsar, Agra, Faizabad , Poona and other cities. Calcutta witnessed for the first time several largely attended open-air meetings.

\begin{center}*\end{center}

\paragraph*{}

But the man who is most frequently associated with the struggle for the freedom of the Press during the nationalist movement is Bal Gangadhar Tilak, the outstanding leader of militant nationalism. Born in 1856, Tilak devoted his entire life to the service of his country. In 1881, along with G.G. Agarkar, he founded the newspaper Kesari (in Marathi) and Mahratta (in English). In 1888, he took over the two papers and used their columns to spread discontent against British rule and to preach national resistance to it. Tilak was a fiery and courageous journalist whose style was simple and direct and yet highly readable.

In 1893, he started the practice of using the traditional religious Ganapati festival to propagate nationalist ideas through patriotic songs and speeches. In 1896, he started the Shivaji festival to stimulate nationalism among young Maharashtrians. In the same year, he organized an all-Maharashtra campaign for the boycott of foreign cloth in protest against the imposition of the excise duty on cotton. He was, perhaps the first among the national leaders to grasp the important role that the lower middle classes, peasants, artisans and workers could play in the national movement and, therefore, he saw the necessity of bringing them into the Congress fold. Criticizing the Congress for ignoring the peasant, he wrote in the Kesari in early 1897: `The country's emancipation can only be achieved by removing the clouds of lethargy and indifference which have been hanging over the peasant, who is the soul of India. We must remove these clouds, and for that we must completely identify ourselves with the peasant --- we must feel that he is ours and we are his.' Only when this is done would `the Government realize that to despise the Congress is to despise the Indian Nation. Then only will the efforts of the Congress leaders be crowned with success.'

In pursuance of this objective, he initiated a no-tax Campaign in Maharashtra during 1896--97 with the help of the young workers of the Poona Sarvajanik Sabha. Referring to the official famine code whose copies he got printed in Marathi and distributed by the thousand, he asked the famine-stricken peasants of Maharashtra to withhold payment of land revenue if their crops had failed. In 1897, plague broke out in Poona and the Government had to undertake severe measures of segregation and house-searches. Unlike many other leaders, Tilak stayed in Poona, supported the Government and organized his own measures against the plague. But he also criticized the harsh and heartless manner in which the officials dealt with the plague-stricken people. Popular resentment against the official plague measures resulted in the assassination of Rand, the Chairman of the Plague Committee in Poona, and Lt. Ayerst by the Chaphekar brothers on 27 June 1898.

The anti-plague measures weren't the only practices that made the people irate. Since 1894, anger had been rising against the Government because of its tariff, currency and famine policy. A militant trend was rapidly growing among the nationalists and there were hostile comments in the Press. The Government was determined to check this trend and teach a lesson to the Press. Tilak was by now well-known in Maharashtra, both as a militant nationalist and as a hostile arid effective journalist. The Government was looking for an opportunity to make an example of him. The Rand murder gave them the opportunity. The British-owned Press and the bureaucracy were quick to portray the Rand murder as a conspiracy by the Poona Brahmins led by Tilak. The Government investigated the possibility of directly involving Tilak in Rand's assassination. But no proof could be found. Moreover, Tilak had condemned the assassination describing it as the horrible work of a fanatic, though he would not stop criticizing the Government, asserting that it was a basic function of the Press to bring to light the unjust state of affairs and to teach people how to defend their rights. And so, the Government decided to arrest him under Section 124A of the Indian Penal Code on the charge of sedition, that is, spreading disaffection and hatred against the Government.

Tilak was arrested on 27 July 1879 arid tried before Justice Strachey and a jury of six Europeans and three Indians. The charge was based on the publication in the Kesari of 15 June of a poem titled `Shivaji's Utterances' `read out by a young man at the Shivaji Festival and on a speech Tilak had delivered at the Festival in defence of Shivaji's killings of Afzal Khan.

In `Shivaji's Utterances,' the poet had shown Shivaji awakening in the present and telling his countrymen: `Alas! Alas! I now see with my own eyes the ruin of my country ... Foreigners are dragging out Lakshmi violently by the hand (kar in Marathi which also means taxes) and by persecution ... The wicked Akabaya (misfortune personified) stalks with famine through the whole country ... How have all these kings (leaders) become quite effeminate like helpless figures on the chess-board?'

Tilak's defence of Shivaji's killing of Afzal Khan was portrayed by the prosecution as an incitement to kill British officials. The overall accusation was that Tilak propagated the views in his newspaper, that the British had no right to stay in India and any and all means could be used to get rid of them.

Looking back, it is clear that the accusation was not wrong. But the days when, under Gandhiji's guidance, freedom fighters would refuse to defend themselves and openly proclaim their sedition were still far off. The politics of sacrifice and open defiance of authority were still at an early stage. It was still necessary to claim that anti-colonial activities were being conducted within the limits of the law. And so Tilak denied the official charges and declared that he had no intention of preaching disaffection against alien rule. Within this `old' style of facing the rulers, Tilak set a high example of boldness and sacrifice. He was aware that he was initiating a new kind of politics which must gain the confidence and faith of the people by the example of a new type of leader, while carefully avoiding premature radicalism which would invite repression by the Government and lead to the cowing down of the people and, consequently, the isolation of the leaders from the people.

Pressure was brought upon Tilak by some friends to withdraw his remarks and apologise. Tilak's reply was: My position (as a leader) amongst the people entirely depends upon my character ... Their (Government's) object is to humiliate the Poona leaders, and I think in me they will not find a ``kutcha'' (weak) reed ... Then you must remember beyond a certain stage we are all servants of the people. You will be betraying and disappointing them if you show a lamentable Want of courage at a critical time.'

Judge Strachey's partisan summing up to the jury was to gain notoriety in legal circles, for he defined disaffection as `simply the absence of affection' which amounted to the presence of hatred, enmity, disloyalty and every other form of ill-will towards the Government! The jury gave a 6 to 3 verdict holding Tilak guilty, the three dissenters being its Indian members. The Judge passed a barbarous sentence of rigorous imprisonment for eighteen months, and this when Tilak was a member of the Bombay Legislative Council! Simultaneously several other editors of Bombay Presidency were tried and given similar harsh sentences.

Tilak's imprisonment led to widespread protests all over the county Nationalist newspapers and political associations, including those run by Tilak's critics like the Moderates, organized a countrywide movement against this attack on civil liberties and the fiefdom of the Press. Many newspapers came out with black borders on the front page. Many published special supplements hailing Tilak as a martyr in the battle for the freedom of the Press. Addressing Indian residents in London, Dadabhai Naoroji accused the Government of initiating Russian (Tsarist) methods of administration and said that gagging the Press was simply suicidal.

Overnight Tilak became a popular all-India leader and the title of Lokamanya (respected and honored by the people) was given to him. He became a hero, a living symbol of the new spirit of self-sacrifice a new leader who preached with his deeds. When at the Indian National Congress session at Amraoti in December 1897, Surendranath Banerjee made a touching reference to Tilak and said that `a whole nation is in tears,' the entire audience stood up and enthusiastically cheered.

In 1898, the Government amended Section 124A and added a new Section 153A to the penal code, making it a criminal offence for anyone to attempt `to bring into contempt' the Government of India or to create hatred among different classes, that is vis-a-vis Englishmen in India. This once again led to nation-wide protest.

\begin{center}*\end{center}

\paragraph*{}
The Swadeshi and Boycott Movement, which we shall look at in more detail later on in CHAPTER \ref{chapter:CH10}, led to a new wave of repression in the country. The people once again felt angry and frustrated. This frustration led the youth of Bengal to take to the path of individual terrorism. Several cases of bomb attacks on officials Occurred in the beginning of 1908. The Government felt unnerved. Once again newspapers became a major target Fresh laws for Controlling the Press were enacted, prosecutions against a large number of newspapers and their editors were launched and the Press was almost completely Suppressed In this atmosphere it was inevitable that the Government's attention would turn towards Lokamanya Tilak, the mainstay of the Boycott movement and militant politics outside Bengal. Tilak wrote a series of articles on the arrival of the `Bomb' on the Indian scene. He condemned the use of violence and individual killings he described Nihilism as `this Poisonous tree' --- but, simultaneously, he held the Government responsible for suppressing criticism and dissent and the urge of the people for greater freedom. In such an atmosphere, he said `violence, however deplorable, became inevitable.' As he wrote in one of his articles: `When the official class begins to overawe the people without any reason and when an endeavour is made to produce despondency among the people unduly frightening them, then the sound of the bomb is spontaneously produced to impart to the authorities the true knowledge that the people have reached a higher stage than the vapid one in which they pay implicit regard to such an illiberal policy of repression.'

Once again, on 24 June 1908, Tilak was arrested and tried on the charge of sedition for having published these articles. Once again Tilak pleaded not guilty and behaved with exemplary courage. A few days before his arrest, a friendly police officer warned him of the coming event and asked Tilak to take precautionary steps. Tilak laughed and said: The Government has converted the entire nation into a prison and we are all prisoners. Going to prison only means that from a big cell one is confined to a smaller one.''In the court, Tilak posed the basic question: `Tilak or no Tilak is not the question. The question is, do you really intend as guardians of the liberty of the Press to allow as much liberty here in India as is enjoyed by the people of England?''

Once again the jury returned a verdict of guilty with only the two Indian members opposing the verdict. Tilak's reply was: `There are higher powers that rule the destiny of men and nations; and it may be the will of Providence that the cause which I represent may prosper more by my sufferings than by my remaining free.' Justice Davar awarded him the sentence of six years' transportation and after some time the Lokamanya was sent to a prison in Mandalay in Burma.

The public reaction was massive. Newspapers proclaimed that they would defend the freedom of the Press by following Tilak's example. All markets in Bombay city were closed on 22 July, the day his was announced, and remained closed for a week. The Workers of all the textile mills and railway workshops went on strike for six days. Efforts to force them to go back to work led to a battle between them and the Police. The army was called out and at the end of the battle sixteen workers lay dead in the streets with nearly fifty others seriously injured. Lenin hailed this as `the entrance of the Indian working class on the political stage.' Echoes of Tilak's trial were to be heard in another not-so-distant court when Gandhiji, his political successor, was tried in 1922 for the same offence of sedition under the same Section 124A for his articles in Young India. When the Judge told him that his offence was similar to Tilak's and that he was giving him the same sentence of six years' imprisonment Gandhiji replied: `Since you have done me the honor of recalling the trial of the late Lokamanya Bal Gangadhar Tilak, I just want to say that I consider it to be proudest privilege and honor to be associated with his name.''

The only difference between the two trials was that Gandhiji had pleaded guilty to the charges. This was also a measure of the distance the national movement had travelled since 1908. Tilak's contribution to this change in politics and journalism had been momentous.
\end{multicols}{2}
