\chapter{The Rise and Growth of Communalism}
\begin{multicols}{2}

Before we discuss the growth of communalism in modern India, it is perhaps useful to define the term and point to certain basic fallacies regarding it. Communalism is basically an ideology with which we have lived so long that it appears to be a simple, easily understood notion. But this is, perhaps, not so. 

Communalism or communal ideology consists of three basic elements or stages, one following the other. First, it is the belief that people who follow the same religion have common secular interests, that is, common political, economic, social and cultural interests. This is the first bedrock of communal ideology. From this arises the notion of socio-political communities based on religion. It is these religion-based communities, and not classes, nationalities, linguistic-cultural groups, nations or such politico- territorial units as provinces or states that are seen as the fundamental units of Indian society. The Indian people, it is believed, can act socially aid politically and protect their collective or corporate or non-individual interests only as members of these religion-based communities. These different communities are alleged to have their own leaders. Those who t.al of being national, regional, or class leaders are merely masquerading; beneath the mask they are only leaders of their own communities. The best they can do is to unite as communal leaders and then serve the wider category of the nation or country. 

The second clement of communal ideology rests on the notion that in multi-religious society like India, the secular interests, that is the social, cultural, economic and political interests, of the followers of one religion are dissimilar and divergent from the interests of the followers of another. 

The third stage of communalism is reached when the interests of the followers of different religions or of different `communities' are seen to be mutually incompatible, antagonistic and hostile. Thus, the communalist asserts this stage that Hindus and Muslims cannot have common secular interests, that their secular interests are bound to be opposed to each other. 

Communalism is, therefore, basically and above all an ideology on which communal politics is based. Communal violence is a conjunctural consequence of communal ideology. Similarly, Hindu, Muslim, Sikh or Christian communalisms are not very different from each other; they belong to a single species; they are varieties of the same communal ideology. 

Communal ideology in a person, party or movement starts with the first stage. Many nationalists fell prey to it or thought within its digits even while rejecting the two other elements of communalism, that is, the notion of the mutual divergence or hostility of the interests of different religion- based communities. These were the persons who saw themselves as Nationalist Hindus, Nationalist Muslims, Nationalist Sikhs, etc., and not as simple nationalists. 

The second stage of communalism may be described as liberal communalism or, in the words of some, moderate communalism. The liberal communalist was basically a believer in and practitioner of communal politics; but he still upheld certain liberal, democratic, humanist and nationalist values. Even while holding that India consisted of distinct religion-based communities, with their own separate and special interests which sometimes came into conflict with each other, he continued to believe and profess publicly that these different communal interests could be gradually accommodated and brought into harmony within the overall, developing national interests, and India built as a nation. Most of the communalists before 1937 --- the Hindu Mahasabha, the Muslim League, the All Brothers after 1925, M.A. Jinnah, Madan Mohan Malaviya, Lajpat Rai, and N.C. Kelkar after 1922 --- functioned within a liberal communal framework. 

Extreme communalism, or communalism functioning broadly within a fascist syndrome, formed the third or last stage of communalism. Extreme communalism was based on fear and hatred, and had a tendency to use violence of language, deed or behaviour, the language of war and enmity against political opponents. It was at this stage that the communalists declared that Muslims, `Muslim culture' and Islam and Hindus, `Hindu culture, and Hinduism were in danger of being suppressed and exterminated. It was also at this stage that both the Muslim and Hindu communalists put forward the theory that Muslims and Hindus constituted separate nations whose mutual antagonism was permanent and irresolvable. The Muslim League and the Hindu Mahasabha after 1937 and the Rashtriya Swayamsevak Sangh (RSS) increasingly veered towards extreme or fascistic communalism. 

Though the three stages of communalism were different from one another, they also interacted and provided a certain continuum. Its first element or stage fed liberal and extreme communalism and made it difficult to carry on a struggle against them. Similarly, the liberal communalist found it difficult to prevent the ideological transition to extreme communalism. 

We may take note of several other connected aspects. While a communalist talked of, or believed in, defending his `community's' interests, in real life no such interests existed outside the field of religion. The economic and political interests of Hindus, Muslims, and others were the same. In that sense they did not even constitute separate communities. As Hindus or Muslims they did not have a separate political-economic life or interests on an all-India or even regional basis. They were divided from fellow Hindus or Muslims by region, language, culture, class, caste, social status, social practices, food and dress habits, etc., and united on these aspects with follower of other religions. An upper class Muslim had far mc in common, even culturally, with an upper class Hindu than with a ka class Muslim. Similarly, a Punjabi Hindu stood closer culturally to a Punjabi Muslim than to a Bengali Hindu; and, of course, the same was true of a Bengali Muslim in relation to a Bengali Hindu and a Punjabi Muslim. The unreal communal division, thus, obscured the real division of the Indian people into linguistic-cultural regions and social classes as well as their real, emerging and growing unity into a nation. 

If communal interests did not exist, then communalism was not a partial or one-sided or sectional view of the social reality; it was its wrong \& unscientific view. It has been suggested, on occasions, that a communalist being narrow-minded, looks after his own community's interests. But if no such interests existed, then he could not be serving his `community's' or co-religionists interests either. He could not be the `representative' of his community. In the name of serving his community's interests, he served knowingly or unknowingly some other interests. He, therefore, either deceived others or unconsciously deceived himself. Thus, communal assumptions, communal logic and communal answers were wrong. What the communalist projected as problems were not the real problems, and what the communalist said was the answer was not the real answer. 

Sometimes, communalism is seen as something that has survived from the past, as something that the medieval period has bequeathed to the present or at least as having roots in the medieval period. But while communalism uses, and is based on, many elements of ancient and medieval ideologies, basically it is a modern technology and political trend that expresses the social urges and serves the political needs of modem social groups, classes and forces. Its social roots as also its social, political and economic objectives lie very much in the modem period of Indian history. It was brought into existence and sustained by contemporary socio-economic structure. 

Communalism emerged as a consequence of the emergence of modern politics which marked a sharp break with the politics of the ancient or medieval or pre-1857 periods. Communalism, as also other modem views such as nationalism and socialism, could emerge as politics and as ideology only after politics based on the people, politics of popular participation and mobilization, politics based on the creation and mobilization of public opinion had come into existence. In pre-modern politics, people were either ignored in upper-class based politics or were compelled to rebel outside the political system and, in case of success, their leaders incorporated into the old ruling classes. This was recognized by many perceptive Indians. Jawaharlal Nehru, for example, noted in 1936: `One must never forget that communalism in India is a latter-day phenomenon which has grown up before our eyes.'' Nor was there anything unique about communalism in the Indian context. It was not an inevitable or inherent product of India's peculiar historical and social development. It was the result of conditions which have in other societies produced similar phenomena and ideologies such as Fascism, anti-Semitism, racism, Catholic-Protestant conflict in Northern Ireland, or Christian- Muslim conflict in Lebanon. 

The communal consciousness arose as a result of the transformation of Indian society under the impact of colonialism and the need to struggle against it. The growing economic, political and administrative unification of regions and the country, the process of making India into a nation, the developing contradiction between colonialism and the Indian people and the formation of modem social classes and strata called for new ways of seeing one's common interests. They made it necessary to have wider links and loyalties among the people and to form new identities. This also followed from the birth of new politics during the last half of the 19th century. The new politics was based on the politicization and mobilization of an ever increasing number of the Indian people. 

The process of grasping the new, emerging political reality and social relations and the adoption of new uniting principles, new social and political identities with the aid of new ideas and concepts was bound to be a difficult and gradual process. The process required the spread of modem ideas of nationalism, cultural-linguistic development and class struggle. But wherever their growth was slow and partial, people inevitably used the old, familiar pre-modern categories of self-identity such as caste, locality, region, race, religion, sect and occupation to grasp the new reality, to make wider connections and to evolve new identities and ideologies. Similar developments have occurred all over the world in similar circumstances. But often such old, inadequate and false ideas and identities gradually give way to the new, historically necessary ideas and identities of nation, nationality and class. This also occurred on a large scale in India, but not uniformly among all the Indian people, in particular, religious consciousness was transformed into communal consciousness in some parts of the country and among some sections of the people. This as because there were some factors in the Indian situation which favoured its growth; it served the needs of certain sections of society and certain social and political forces. The question is why did communalism succeed in growing during the 20th century? What aspects of the Indian situation favoured this process? Which social classes and political forces did it serve? Why did it become such a pervasive pan of Indian reality? Though it as n inherent or inevitable in the situation, it was not a mere conspiracy of power-hungry politicians and crafty administrators either. It had socio-economic and political roots. There was a social situation which was funnelling it and without which it could not have survived for long.

\begin{center}*\end{center}

\paragraph*{}

Above all, communalism was one of the by-products of the colonial character of Indian economy, of colonial underdevelopment, of the incapacity of colonialism to develop the Indian economy. The resulting economic stagnation and its impact on the lives of the Indian people, especially the middle classes, produced conditions which were conducive to division and antagonism within Indian society as also to its radical transformation. 

Throughout the 20th century, in the absence of modem industrial development and the development of education, health and other social and cultural Services, unemployment was an acute problem in India, especially for the educated middle and lower middle classes who could not fall back on land and whose socio-economic conditions suffered constant deterioration. These economic opportunities declined further during the Great Depression after 1928 when large scale unemployment prevailed. 

In this social situation, the nationalist and other popular movements worked for the long-term solution to the people's problems by fighting for the overthrow of colonialism and radical social transformation. In fact, the middle classes formed the backbone both of the militant national movement from 1905 to 1947 and the left-wing parties and groups since the 1920s. Unfortunately there were some who lacked a wider social vision and political understanding and looked to their narrow immediate interests and short-term solutions to their personal or sectional problems such as communal, caste, or provincial reservation in jobs or in municipal committees, legislatures, and so on. 

Because of economic stagnation, there was intense competition among individuals for government jobs, in professions like law and medicine, and in business for customers and markets. In an attempt to get a larger share of existing economic opportunities, middle class individuals freely used all the means at their disposal --- educational qualifications, personal merit as also nepotism, bribery, and so on. At the same time, to give their struggle a wider base, they also used other group identities such as caste, province and religion to enhance their capacity to compete. Thus, some individuals from the middle classes could, and did, benefit, in the short run, from communalism, especially in the field of government employment. This gave a certain aura of validity to communal politics. The communalist could impose his interpretation of reality on middle class' individuals because it did have a basis, however partial, perverted and short-term, in the social existence and social experience of the middle classes. 

Gradually, the spread of education to well-off peasants and small landlords extended the boundaries of the job-seeking middle class to the rural areas. The newly educated rural youth could not be sustained by land whether as land lords or peasants, especially as agriculture was totally stagnant because of the colonial impact. They flocked on the towns and cities for opening in government jobs and professions and tried to save themselves by fighting for jobs through the system of communal reservations and nominations. This development gradually widened the social base of communalism to cover the rural upper strata of peasants and landlords. 

Thus, the crisis of the colonial economy constantly generated two opposing sets of ideologies and political tendencies among the middle classes. When anti-imperialist revolution and social change appeared on the agenda, the middle classes enthusiastically joined the national and other popular movements. They then readily advocated the cause and demands of the entire society from the capitalists to the peasants and workers. Individual ambitions were then sunk in the wider social vision. But, when prospects of revolutionary change receded, when the anti-imperialist struggle entered a more passive phase, many belonging to the middle classes shifted to short-term solutions of their personal problems, to politics based on communalism and other similar ideologies. Thus with the same social causation, large sections of the middle classes in several parts of the country constantly oscillated between anti- imperialism and communalism or communal-type politics. But, there was a crucial different in the two cases. In the first case, their own social interests merged with interests of general social development and their politics formed a part of the broader anti- imperialist struggle. In the second case, they functioned as a narrow and selfish interest group, accepted the socio- political status and objectively served colonialism. 

To sum up this aspect: communalism was deeply rooted in and was an expression of the interests and aspirations of the middle classes in a social situation in which opportunities for them were grossly inadequate. The communal question was, therefore a middle class question par excellence. The main appeal of communalism and its main social base also lay among the middle classes. It is, however, important to remember that a large number of middle class individuals remained, on the whole, free of communalism even in the l930s and 1940s. This was, in particular, true of most of the intellectuals, whether Hindu, Muslim or Sikh. In fact, the typical Indian intellectual of the l930s tended to be both secular and broadly left-wing.

\begin{center}*\end{center}

\paragraph*{}

There was another aspect of the colonial economy that favoured communal politics. In the absence of openings in industry, commerce, education and other social services, and the cultural and entertainment fields, the Government service was the main avenue of employment for the middle classes. Much of the employment for teachers, doctors and engineers was also under government control. As late as 1951, while 1.2 million persons were covered by the Factory Acts, 3.3 millions got employment in government service. And communal politics could be used to put pressure on the Government to reserve and allocate its jobs as also seats in professional colleges on communal and caste lines. Consequently, communal politics till 1937 was organized around government jobs, educational concessions, and the like as also political positions --- seats in legislative councils, municipal bodies, etc. --- which enabled control over these and other economic opportunities. It may also be noted that though the communalists spoke in the name of their `communities,' the reservations, guarantees and other `rights' they demanded were virtually confined to these two aspects. They did not take up any issues which were of interest to the masses.

\begin{center}*\end{center}

\paragraph*{}

At another plane, communalism often distorted or misinterpreted social tension and class conflict between the exploiters and the exploited belonging to different religions as communal conflict. While the discontent and clash of interests was real and was due to non-religious or non- communal factors, because of backward political consciousness it found a distorted expression in communal conflict. As C.G. Shah has put it: `Under the pressure of communal propaganda, the masses are unable to locate the real causes of their exploitation, oppression, and suffering and imagine a fictitious communal source of their origin.' What made such communal (and later casteist) distortion possible specific feature of Indian social development --- in several parts of the country the religious distinction coincided with social, and class distinctions. Here most often the exploiting sections --- landlords, merchants and moneylenders, were upper caste Hindus while the poor and exploited were Muslims or lower caste Hindus. Consequently, propaganda by the Muslim communalists that Hindus were exploiting Muslims or by the Hindu communalists that Muslims were threatening Hindu property or economic interests could succeed even while wholly incorrect. Thus, for example, the struggle between tenant and landlord in East Bengal and Malabar and the peasant-debtor and the merchant-moneylender in Punjab could be portrayed by the communalists as a struggle between Muslims and Hindus. Similarly, the landlord-moneylender oppression was represented as the oppression of Muslims by Hindus, and the attack by the rural poor on the rural rich as an attack by Muslims on Hindus. For example, one aspect of the growth of communalism in Punjab was the effort of the big Muslim landlords to protect their economic and social position by using communalism to turn the anger of their Muslim tenants against Hindu traders and moneylenders, and the use of communalism by the latter to protect their threatened class interests by raising the cry of Hindu interests in danger. In reality, the struggle of the peasants for their emancipation was inevitable. The question was what type of ideological- political content it would acquire. Both the communalists and the colonial administrators stressed the communal as against the class aspects of agrarian exploitation and oppression. Thus, they held that the Muslim peasants and debtors were being exploited not as peasants and debtors but because they were Muslims. 

In many cases, a communal form is given to the social conflict not b the participants but by the observer, the official, the journalist, the politician, and, finally, the historian, all of whom provide a post-facto communal explanation for the conflict because of their own conscious or unconscious outlook. It is also important to note that agrarian conflicts did not assume a communal colour until the 20th century and the rise of communalism and that too not in most cases, in the Pabna agrarian riots of 1873, both Hindu and Muslim tenants fought zamindars together. Similarly, as brought out in earlier chapters, most of the agrarian struggles in 1919 stayed clear of communal channels. The peasants' and workers' --- the radial intelligentsia succeeded in creating powerful secular wit arid %ken movements and organizations which became important constituents of the anti-imperialist struggle. 

It is important to note in this context that Hindu zamindars in Bengal had acquired control over land not because they were Hindus but as a result of the historical process of the spread of Islamic religion in Bengal among the lower castes and classes. Hindu zamindars and businessmen acquired economic dominance over landed capital in Bengal at the beginning of the 18th century during the rule of Murshid Quli Khan, religiously the most devout of Aurangzeb's officials and followers. Under his rule, more than seventy-five per cent of the zamindars and most of the taluqdars were Hindus. The Permanent Settlement of 1793 further strengthened the trend by eliminating on a large scale both the old Hindu and Muslim zamindar families and replacing them with new men of commerce who were Hindus. Similarly, the predominance of Hindus among bankers, traders and moneylenders in northern India dated to the medieval period. The dominance these strata acquired over rural society under British rule was the result not of their being Hindu but of the important economic role they acquired in the colonial system of exploitation. In other words, colonial history guaranteed the growth and economic domination of merchant-moneylenders; medieval history had guaranteed that they would be mostly Hindus. 

Communalism represented, at another level, a struggle between two upper classes or strata for power, privileges and economic gains. Belonging to different religions (or castes) these classes or strata used communalism to mobilize the popular support of their co-religionists in their mutual struggles. This was, for example, the case in Western Punjab where the Muslim landlords opposed the Hindu moneylenders and in East Bengal where the Muslim jotedars (small landlords) opposed the Hindu zamindars.

\begin{center}*\end{center}

\paragraph*{}

Above all, communalism developed as a weapon of economically and politically reactionary social classes and political forces --- and semi- feudal landlords and ex-bureaucrats (whom Dr. K.M. Ashraf has called the jagirdari classes) merchants and moneylenders and the colonial state. Communal leaders and parties were, in general, allied with these classes and forces. The social, economic and political vested interests deliberately encouraged or unconsciously adopted communalism because of its capacity to distort and divert popular struggles, to prevent the masses from understanding the socio-economic arid political forces responsible for their social condition, to prevent unity on national and class lines, and to turn them away from their real national and socio-economic interests and issues and mass movements around them. Communalism also enabled the upper classes and the colonial rulers to unite with sections of the middle (lasses and to utilize the latter's politics t serve their own ends.

\begin{center}*\end{center}

\paragraph*{}

British rule and its policy of Divide and Rule bore special responsibility for the growth communalism in modem India, though it is also true that it could succeed only because of internal social and political conditions. The fact was that the state, with its immense power, could promote either national integration or all kinds of divisive forces. The colonial state chose the latter course. It used communalism to counter and weaken the growing national movement and the welding of the Indian people into a nation, communalism was presented by the colonial rulers as the problem of the defence of minorities. Hindu-Muslim disunity --- and the need to protect minorities from domination and suppression by the majority --- was increasingly offered as the main justification for the maintenance of British rule, especially as theories of civilizing mission, white man's burden, welfare of the ruled, etc., got increasingly discredited. 

Communalism was, of course, not the only constituent of the policy of Divide and Rule. Every existing division of Indian society was encouraged to prevent the emerging unity of the Indian people. An effort was made to set region against, region, province against province, caste against caste, language against language, reformers against the orthodox, the moderate against the militant, leftist against rightist, and even class against class. It was, of course, the communal division which survived to the end and proved the most serviceable. In fact, near the end, it was to become the main prop of colonialism, and colonial authorities were to stake their all on it. On the other hand, communalism could not have developed to such an extent as to divide the country, if it did not have the powerful support of the colonial state. In this sense, communalism may be described as the channel through which the politics of the middle classes were placed at the service of colonialism and the jagirdari classes. In fact, communalism was the route through which colonialism was able to extend its narrow social base to sections of workers, peasants, the middle classes and the bourgeoisie whose interests were otherwise in contradiction with colonialism. 

What were the different ways and policies, or acts of omission and commission, through which the British encouraged and nurtured communalism? First, by consistently treating Hindus, Muslims and Sikhs as separate communities and socio- political entities which had little in common. India, it was said, was neither a nation or a nation-in-the- making, nor did it consist of nationalities or local societies, but consisted of structured, mutually exclusive and antagonistic religion-based communities. Second, official favour and patronage were extended to the communalists. Third, the communal Press and persons and agitations were shown extraordinary tolerance. Fourth, communal demands were readily accepted, thus politically strengthening communal organizations and their hold over the people. For example, while the Congress could get none of its demands accepted from 1885--1905, the Muslim communal demands were accepted in 1906 as soon as they were presented to the Viceroy. Similarly, in 1932, the Communal Award accepted all the major communal demands of the time. During World War II, the Muslim communalists ere given a complete veto on any political advance. Fifth, the British readily accepted communal organizations and leaders as the real spokesperson for their `communities,' while the nationalist leaders were treated as representing a microscopic minority --- the elite. Sixth, separate electorates served as an important instrument for the development of communal politics. Lastly, the colonial government encouraged communalism through a policy of non- action against it. Certain positive measures which the state alone could undertake were needed to check the growth of communalism. The failure to undertake them served as an indirect encouragement to communalism. The Government refused to take action against the propagation of `virulent communal ideas and communal hatred through the Press, pamphlets, leaflets, literature, public platform and rumours. This was in sharp contrast with the frequent suppression of the nationalist Press, literature, civil servants, propaganda, and so on. On the contrary, the Government freely rewarded communal leaders, intellectuals and government servants with titles, positions of profit, high salaries, and so on. The British administrators also followed a policy of relative inactivity and irresponsibility in dealing with communal riots. When they occurred, they were not crushed energetically. The administration also seldom made proper preparations or took preventive measures to meet situations of communal tension, as they did in case of nationalist and other popular protest movements. 

To sum up: So long as the colonial state supported communalism, a solution to the communal problem was not easily possible while the colonial state remained; though, of course, the overthrow of the colonial state was only the necessary but not a sufficient condition for a successful struggle against communalism.

\begin{center}*\end{center}

\paragraph*{}

A strong contributory factor in the growth of communalism was the pronounced Hindu tinge in much of nationalist thought and propaganda in the beginning of the 20th century. 

Many of the Extremists introduced a strong Hindu religious element in nationalist thought and propaganda. They tended to emphasize ancient Indian culture to the exclusion of medieval Indian culture. They tried to provide a Hindu ideological underpinning to Indian nationalism or at least a Hindu idiom to its day-to-day political agitation. Thus, Tilak used the Ganesh Puja and the Shivaji Festival to propagate nationalism; and the anti-partition of Bengal agitation was initiated with dips in the Ganges. What was much worse, Bankim Chandra Chatterjea and many other writers in Bengali, Hindi, Urdu and other languages often referred to Muslims as foreigners in their novels, plays, poems, and stories, and tended to identify nationalism with Hindus. This type of literature, in which Muslim rulers and officials were often portrayed as tyrants, tended to produce resentment among literate Muslims and alienate them from the emerging national movement. Moreover, a vague Hindu aura pervaded much of the nationalist agitation because of the use of Hindu symbols, idioms, and myths. 

Of course, the nationalist movement remained, on the whole, basically secular in its approach and ideology, and young nationalist Muslims like M.A. Jinnah and Maulana Abul Kalam Azad had little difficulty in accepting it as such and in joining it. This secularism became sturdier when leaders like Gandhi, C.R. Das, Motilal Nehru, Jawaharlal Nehru, Maulana Azad, Dr. M.A. Ansari, Subhas Bose, Sardar Patel and Rajendra Prasad came to the helm. The Hindu tinge was not so much a cause of communalism as a cause of the nationalist failure to check the growth. It made it slightly more difficult to win over Muslims to the national movement. It enabled the Government and Muslim communalists to use it to keep large sections of Muslims away from the nationalist movement and to instil among them the feeling that the success of the movement would mean `Hindu supremacy' in the country. 

This Hindu tinge also created ideological openings for Hindu communalism and made it difficult for the nationalist movement to eliminate Hindu communal political and ideological elements within its own ranks. It also helped the spread of a Muslim tinge among Muslim nationalists.

\begin{center}*\end{center}

\paragraph*{}

A communal and distorted unscientific view of Indian history, especially of its ancient and medieval periods, was a major instrument for the spread of communal consciousness as also a basic constituent of communal ideology. The teaching of Indian history in schools and colleges from a basically communal point of view made a major contribution to the rise and growth of communalism. For generations, almost from the beginning of the modern school system, communal interpretations of history of varying degrees of virulence were propagated, first by imperialist writers and then by others. So deep and widespread was the penetration of the communal view of history that even sturdy nationalists accepted, however unconsciously, some of its basic digits. All this was seen by many contemporary observers. Gandhiji, for example, wrote: `Communal harmony could not be permanently established in our country so long as highly distorted versions of history were being taught in her schools and colleges, through the history textbooks.' Over and above the textbooks, the communal view of history was spread widely through poetry, drama, historical novels and short stories, newspapers and popular magazines, pamphlets, and above all, orally through the public platform, classroom teaching, socialization through the family, and private discussion and conversation. 

A beginning was made in the early 19th century by the British historian, James Mill, who described the ancient period of Indian history as the Hindu period and the medieval period as the Muslim period. (Though he failed to characterize the modern period as the Christian period!). Other British and Indian historians followed him in this respect. Furthermore, though the Muslim masses were as poor, exploited and oppressed as the Hindu masses, and there were Hindu zamindars, nobles and rulers along with Muslim ones, these writers declared that all Muslims were rulers in medieval India and all Hindus were the ruled. Thus, the basic character of a polity in India was identified with the religion of the ruler Later the culture and society of various periods were also declared to be either Hindu or Muslim in character. 

The Hindu communalist readily adopted the imperialist view that medieval rulers in India were anti-Hindu, tyrannized Hindus and converted them forcibly. All communalist, as also imperialist, historians saw medieval history as one long story of Hindu- Muslim conflict and believed that throughout the medieval period there existed distinct and separate Hindu and Muslim cultures. The Hindu communalists described the rule of medieval Muslim rulers as foreign rule because of their religion. The talk of `a thousand years of slavery' and `foreign rule' was common rhetoric, sometimes even used by nationalists. Above all, the Hindu communal view of history relied on the myth that Indian society and culture had reached great, ideal heights in the ancient period from which they fell into permanent and continuous decay during the medieval period because of `Muslim' rule and domination. The basic contribution of the medieval period to the development of the Indian economy and technology, religion and philosophy, arts and literature, and culture and society was denied. 

In turn the Muslim communalists harked back to the `Golden Age of Islamic achievement' in West Asia and appealed to its heroes, myths and cultural traditions. They propagated the notion that all Muslims were the rulers in medieval India or at least the beneficiaries of the so-called Muslim rule. They tended to defend and glorify all Muslim rulers, including religious bigots like Aurangzeb. They also evolved their own version of the `fall' theory. While Hindus were allegedly in the ascendant during the 19th century, Muslims, it was said, `fell' or declined as a `community' throughout the 19th century after `they' lost political power.

\begin{center}*\end{center}

\paragraph*{}

A major factor in the growth of communalism according to some authors was the religious pluralism or the existence of several religions in India. This is not so. It is not true that communalism must arise inevitably in a multi-religious society. Religion was not an underlying or basic cause of communalism, whose removal was basic to tackling or solving the communal problem. Here we must distinguish between religion as a belief system, which people follow as part of their personal belief, and the ideology of a religion-based socio-political identity, that is, communalism. In other words, religion is not the `cause' of communalism, even though communal cleavage is based by the communalist on differences in religion --- this difference is then used to mask or disguise the social needs, aspirations, conflicts, arising in non-religious fields. Religion comes into communalism to the extent that it serves politics arising in spheres other than religion. K.M. Ashraf put this aspect in an appropriate phrase when he described communalism as `Mazhab ki siyasi dukadari' (political trade in religion). Communalism was not inspired by religion, nor was religion the object of communal politics --- it was only its vehicle. 

Religion was, however, used as a mobilizing factor by the communalists. Communalism could become a popular movement after 1939, and in particular during 1945--47, only when it adopted the inflammable cry of religion in danger. Moreover, differing religious practices were the immediate cause of situations of communal tension and riots. We may also note that while religion was not responsible for communalism, religiosity was a major contributory factor. (Religiosity may be defined as intense emotional commitment to matters of religion and the tendency to let religion and religious emotions intrude into non-religious or non-spiritual areas of life and beyond the individual's private and moral world.) Religiosity was not communalism but it opened a person to the appeal of communalism in the name of religion. Secularization did not, therefore, mean removing religion but it did mean reducing religiosity or increasingly narrowing down the sphere of religion to the private life of the individual.
\end{multicols}