
\chapter{Communalism: The Liberal Phase}



There was hardly any communalism in India before the last quarter of the 19th century. As is well-known, Hindus and Muslims had fought shoulder to shoulder in the Revolt of 1857. The notion of Hindu-Muslim distinction at the non-religious plane, not to speak of the clash of interests of Hindus and Muslims was by and large non-existent in the Press during the 1860s. The identity that the North Indian newspapers emphasised was that of the Hindustanis, especially vis-a-vis European or British rulers 

Even when some Muslim intellectuals began to notice that Muslims in some parts of the country were tagging behind Hindus in modern education and in government jobs, they blamed not Hindus but the Government's anti-Muslim policy and the neglect of modem education by upper class Muslims. Syed Ahmed Khan, undoubtedly one of the outstanding Indians of the l9thiitury, began his educational activities without any communal bias. The numerous scientific societies he founded in 1860s involved both Hindus and Muslims. The Aligarh College he specially founded to fight the bias against modern education among Muslims, received financial support from moneyed Hindus; and its faculty and students had a large Hindu component Syed Ahmed loudly preached the commonness of Hindus and Muslims till the founding of the Congress in 1885. Thus, for example, he said in 1884: `Do you not inhabit this land? Are you not buried in it or cremated on it? Surely you live and die on the same land. Remember that Hindus and Muslims are religious terms. Otherwise Hindus, Muslims and Christians who live in this country are by virtue of this fact one qawm' (nation or community). 

Ironically, communalism in India got its initial start in the 1 880s when Syed Ahmed Khan counterposed it to the national movement initiated by the National Congress. In 1887, Dufferin, the Viceroy, and A. Colvin the Lieutenant-Governor of U.P., launched a frontal public attack on the National Congress, once its anti-imperialist edge became clear. Syed Ahmed, believing that the Muslims' share in administrative posts and in profession could be increased only by professing and proving loyalty to the colonial rulers, decided to join in the attack. Furthermore, he felt that he needed the active support of big zamindars and the British officials for the Aligarh College. Initially he made an attempt with the help of Shiva Prasad, Raja of Bhinga, and others to organize along caste, birth, class and status lines the feudal (jagirdari) and bureaucratic elements in opposition to the rising democratic national movement. However, this attempt failed to get off the ground. 

Syed Ahmed now set out to organize the jagirdari elements among Muslims as Muslims or the Muslim qawm (community). He and his fo1lowrs gradually laid down the foundation of all the basic themes of the communal ideology as it was to be propagated in the first half of the 20th century. A basic theme was that Hindus, because they were a majority, would dominate Muslims and `totally override the interests of the smaller community' if representative, democratic government was introduced or if British rule ended and power was transferred to Indians. The British were needed to safeguard Muslims as a minority. In the Indian context, said Syed Ahmed, they were the best guardians of Muslim interests. Muslims must, therefore, remain loyal and oppose the National Congress. The theme of a permanent clash of `Hindu' and `Muslim' interests was also brought forth. Giving up his earlier views, he now said that India could not be considered a nation. He declared that the Congress was a Hindu body whose major objectives were `against Muslim interest.' Simultaneously, he criticized the Congress for basing itself on the principle of social equality among the `lowly' and the `highly' born. Objecting to the Congress demand for democratic elections, Syed Ahmed said that this would `mean that Muslims would not be able to guard their interests, for 9t would be like a game of dice in which one man had four dice and the other only one.' Any system of elections, he said, would put power into the hands of `Bengalis or of Hindus of the Bengali type' which would lead to Muslims falling into `a condition of utmost degradation' and `the ring of slavery' being put on them by Hindus. Syed Ahmed and his co-workers also demanded safeguards for Muslims in Government jobs, legislative councils, and district boards and recognition of the historical role and political importance of Muslims so that their role in legislative councils should not be less than that of Hindus. At the same time, Syed Ahmed and his followers did not create a counter command political organization, because the British authorities at the time frowned upon any politicization of the Indian people. Syed Ahmed held that any agitational politics would tend to become anti- government and seditious and to create suspicion disloyalty among the rulers. He, therefore, asked Muslims to shun all polities and remain politically passive, i.e., non-agitational, in their approach. The co1onial rulers were quick to see the inherent logic or communalism and the theory of the official protection of the minorities and from the beginning actively promoted and supported communalism. 

The Muslim communalists continued to follow the politics of loyalty after Syed Ahmed's death. They openly sided with the Government during the Swadeshi Movement in Bengal during 1905-6 and condemned the Muslim supporters of the movement as `vile traitors' to Islam and as `Congress touts.' But the attempt to keep the growing Muslim intelligentsia politically passie or loyalist was not wholly successful. Badruddin Tyabji presided over the Congress session in 1887, and the number of Muslim delegates to the Congress increased in the succeeding years. R.M. Sayani, A. Bhimji, Mir Musharaff Hussain, Hamid Ali Khan and numerous other Muslim intellectuals from Bombay, Bengal and Northern India joined the Congress. They pointed out that not even one of the Congress demands was communal or for Hindus only. The nationalist trend continued to spread among Muslims all over the country till the end of the 19th century. Abdul Rasul and a large number of other Bengali Muslim intellectuals gave active support to the Swadeshi agitation against the partition of Bengal. In fact, the nationalist trend remained dominant among Muslims in Bengal till the late 1920s. 

Once the Swadeshi Movement brought mass politics to India, a large section of the Muslim intelligentsia could not be kept away from the Congress; the British Government felt compelled to difer some constitutional concessions, and it became impossible to continue to follow the policy of political passivity. The communalists, as also their official supporters, felt that they had to enter the political arena. At the end of 1907 the 

All India Muslim League was founded by a group of big zamindars ex-bureaucrats and other upper class Muslims like the Aga Khan, the Nawab of Dacca and Nawab Mohsin-uI-Muk. Founded as a loyalist, communal and conservative political organization, the League supported the partition of Bengal, raised the slogan of separate Muslim interests, demanded separate electorates and safeguards for Muslims in government services, and reiterated all the major themes of communal politics and ideology enunciated earlier by Syed Ahmed and his followers. Viqar-ul-Mulk for example, said: `God forbid, if the British rule disappears from India, Hindus will lord over it; and we will be in constant danger of our life, property and honour. The only way for the Muslims to escape this danger is to help in the continuance of the British rule.'9 He also expressed the fear `of the minority losing its identity.' One of the major objectives of the Muslim League was to keep the emerging intelligentsia among Muslims from joining the Congress. Its activities were directed against the National Congress and Hindus and not against the colonial regime. 

Simultaneously, Hindu communalism, was also being born. From the 1870s, a section of Hindu zamindars, moneylenders and middle class professionals began to arouse anti-Muslim sentiments. Fully accepting the colonial view of Indian history, they talked of the `tyrannical' Muslim rule in the medieval period and the `liberating' role of the British in `saving' Hindus from `Muslim oppression.' In U.P. and Bihar, they took up the question of Hindi and gave it a communal twist, declaring that Urdu was the language of Muslims and Hindi of Hindus. All over India, anti-cow slaughter propaganda was undertaken in the early 1890s, the campaign being primarily directed not against the British but against Muslims; the British cantonments, for example, were left free to carry on cow slaughter on a large scale. Consequently, this agitation invariably took a communal turn, often resulting in communal riots. The anti-cow slaughter agitation died down by 1896, to be revived again in a more virulent form in the second decade of the 20th century. The Hindu communalists also carried on a regular agitation for a `Hindu' share of seats in legislatures and in government services. 

The Punjab Hindu Sabha was founded in 1909. Its leaders, 

U.N. Mukherji and Lal Chand, were to lay down the foundations of Hindu communal ideology and politics. They directed their anger primarily against the National Congress for trying to unite Indians into a single nation and for `sacrificing Hindu interests' to appease Muslims. In his booklet, Self-Abnegation in Politics, Lal Chand described the Congress as the `self-inflicted misfortune' of Hindus. Hindus, he wrote, were moving towards extinction because of `the poison imbibed for the last 25 years.' They could be saved only if they were willing to `purge' the poison and get rid of the `evil.' He accused the Congress of making `impossible' demands on the Government, leading to its justifiable anger against the Congress and Hindus. Instead Hindus should try to neutralize the third party, the Government, in their fight against Muslims. It was also essential that Hindus abandon and `end' the Congress. `A Hindu,' Lal Chand declared, `should not only believe but make it a part and parcel of his organism, of his life and of his conduct, that he is a Hindu first and an Indian after.'' 

The first session of the All-India Hindu Mahasabha was held in April 1915 under the presidentship of the Maharaja of Kasim Bazar. But it remained for many years a rather sickly child compared to the Muslim League. This was for several reasons. The broader social reason was the greater and even dominant role of the zamindars, aristocrats and ex-bureaucrats among Muslims in general and even among the Muslim middle classes. While among Parsis and Hindus, increasingly, it was the modern intelligentsia, with its emphasis on science, democracy and nationalism, and the bourgeois elements in general, which rapidly acquired intellectual, social, economic and political influence and hegemony, among Muslims the reactionary landlords and mullahs continued to exercise dominant influence or hegemony. Landlords and traditional religious priests, whether Hindu or Muslim, were conservative and supporters of established, colonial authority. But while among Hindus, they were gradually losing positions of leadership, they continued to dominate among Muslims. In this sense the weak position of the middle class among Muslims and its social and ideological backwardness contributed to the growth of Muslim communalism. 

There were other reasons for the relative weakness of Hindu communalism. The colonial Government gave Hindu communalism few concessions and little support, for it banked heavily on Muslim communalism and could not easily simultaneously placate both communalisms. 

The colonial authorities and the communalists together evolved another powerful instrument for the spread and consolidation of communalism in separate electorates which were introduced in the Morley-M into Reforms of 1907. Under this system, Muslim voters (and later Sikhs and others) were put in separate constituencies from which only Muslims could stand as candidates and for which only Muslims could vote. Separate electorates turned elections and legislative councils into arenas for communal conflicts. Since the voters were exclusively the followers of one religion, the candidates did not have to appeal to voters belonging to other religions. They could, therefore, make blatantly communal appeals and voters and others who listened to these appeals were gradually trained to think and vote communally and in general to think in terms of `communal' power and progress and to express their socio-economic grievances in communal terms. The system of reservation of seats and weightage in legislatures, government services, educational institutions etc., also had the same consequences. 

A slight detour at this stage is perhaps necessary. When discussing the history of the origins and growth of communalism and communal organizations, one particular error is to be avoided. Often a communalist ascribed --- or even now ascribes in historical writings --- the origins of one communalism to the existence of and as a reaction to the other communalism. Thus, by assigning the `original' blame to the other communalism a sort of backdoor justification for one's own communalism is (or was) provided. Thus the Hindu, Muslim or Sikh communalists justified their own communalism by arguing that they were reacting to the communalism initiated by others. In fact, to decide which communalism came first is like answering the question: which came first, the chicken or the egg? Once communalism arose and developed, its different variants fed and fattened on each other. 

The younger Muslim intellectuals were soon dissatisfied with the loyalist, anti-Hindu and slavish mentality of the upper class leadership of the Muslim League. They were increasingly drawn to modern and radical nationalist ideas. The militantly nationalist Ahrar movement was founded a this time under the leadership of Maulana Mohammed Au, Hakim Ajmal Khan, Hasan Imam, Maulana Zafar Ali Khan, and Mazhar-uI-Haq. In their efforts, they got support from a section of orthodox uiwna scholars) especially those belonging to the Deoband school. Another orthodox scholar to be attracted to the national movement was the young Maulana Abul Kalam Azad, who was educated at the famous Al Azhar University at Cairo and who propagated his rationalist and nationalist ideas in his newspaper Al Hilal which he brought out in 1912 at the age of twenty-four After an intense struggle, the nationalist young Muslims came to the fore in the Muslim League. They also became active in the Congress. In 1912, the brilliant Congress leader, M.A. Jinnah, was invited to join the League which adopted self-government as one of its objectives, in the same year, the Aga Khan resigned as the President of the League. 

From 1912 to 1924, the young nationalists began to overshadow the loyalists in the League which began to move nearer to the policies of the Congress. Unfortunately, their nationalism was flawed in so far as it was not fully secular (except with rare exceptions like Jinnah). It had a strong religious and pan-Islamic tinge. Instead of understanding and opposing the economic and political consequences of modern imperialism they fought it on the ground that it threatened the Caliph (khalifa) and the holy places. Quite often their appeal was to religious sentiments. This religious tinge or approach did not immediately clash with nationalism. Rather, it made as adherents anti-imperialist; and it encouraged the nationalist trend among urban Muslims. But in the long run it proved harmful as it inculcated arid encouraged the habit of looking at political questions from a religious point of view. The positive development within the Congress --- discussed in an earlier CHAPTER - and within the Muslim League soon led to broad political unity among the two, an important role in this being played by Lokmanya Tilak and M.A. Jinnah. The two organizations held their sessions at the d of 1916 at Lucknow, signed a pact known as the Lucknow Pact, and put forward common political demands before the Government including the demand for self-government for India after the war. The Pact accepted separate electorates and the system of weightage and reservation of seats for the minorities in the legislatures. While a step forward in many respects --- and it enthused the political Indian --- the Pact was also a step back. The Congress had accepted separate electorates and formally recognized communal politics. Above all, the Pact was tacitly based on the assumption that India consisted of different communities with separate interests of their own. It, therefore, left the way open to the future resurgence of communalism in Indian politics. 

The nationalist movement and Hindu-Muslim unity took giant steps forward after World War I during the agitation against the Rowlatt Acts, and the Khilafat and the Non-Cooperation Movements. As if to declare before the world the principle of Hindu-Muslim unity in political action, Swami Shradhanand, a staunch Arya Samajist, was asked by Muslims to preach from the pulpit of the Jama Masjid at Delhi, while Dr. Saifuddin Kitchlu, a Muslim, was given the keys to the Golden Temple, the Sikh shrine at Amritsar. The entire country resounded to the cry of `Hindu-Muslim ki Jai'. The landlord-communalists and ex- bureaucrats increasingly disassociated themselves from the Muslim League, while the League itself was overshadowed by the Khilafat Committee as many of the League leaders --- as also many of the old Congress leaders --- found it difficult to keep pace with the politics of a mass movement. Even though the Khilafat was a religious issue, it resulted in raising the national, anti- imperialist consciousness of the Muslim masses and middle classes. Moreover, there was nothing wrong in the nationalist movement taking up a demand that affected Muslims only, just as the Akali Movement affected the Sikhs only and the anti-untouchability campaign Hindus only. 

But there were also certain weaknesses involved. The nationalist leadership failed to some extent in raising the religious political consciousness of Muslims to the higher plane of secular political consciousness. The Khilafat leaders, for example, made appeals to religion and made full use of fatwas (opinion or decision on a point of Islamic law given by a religious person of standing) and other religious sanctions. Consequently, they strengthened the hold of orthodoxy and priesthood over the minds of men and women and encouraged the habit of looking at political questions from the religious point of view. By doing so and by emphasizing the notion of Muslim solidarity, they kept an opening for communal ideology and politics to grow at a later stage. 

The Non-Cooperation Movement was withdrawn in February 1922. As the people felt disillusioned and frustrated and the Dyarchy became operational, communalism reared its ugly head and in the post-1922 years the country was repeatedly plunged into communal riots. Old communal organizations were revived and fresh ones founded. The Muslim League once again became active and was cleansed of radical and nationalist elements. The upper class leaders with their open loyalism and frankly communal ideology once again came to the fore. The Hindu Mahasabha was revived in 1923 and openly began to cater to anti-Muslim sentiments. Its proclaimed objective became `the maintenance, protection and promotion of Hindu race, Hindu culture and Hindu civilization for the advancement of Hindu Rashtra.' The Hindu as well as Muslims communalists tried to inculcate the psychology of fear among Hindus and Muslims --- the fear of being deprived, surpassed, threatened, dominated, suppressed, beaten down, and exterminated. It was during these years that Sangathan and Shuddhi movements among Hindus and Tanzeem and Tabligh movements among Muslims, working for communal consolidation and religious conversion, came up. The nationalists were openly reviled as apostates and as enemies of their own religion and co-religionists. A large number of nationalists were not able to withstand communal pressure and began to adopt communal or semi- communal positions. The Swarajists were split by communalism. A group known as `responsivists' offered cooperation to the Government so that the so-called Hindu interests might be safeguarded. Lajpat Rai, Madan Mohan Malaviya and N.C. Kelkar joined the Hindu Mahasabha and argued for Hindu communal solidarity. The less responsible `responsivists' and Hindu Mahasabhaites carried on a virulent campaign against secular Congressmen. They accused MOWSI Nehru of letting down Hindus, of being anti-Hindu and an Islam- lover, of favo.xmg cow-slaughter, and of eating beef. Many old Khilafatists also now turned communal. The most dramatic shift was that of Maulanas 

Mohammed All and Shaukat All who now accused the Congress of trying to establish a Hindu Government and Hindus of wanting to dominate and suppress Muslims. The most vicious expression of communalism were communal riots which broke out in major North Indian cities during 1923--24. According to the Simon Commission Report, nearly 112 major communal riots occurred between 1922 and 1927. 

The nationalist 1eadership made strenuous efforts to oppose communal political forces, but was not able to evolve an effective line of action. What was the line of action that it adopted and why did it fail? Its basic strategy was to try to bring about unity at the top with communal leaders through negotiations. This meant that either the Congress leaders acted as mediators or intermediaries between different communal groups or they themselves tried to arrive at a compromise with Muslim communal leaders on questions of `protection' to and `safeguards' of the interests of the minorities in terms of reservation of seats in the legislatures and of jobs in the government. 

The most well-known of such efforts was made during 1928. As an answer to the challenge of the Simon Commission, Indian political leaders organized several all-India conferences to settle communal issues and draw up an agreed constitution for India. A large number of Muslim communal leaders met at Delhi in December 1927 and evolved four basic demands known as the Delhi Proposals. These proposals were: (1) Sind should be made a separate province; (2) the North-West Frontier Province should be treated constitutionally on the same footing as other provinces; Muslims should have 33 1/3 per cent representation in the central legislature; (4) in Punjab and Bengal, the proportion of representation should be in accordance with the population, thus guaranteeing a Muslim majority, and in other provinces, where Muslims were a minority, the existing reservation of seats for Muslims should continue. 

The Congress proposals came in the form of the Nehru Report drafted by an all-parties committee. The Report was put up for approval before an All-Party Convention at Calcutta at the end of December 1928. Apart from other aspects, the Nehru Report recommended that India should be a federation on the basis of linguistic provinces and provincial autonomy, that elections be held on the basis of joint electorates and that seats in central and provincial legislatures be reserved for religious minorities in proportion to their population. The Report recommended the separation of Sind from Bombay and constitutional reform in the North-West Frontier Province. 

The Report could not be approved unanimously at the Calcutta Convention. While there were wide differences among Muslims communalists. a section of the League and the Khilafatists were willing to accept joint electorates and other proposals in the Report provided three amendments, moved by 

M.A. Jinnah, were accepted. Two of these were the same as the third and fourth demands in the Delhi Proposals, the first and the second of these demands having been conceded by the Nehru Report. The third was a fresh demand that residuary powers should vest in the provinces. A large section of the League led by Mohammed Shafi and the Aga Khan and many other Muslim communal groups refused to agree to these amendments; they were not willing to give up separate electorates. The Hindu Mahasabha and the Sikh League raised vehement objections to the parts of the Report dealing with Sind, North-West Frontier Province, Bengal and Punjab. They also refused to accept the Jinnah amendments. The Congress leaders were not willing to accept the weak centre that the Jinnah proposals envisioned. 

Most of the Muslim communalists now joined hands and Jinnah too decided to fall in line. Declaring that the Nehru Report represented Hindu interests, he consolidated all the communal demands made by different communal organizations at different times into a single document which came to be known as Jinnah's Fourteen Points. The Fourteen Points basically consisted of the four Delhi Proposals, the three Calcutta amendments and demands for the continuation of separate electorates arid reservation of seats for Muslims in government services and self- governing bodies. The Fourteen Points were to form the basis of all future communal propaganda in the subsequent years. 

This strategy of trying to solve the communal problem through an agreement or pact with the Hindu, Muslim and Sikh communal leaders proved a complete failure and suffered from certain inherent weaknesses. Above all it meant that the 

Congress tacitly or by implication accepted, to a certain extent, the claim of the communal leaders that they were representatives of the communal interests of their respective `communities,' and, of course, that such communal interests and religious communities existed in real life. By negotiating with communal leaders, the Congress legitimized their politics and made them respectable. It also weakened its right, as well as the will, to carry on a hard political-ideological campaign against communal parties and individuals. Constant negotiations with Muslim communal leaders wakened the position of secular, anti- imperialist Muslims and Muslim leaders like Azad, Ansari and Asaf Ali. They also made it difficult to oppose and expose the communalism and semi- communalism of leaders like Madan Mohan Malaviya, Lajpat Rai and Maulana Mohammed Ali who often worked within the Congress ranks. 

The strategy of negotiations at the top required generous concessions by the majority to the minority communalism on the question of jobs and seats in the legislatures. But communalism was quite strong among the Hindu middle classes which too suffered from the consequences of colonial underdevelopment. The Congress leadership found it politically difficult to force concessions to Muslim communalism down the throat of Hindu and Sikh communalists. Thus, the failure to conciliate the Muslim communalists helped them gain strength, while any important concessions to them tended to produce a Hindu communal backlash. In any case, even if by a supreme effort in generosity and sagacity a compromise with communal leaders had been arrived at, it was likely to prove temporary as was the case with the Lucknow Pact and to some extent the Nehru Report. Not one communal leader or group or party had enough authority over other communal groups and individuals to sign a lasting agreement. Concessions only whetted the appetite of the communalists. A soon as one group was appeased, a more `extreme' or recalcitrant leader or group emerged and pushed up the communal demands. Consequently, often the more `reasonable' leader or group felt his communal hold over the followers weakening and found it necessary to go back even on the earlier partial or fuller agreement. This is what repeatedly happened during 1928--29 --- and Jinnah's was a typical example. The fact was that so long as communal ideology flourished or the socio-political conditions favouring communal politics persisted, it was difficult to appease or conciliate communal leaders permanently or for any length of time. The real answer lay in an all-out opposition to communalism in all arenas --- ideological, cultural, social and political. Based on a scientific understanding of its ideology, its social and ideological sources and roots, its social base, and the reasons for its growth in the face of the nationalist work in favour of Hindu-Muslim unity, an intense political-ideological struggle had to be waged against communalism and communal political forces. Moreover, it was necessary to take up the peasants' cause where their class struggle was being distorted into communal channels. All this was not done, despite the deep commitment to secularism of the bulk of the nationalist leadership from Dadabhai Naoroji to Gandhiji and Nehru. 

The need was to direct the debate with the communalists into hard, rational, analytical channels so that the latter were forced to fight on the terrain of reason and science and not of emotion and bias. Gandhiji and the Congress did make Hindu- Muslim unity one of the three basic items of the nationalist political platform. They also, at crucial moments, refused to appease the Hindu communalists. Gandhiji several times staked his life for the secular cause. But Gandhiji and the Congress provided no deeper analysis of the communal phenomenon. 

Despite the intensified activities of communal parties and groups during the 1920s, communalism was not yet very pervasive in Indian society. Communal riots were largely confined to cities and their number, keeping in view the size of the country, was not really large. The Hindu communalists commanded little support among the masses. The social base of the Muslim communalists was also quite narrow. The nationalist Muslims, who were part of the Congress, still represented a major political force. The rising trade union, peasant and youth movements were fully secular. The reaction to the Simon Commission further revealed the weakness of communal forces when both the Muslim League and the Hindu Mahasabha got divided, some in favour of a boycott of the Commission and others for cooperating with it. 

The anti-Simon Commission protest movement and then the Second Civil Disobedience Movement from 1930 to 1934 swept the entire country and once again pushed the communalists as a whole into the background. Led by the Congress, Jamait-ul-Ulama-i-Hind, Khudai Khidmatgars and other organizations, thousands of Muslims went to jail. The national movement engulfed for the first time two new major areas with a Muslim majority --- the North-West Frontier Province and Kashmir. The communal leaders got a chance to come into the limelight during the Round Table Conferences of the early 1930s. At these conferences, the communalists joined hands with the most reactionary sections of the British ruling classes. Both the Muslim and Hindu communalists made efforts to win the support of British authorities to defend their so-called communal interests. In 1932, at a meeting in the House of Commons, the Aga Khan, the poet Mohammed lqbal and the historian Shafaat Ahmad Khan stressed `the inherent impossibility of securing any merger of Hindu and Muslim, political, or indeed social interests' and `the impracticability of ever governing India through anything but a British agency.' Similarly, in 1933, presiding over the Hindu Mahasabha session, Bhai Parmanand made a plea for cooperation between Hindus and the British Government and said: `I feel an impulse in me that Hindus would willingly cooperate with Great Britain if their status and responsible position as the premier community in India is recognized in the political institutions of new India.' 

The communal parties and groups remained quite weak and narrow based till 1937. Most of the Muslim as also Hindu young intellectuals, workers and peasants joined the mainstreams of nationalism and socialism in the early 193 Os. In Bengal, many joined the secular and radical Krishak Praja Party. Moreover, in 1932, in an effort to bolster the sagging Muslim communalism, the British Government announced the Communal Award which accepted virtually all the Muslim communal demands embodied in the Delhi Proposals of 1927 and Jinnah's Fourteen Points of 1929. The communal forces were faced with an entirely new situation; they could not carry on as before. The question was where would they go from here. 