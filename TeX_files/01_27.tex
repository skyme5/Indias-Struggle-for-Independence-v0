\cleardoublepage
\chapter{Peasant Movements in the 1930s and '40s}



The 1930s bore witness to a new and nation-wide awakening of Indian peasants to their own strength and capacity to organize for the betterment of their living conditions. This awakening was largely a result of the combination of particular economic and political developments: the great Depression that began to hit India from 1929-30 and the new phase of mass struggle launched by the Indian National Congress in 1930.

The Depression which brought agricultural prices crashing down to half or less of their normal levels dealt a severe blow to the already impoverished peasants burdened with high taxes and rents. The Government was obdurate in refusing to scale down its own rates of taxation or in asking zamindars to bring down their rents. The prices of manufactured goods, too, didn't register comparable decreases. All told, the peasants were placed in a situation where they had to continue to pay taxes, rents, and debts at pre-Depression rates while their incomes continued to spiral steadily downward.

The Civil Disobedience Movement was launched in this atmos1here of discontent in 1930, and in many parts of the country it soon took on the form of a no-tax and no-rent campaign. Peasants, emboldened by the recent success of the Bardoli Satyagraha (1928), joined the protest in large numbers. In Andhra, for example, the political movement was soon enmeshed with the campaign against re-settlement that threatened an increase in land revenue. In U.P., no-revenue soon turned into no-rent 3nd the movement continued even during the period of truce following the Gandhi-Irwin Pact. Gandhiji himself issued a manifesto to the U.P. kisans asking them to pay only fifty per cent of the legal rent and get receipts for payment of the full amount. Peasants in Gujarat, especially in Surat and Kheda, refused to pay their taxes and went hijrat to neighbouring Baroda territory to escape government repression. Their lands and movable property were confiscated. In Bihar and Bengal, powerful movements were launched against the hated chowkidara tax by which villagers were made to pay for the upkeep of their own oppressors. In Punjab, a no-revenue campaign was accompanied by the emergence of kisan sabhas that demanded a reduction in land revenue and water-rates and the scaling down of debts. Forest satyagrahas by which peasants, including tribals, defied the forest laws that prohibited them from use of the forests were popular in Maharashtra, Bihar and the Central Provinces. Anti-zamindari struggles emerged in Andhra, and the first target was the Venkatagiri zamindari, in Nellore district.

\begin{center}*\end{center}

\paragraph*{}


The Civil Disobedience Movement contributed to the emerging peasant movement in another very important way; a whole new generation of young militant, political cadres was born from its womb. This new generation of political workers, which first received its baptism of fire in the Civil Disobedience Movement, was increasingly brought under the influence of the Left ideology that was being propagated by Jawaharlal Nehru, Subhas Bose. the Communists and other Marxist and Left individuals and groups. With the decline of the Civil Disobedience Movement, these men and women began to search for an outlet of their political energies and many of them found the answer in organizing the peasants.

Also, in 1934, with the formation of the Congress Socialist Party (CSP). the process of the consolidation of the Left forces received a significant push forward. The Communists, too, got the opportunity, by becoming members of the CSP to work in an open and legal fashion. This consolidation of the Left acted as a spur to the formation of an all-India body to coordinate the kisan movement, a process that was already under way through the efforts of N.G. Ranga and other kisan leaders. The culmination was the establishment of the All-India Kisan Congress in Lucknow in April 1936 which later changed its name to the All- India kisan Sabha. Swami Sahajanand, the militant founder of the Bihar Provincial Kisan Sabha (1929), was elected the President, and N.G. Ranga, the pioneer of the kisan movement in Andhra and a renowned scholar of the agrarian problem, the General Secretary. The first session was greeted in person by Jawaharlal Nehru. Other participants included Ram Manohar Lohia, Sohan Singh Josh, Indulal Yagnik, Jayaprakash Narayan, Mohanlal Gautam, Kamal Sarkar, Sudhin Pramanik and Ahmed Din. The Conference resolved to bring out a Kisan Manifesto and a periodic bulletin edited by Indulal Yagnik.

A Kisan Manifesto was finalized at the All-India Kisan Committee session in Bombay and formally presented to the Congress Working Committee to be incorporated into its forthcoming manifesto for the 1937 elections. The Kisan Manifesto considerably influenced the agrarian programme adopted by the Congress at its Faizpur session, which included demands for fifty per cent reduction in land revenue and rent, a moratorium on debts, the abolition of feudal levies, security of tenure for tenants, a living wage for agricultural labourers, and the recognition of peasant unions.

At Faizpur, in Maharashtra, along with the Congress session, was held the second session of the All India Kisan Congress presided over by N.G. Ranga. Five hundred kisans marched for over 200 miles from Manmad to Faizpur educating the people along the way about the objects of the Kisan Congress. They were welcomed at Faizpur by Jawaharlal Nehru, Shankar Rao Deo, M.N. Roy, Narendra Dev, S.A. Dange, M.R. Masani, Yusuf Meherally, Bankim Mukherji and many other Kisan and Congress leaders. Ranga, in his Presidential Address, declared: `We arc organizing ourselves in order to prepare ourselves for the final inauguration of a Socialist state and society.'

\begin{center}*\end{center}

\paragraph*{}


The formation of Congress Ministries in a majority of the provinces in early 1937 marked the beginning of a new phase in the growth of the peasant movement. The political atmosphere in the country underwent a marked change: increased civil liberties, a new sense of freedom born of the feeling that `our own people are in power', a heightened sense of expectation that the ministries would bring in pro-people measures — all combined to make the years 1937-39 the high-water mark of the peasant movement. The different Ministries also introduced varying kinds of agrarian legislation — for debt relief, restoration of lands lost during the Depression, for security of tenure to tenants and this provided an impetus for the mobilization of the peasantry either in support of proposed legislation or for asking for changes in its content.

The chief form of mobilization was through the holding of kisan conferences or meetings at the thana, taluqa. district and provincial levels at winch peasants' demands would be aired and resolutions passed. These conferences would be addressed by local, provincial and all-India leaders. These would also usually be preceded by a campaign of mobilization at the village level when kisan workers would tour the villages, hold meetings, enrol Congress and kisan Sabha members, collect subscriptions in money and kind and exhort the peasants to attend the conferences in large numbers. Cultural shows would be organized at these conferences to carry the message of the movement to the peasants in an appealing manner. The effect on the surrounding areas was powerful indeed, and peasants returned from these gatherings with a new sense of their own strength and a greater understanding of their own conditions.

In Malabar, in Kerala, for example, a powerful peasant movement developed as the result of the efforts mainly of CSP activists, who had been working among the peasants since 1934, touring villages and setting up Karshaka Sanghams (peasant associations). The main demands, around which the movement cohered, were for the abolition of feudal levies or akramapirivukal, renewal fees or the practice of policceluthu, advance rent, and the stopping of eviction of tenants by landlords on the ground of personal cultivation. Peasants also demanded a reduction in the tax, rent, and debt burden, and the use of proper measures by landlords when measuring the grain rent, and an end to the corrupt practices of the landlords' managers. The main forms of mobilization and agitation were the formation of village units of the Karshaka Sanghams, conferences and meetings. But a form that became very popular and effective was the marching of jat has or large groups of peasants to the houses of big jenmies or landlords, placing the demands before them and securing immediate redressal. The main demand of these jathas was for the abolition of feudal levies such as vasi, nuri, etc.

The Karshaka Sanghams also organized a powerful campaign around the demand for amending the Malabar Tenancy Act of 1929. The 6th of November, 1938 was observed as the Malabar Tenancy Act Amendment Day and meetings all over the district passed a uniform resolution pressing the demand. A committee headed by R. Ramachandra Nedumgadi was appointed by the All Malabar Karshaka Sangham to enquire into the tenurial problem and its recommendations were endorsed by the Kerala Pradesh Congress Committee on 20 November 1938. In December, two jathas of five hundred each started from Karivallur in north Malabar and Kanjikode in the south and, after being received and hosted by local Congress Committees en route converged at Chevayur near Calicut where the All Malabar Karshaka Sangham was holding its conference. A public meeting was held the same evening at Calicut beach presided over by P. Krishna Pillai, the CSP and later Communist leader, and resolutions demanding amendments in the Tenancy Act were passed. In response to popular pressure, T. Prakasam, the Andhra Congress leader who was the Revenue Minister in the Congress Ministry in Madras Presidency, toured Malabar in December 1938 to acquaint himself with the tenant problem. A Tenancy Committee was set up which included three left-wing members. The Karshaka Sangham units and Congress committees held a series of meetings to mobilize peasants to present evidence and to submit memoranda to the Committee. But, by the time the Committee submitted its report in 1940, the Congress Ministries had already resigned and no immediate progress was possible. But the campaign had successfully mobilized the peasantry on the tenancy question and created an awareness that ensured that in later years these demands would inevitably have to be accepted. Meanwhile, the Madras Congress Ministry had passed legislation for debt relief, and this was welcomed by the Karshaka Sangham.

In coastal Andhra, too, the mobilization of peasants proceeded on an unprecedented scale. The Andhra Provincial Ryots Association and the Andhra Zamin Ryots Association already had a long history of successful struggle against the Government and zamindars. In addition, N.G. Ranga had, since 1933, been running the Indian Peasants' Institute in his home village of Nidobrolu in Guntur district which trained peasants to become active workers of the peasant movement. After 1936, left- wing Congressmen, members of the CSP, many of whom were to latter join the CPI also joined in the effort to organize the peasants, and the name of P. Sundarayya was the foremost among them.

The defeat of many zamindar and pro-zamindar candidates in the 1937 elections by Congress candidates dealt a blow to the zamindars prestige and gave confidence to the zamindari ryots. Struggles were launched against the Bobbili and Mungala zamindaris, and a major struggle erupted against the Kalipatnam zamindari over cultivation and fishing rights.

In coastal Andhra, the weapon of peasant marches had already been used effectively since 1933. Peasant marchers would converge on the district or taluqa headquarters and present a list of demands to the authorities. But, in 1938, the Provincial Kisan Conference organized, for the first time, a march on a massive scale — a true long march in which over 2.000 kisans marched a distance of over 1,500 miles, starting from Itchapur in the north, covering nine districts and walking for a total of 130 days En route, they held hundreds of meetings attended by lakhs of peasants and collected over 1,100 petitions; these were then presented to the provincial legislature in Madras on 27 March 1938. One of their main demands was for debt relief, and this was incorporated in the legislation passed by the Congress Ministry and was widely appreciated in Andhra. In response to the peasants' demands the Ministry had appointed a Zamindari Enquiry Committee, but the legislation based on its recommendations could not be passed before the Congress Ministries resigned.

Another notable feature of the movement in Andhra was the organization of Summer Schools of Economics and Politics for peasant activists. These training camps, held at Kothapatnam, Mantenavaripalarn and other places were addressed by many of the major Left Communist leaders of the time including P.C. Joshi, Ajoy Ghosh and R.D. Bhardwaj. Lectures were delivered on Indian history, the history of the national struggle on Marxism, on the Indian economy and numerous associated subjects. Money and provisions for running these training camps were collected from the peasants of Andhra. The celebration of various kisan and other `days,' as well as the popularization of peasant songs, was another form of mobilization.

Bihar was another major area of peasant mobilization in this period. Swami Sahajanand., the founder of the Bihar Provincial Kisan Sabha and a major leader 3f the All India Kisan Sabha, was joined by many other left-wing leaders like Karyanand Sharma, Rahul Sankritayan, Panchanan Sharma and Yadunandan Sharma in spreading the kisan sabha organization to the village of Bihar.

The Bihar Provincial Kisan Sabha effectively used meetings, conferences, rallies, and mass demonstrations, including a demonstration of one lakh peasants at Patna in 1938, to popularize the kisan Sabha programme. The slogan of zamindari abolition, adopted by the Sabha in 1935, was popularized among the peasants through resolutions passed at these gatherings. Other demands included the stopping of illegal levies, the prevention of eviction of tenants and the return of Bakasht lands.

The Congress Ministry had initiated legislation for the reduction of rent and the restoration of Bakasht lands. Bakasht lands were those which the occupancy tenants had lost to zamindars, mostly during the Depression years, by virtue of non-payment of rent, and which they often continued to cultivate as share-croppers. But the formula that was finally incorporated in the legislation on the basis of an agreement with the zamindars did not satisfy the radical leaders of the kisan Sabha. The legislation gave a certain proportion of the lands back to the tenants on condition that they pay half the auction price of the land. Besides, certain categories of land had been exempted from the operation of the law.

The Bakasht lands issue became a major ground of contention between the Kisan Sabha and the Congress Ministry. Struggles, such as the one already in progress in Barahiya tal in Monghyr district under the leadership of Karyanand Shanna, were continued and new ones emerged. At Reora, in Gaya district, with Yadunandan Sharma at their head, the peasants won a major victory when the District Magistrate gave an award restoring 850 out of the disputed 1,000 bighas to the tenants. This gave a major fillip to the movement elsewhere. In Darbhanga, movements emerged in Padri, Raghopore, Dekuli and Pandoul. Jamuna Karjee led the movement in Saran district, and Rahul Sankritayan in Annawari. The movements adopted the methods of Saiyagraha, and forcible sowing and harvesting of crops. The zamindars retaliated by using lathials to break up meetings and terrorize the peasants. Clashes with the zamindars' men became the order of the day and the police often intervened to arrest the leaders and activists. In some places, the government and other Congress leaders intervened to bring a compromise. The movement on the Bakasht issue reached its peak in late 1938 and 1939, but by August 1939 a combination of concessions, legislation and the arrest of about 600 activists succeeded in quietening the peasants. The movement was resumed in certain pockets in 1945 and continued in one form or another till zamindari was abolished.

Punjab was another centre of kisan activity. Here, too, the kisan sabhas that had emerged in the early 1930s, through the efforts of Naujawan Bharat Sabha, Kirti Kisan. Congress and Akali activists, were given a new sense of direction and cohesion by the Punjab Kisan Committee formed in 1937. The pattern of mobilization was the familiar one — kisan workers toured villages enrolling kisan Sabha and Congress members, organizing meetings, mobilizing people for the tehsils, district and provincial level conferences (which were held with increasing frequency and attended by an array of national stars). The main demands related to the reduction of taxes and a moratorium on debts. The target of attack was the Unionist Ministry, dominated by the big landlords of Western Punjab.

The two issues that came up for an immediate struggle were the resettlement of land revenue of Amritsar and Lahore districts and the increase in the canal tax or water-rate. Jathas marched to the district headquarters and huge demonstrations were held. The culmination was the Lahore Kisan Morcha in 1939 in which hundreds of kisans from many districts of the province courted arrest. A different kind of struggle broke out in the Multan and Montgomery canal colony areas. Here large private companies that had leased this recently-colonized land from the government and some big landlords insisted on recovering a whole range of feudal levies from the share-croppers who tilled the land. The kisan leaders organized the tenants to resist these exactions which had recently been declared illegal by a government notification and there were strikes by cultivators in some areas in which they refused to pick cotton and harvest the crops. Many concessions were won as a result. The tenants' struggle, I suspended as a result of the War, was resumed in 1946-47.

The peasant movement in Punjab was mainly located in the Central districts, the most active being the districts of Jullundur, Amritsar, Hoshiarpur, Lyalipur and Sheikhupura. These districts were the home of the largely self-cultivating Sikh peasantry that had already been mobilized into the national struggle via the Gurdwara Reform Movement of the early 1920s and the Civil Disobedience Movement in 1930-32. The Muslim tenants-at-will of Western Punjab, the most backward part of the province, as well as the Hindu peasants of South-eastern Punjab (the present- day Haryana) largely remained outside the ambit of the Kisan Movement. The tenants of Montgomery and Multan districts mobilized by the kisan leaders were also mostly emigrants from Central Punjab, Baba Sohan Singh. Teja Singh Swatantar, Baba Rur Singh, Master Han Singh, Bhagat Singh Bilga, and Wadhawa Ram were some of the important peasant leaders.'

The princely states in Punjab also witnessed a major outbreak of peasant discontent. The most powerful movement emerged in Patiala and bas based on the demand for restoration of lands illegally seized by a landlord-official combine through various forms of deceit and intimidation. The muzaras (tenants) refused to pay the batai (share rent) to their biswedars (landlords) and in this they were led by Left leaders like Bhagwan Singh Longowalia and Jagir Singh Joga and in later years by Teja Singh Swatantar. This struggle continued intermittently till 1953 when legislation enabling the tenants to become owners of their land was passed.

In other parts of the country as well, the mobilization of peasants around the demands for security of tenure, abolition of feudal levies, reduction of taxes and debt relief, made major headway. In Bengal, under the leadership of Bankim Mukherji, the peasants of Burdwan agitated against the enhancement of the canal tax on the Damodar canal and secured major concessions. Kisans of the 24-Parganas pressed their demands by a march to Calcutta in April 1938. In Surma Valley, in Assam, a no-rent struggle continued for six months against zamindari oppression and Karuna Sindhu Roy conducted a major campaign for amendment of the tenancy law. In Orissa, the Utkal Provincial Kisan Sabha, organized by Malati Chowdhury and others in 1935, succeeded in getting the kisan manifesto accepted by the PCC as part of its election manifesto, and the Ministry that followed introduced significant agrarian legislation. In the Orissa States, a powerful movement in which tribals also participated was led on the question of forced labour, rights in forests, and the reduction of rent. Major clashes occurred in Dhenkanal and thousands fled the state to escape repression. The kisans of Ghalla Dhir state in the North-West Frontier Province protested against evictions and feudal exactions by their Nawab. In Gujarat the main demand was for the abolition of the system of hail (bonded labour) and a significant success was registered. The Central Provinces Kisan Sabha led a march to Nagpur demanding the abolition of the malguzari system, reduction of taxes and moratorium on debts.

\begin{center}*\end{center}

\paragraph*{}


The rising tide of peasant awakening was checked by the outbreak of World War II which brought about the resignation of the Congress Ministries and the launching of severe repression against left-wing and kisan Sabha leaders and workers because of their strong anti-War stance. The adoption by the CM of the Peoples' War line in December 1941 following Hitler's attack on the Soviet Union created dissensions between the Communist and non-Communist members of the kisan Sabha. These dissensions came to a head with the Quit India Movement, in which Congress Socialist members played a leading role. The CPI because of its pro-War People's War line asked its cadres to stay away, and though mans local level workers did join the Quit India Movement, the party line sealed the rift in the kisan sabha ranks, resulting in a split in 1943. In these year' three major leaders of the All India Kisan Sabha, N.G. Ranga, Swam, Sahajanand Saraswati and Indulal Yagnik, left the organization. Nevertheless, during the War years the kisan Sabha continued to play and important role in various kinds of relief work, as for example in the Bengal Famine of 1943 and helped to lessen the rigour of shortages of essential goods, rationing and the like. It also continued its organizational work, despite being severely handicapped by its taking the unpopular pro- War stance which alienated it from various sections of the peasantry.

\begin{center}*\end{center}

\paragraph*{}


The end of the War, followed by the negotiations for the transfer of power and the anticipation of freedom, marked a qualitatively new stage in the development of the peasant movement. A new spirit was evident and the certainty of approaching freedom with the promise of a new social order encouraged peasants, among other social groups, to assert their rights and claims with a new vigour.

Many struggles that had been left off in 1939 were renewed. The demand for zamindari abolition was pressed with a greater sense of urgency. The organization of agricultural workers in Andhra which had begun a few years earlier took on the form of a struggle for higher wages and use of standard measures for payment of wages in kind.

The peasants of Punnapra-Vayalar in Travancore fought bloody battles with the administration. In Telengana, the peasants organized thcmseh'es to resist the landlords' oppression and played an important role in the anti-Nizam struggle. Similar events took place in other parts of the country. But in British India, it was the tebhaga struggle in Bengal that held the limelight. in late 1946, the share-croppers of Bengal began to assert that they would no longer pay a half share of their crop to the jotedars but only one-third and that before division the crop would be stored in their khamars (godowns) and not that of the jotedars. They were no doubt encouraged by the fact that the Bengal Land Revenue Commission, popularly known as the Floud Commission, had already made this recommendation in its report to the government. The Hajong tribals were simultaneously demanding commutation of their kind rents into cash rents. The tebhaga movement, led by the Bengal Provincial Kisan Sabha, soon developed into a clash between jotedars and bargadars with the bargadars insisting on storing the crop in their own khamars.

The movement received a great boost in late January 1947 when the Muslim League Ministry led by Suhrawardy published the Bengal Bargadars Temporary Regulation Bill in the Calcutta Gazette on 22 January 1947. Encouraged by the fact that the demand for tebhaga could no longer be called illegal, peasants in hitherto untouched villages and areas joined the struggle. In many places, peasants tried to remove the paddy already stored in the jotedars' khamars to their own, and this resulted in innumerable clashes.

The jotedars appealed to the Government, and the police came in to suppress the peasants. Major clashes ensued at a few places, the most important being the one at Khanpur in which twenty peasants were killed. Repression continued and by the end of February the movement was virtually dead. A few incidents occurred in March as well, but these were only the death pangs of a dying struggle.

The Muslim League Ministry failed to pursue the bill in the Assembly and it was only in 1950 that the Congress Ministry passed a Bargadars Bill which incorporated, in substance, the demands of the movement. The main centres of the movement were Dinajpur, Rangpur, Jalpaiguri, Mymensingh, Midnapore, and to a lesser extent 24-Parganas and Khulna. Initially, the base was among the Rajbansi Kshatriya peasants, but it soon spread to Muslims, Hajongs, Santhals and Oraons. Among the important leaders of this movement were Krishnobinode Ray, Abani Lahiri, Sunil Sen, Bhowani Sen, Moni Singh, Ananta Singh, Bhibuti Guha, Ajit Ray, Sushil Sen, Samar Ganguli, and Gurudas Talukdar.

\begin{center}*\end{center}

\paragraph*{}


To draw up a balance sheet of such a diverse and varied struggle is no easy task, but it can be asserted that perhaps the most important contribution of the peasant movements that covered large areas of the subcontinent in the 30s and 40s was that even when they did not register immediate successes, they created the climate which necessitated the post-Independence agrarian reforms. Zamindari abolition, for example, did not come about as a direct culmination of any particular struggle, but the popularization of the demand by the kisan sabha certainly contributed to its achievement.

The immediate demands on which struggles were fought in the pre-Independence days were the reduction of taxes, the abolition of illegal cesses or feudal levies and begar or vethi, the ending of oppression by landlords and their agents, the reduction of debts, the restoration of illegally or illegitimately seized lands, and security of tenure for tenants. Except in a few pockets like Andhra and Gujarat, the demands of agricultural labourers did not really become part of the movement. These demands were based on the existing consciousness of the peasantry of their just or legitimate rights, which was itself a product of tradition, custom, usage, and legal rights. When landlords or the Government demanded what was seen by peasants as illegitimate — high taxes, exorbitant rents, illegal cesses, forced labour or rights over land which peasants felt was theirs — they were willing to resist if they could muster the necessary organizational and other resources. But they were also willing to continue to respect what they considered legitimate demands. The struggles based on these demands were clearly not aimed at the overthrow of the existing agrarian structure but towards alleviating its most oppressive aspects. Nevertheless, they corroded the power of the landed classes in many ways and thus prepared the ground for the transformation of the structure itself. The kisan movement was faced with the task of transforming the peasants' consciousness and building movements based on a transformed consciousness.

It is also important to note that, by and large, the forms of struggle and mobilization adopted by the peasant movements in diverse areas were similar in nature as were their demands. The main focus was on mobilization through meetings, conferences, rallies, demonstrations, enrolment of members, formation of kisan sabhas or ryotu and karshaka sanghams. Direct action usually involved Satyagraha or civil disobedience, and non-payment of rent and taxes. All these forms had become the stock-in-trade of the national movement for the past several years. As in the national movement, violent clashes were the exception and not the norm. They were rarely sanctioned by the leadership and were usually popular responses to extreme repression.

The relationship of the peasant movement with the national movement continued to be one of a vital and integral nature. For one, areas where the peasant movement was active were usually the ones that had been drawn into the earlier national struggles. This was true at least of Punjab, Kerala, Andhra, U.P. and Bihar. This was hardly surprising since it was the spread of the national movement that had created the initial conditions required for the emergence of peasant struggles — a politicized and conscious peasantry and a band of active political workers capable of and willing to perform the task of organization and leadership.

In its ideology as well, the kisan movement accepted and based itself on the ideology of nationalism. Its cadres and leaders carried the message not only of organization of the peasantry on class lines but also of national freedom. As we have shown earlier, in most areas kisan activists simultaneously enrolled kisan sabha and Congress members.

True, in some regions, like Bihar, serious differences emerged between sections of Congressmen and the kisan sabha and at times the kisan movement seemed set on a path of confrontation with the Congress, but this tended to happen only when both left-wing activists and right- wing or conservative Congressmen took extreme positions and showed an unwillingness to accommodate each other. Before 1942 these differences were usually contained and the kisan movement and the national movement occupied largely common ground. With the experience of the split of 1942, the kisan movement found that if it diverged too far and too clearly from the path of the national movement, it tended to lose its mass base, as well as create a split within the ranks of its leadership. The growth and development of the peasant movement was thus indissolubly linked with the national struggle for freedom.
