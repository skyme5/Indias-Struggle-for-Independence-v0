
\chapter{Socio-Religious Reforms and the National Awakening}

`I regret to say,' wrote Raja Rammohan Roy in 1828, `that the present system of religion adhered to by the Hindus is not well calculated to promote their political interest. The distinctions of castes introducing innumerable divisions and sub-divisions among them has entirely deprived them of patriotic feeling, and the multitude of religious rites and ceremonies and the laws of purification have totally disqualified them from undertaking any difficult enterprise. It is, I think, necessary that some change should take place in their religion at least for the sake of their political advantage and social comfort.'' Written at a time when Indians had just begun to experience the `intellectual and cultural turmoil that characterized social life in nineteenth century India this represented the immediate Indian response. The British conquest and the consequent dissemination of colonial culture and ideology had led to an inevitable introspection about the strengths and weaknesses of indigenous culture and institutions. The response, indeed, was varied but the need to reform social and religious life was a commonly shared conviction. The social base of this quest which has generally, but not altogether appropriately been called the renaissance, was the newly emerging middle class and the traditional as well as western educated intellectuals. The socio-cultural regeneration in nineteenth century India was occasioned by the colonial presence, but not created by it.

The spirit of reform embraced almost the whole of India beginning with the efforts of Raja Rammohan Roy in Bengal leading to the formation of the Brahmo Samaj in 1828. Apart from the Brahmo Samaj, which has branches in several parts of the country, the Paramahansa Mandali and the Prarthana Samaj in Maharashtra and the Arya Samaj in Punjab and North India were some of the prominent movements among the Hindus. There were several other regional and caste movements like the Kayasth Sabha in Uttar Pradesh and the Sarin Sabba in Punjab. The backward castes also started the work of reformation with the Satya Sodhak Samaj in Maharashtra and the Sri Narayana Dharma Paripalana Sabha in Kerala. The Ahmadiya and Aligarh movements, the Singh Sabha and the Rehnumai Mazdeyasan Sabha represented the spirit of reform among the Muslims, the Sikhs and the Parsees respectively. Despite being regional in scope and content and confined to a particular religion, their general perspectives were remarkably similar; they were regional and religious manifestations of a common Consciousness.

Although religious reformation `was a major concern of these movements, none of them were exclusively religious in character. Strongly humanist in inspiration, the idea of otherworldliness and salvation were not a part of their agenda; instead their attention was focused on worldly existence. Raja Rammohan Roy was prepared to concede the possible existence of the other world mainly due to its utilitarian value. Akshay Kumar Dutt and Ishwarchandra Vidyasagar were agnostics who refused to be drawn into any discussion on supernatural questions. Asked about the existence of God, Vidyasagar quipped that he had no time to think about God, since there was much to be done on earth. Bankim Chandra Chatterjee and Vivekananda emphasized the secular use of religion and used spirituality to take cognizance of the material conditions of human existence. Given the inter-connection between religious beliefs and social practices, religious reformation was a necessary pre-requisite for social reform. `The Hindu meets his religion at every turn. In eating, in drinking, moving, sitting, standing, he is to adhere to sacred rules, to depart from which is sin and impiety.' Similarly, the social life of the Muslims was strongly influenced by religious tenets. Religion was the dominant ideology of the times and it was not possible to undertake any social action without coming to grips with it.

\begin{center}*\end{center}

\paragraph*{}

Indian society in the nineteenth century was caught in a vicious web created by religious superstitions and social obscurantism. Hinduism, as

Max Weber observed, had `become a compound of magic, animism and superstition' and abominable rites like animal sacrifice and physical torture had replaced the worship of God. The priests exercised an overwhelming and, indeed, unhealthy influence on the minds of the people. Idolatry and polytheism helped to reinforce their Position. As suggested by Raja Rammohan Roy, their monopoly of scriptural knowledge and of ritual interpretation imparted a deceptive character to all religious systems. The faithful lived in submission, not only to God, the powerful and unseen, but even to the whims, fancies and wishes of the priests. There was nothing that religious ideology could not persuade people to do --- women even went to the extent of offering themselves to priests to satisfy their carnal pleasures.

Social conditions were equally depressing. The most distressing was the position of women. The birth of a girl was unwelcome, her marriage a burden and her widowhood inauspicious. Attempts to kill girl infants at birth were not unusual. Those who escaped this initial brutality were subjected to the violence of marriage at a tender age. Often the marriage was a device to escape social ignominy and, hence, marital life did not turn out to be a pleasant experience. An eighty-year-old Brahmin in Bengal had as many as two hundred wives, the youngest being just eight years old. Several women hardly had a married life worth the name, since their husbands participated in nuptial ceremonies for a consideration and rarely set eyes on their wives after that. Yet when their husbands died they were expected to commit Sati which Rammohan described as `murder according to every shasfra.' If they succeeded in overcoming this social coercion, they were condemned, as widows, to life-long misery, neglect and humiliation.

Another debilitating factor was caste; it sought to maintain a system of segregation, hierarchically ordained on the basis of ritual status. The rules and regulations of caste hampered social mobility, fostered social divisions and sapped individual initiative. Above all was the humiliation of untouchability which militated against human dignity.

There were innumerable other practices marked by constraint, credulity, status, authority, bigotry and blind fatalism. Rejecting them as features of a decadent society, the reform movements sought to create a social climate for modernization. In doing so, they referred to a golden past when no such malaise existed. The nineteenth century situation was the result of an accretionary process; a distortion of a once ideal past. The reformers' vision of the future, however, was not based on this idealization. It was only an aid and an instrument --- since practices based on faith cannot be challenged without bringing faith itself into question. Hence, Raja Rammohan Roy, demonstrated that sati had no religious sanction, Vidyasagar did not `take up his pen in defence of widow marriage' without being convinced about Scriptural support and Dayanand based his anti-casteism on Vedic authority.

This, however, did not mean a subjection of the present to the past nor a blind resurrection of tradition `The dead and the buried,' maintained Mahadev Govind Ranade, the doyen of reformers in Maharashtra, `are dead, buried, and burnt once for all and the dead past cannot, therefore, be revived except by a reformation of the old materials into new organized forms.' Neither a revival of the past nor a total break with tradition was contemplated.

\begin{center}*\end{center}

\paragraph*{}

Two important intellectual criteria which informed the reform movements were rationalism and religious universalism. Social relevance was judged by a rationalist critique. It is difficult to match the uncompromising rationalism of the early Raja Rammohan Roy or Akshay Kumar Dutt. Rejecting supernatural explanations, Raja Rammohan Roy affirmed the principle of causality linking the whole phenomenal universe. To him demonstrability was the sole criterion of truth. In proclaiming that rationalism is our only preceptor,' Akshay Kumar went a step further. All natural and social phenomena, he held, could be analyzed and understood by purely mechanical processes. This perspective not only enabled them to adopt a rational approach to tradition but also to evaluate the contemporary socio-religious practices from the standpoint of social utility and to replace faith with rationality. In the Brahmo Samaj, it led to the repudiation of the infallibility of the Vedas, and in the Aligarh Movement, to the reconciliation of the teachings of Islam with the needs of the modern age. Holding that religious tenets were not immutable, Syed Ahmed Khan emphasized the role of religion in the progress of society: if religion did not keep pace with and meet the demands of the time. It would get fossilized as in the case of Islam in India.

The perspectives on reform were not always influenced by religious Considerations A rational and secular outlook was very much evident in Posing an alternative to prevalent social practices. In advocating widow marriage and opposing polygamy and child marriage, Akshay Kumar was not concerned about religious sanction or whether they existed in the pa His arguments were mainly based on their effects of Society. Instead of depending on the scriptures, he cited medical Opinion against Child marriage. He held very advanced ideas about marriage and family: courtship before marriage, partnership and equality as the basis of married life and divorce by both law and custom. In Maharashtra, as compared to other regions, there was less dependence on religion as an aid to social reform. To Gopal Han Deshmukh, popularly known as Lokahitavadi whether social reforms had the sanction of religion was immaterial. If religion did not sanction these, he advocated that religion itself should be changed as it was made by man and what was laid down, in the scriptures need not necessarily be of contemporary relevance.

Although the ambit of reforms was particularistic, their religious perspective was universalistic. Raja Rammohan Roy considered different religions as national embodiments of universal theism. The Brahmo Samaj was initially conceived by him as a universalist church. He was a defender of the basic and universal principles of all religions --- the monotheism of the Vedas and the Unitarianism of Christianity --- and at the same time attacked polytheism of Hinduism and the trinitarianism of Christianity. Syed Ahmed Khan echoed the same idea: all prophets had the same din (faith) and every country and nation had different prophets. This perspective found clearer articulation in Keshub Chandra Sen's ideas. He said `our position is not that truths are to be found in all religions, but all established religions of the world are true.' He also gave expression to the social implications of this universalist perspective: `Whoever worships the True God daily must learn to recognize all his fellow countrymen as brethren. Caste would vanish in such a state of society. If I believe that my God is one, and that he has created us all, I must at the same time instinctively, and with all the warmth of natural feelings, look upon all around me --- whether Parsees, Hindus, Mohammadans or Europeans --- as my brethern.'

The universalist perspective was not a purely philosophic concern; it strongly influenced the political and social outlook of the time, till religious particularism gained ground in the second half of the nineteenth century. For instance, Raja Rammohan Roy considered Muslim lawyers to be more honest than their Hindu counterparts and Vidyasagar did not discriminate against Muslims in his humanitarian activities. Even to Bankim, who is credited with a Hindu outlook, dharma rather than religious belonging was the criterion for determining superiority. Yet, `Muslim yoke' and `Muslim tyranny' were epithets often used to describe the pre-colonial rule. This, however, was not a religious but a political attitude, influenced by the arbitrary character of pre-colonial political institutions. The emphasis was not on the word `Muslim' but on the word `tyranny.' This is amply clear from Syed Ahmed Khan's description of the pre-colonial system: `The rule of the former emperors and rajas was neither in accordance with the Hindu nor the Mohammadan religion. It was based upon nothing but tyranny and oppression; the law of might was that of right; the voice of the people was not listened to'. The yardstick obviously was not religious identity, but liberal and democratic principles. This, however, does not imply that religious identity did not influence the social outlook of the people; in fact, it did very strongly. The reformers' emphasis on universalism was an attempt to contend with it. However, faced with the challenge of colonial culture and ideology, universalism, instead of providing the basis for the development of a secular ethos, retreated into religious particularism.

\begin{center}*\end{center}

\paragraph*{}

The nineteenth century witnessed a cultural-ideological struggle against the backward elements of traditional culture, on the one hand, and the fast hegemonizing colonial culture and ideology on the other. The initial refonning efforts represented the former. In the religious sphere they sought to remove idolatry, polytheism and priestly monopoly of religious knowledge and to simplify religious rituals. They were important not for purely religious reasons but equally for their social implications. They contributed to the liberation of the individual from conformity born out of fear and from uncritical submission to the exploitation of the priests. The dissemination of religious knowledge through translation of religious texts into vernacular languages and the right granted to the laity to interpret scriptures represented an important initial breach in the stranglehold of misinterpreted religious dogmas. The simplification of rituals made worship a more intensely personal experience without the mediation of intermediaries. The individual was, thus, encouraged to exercise his freedom.

The socially debilitating influence of the caste system which perpetuated social distinctions was universally recognized as an area which called for urgent reform. It was morally and ethically abhorrent, more importantly, it militated against patriotic feelings and negated the growth of democratic ideas. Raja Rammohan Roy initiated, in ideas but not in practice, the opposition which became loud and clear as the century progressed. Ranade, Dayanand and Vivekananda denounced the existing system of caste in no uncertain terms. While the reform movements generally stood for its abolition, Dayanand gave a utopian explanation for chaturvarna (four-fold varna division of Hindu society) and sought to maintain it on the basis of virtue. `He deserves to be a Brahman who has acquired the best knowledge and character, and an ignorant person is fit to be classed as a shudra,' he argued. Understandably the most virulent opposition to caste came from lower caste movements. Jyotiba Phule and Narayana Guru were two unrelenting critics of the caste system and its consequences. A conversation between Gandhiji and Narayana Guru is significant. Gandhiji, in an obvious reference to Chaturvarna and the inherent differences in quality between man and man, observed that all leaves of the same tree are not identical in shape and texture. To this Narayana Guru pointed out that the difference is only superficial, but not in essence: the juice of all leaves of a particular tree would be the same in content. It was he who gave the call --- `one religion, one caste and one God for mankind' which one of his disciples, Sahadaran Ayyapan, changed into `no religion, no caste and no God for mankind.'

The campaign for the improvement of the condition and status of women was not a purely humanitarian measure either. No reform could be really effective without changes in the domestic conditions, the social space in which the initial socialization of the individual took place. A crucial role in this process was played by women. Therefore, there could be no reformed men and reformed homes without reformed women. Viewed from the standpoint of women, it was, indeed, a limited perspective. Nevertheless it was realized that no country could ever make `significant progress in civilization whose females were sunk in ignorance.'

If the reform movements had totally rejected tradition, Indian society would have easily undergone a process of westernization. But the reformers were aiming at modernization rather than westernization. A blind initiation of western cultural norms was never an integral part of reform.

To initiate and undertake these reforms which today appear to be modest, weak and limited was not an easy proposition. It brought about unprecedented mental agony and untold domestic and social tension. Breaking the bonds of tradition created emotional and sentimental crises for men and women caught between two worlds. The first widow marriage in Bengal attracted thousands of curious spectators. To the first such couple in Maharashtra the police had to give lathis to protect themselves! Rukmabhai, who refused to accept her uneducated and unaccomplished husband, virtually unleashed a storm. Faced with the prospect of marrying a young girl much against his conviction, Ranade spent several sleepless nights. So did Lokahitavadi, Telang and a host of others who were torn between traditional sentiments and modern commitments. Several however succumbed to the former, but it was out of this struggle that the new men and the new society evolved in India.

\begin{center}*\end{center}

\paragraph*{}

Faced with the challenge of the intrusion of colonial culture and ideology, an attempt to reinvigorate traditional institutions and to realize the potential of traditional culture developed during the nineteenth century. The initial expression of the struggle against colonial domination manifested itself in the realm of culture as a result of the fact that the principles on which the colonial state functioned were not more retrogressive than those of the pre-colonial state. All intrusions into the cultural realm were more intensely felt. Therefore, a defence of indigenous culture developed almost simultaneously with the colonial conquest.

This concern embraced the entire cultural existence, the way of life and all signifying practices like language, religion, art and philosophy. Two features characterized this concern; the creation of an alternate cultural-ideological system and the regeneration of traditional institutions. The cultivation of vernacular languages, the creation of an alternate system of education, the efforts to regenerate Indian art and literature, the emphasis on Indian dress and food, the defence of religion and the attempts to revitalize the Indian system of medicine, the attempt to probe the potentialities of pre-colonial technology and to reconstruct traditional knowledge were some of the expressions of this concern. The early inklings of this can be discerned in Raja Rammohan Roy's debates with the Christian missionaries, in the formation and activities of Tattvabodhini Sabha, in the memorial on education signed by 70,000 inhabitants of Madras and in the general resentment against the Lex Loci Act (the Act proposed in 1845 and passed in 1850 provided the right to inherit ancestral property to Hindu converts to Christianity). A more definite articulation, however, was in the ideas and activities of later movements generally characterized as conservative and revivalist. Strongly native in tendency, they were clearly influenced by the need to defend indigenous culture against colonial cultural hegemony. In this specific historical sense, they were not necessarily retrogressive, for underlying these efforts was the concern with the revival of the cultural personality, distorted, if not destroyed, by colonial domination. More so because it formed an integral element in the formation of national consciousness. Some of these tendencies however, were not able to transcend the limits of historical necessity and led to a sectarian and obscurantist outlook. This was possibly a consequence of the lack of integration between the cultural and political struggles, resulting in cultural backwardness, despite political advance. The cultural-ideological struggle, represented by the socio- religious movements, was an integral part of the evolving national consciousness. This was so because it was instrumental in bringing about the initial intellectual and cultural break which made a new vision of the future possible. Second, it was a part of the resistance against colonial cultural and ideological hegemony. Out of this dual struggle evolved the modern cultural situation: new men, new homes and a new society.
