\chapter{Indian Capitalists and the National Movement}
\begin{multicols}{2}

Among the various groups that participated in the national movement were several individual capitalists who joined the Congress. They fully identified with the movement, went to jails and accepted the hardships that were the lot of Congressmen in the colonial period. The names of Jamnalal Bajaj, Vadilal Lallubhai Mehta, Samuel Aaron, Lala Shankar Lal, and others are well known in this regard. There were other individual capitalists who did not join the Congress but readily gave financial and other help to the movement. People like G.D. Birla, Ambalal Sarabhai and Waichand Hirachand, fall into this category. There were also a large number of smaller traders and merchants who at various points came out in active support of the national movement. On the other hand, there were several individual capitalists or sections of the class who either remained neutral towards the Congress and the national movement or even actively opposed it. 

In this chapter, we shall examine the overall strategy of the Indian capitalist class, as a class, towards the national movement, rather than highlight the role of various individuals or sections within the class who did not necessarily represent the class as a whole, or even its dominant section.

\begin{center}*\end{center}

\paragraph*{}

At the outset it must be said that the economic development of the Indian capitalist class in the colonial period was substantial and in many ways the nature of its growth was quite different from the usual experience in other colonial countries. This had important implications regarding the class's position vis-a-vis imperialism. First, the Indian capitalist class grew from about the mid 19th century with largely an independent capital base and not as junior partners of foreign capital or as compradors. Second, the capitalist class on the whole was not tied up in a subservient position with pro-imperialist feudal interests either economically or politically. In fact, a wide cross section of the leaders of the capitalist class actually argued, m 1944--45, in their famous Bombay plan (the signatories to which were Purshottamdas Thakurdas, J.R. D. Tata, G.D. Birla, Ardeshir Dalal, Sri Ram, Kasturbhai Lalbhai, A.D. Shroff and John Mathai) for comprehensive land reform, including cooperativization of production, finance and marketing.' 

Third, in the period 1914--1947, the capitalist class grew rapidly, increasing its strength and self-confidence. This was achieved primarily through import substitution; by edging out or encroaching upon areas of European domination, and by establishing almost exclusive control over new areas thus accounting for the bulk of the new investments made since the 1920s. Close to independence, indigenous enterprise had already cornered seventy two to seventy three per cent of the domestic market and over eighty per cent of the deposits in the organized banking sector. 

However, this growth, unusual for a colonial capitalist class, did not occur, as is often argued, as a result or by-product of colonialism or because of a policy of decolonization. On the contrary it was achieved in spite of and in opposition to colonialism --- by waging a constant struggle against colonialism and colonial interests, i.e., by wrenching space from colonialism itself. 

There was, thus, nothing in the class position or the economic interest of the Indian capitalists which, contrary to what is so often argued,4 inhibited its opposition to imperialism. In fact, by the mid 1920s, Indian capitalists began to correctly perceive their long-term class interest and felt strong enough to take a consistent and openly anti-imperialist position. The hesitation that the class demonstrated was not in its opposition to ampenalism but in the choice of the specific path to fight imperialism. It was apprehensive that the path chosen should not be one which, while opposing imperialism, would threaten its own existence, i.e., undermine capitalism itself.

\begin{center}*\end{center}

\paragraph*{}
Before we go on to discuss the capitalist class's position vis- a-vis imperialism and vis-a-vis the course of the anti-imperialist movement, we should look at the emergence of the class as a political entity --- a `class for itself.' 

Since the early 1920s, efforts were being made by various capitalists like G.D. Birla and Purshottamdas Thakurdas to establish a national level organization of Indian commercial, industrial and financial interests (as opposed to the already relatively more organized European interests in India) to be able to effectively lobby with the colonial government. This effort culminated in the formation of the Federation of Indian Chambers of Commerce and Industry (FICCI) in 1927, with a large and rapidly increasing representation from all parts of India. The FICCI was soon recognized by the British government as well as the Indian public in general, as representing the dominant opinion as well as the overall consensus within the Indian capitalist class. 

The leaders of the capitalist class also clearly saw the role of the FICCI as being that of `national guardians of trade, commerce and industry,' performing in the economic sphere in colonial India the functions of a national government.5 In the process, Indian capitalists, with some of the most astute minds of the period in their ranks, developed a fairly comprehensive economic critique of imperialism in all its manifestations, whether it be direct appropriation through-home charges or exploitation through trade, finance, currency manipulation or foreign investments, including in their sweep the now fashionable concept of unequal exchange occurring in trade between countries with widely divergent productivity levels. (G.D. Birla and S.P. Jam were talking of unequal exchange as early as the 1930s).6 The Congress leaders quite often saw their assistance as invaluable and treated their opinions and expertise on many national economic issues with respect. 

The FICCI was, however, not to remain merely a sort of trade union organization of the capitalist class fighting for its own economic demands and those of the nation. The leaders of the capitalist class now clearly saw the necessity of, and felt strong enough for, the class to effectively intervene in politics. As Sir Purshottamdas, President of FICCI, declared at its second annual session in 1928: `We can no more separate our politics from our economics.' Further involvement of the class in politics meant doing so on the side of Indian nationalism. `Indian commerce and industry are intimately associated with and are, indeed, an integral part of the national movement --- growing with its growth and strengthening with its strength.' Similarly G.D. Birla was to declare a little later in 1930: `It is impossible in the present ... political condition of our country to convert the government to our views ... the only solution ... lies in every Indian businessman strengthening the hands of those who are fighting for the freedom of our country.'

\begin{center}*\end{center}

\paragraph*{}

However, as mentioned earlier, the Indian capitalist class had its own notions of how the anti-imperialist struggle ought to be waged. It was always in favour of not completely abandoning the constitutional path and the negotiating table and generally preferred to put its weight behind constitutional forms of struggle as opposed to mass civil disobedience. This was due to several reasons. 

First, there was the fear that mass civil disobedience, especially if it was prolonged, would unleash forces which could turn the movement revolutionary in a social sense (i.e., threaten capitalism itself). As Laiji Naranji wrote to Purshottamdas in March 1930, `private property,' itself could be threatened and the `disregard for authority' created could have `disastrous after effects' even for the `future government of Swaraj.' Whenever the movement was seen to be getting too dangerous in this sense, the capitalists tried their best to bring the movement back to a phase of constitutional opposition. 

Second, the capitalists were unwilling to support a prolonged all-out hostility to the government of the day as it prevented the continuing of day-to-day business and threatened the very existence of the class. 

Further, the Indian capitalists' support to constitutional participation, whether it be in assemblies, conferences or even joining the Viceroy's Executive Council, is not to be understood simply as their getting co-opted into the imperial system or surrendering to it. They saw all this as a forum for maintaining an effective opposition fearing that boycotting these forums completely would help `black legs' and elements who did not represent the nation to, without any opposition, easily pass measures which could severely affect the Indian economy and the capitalist class. However, there was no question of unconditionally accepting reforms or participating in conferences or assemblies. The capitalists were to `participate on (their) own terms,' with `no compromise on fundamentals,' firmly rejecting offers of cooperation which fell below their own and the minimum national demands.' It was on this ground that the FICCI in 1934 rejected the `Report of the Joint Parliamentary Committee on Constitutional Reforms for India' as `even more reactionary than the proposals contained in the White paper.'' 

Further, however keen the capitalists may have been to keep constitutional avenues open, they clearly recognized the futility of entering councils, etc., `unless,' as N.R. Sarkar, the President of FICCI, noted in 1934, `the nation also decides to enter tliem.'' They also generally refused to negotiate with the British Government, and certainly to make any final commitments, on constitutional as well as economic issues, behind the back of the Congress, i.e., without its participation or at least approval. In 1930, the FICCI (in sharp contrast to the Liberals) advised its members to boycott the Round Table Conference (RTC) stating that `... no conference ... convened for the purpose of discussing the problem of Indian constitutional advance can come to a solution ... unless such a conference is attended by Mahatma Gandhi, as a free man, or has at least his aproval.''3 This was partially because the capitalists did not want India to present a divided front at the RTC and because they knew only the Congress could actually deliver the goods. As Ambalal Sarabhai put it in November 1929, `Minus the support of the Congress, the government will not listen to you.'' 

Finally, it must be noted that for the capitalist class constitutionalism was not an end in itself, neither did it subscribe to what has often been called `gradualism,' in which case it would have joined hands with the Liberals and not supported the Congress which repeatedly went in for non- constitutional struggle including mass civil disobedience. The capitalist class itself did not rule out other forms of struggle, seeing constitutional participation as only a step towards the goal, to achieve which other steps could be necessary. For example, GD. Birla, who had worked hard for a compromise leading to the Congress accepting office in 1937, warned Lord Halifax and Lord Lothian that the `Congress was not coming in just to work the constitution, but to advance towards their goal,' and if the `Governors and the Services' did not play `the game' or `in case there was no (constitutional) advance after two or three years, then India would be compelled to take direct action,' by which he meant `non-violent mass civil disobedience.''

\begin{center}*\end{center}

\paragraph*{}

This brings us to the Indian capitalists' attitude towards mass civil disobedience, which was very complex. While, on the one hand, they were afraid of protracted mass civil disobedience, on the other hand, they clearly saw the utility, even necessity of civil disobedience in getting crucial concessions for their class and the nation. In January 1931, commenting on the existing Civil Disobedience Movement,. G.D. Birla wrote to Purshottamdas, `There could be no doubt that what we are being offered at present is entirely due to Gandhiji ... if we are to achieve what we desire, the present movement should not be allowed to slacken.'' 

When, after the mass movement had gone on for considerable time, the capitalists, for reasons discussed above, sought the withdrawal of the movement and a compromise (often mediating between the Government and Congress to secure peace), they were quite clear that this was to be only after extracting definite concessions, using the movement, or a threat to re-launch it, to bargain. In their `anxiety for peace,' they were not to surrender or `reduce (their) demands.'' The dual objective of achieving conciliation without weakening the national movement, which after all secured the concessions, was aptly described by G.D. Birla in January 1931: `We should ... have two objects in view: one is that we should jump in at the most opportune time to try for a conciliation and the other is that we should not do anything which might weaken the hands of those through whose efforts we have arrived at this stage.'' 

Further, however opposed the capitalist class may have been at a point of time to mass civil disobedience, it never supported the colonial Government in repressing it. In fact, the capitalists throughout pressurized the Government to stop repression, remove the ban on the Congress and the press, release political prisoners and stop arbitrary rule with ordinances as a first step to any settlement, even when the Congress was at the pitch of its non-constitutional mass phase. The fear of Congress militancy or radicalization did not push the capitalists (especially after the late 1920s) to either supporting imperialism in repressing it or even openly condemning or dissociating themselves from the Congress. 

The Indian capitalists' attitude had undergone significant changes on this issue over time. During the Swadeshi Movement (1905--08), the capitalists remained opposed to the boycott agitation. Even during the Non-Cooperation Movement of the early `20s, a small section of the capitalists, including Purshottamdas, openly declared themselves enemies of the Non- Cooperation Movement. However, during the I 930s' Civil Disobedience Movement, the capitalists largely supported the movement and refused to respond to the Viceroy's exhortations (in September 1930) to publicly repudiate the Congress stand and his offer of full guarantee of government protection against any harassment for doing so.'9 In September 1940, Purshottamdas felt that, given the political stance of the British, the Congress was `left with no other alternative than to launch non-cooperation.'20 On 5 August 1942, four days before the launching of the Quit India Movement, Purshottamdas, J.R. D. Tata and G.D. Birla wrote to the Viceroy that the only solution to the present crisis, the successful execution of the war and the prevention of another civil disobedience movement was `granting political freedom to the country ... even during the midst of war.'

\begin{center}*\end{center}

\paragraph*{}

It must be emphasized at this stage that though, by the late 1920s, the dominant section of the Indian capitalist class began to support the Congress, the Indian national movement was not created, led or in any decisive way influenced by this class, nor was it in any sense crucially dependent on its support. In fact, it was the capitalist class which reacted to the existing autonomous national movement by constantly trying to evolve a strategy towards it. Further, while the capitalist class on the whole stayed within the nationalist camp (as opposed to lining up with the loyalists), it did so on the most conservative end of the nationalist spectrum, which certainly did not call the shots of the national movement at any stage. 

However, the relative autonomy of the Indian national movement has been repeatedly not recognized, and it has been argued that the capitalists, mainly by using the funds at their command, were able to pressurize the Congress into making demands such as a lower Rupee-Sterling ratio, tariff protection, reduction in military expenditure, etc., which allegedly suited only their class? Further, it is argued that the capitalists were able to exercise a decisive influence over the political course followed by the Congress, even to the extent of deciding whether a movement was to be launched, continued or withdrawn. The examples quoted are of the withdrawal of civil disobedience in 1931 with the Gandhi-Irwin Pact and the non-launching of another movement between 1945--47. These formulations do not reflect the reality and this for several reasons. First, a programme of economic nationalism vis- a-vis imperialism, with demands for protection, fiscal and monetary autonomy, and the like, did not represent the interest of the capitalist class alone, it represented the demands of the entire nation which was subject to imperialist exploitation. Even the leftists --- Nehru, Socialists, and Communists --- in their struggle against imperialism had to and did fight for these demands. 

Second, the detailed working out of the doctrine of economic nationalism was done by the early nationalism nearly haifa century before the Indian capitalists got constituted as a class and entered the political arena and began fighting for these demands. So there was no question of the Congress being bought, manipulated or pressurized into these positions by the capitalist class. 

Third, while it is true that the Congress needed and accepted funds from the business community, especially during constitutional (election) phases. there is no evidence to suggest that through these funds the businessmen were able to, in any basic way, influence the party's policy and ideology along lines which were not acceptable to it independently. Even the Congress dependence on funds (in the days when it was a popular movement) has been grossly exaggerated. The Director of the Intelligence Bureau, in reply to a query from the Viceroy, noted in March 1939, `Congress has also very important substitutes for regular finance. The ``appeal to patriotism'' saves a lot of cash expenditures ... Both for normal Congress activities and for election purposes, the moneybags (capitalists) are less important than the Gandhian superstition ... local Congress organizations can command so much support from the public that they are in a position to fight elections without needing much money. `In non-election phases, an overwhelming majority of Congress cadres maintained themselves on their own and carried on day-to-day agitations with funds raised through membership fees and small donations. 

Gandhiji's position on capitalist support is very revealing in this context. As early as 1922, while welcoming and even appealing for support from merchants and mill owners; he simultaneously maintained that, `whether they do so or not, the country's march to freedom cannot be made to depend on any corporation or groups of men. This is a mass manifestation. The masses are moving rapidly towards deliverance and they must move whether with the aid of the organized capital or without. This must therefore be a movement independent of capital and yet not antagonistic to it. Only if capital came to the aid of the masses, it would redound to the credit of the capitalists and hasten the advent of the happy day.'2S (Gandhiji's attitude towards the capitalists was to harden further over time, especially during World War II when a large number of them were busy profiteering while the national movement was facing untold repression and the people shortages and famines). Lastly, as for the capitalists' determining the course of the Congress- led movements (many of them in specific areas led or supported by socialists and Communists), again there is little evidence to support this view. The Congress launched or withdrew movements based on its own strategic perceptions arising out of its understanding of the nature of the colonial state and its current postures, the organizational, political and ideological preparedness of the people, the staying power of the masses, especially when faced with repression, and so on. It did not do so at the behest, and not even on behalf of the capitalist class. In fact, almost each time the Congress launched mass movements, e.g., in 1905--08, 1920--22, 1930, 1932 and 1942, it did so without the approval of either the capitalist class as a whole or a significant section of it. However, once the movements were launched, the capitalist class reacted to it in a complex and progressively changing fashion, as discussed above. 

Quite significantly, the Indian capitalists never saw the Congress as their class party or even as a party susceptible only to their influence. On the contrary, they saw the Congress as an open-ended organization, heading a popular movement, and in the words of J.K. Mehta, Secretary, Indian Merchants' Chamber, with `room in it for men of all shades of political opinion and economic views,' and therefore, open to be transformed in either the Left or the Right direction.

\begin{center}*\end{center}

\paragraph*{}

In fact, it was precisely the increasing radicalization of the Congress in the Left direction in the 1930s, with the growing influence of Nehru, and the Socialists and Communists within the Congress, which spurred the capitalists into becoming more active in the political field. The fear of radicalization of the national movement, however, did not push the capitalists into the `lap of imperialism,' as predicted by contemporary radicals and as actually happened in some other colonial and semi-colonial countries. Instead, the Indian capitalists evolved a subtle, many- sided strategy to contain the Left, no part of which involved a sell-out to imperialism or imperial interests. 

For example, when in 1929 certain capitalists, to meet the high pitch of Communist activity among the trade unions, attempted to form a class party, where European and Indian capitalists would combine, the leaders of the capitalist class firmly quashed such a move. As G.D. Birla put it, The salvation of the capitalists does not lie in joining hands with reactionary elements' (i.e., pro-imperialist European interests in India) but in `cooperating with those who through constitutional means want to change the government for a national one' (i.e. conservative nationalists). Similarly, in 1928, the capitalists refused to support the Government in introducing the Public Safety Bill, which was intended to contain the Communists, on the ground that such a provision would be used to attack the national movement. 

Further, the capitalists were not to attempt to `kill Bolshevism and Communism with such frail weapons' as frontally attacking the Left with their class organizations which would carry no weight with `the masses' or even the `middle classes.' As Birla explained, `I have not the least doubt in my mind that a purely capitalist organization is the last body to put up an effective fight against communism.' A much superior method, he argued later (in 1936), when Nehru's leftist attitude was seen as posing a danger, was to `let those who have given up property say what you want to say.' The strategy was to `strengthen the hands' of those nationalists who, in their' ideology, did not transcend the parameters of capitalism or, preferably, even opposed socialism. 

The capitalists also realized, as G.L. Mehta, the president of FICCI, argued in 1943, that `A consistent ... programme of reforms' was the most effective remedy against social upheavals.' It was with this reform perspective that the `Post War Economic Development Committee,' set up by the capitalists in 1942, which eventually drafted the Bombay Plan, was to function. Its attempt was to incorporate `whatever is sound and feasible in the socialist movement' and see `how far socialist demands could be accommodated without capitalism surrendering any of its essential features.' The Bombay Plan, therefore, seriously took up the question of rapid economic growth and equitable distribution, even arguing for the necessity of partial nationalization, the public sector, land reform and a series of workers' welfare schemes. One may add that the basic assumption made by the Bombay planners was that the plan could be implemented only by an independent national Government. 

Clearly the Indian capitalist class was anti-socialist and bourgeois but it was not pro-imperialist The maturity of the Indian capitalist class in identifying its long term interests, correctly understanding the nature of the Congress and its relationship with the different Classes in Indian society, its refusal to abandon the side of Indian nationalism even when threatened by the Left or tempted by imperialism, its ability to project its own class interests as societal interests, are some of the reasons (apart from the failure of the Left in several of the above directions) which explains why, on the whole, the Indian national movement remained, till independence under bourgeois ideological hegemony, despite strong contending trends within it.
\end{multicols}