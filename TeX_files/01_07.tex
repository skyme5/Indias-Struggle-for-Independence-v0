\chapter{An Economic Critique of Colonialism}
\begin{multicols}{2}

Of all the national movements in colonial countries, the Indian national movement was the most deeply and firmly rooted in an understanding of the nature and character of colonial economic domination and exploitation. Its early leaders, known as Moderates, were the first in the 19th century to develop an economic critique of colonialism. This critique was, also, perhaps their most important contribution to the development of the national movement in India --- and the themes built around it were later popularized on a massive scale and formed the very pith and marrow of the nationalist agitation through popular lectures, pamphlets, newspapers, dramas, songs, and prabhat pheries.

Indian intellectuals of the first half of the 19th century had adopted a positive attitude towards British rule in the hope that Britain, the most advanced nation of the time, would help modernize India. In the economic realm, Britain, the emerging industrial giant of the world, was expected to develop India's productive forces through the introduction of modern sciences and technology and capitalist economic organization. It is not that the early Indian nationalists were unaware of the many political, psychological and economic disabilities of foreign domination, but they still supported colonial rule as they expected it to rebuild India as a spit image of the Western metropolis.

The process of disillusionment set in gradually after 1860 as the reality of social development in India failed to conform to their hopes. They began to notice that while progress in new directions was slow and halting; overall the country was regressing and under-developing. Gradually, their image of British rule began to take on darker hues; and they began to probe deeper into the reality of British rule and its impact on India.

Three names stand out among the large number of Indians who initiated and carried out the economic analysis of British rule during the years 1870--1905. The tallest of the three was Dadabhai Naoroji, known in the pre-Gandhian era as the Grand Old Man of India. Born in 1825, he became a successful businessman but devoted his entire life and wealth to the creation of a national movement in India. His near contemporary Justice Mahadev Govind Ranade, taught an entire generation of Indians the value of modem industrial development. Romesh Chandra Dutt, a retired ICS officer, published The Economic History of India at the beginning of the 20th century in which he examined in minute detail the entire economic record of colonial rule since 1757.

These three leaders along with G.V. Joshi, G. Subramaniya lyer, G.K. Gokhale, Prithwis Chandra Ray and hundreds of other political workers and journalists analysed every aspect of the economy and subjected the entire range of economic issues and colonial economic policies to minute scrutiny. They raised basic questions regarding the nature and purpose of British rule. Eventually, they were able to trace the process of the colonialization of the Indian economy and conclude that colonialism was the main obstacle to India's economic development.

They clearly understood the fact that the essence of British imperialism lay in the subordination of the Indian economy to the British economy. They delineated the colonial structure in all its three aspects of domination through trade, industry and finance. They were able to see that colonialism no longer functioned through the crude tools of plunder and tribute and mercantilisin but operated through the more disguised and complex mechanism of free trade and foreign capital investment. The essence of 19th century colonialism, they said, lay in the transformation of India into a supplier of food stuffs and raw materials to the metropolis, a market for the metropolitan manufacturers, and a field for the investment of British capital.

The early Indian national leaders were simultaneously learners and teachers. They organized powerful intellectual agitations against nearly all the important official economic policies. They used these agitations to both understand and to explain to others the basis of these policies in the colonial structure. They advocated the severance of India's economic subservience to Britain in every sphere of life and agitated for an alternative path of development which would lead to an independent economy. An important feature of this agitation was the use of bold, hard-hitting and colorful language.

\begin{center}*\end{center}

\paragraph*{}

The nationalist economic agitation started with the assertion that Indians were poor and were growing poorer every day. Dadabhai Naoroji made poverty his special subject and spent his entire life awakening the Indian and British public to the `continuous impoverishment and exhaustion of the country' and `the wretched, heart-rending, blood-boiling condition of India.' Day after day he declaimed from public platforms and in the Press that the Indian `is starving, he is dying off at the slightest touch, living on insufficient food.'

The early nationalists did not see this all-encompassing poverty as inherent and unavoidable, a visitation from God or nature. It was seen as man-made and, therefore, capable of being explained and removed. As R.C. Dutt put it: `If India is poor today, it is through the operation of economic causes.' In the course of their search for the causes of India's poverty, the nationalists underlined factors and forces which had been brought into play by the colonial rulers and the colonial structure.

The problem of poverty was, moreover, seen as the problem of increasing of the `productive capacity and energy' of the people, in other words as the problem of national development. This approach made poverty a broad national issue and helped to unite, instead of divide, different regions and sections of Indian society.

Economic development was seen above all as the rapid development of modern industry. The early nationalists accepted with remarkable unanimity that the complete economic transformation of the country on the basis of modem technology and capitalist enterprise was the primary goal of all their economic policies. Industrialism, it was further believed, represented, to quote G.V. Joshi, `a superior type and a higher stage of civilization;' or, in the words of Ranade, factories could `far more effectively than Schools and Colleges give a new birth to the activities of the Nation.' Modem industry was also seen as a major force which could help unite the diverse peoples of India into a single national entity having common interests. Surendranath Banerjee's newspaper the Bengalee made the point on 18 January 1902: `The agitation for political rights may bind the various nationalities of India together for a time. The community of interests may cease when these rights are achieved. But the commercial union of the various Indian nationalities, once established, will never cease to exist. Commercial and industrial activity is, therefore, a bond of very strong union and is, therefore, a mighty factor in the formation of a great Indian union.'

Consequently, because of their whole-hearted devotion to the cause of industrialization, the early nationalists looked upon all other issues such as foreign trade, railways, tariffs, currency and exchange, finance, and labour legislation in relation to this paramount aspect.

\begin{center}*\end{center}

\paragraph*{}

At the same time, nearly all the early nationalists were clear on one question: However great the need of India for industrialization, it had to be based on Indian capital and not foreign capital. Ever since the 1840s, British economists, statesman and officials had seen the investment of foreign capital, along with law and order, as the major instrument for the development of India. John Stuart Mill and Alfred Marshall had put forward this view in their economic treatises. In 1899, Lord Curzon, the Viceroy, said that foreign capital was `a sine qua non to the national advancement' of India.

The early nationalists disagreed vehemently with this view. They saw foreign capital as an unmitigated evil which did not develop a country but exploited and impoverished it. Or, as Dadabhai Naoroji popularly put it, foreign capital represented the `despoilation' and `exploitation' of Indian resources. Similarly, the editor of the Hindustan Review and Kayastha Samachar described the use of foreign capital as `a system of international depradation. `

They further argued that instead of encouraging and augmenting Indian capital foreign capital replaced and suppressed it, led to the drain of capital from India and further strengthened the British hold over the Indian economy. To try to develop a country through foreign capital, they said, was to barter the entire future for the petty gains of today. Bipin Chandra Pal summed up the nationalist point of view in 1901 as follows: `The introduction of foreign, and mostly British, capital for working out the natural resources of the Country, instead of being a help, is, in fact, the greatest of hindrances to all real improvements in the economic condition of the people. It is as much a political, as it is an economic danger. And the future of New India absolutely depends upon as early and radical remedy of this two-edged evil.'

In essence, the early nationalists asserted that genuine economic development was possible only if Indian capital itself initiated and developed the process of industrialization. Foreign capital would neither undertake nor could it fulfill this task.

According to the early nationalists, the political consequences of foreign capital investment were no less harmful for the penetration of a country by foreign capital inevitably led to its political subjugation. Foreign capital investment created vested interests which demanded security for investors and, therefore, pert foreign rule. `Where foreign capital has been sunk in a country,' wrote the Hindu in its issue dated 23 September 1889, `the administration of that country becomes at once the concern of the bondholders.' It added: `(if) the influence of foreign capitalists in the land is allowed to increase, then adieu to all chances of success of the Indian National Congress whose voice will be drowned in the tremendous uproar of ``the empire in danger'' that will surely be raised by the foreign capitalists.'

\begin{center}*\end{center}

\paragraph*{}

A major problem the early nationalists highlighted was that of the progressive decline and ruin of India's traditional handicrafts. Nor was this industrial prostration accidental they said. It was the result of the deliberate policy of stamping out Indian industries in the interests of British manufacturers.

The British administrators, on the other hand, pointed with pride to the rapid growth of India's foreign trade and the rapid construction of railways as instruments of India's development as well as proof of its growing prosperity. However, the nationalists said that because of their negative impact on indigenous industries, foreign trade and railways represented not economic development but colonialization and Under-development of the economy. What mattered in the case of foreign trade, they maintained, was not its volume but its pattern or the nature of goods internationally exchanged and their impact on national industry and agriculture. And this pattern had undergone drastic changes during the 19th Century, the bias being overwhelmingly towards the export of raw materials and the import of manufactured goods.

Similarly, the early nationalists pointed out that the railways had not been coordinated with India's industrial needs. They had therefore, ushered in a commercial and not an industrial revolution which enabled imported foreign goods to undersell domestic industrial products. Moreover, they said that the benefits of railway construction in terms of encouragement to the steel and machine industry and to capital investment --- what today we would call backward and forward linkages --- had been reaped by Britain and not India. In fact, remarked G.V. Joshi, expenditure on railways should be seen as Indian subsidy to British industries.' Or, as Tilak put it, it was like `decorating another's wife.''

According to the early nationalists, a major obstacle to rapid industrial development was the policy of free trade which was, on the one hand, ruining India's handicraft industries and, on the other, forcing the infant and underdeveloped modem industries into a premature and unequal and, hence, unfair and disastrous competition with the highly organized and developed industries of the West. The tariff policy of the Government convinced the nationalists that British economic policies in India were basically guided by the interests of the British capitalist class. The early nationalists strongly criticized the colonial pattern of finance. Taxes were so raised, they averred, as to overburden the poor while letting the rich, especially the foreign capitalists and bureaucrats, go scot-free. To vitiate this, they demanded the reduction of land revenue and abolition of the salt tax and supported the imposition of income tax and import duties on products which the rich and the middle classes consumed.

On the expenditure side, they pointed out that the emphasis was on serving Britain's imperial needs while the developmental and welfare departments were starved. In particular, they condemned the high expenditure on the army which was used by the British to conquer and maintain imperialist control over large parts of Asia and Africa.

\begin{center}*\end{center}

\paragraph*{}

The focal point of the nationalist critique of colonialism was the drain theory. The nationalist leaders pointed out that a large part of India's capital and wealth was being transferred or `drained' to Britain in the form of salaries and pensions of British civil and military officials working in India, interest on loans taken by the Indian Government, profits of British capitalists in India, and the Home Charges or expenses of the Indian Government in Britain.

The drain took the form of an excess of exports over imports for which India got no economic or material return. According to the nationalist calculations, this drain amount to one-half of government revenues, more than the entire land revenue collection and over one-third of India's total savings. (In today's terms this would amount to 8\% of India's national income).

The acknowledged high-priest of the drain theory was Dadabhai Naoroji. It was in May 1867 that Dadabhai Naoroji put forward the idea that Britain was draining and `bleeding' India. From then on for nearly half a century he launched a raging campaign against the drain, hammering at the theme through every possible form of public communication.

The drain, he declared, was the basic cause of India's poverty and the fundamental evil of British rule in India. Thus, he argued in 1880: it is not the pitiless operations of economic laws, but it is the thoughtless and pitiless action of the British policy; it is the pitiless eating of India's substance in India, and the further pitiless drain to England; in short, it is the pitiless perversion of economic laws by the sad bleeding to which India is subjected, that is destroying India.'

Other nationalist leaders, journalists and propagandists followed in the foot-steps of Dadabhai Naoroji. R.C. Dutt, for example, made the drain the major theme of his Economic History of India. He protested that `taxation raised by a king, says the Indian poet, is like the moisture sucked up by the sun, to be returned to the earth as fertilizing rain; but the moisture raised from the Indian soil now descends as fertilizing rain largely on other lands, not on India ... So great an Economic Drain out of the resources of a land would impoverish the most prosperous countries on earth; it has reduced India to a land of famines more frequent, more widespread, and more fatal, than any known before in the history of India, or of the world.'

The drain theory incorporated all the threads of the nationalist critique of Colonialism, for the drain denuded India of the productive capital its agriculture and industries so desperately needed. Indeed, the drain theory was the high water-mark of the nationalist leaders' comprehensive, interrelated and integrated economic analysis of the colonial situation. Through the drain theory, the exploitative character of British rule could be made Visible. By attacking the drain, the nationalists were able to call into question in an uncompromising manner, the economic essence of imperialism.

Moreover, the drain theory possessed the great political merit of being easily grasped by a nation of peasants. Money being transferred from one country to another was the most easily understood of the theories of economic exploitation, for the peasant daily underwent this experience vis-a-vis the state, landlords, moneylenders, lawyers and priests. No other idea could arouse people more than the thought that they were being taxed so that others in far off lands might live in comfort. `No drain' was the type of slogan that all successful movements need --- it did not have to be proved by sophisticated and complex arguments. It had a sort of immanent quality about it; it was practically self-evident. Nor could the foreign rulers do anything to appease the people on this question. Modem colonialism was inseparable from the drain. The contradiction between the Indian people and British imperialism was seen by people to be insoluble except by the overthrow of British rule. It was, therefore, inevitable that the drain theory became the main staple of nationalist political agitation during the Gandhian era.

\begin{center}*\end{center}

\paragraph*{}

This agitation on economic issues contributed to the undermining of the ideological hegemony of the alien rulers over Indian minds, that is, of the foundations of colonial rule in the minds of the people. Any regime is politically secure only so long as the people have a basic faith in its moral purpose, in its benevolent character --- that is, they believe that the rulers are basically motivated by the desire to work for their welfare. It is this belief which leads them to support the regime or to at least acquiesce in its continuation. It provides legitimacy to a regime in this belief lie its moral foundations.

The secret of British power in India lay not only in physical force but also in moral force, that is; in the belief sedulously inculcated by the rulers for over a century that the British were the Mai-Baap of the common people of India --- the first lesson in primary school language textbooks was most often on `the benefits of British rule.' The nationalist economic agitation gradually undermined these moral foundations. It corroded popular confidence in the benevolent character of British rule --- in its good results as well as its good intentions.

The economic development of India was offered as the chief justification for British rule by the imperialist rulers and spokesmen. The Indian nationalists controverted it forcefully and asserted that India was economically backward precisely because the British were ruling it in the interests of British trade, industry and capital, and that poverty and backwardness were the inevitable Consequences of colonial rule. Tilak's newspaper, the Kesari, for example, wrote on 28 January 1896: `Surely India is treated as a vast pasture for the Europeans to feed upon.' And

P. Ananda Charlu, an ex-President of the Congress, said in the Legislative Council: `While India is safe-guarded against foreign inroads by the strong arm of the British power, she is defenceless in matters where the English and Indian interests clash and where (as a Tamil saying puts it) the very fence begins to feed on the crop.'

The young intellectual from Bihar, Sachidanand Sinha, summed up the Indian critique in a pithy manner in Indian People on 27 February 1903: `Their work of administration in Lord Curzon's testimony is only the handmaid to the task of exploitation. Trade cannot thrive without efficient administration, while the latter is not worth attending to in the absence of profits of the former. So always with the assent and often to the dictates of the Chamber of Commerce, the Government of India is carried on, and this is the ``White Man's Burden.'''

It was above all Dadabhai Naoroji who in his almost daily articles and speeches hammered home this point. `The face of beneficence,' he said, was a mask behind which the exploitation of the country was carried on by the British though `unaccompanied with any open compulsion or violence to person or property which the world can see and be horrified with.' And, again: `Under the present evil and unrighteous administration of Indian expenditure, the romance is the beneficence of the British Rule, the reality is the ``bleeding'' of the British Rule.'' Regarding the British claim of having provided security of life and property, Dadabhai wrote: `The romance is that there is security of life and property in India; the reality is that there is no such thing. There is security of life and property in one sense or way, i.e., the people are secure from any violence from each other or from Native despots ... But from England's own grasp there is no security of property at all, and, as a consequence, no security for life… What is secure, and well secure, is that England is perfectly safe and secure… to carry away from India, and to eat up in India, her property at the present rate of 30,000,000 or 40,000,000 \pounds a year ... To millions in India life is simply ``half-feeding,'' or starvation, or famine and disease `. With regard to the benefits of law and order, Dadabhai said: `There is an Indian saying: ``Pray strike on the back, but don't strike on the belly.''' Under the `native despot the people keep and enjoy what they produce, though at times they suffer some violence on the back. Under the British Indian despot the man is at peace, there is no violence; his substance is drained away, unseen, peaceably and subtly --- he starves in peace, and peaceably perishes in peace, with law and order.

\begin{center}*\end{center}

\paragraph*{}

The corrosion of faith in British rule inevitably spread to the political field. In the course of their economic agitation, the nationalist leaders linked nearly every important economic question with the politically subordinated status of the country. Step by step, issue by issue, they began to draw the conclusion that since the British Indian administration was `only the handmaid to the task of exploitation,' pro-Indian and developmental policies would be followed only by a regime in which Indians had control over political power.

The result was that even though most of the early nationalist leaders were moderate in politics and political methods, and many of them still professed loyalty to British rule, they cut at the political roots of the empire and sowed in the land the seeds of disaffection and disloyalty and even sedition. This was one of the major reasons why the period 1875--1905 became a period of intellectual unrest and of spreading national consciousness --- the seed-time of the modem Indian national movement.

While until the end of the 19th century, Indian nationalists confined their political demands to a share in political power and control over the purse, by 1905 most of the prominent nationalists were putting forward the demand for some form of self-government. Here again, Dadabhai Naoroji was the most advanced. Speaking on the drain at the International Socialist Congress in 1904, he put forward the demand for `self-government' and treatment of India `like other British Colonies.'' A year later in 1905, in a message to the Benares session of the Indian National Congress, Dadabhai categorically asserted: `Self-government is the only remedy for India's woes and wrongs.' And, then, as the President of the 1906 session of the Congress at Calcutta, he laid down the goal of the national movement as ``self-government or Swaraj,'' like that of the United Kingdom or the Colonies.'

While minds were being prepared and the goal formed, the mass struggle for the political emancipation of the country was still in the womb of time. But the early nationalists were laying Strong and enduring foundations for the national movement to grow upon. They sowed the seeds of nationalism well and deep. They did not base their nationalism primarily on appeals to abstract or shallow Sentiments or on obscurantist appeals to the past. They rooted their nationalism in a brilliant scientific analysis of the complex economic mechanism of modern colonialism and of the chief contradiction between the interests of the Indian people and British rule.

The nationalists of the 20th century were to rely heavily on the main themes of their economic critique of colonialism. These themes were then to reverberate in Indian cities, towns and villages, carried there by the youthful agitators of the Gandhian era. Based on this firm foundation, the later nationalists went on to stage powerful mass agitations and mass movements. At the same time, because of this firm foundation, they would not, unlike in China, Egypt and many other colonial and semi-colonial countries, waver in their anti-imperialism.
\end{multicols}