
\chapter{Gandhiji's Early Career and Activism}

When Mohandas Karamchand Gandhi called for a nation-wide Satyagraha against the Rowlatt Act in March 1919, his first attempt at leading an all India struggle, he was already in his fiftieth year. To understand the man who was about to take over the reins of the Indian national movement and guide its destinies through its most climactic years, it is necessary to begin his story at least twenty-five years earlier, in 1893, when as a twenty-four old barrister, he began the struggle of Indians against racial discrimination in South Africa.

\begin{center}*\end{center}

\paragraph*{}


The young barrister who landed at Durban in 1893 on a one-year contract to sort out the legal problems of Dada Abdullah, a Gujarati merchant, was to all appearances an ordinary young man trying to make a living. But he was the first Indian barrister, the first highly-educated Indian, to have come to South Africa.

Indian immigration to South Africa had begun in 1890 when the White settlers recruited indentured Indian labour, mainly from South India, to work on the sugar plantations. In their wake had come Indian merchants, mostly Meman Muslims. Ex- indentured labourers, who had settled down in South Africa after the expiry of their contract, and their children, many born in South Africa itself, constituted the third group of Indians that was in South Africa prior to Gandhiji's arrival. None of these groups of Indians had much access to education and certainly very little education in English; even the wealthy merchants often knew only a smattering of English necessary to carry on their trade. The racial discrimination to which they were subjected, as part of their daily existence, they had come to accept as a way of life, and even if they resented it, they had little idea about how to challenge it.

But young Mohandas Gandhi was not used to swallowing racial insults in order to carry on with the business of making a living. He was the son of a Dewan (Minister) of an Indian state whose family, though in straitened economic circumstances, was widely respected in his native Kathiawad. Further, he had spent three years in London studying for the Bar. Neither m India nor in England had he ever come in contact with the overt racism that confronted him within days of his arrival in South Africa.

His journey from Durban to Pretoria, which he undertook within a week of his arrival on the continent, consisted of a series of racial humiliations. Apart from the famous incident in which he was bundled out of a first-class compartment by a White man and left to spend the night shivering in the waiting room, he was made to travel in the driver's box in a coach for which he had bought a first-class ticket, when he ignored the coach leader's order to vacate even that seat and sit on the foot-board, he was soundly thrashed. On reaching Johannesburg, he found that all the hotels became full up the moment he asked for a room to stay the night. Having succeeded in securing a first-class train ticket from Johannesburg to Pretoria (after quoting extensively from railway regulations), he was almost pushed out again from his railway compartment and was only saved this humiliation by the intervention of a European passenger.'

On his arrival in Pretoria, where he was to work on the civil suit that had brought him to South Africa, he immediately convened a meeting of the Indians there. He offered to teach English to anybody who wanted to learn and suggested that they organize themselves and protest against oppression. He voiced his protest through the Press as well. In an indignant letter to the Natal Advertiser, he asked: `Is this Christian-like, is this fair play, is this justice, is this civilization? I pause for a reply.' Even though he had no plans of staying in South Africa at that stage, he tried his best to arouse the Indians in Pretoria to a sense of their own dignity as human beings and persuade them to resist all types of racial disabilities.

Having settled the law suit for which he had come, Gandhiji prepared to leave for India. But on the eve of his departure from Durban, he raised the issue of the bill to disenfranchise Indians which was in the process of being passed by the Natal legislature. The Indians in South Africa begged Gandhiji to stay on for a month and organize their protest as they could not do so on their own, not knowing even enough English to draft petitions, and so on. Gandhiji agreed to stay on for a month and stayed for twenty years. He was then only twenty-five; when he left, he was forty- five.

Gandhiji's experience in South Africa was unique in one respect. By virtue of being a British-educated barrister, he demanded many things as a matter of right, such as first-class train tickets and rooms in hotels, which other Indians before him had never probably even had the courage to ask for. Perhaps, they believed that they were discriminated against because they were not `civilized,' that is, `westernized.' Gandhiji's experience, the first of a westernized Indian in South Africa, demonstrated clearly, to him and to them, that the real cause lay elsewhere, in the assumption of racial superiority by the White rulers.

His uniqueness in being the only western-educated Indian also simultaneously placed on his shoulders the responsibility of leading the struggle of the Indians against increasing racial discrimination. Wealthy Indian merchants, senior to the twenty-five-year-old barrister in experience and age, appointed him as their leader because he was the only one who could speak to the rulers in their own language, the only one who understood the intricacies of their laws and their system of government, the only one who could draft their petitions, create their organizations, and represent them before their rulers.

\begin{center}*\end{center}

\paragraph*{}


The story of Gandhiji in South Africa is a long one and we present it here in its briefest outline only to highlight the wide experience that Gandhiji had undergone before he came back to India.

Gandhiji's political activities from 1894 to 1906 may be classified as the `Moderate' phase of the struggle of the South African Indians. During this phase, he concentrated on petitioning and sending memorials to the South African legislatures, the Colonial Secretary in London and the British Parliament. He believed that if all the facts of the case were presented to the Imperial Government, the British sense of justice and fair play would be aroused and the Imperial Government would intervene on behalf of Indians who were, after all, British subjects. His attempt was to unite the different sections of Indians, and to give their demands wide publicity. This he tried to do through the setting up of the Natal Indian Congress and by starting a paper called Indian Opinion. Gandhiji's abilities as an organizer, as a fund-raiser, as a journalist and as a propagandist, all came to the fore during this period. But, by 1906, Gandhiji, having fully tried the `Moderate' methods of struggle, was becoming convinced that these would not lead anywhere.

The second phase of the struggle in South Africa, which began in 1906, was characterized by the use of the method of passive resistance or civil disobedience, which Gandhiji named Satyagraha. It was first used when the Government enacted legislation making it compulsory for Indians to take out certificates of registration which held their finger prints. It was essential to carry these on person at all times. At a huge public meeting held on 11 September, 1906, in the Empire Theatre in Johannesburg, Indians resolved that they would refuse to submit to this law and would face the consequences. The Government remained adamant, and so did the Indians. Gandhiji formed the Passive Resistance Association to conduct the campaign. The last date for registration being over, the Government started proceedings against Gandhiji and twenty-six others. The passive resisters pleaded guilty, were ordered to leave the country and, on refusing to do so, were sent to jail. Others followed, and their numbers swelled to 155. The fear of jail had disappeared, and it was popularly called King Edward's Hotel.

General Smuts called Gandhiji for talks, and promised to withdraw the legislation if Indians voluntarily agreed to register themselves. Gandhiji accepted and was the first to register. But Smuts had played a trick; he ordered that the voluntary registrations be ratified under the law. The Indians under the leadership of Gandhiji retaliated by publicly burning their registration certificates.

Meanwhile, the Government brought in new legislation, this time to restrict Indian immigration. The campaign, widened to oppose this. In August 1908, a number of prominent Indians from Natal crossed the frontier into Transvaal to defy the new immigration laws and were arrested. Other Indians from Transvaal opposed the laws by hawking without a license; traders who had Licenses refused to produce them. All of them were jailed. Gandhiji himself landed in jail in October 1908 and, along with the other Indians, was sentenced to a prison term involving hard physical labour and miserable conditions. But imprisonment failed to crush the spirit of the resisters, and the Government resorted to deportation to India, especially of the poorer Indians. Merchants were pressurized by threats to their economic interests.

At this stage, the movement reached an impasse. The more committed Satyagrahis continued to go in and out of jail, but the majority were showing signs of fatigue. The struggle was obviously going to be a protracted one, and the Government was in no mood to relent. Gandhiji's visit to London in 1909 to meet the authorities there yielded little result. The funds for supporting the families of the Satyagrahis and for running Indian Opinion were fast running out. Gandhiji's own legal practice had virtually ceased since 1906, the year he had started devoting all his attention to the struggle. At this point, Gandhiji set up Tolstoy Farm, made possible through the generosity of his German architect friend, Kallenbach, to house the families of the Satyagrahis and give them a way to sustain themselves. Tolstoy Farm was the precursor of the later Gandhian ashrams that were to play so important a role in the Indian national movement. Funds also came from India — Sir Ratan Tata sent Rs. 25,000 and the Congress and the Muslim League, as well as the Nizam of Hyderabad, made their contributions.

In 1911, to coincide with the coronation of King George V, an agreement was reached between the Government and the Indians which, however, lasted only till the end of 1912. Meanwhile, Gokhale paid a visit to South Africa, was treated as a guest of the Government and was made a promise that all discriminatory laws against Indians would be removed. The promise was never kept, and Satyagraha was resumed in 1913. This time the movement was widened further to include resistance to the poll tax of three pounds that was imposed on all ex-indentured Indians. The inclusion of the demand for the abolition of this tax, a particularly heavy charge on poor labourers whose wages hardly averaged ten shillings a month, immediately drew the indentured and ex-indentured labourers into the struggle, and Satyagraha could now take on a truly mass character. Further fuel was added to the already raging fire by a judgement of the Supreme Court which invalidated all marriages not conducted according to Christian rites and registered by the Registrar of Marriages. By implication, Hindu, Muslim and Parsi marriages were illegal and the children born through these marriages illegitimate. The Indians treated this judgment as an insult to the honor of their women and many women were drawn into the movement because of this indignity.

Gandhiji decided that the time had now come for the final struggle into which all the resisters' resources should be channelled. The campaign was launched by the illegal crossing of the border by a group of sixteen Satyagrahis, including Kasturba, Gandhiji's wife, who marched from Phoenix Settlement in Natal to Transvaal, and were immediately arrested. A group of eleven women then marched from Tolstoy Farm in Transvaal and crossed the border into Natal without a permit, and reached New Castle, a mining town. Here, they talked to the Indian mine workers, mostly Tamils, and before being arrested persuaded them to go on strike.

Gandhiji reached New Castle and took charge of the agitation. The employers retaliated by cutting off water and electricity to the workers' quarters, thus forcing them to leave their homes. Gandhiji decided to march this army of over two thousand men, women and children over the border and thus see them lodged in Transvaal jails. During the course of the march, Gandhiji was arrested twice, released, arrested a third time and sent to jail. The morale of the workers, however, was very high and they continued the march till they were put into trains and sent back to Natal, where they were prosecuted and sent to jail. The treatment that was meted out to these brave men and women in jail included starvation and whipping, and being forced to work in the mines by mounted military police. Gandhiji himself was made to dig stones and sweep the compound. He was kept in a dark cell, and taken to court handcuffed and manacled.

The Governments' action inflamed the entire Indian community; workers on the plantations and the mines went on a lightning strike. Gokhale toured the whole of India to arouse Indian public opinion and even the Viceroy, Lord Hardinge, condemned the repression as `one that would not be tolerated by any country that calls itself civilized' and called for an impartial enquiry into the charges of atrocities. The use of brutal force on unarmed and peaceful men and women aroused widespread indignation and condemnation.

Eventually, through a series of negotiations involving Gandhiji, the Viceroy, Lord Hardinge, C.F. Andrews and General Smuts, an agreement was reached by which the Government of South Africa conceded the major Indian demands relating to the poll tax, the registration certificates and marriages solemnized according to Indian rites, and promised to treat the question of Indian immigration in a sympathetic manner.

\begin{center}*\end{center}

\paragraph*{}


Non-violent civil disobedience had succeeded in forcing the opponents to the negotiating table and conceding the substance of the demands put forward by the movement. The blueprint for the `Gandhian' method of struggle had been evolved and Gandhiji started back for his native land. The South African `experiment' was now to be tried on a much wider scale on the Indian sub-continent.

In other respects, too, the South African experiment prepared Gandhiji for leadership of the Indian national struggle. He had had the invaluable experience of leading poor Indian labourers, of seeing their capacity for sacrifice and for bearing hardship, their morale in the face of repression. South Africa built up his faith in the capacity of the Indian masses to participate in and sacrifice for a cause that moved them. Gandhiji also had had the opportunity of leading Indians belonging to different religions: Hindus, Muslims, Christians and Parsis were all united under his leadership in South Africa. They also came from different regions, being mainly Gujaratis and Tamils. They belonged to different social classes; rich merchants combined with poor indentured labourers. Women came along with the men.

Another aspect of the South African experience also stood Gandhiji in good stead. He learnt, the hardest way, that leadership involves facing the ire not only of the enemy but also of one's followers. There were two occasions on which Gandhiji was faced with a serious threat to his life. Once, when a white mob chased him down a street in Durban in 1896 and surrounded the house where he was staying, asking for his blood; he had to be whisked out in disguise. The second, when an Indian, a Pathan, who was angry with him because of an agreement he had reached with the Government assaulted him on the street. Gandhiji learnt that leaders often have to take hard decisions that are unpopular with enthusiastic followers.

South Africa, then, provided Gandhiji with an opportunity for evolving his own style of politics and leadership, for trying out new techniques of struggle, on a limited scale, untrammelled by the opposition of contending political currents. In South Africa, he had already taken the movement from its `Moderate' phase into its `Gandhian' phase. He already knew the strengths and the weaknesses of the Gandhian method and he was convinced that it was the best method around. It now remained for him to introduce it into India.

Gandhiji returned to India, in January 1915, and was warmly welcomed. His work in South Africa was well-known, not only to educated Indians, but, as he discovered on his visit to the Kumbh Mela at Hardwar, even to the masses who flocked to him for his `darshan.' Gokhale had already hailed him as being `without doubt made of the stuff of which heroes and martyrs are made.' The veteran Indian leader noticed in Gandhiji an even more important quality: `He has in him the marvelous spiritual power to turn ordinary men around him into heroes and martyrs.'

On Gokhale's advice, and in keeping with his own style of never intervening in a situation without first studying it with great care, Gandhiji decided that for the first year he would not take a public stand on any political issue. He spent the year travelling around the country, seeing things for himself, and in organizing his ashram in Ahmedabad where he, and his devoted band of followers who had come with him from South Africa, would lead a community life. The next year as well, he continued to maintain his distance from political affairs, including the Home Rule Movement that was gathering momentum at this time. His own political understanding did not coincide with any of the political currents that were active in India then. His faith in `Moderate' methods was long eroded, nor did he agree with the Home Rulers that the best time to agitate for Home Rule was when the British were in difficulty because of the First World War.

Further, he was deeply convinced that none of these methods of political struggle were really viable; the only answer lay in Satyagraha. His reasons for not joining the existing political organizations are best explained in his own words: `At my time of life and with views firmly formed on several matters, I could only join an organization to affect its policy and not be affected by it. This does not mean that I would not now have an open mind to receive new light. I simply wish to emphasize the fact that the new light will have to be specially dazzling in order to entrance me.'' In other words, he could only join an organization or a movement that adopted non-violent Satyagraha as its method of struggle.

That did not, however, mean that Gandhiji was going to remain politically idle. During the course of 1917 and early 1918, he was involved in three significant struggles — in Champaran in Bihar, in Ahmedabad and in Kheda in Gujarat. The common feature of these struggles was that they related to specific local issues and that they were fought for the economic demands of the masses. Two of these struggles, Champaran and Kheda, involved the peasants and the one in Ahmedabad involved industrial workers.

\begin{center}*\end{center}

\paragraph*{}


The story of Champaran begins in the early nineteenth century when European planters had involved the cultivators in agreements that forced them to cultivate indigo on 3/20th of their holdings (known as the tinkathia system). Towards the end of the nineteenth century, German synthetic dyes forced indigo out of the market and the European planters of Champaran, keen to release the cultivators from the obligation of cultivating indigo, tried to turn their necessity to their advantage by securing enhancements in rent and other illegal dues as a price for the release. Resistance had surfaced in 1908 as well, but the exactions of the planters continued till Raj Kumar Shukla, a local man, decided to follow Gandhiji all over the country to persuade him to come to Champaran to investigate the problem. Raj Kumar Shukla's decision to get Gandhiji to Champaran is indicative of the image he had acquired as one who fought for the rights of the exploited and the poor.

Gandhiji, on reaching Champaran, was ordered by the Commissioner to immediately leave the district. But to the surprise of all concerned, Gandhiji refused and preferred to take the punishment for his defiance of the law. This was unusual, for even Tilak and Annie Besant, when externed from a particular province, obeyed the orders even though they organized public protests against them. To offer passive resistance or civil disobedience to an unjust order was indeed novel. The Government of India, not willing to make an issue of it and not yet used to treating Gandhiji as a rebel, ordered the local Government to retreat and allow Gandhiji to proceed with his enquiry.

A victorious Gandhiji embarked on his investigation of the peasants' grievances. Here, too, his method was striking. He and his colleagues, who now included Brij Kishore, Rajendra Prasad and other members of the Bihar intelligentsia, Mahadev Desai and Narhari Parikh, two young men from Gujarat who had thrown in their lot with Gandhiji, and J.B. Kripalani, toured the villages and from dawn to dusk recorded the statements of peasants, interrogating them to make sure that they were giving correct information.

Meanwhile, the Government appointed a Commission of Inquiry to go into the whole issue, and nominated Gandhiji as one of its members. Armed with evidence collected from 8,000 peasants, he had little difficulty in convincing the Commission that the tinkathia system needed to be abolished and that the peasants should be compensated for the illegal enhancement of their dues. As a compromise with the planters, he agreed that they refund only twenty-five per cent of the money they had taken illegally from the peasants. Answering critics who asked why he did not ask for a full refund, Gandhiji explained that even this refund had done enough damage to the planters' prestige and position. As was often the case, Gandhiji's assessment was correct and, within a decade, the planters left the district altogether.

\begin{center}*\end{center}

\paragraph*{}


Gandhiji then turned his attention to the workers of Ahmedabad. A dispute was brewing between them and the mill owners over the question of a `plague bonus' the employers wanted to withdraw once the epidemic had passed but the workers insisted it stay, since the enhancement hardly compensated for the rise in the cost of living during the War. The British Collector, who feared a showdown, asked Gandhiji to bring pressure on the mill owners and work out a compromise. Ambalal Sarabhai, one of the leading mill owners of the town, was a friend of Gandhiji, and had just saved the Sabarmati Ashram from extinction by a generous donation. Gandhiji persuaded the mill owners and the workers to agree to arbitration by a tribunal, but the mill owners, taking advantage of a stray strike, withdrew from the agreement. They offered a twenty per cent bonus and threatened to dismiss those who did not accept it.

The breach of agreement was treated by Gandhiji as a very serious affair, and he advised the workers to go on strike. He further suggested, on the basis of a thorough study of the production costs and profits of the industry as well as the cost of living, that they would be justified in demanding a thirty-five per cent increase, in wages.

The strike began and Gandhiji addressed the workers every day on the banks of the Sabarmati River. He brought out a daily news bulletin, and insisted that no violence be used against employers or blacklegs. Ambalal Sarabhai's sister, Anasuya Behn, was one of the main lieutenants of Gandhiji in this struggle in which her brother, and Gandhiji's friend, was one of the main adversaries.

After some days, the workers began to exhibit signs of weariness. The attendance at the daily meetings began to decline and the attitude towards blacklegs began to harden. In this situation, Gandhiji decided to go on a fast, to rally the workers and strengthen their resolve to continue. Also, he had promised that if the strike led to starvation he would be the first to starve, and the fast was a fulfillment of that promise. The fast, however, also had the effect of putting pressure on the mill owners and they agreed to submit the whole issue to a tribunal. The strike was withdrawn and the tribunal later awarded the thirty-five per cent increase the workers had demanded

\begin{center}*\end{center}

\paragraph*{}


The dispute in Ahmedabad had not yet ended when Gandhiji learnt that the peasants of Kheda district were in extreme distress due to a failure of crops, and that their appeals for the remission of land revenue were being ignored by the Government. Enquiries by members of the Servants of India Society, Vithalbhai Patel and Gandhiji confirmed the validity of the peasants' case. This was that as the crops were less than one-fourth of the normal yield, they were entitled under the revenue code to a total remission of the land revenue.

The Gujarat Sabha, of which Gandhiji was the President, played a leading role in the agitation. Appeals and petitions having failed, Gandhiji advised the withholding of revenue, and asked the peasants to `fight unto death against such a spirit of vindictiveness and tyranny,' and show that `it is impossible to govern men without their consent.' Vallabhbhai Patel, a young lawyer and a native of Kheda district, and other young men, including Indulal Yagnik, joined Gandhiji in touring the villages and urging the peasants to stand firm in the face of increasing Government repression which included the seizing of cattle and household goods and the attachment of standing crops. The cultivators were asked to take a solemn pledge that they would not pay; those who could afford to pay were to take a vow that they would not pay in the interests of the poorer ryots who would otherwise panic and sell off their belongings or incur deb4s in order to pay the revenue. However, if the Government agreed to suspend collection of land revenue, the ones who could afford to do so could pay the whole amount.

The peasants of Kheda, already hard pressed because of plague, high prices arid drought, were beginning to show signs of weakness when Gandhiji came to know that the Government had issued secret instructions directing that revenue should be recovered only from those peasants who could pay. A public declaration of this decision would have meant a blow to Government prestige, since this was exactly what Gandhiji had been demanding. In these circumstances, the movement was withdrawn. Gandhiji later recalled that by this time `the people were exhausted' and he was actually `casting about for some graceful way of terminating the struggle.

Champaran, Ahmedabad and Kheda served as demonstrations of Gandhiji's style and method of politics to the country at large. They also helped him find his feet among the people of India and study their problems at close quarters. He came to possess, as a result of these struggles, a surer understanding of the strengths and weaknesses of the masses, as well as of the viability of his own political style. He also earned the respect and commitment of many political workers, especially the younger ones, who were impressed by his identification with the problems of ordinary Indians, and his willingness to take up their cause.

\begin{center}*\end{center}

\paragraph*{}


It was this reservoir of goodwill, and of experience, that encouraged Gandhiji, in February 1919, to call for a nation-wide protest against the unpopular legislation that the British were threatening to introduce. Two bills, popularly known as the Rowlatt Bills after the man who chaired the Committee that suggested their introduction, aimed at severely curtailing the civil liberties of Indians in the name of curbing terrorist violence, were introduced in the Legislative Council. One of them was actually pushed through in indecent haste in the face of opposition from all the elected Indian members. This act of the Government was treated by the whole of political India as a grievous insult, especially as it came at the end of the War when substantial constitutional concessions were expected.

Constitutional protest having failed, Gandhiji stepped in and suggested that a Satyagraha be launched. A Satyagraha Sabha was formed, and the younger members of the Home Rule Leagues who were more than keen to express their disenchantment with the Government flocked to join it. The old lists of the addresses of Home Rule Leagues and their members were taken out, contacts established and propaganda begun. The form of protest finally decided upon was the observance of a nation-wide hartal (strike) accompanied by fasting and prayer. In addition, it was decided that civil disobedience would be offered against specific laws.

The sixth of April was fixed as the date on which the Satyagraha would be launched. The movement that emerged was very different from the one that had been anticipated or planned. Delhi observed the hartal on 30 March because of some confusion about dates, and there was considerable violence in the streets. This seemed to set the pattern in most other areas that responded to the call; protest was generally accompanied by violen4ce and disorder. Punjab, which was suffering from the after effects of severe war-time repression, forcible recruitment, and the ravages of disease, reacted particularly strongly and both in Amritsar and Lahore the situation became very dangerous for the Government. Gandhiji tried to go to Punjab to help quieten the people, but the Government deported him to Bombay. He found that Bombay and even his native Gujarat, Including Ahmedabad, were up in flames and he decided to stay and try and pacify the people.

Events in Punjab were moving in a particularly tragic direction. In Amritsar, the arrest of two local leaders on 10 April led to an attack on the town hail and the post office: telegraph wires were cut and Europeans including women were attacked. The army was called in and the city handed over to General Dyer, who issued an order prohibiting public meetings and assemblies. On 13 April, Baisakhi day, a large crowd of people, many of whom were visitors from neighbouring villages who had come to the town to attend the Baisakhi celebrations, collected in the Jallianwala Bagh to attend a public meeting. General Dyer, incensed that his orders were disobeyed, ordered his troops to fire upon the unarmed crowd. The shooting continued for ten minutes. General Dyer had not thought It necessary to issue any warning to the people nor was he deterred by the fact that the ground was totally hemmed in from all sides by high walls which left little chance for escape. The Government estimate was 379 dead, other estimates were considerably higher.

The brutality at Jallianwala Bagh stunned the entire nation. The response would come, not immediately, but a little later. For the moment, repression was intensified, Punjab placed under martial law and the people of Amritsar forced into indignities such as crawling on their bellies before Europeans Gandhiji, overwhelmed by the total atmosphere of violence, withdrew the movement on 18 April.

That did not mean, however, that Gandhiji had lost faith either in his non-violent Satyagraha or in the capacity of the Indian people to adopt it as a method of struggle. A year later, he launched another nation-wide struggle, on a scale bigger than that of the Rowlatt Satyagraha. The wrong Inflicted on Punjab was one of the major reasons for launching it. The Mahatma's `Indian Experiment' had begun.
