\cleardoublepage
\chapter{The Gathering Storm 1927-29}



In the years following the end of the Non-Cooperation Movement in 1922, the torch of nationalism had been kept alive by the Gandhian constructive workers who dug their roots deep into village soil, by the Swarajists who kept the Government on its toes in the legislatures, by the Koya tribals in Andhra who heroically fought the armed might of the colonial state under the leadership of Ramachandra Raju from 1922-24, by the Akalis in Punjab, by the Satyagrahis who flocked to defend the honour of the national flag in Nagpur in 1923, and countless others who engaged themselves in organizational, ideological and agitational activities at a variety of levels. 

It was, however, from the latter part of 1927 that the curve of the mass anti-imperialist upsurge began to take a marked upward turn. As with the Rowlatt Bills in 1919, it was the British Government that provided a catalyst and a rallying ground by an announcement on 8 November 1927 of an all-White commission to recommend whether India was ready for further constitutional progress and on which lines. Indian nationalists had for many years declared the constitutional reforms of 1919 as inadequate and had been clamouring for an early reconsideration of the constitutional question, but the Government had been adamant that the declared period of ten years must lapse before fresh proposals were considered. In 1927, however, the Conservative Government of Britain, faced with the prospect of electoral defeat at the hands of the Labour Party, suddenly decided that it could not leave an issue which concerned the future of the British Empire in the irresponsible hands of an inexperienced Labour Government and it was thus that the Indian Statutory Commission, popularly known as the Simon Commission after its Chairman, was appointed. 

The response in India was immediate and unanimous. That no Indian should be thought fit to serve on a body that claimed the right to decide the political future of India was an insult that no Indian of even the most moderate political opinion was willing to swallow. The call for a boycott of the Commission was endorsed by the Liberal Federation led by Tej Bahadur Sapru, by the Indian Industrial and Commercial Congress, arid by the Hindu Mahasabha the Muslim League even split on the issue, Mohammed Ali Jinnah carrying the majority with him in favour of boycott. 

It was the Indian National Congress, however, that turned the boycott into a popular movement. The Congress had resolved on the boycott at its annual session in December 1927 at Madras, and in the prevailing excitable atmosphere, Jawaharlal Nehru had even succeeded in getting passed a snap resolution declaring complete independence as the goal of the Congress. But protest could not be confined to the passing of resolutions, as Gandhiji made clear in the issue of Young India of 12 January 1928: `It is said that the Independence Resolution is a fitting answer. The act of appointment (of the Simon Commission) needs for an answer, not speeches, however heroic they may be, not declarations, however brave they may be, but corresponding action ...' 

The action began as soon as Simon and his friends landed at Bombay on 3 February 1928. That day, all the major cities and towns observed a complete hartal, and people were out on the streets participating in mass rallies, processions and black-flag demonstrations. In Madras, a major clash with the police resulted in firing and the death of one person. T. Prakasam symbolized the defiant spirit of the occasion by baring his chest before the armed policemen who tried in vain to stop him from going to the scene of the killing. Everywhere that Simon went — Calcutta, Lahore, Lucknow, Vijayawada, Poona — he was greeted by a sea of black-flags carried by thousands of people. And ever new ways of defiance were being constantly invented. The youth of Poona, for example, took advantage of the fact that for a long stretch between Lonavala and Poona the road and the rail-track ran within sight of each other. They climbed into a lorry and drove alongside the train that was carrying Simon and Company, waving black flags at them all the way from Lonavala to Poona. In Lucknow, Khaliquzzaman executed the brilliant idea of floating kites and balloons imprinted with the popular slogan `Go Back Simon' over the reception organized in Kaiserbagh by the taluqdars for members of the Commission. 

If humour and creativity was much in evidence, so too was popular anger at the manner in which the police dealt with the protesters. Lathi charges were becoming all too frequent, and even respected and senior leaders were not spared the blows. In Lucknow, Jawaharlal and Govind Ballabh Pant were beaten up by the police. But the worst incident happened in Lahore where Lala Lajpat Rai, the hero of the Extremist days and the most revered leader of Punjab, was hit on the chest by lathis on 30 October and succumbed to the injuries on 17 November 1928. It was his death that Bhagat Singh and his comrades were seeking to avenge when they killed the white police official, Saunders, in December 1928.

\begin{center}*\end{center}

\paragraph*{}


The Simon boycott movement provided the first taste of political action to a new generation of youth. They were the ones who played the most active role in this protest, and it was they who gave the movement its militant flavour. And although a youth movement had already begun to take shape by 1927, it was participation in the Simon agitation that gave a real fillip to the formation of youth leagues and associations all over the country. Jawaharlal Nehru and Subhas Bose emerged as the leaders of this new wave of youth and students, and they travelled from one province to another addressing and presiding over innumerable youth conferences. 

The upsurge among the youth also proved a fruitful ground for the germination and spread of the new radical ideas of socialism that had begun to reach Indian shores. Jawaharlal Nehru had returned from Europe in 1927 after representing the Indian National Congress at the Brussels Congress of the League against Imperialism. He also visited the Soviet Union and was deeply impressed by socialist ideas. It was with the youth that he first shared his evolving perspective. Although Jawaharlal Nehru's was undoubtedly the most important role, other groups and individuals too played a crucial part in the popularization of the socialist vision. Subhas Bose was one such individual, though his notion of socialism was nowhere as scientific and clear as Jawaharlal's. Among groups, the more important ones were the Naujawan Bharat Sabha in Lahore, and the small group of Communists who had formed the Workers' and Peasants' Parties with the specific aim of organizing workers and peasants and radicalizing the Congress from within. As a result, the young people who were being drawn into the anti-imperialist movement were also simultaneously becoming sympathetic to the ideas of socialism, and youth groups in some areas even developed links with workers' and peasants' struggles.

\begin{center}*\end{center}

\paragraph*{}


Lord Birkenhead, the Conservative Secretary of State responsible for the appointment of the Simon Commission, had constantly harped on the inability of Indians to formulate a concrete scheme of constitutional reforms which had the support of wide sections of Indian political opinion. This challenge, too, was taken up and meetings of the All-Parties Conference were held in February, May and August 1928 to finalize a scheme which popularly came to be known as the Nehru Report after Motilal Nehru, its principal author. This report defined Dominion Status as the form of government desired by India. It also rejected the principle of separate communal electorates on which previous constitutional reforms had been based. Seats would be reserved for Muslims at the Centre and in provinces in which they were in a minority, but not in those where they had a numerical majority. The Report also recommended universal adult suffrage, equal rights for women, freedom to form unions, and dissociation of the state from religion in any form. A section of the Muslim League had in any case dissociated itself from these deliberations, but by the end of the year it became clear that even the section led by Jinnah would not give up the demand for reservation of seats for Muslims especially in Muslim majority provinces. The dilemma in which Motilal Nehru and other secular leaders found themselves was not one that was easy to resolve: if they conceded more to Muslim communal opinion, then Hindu communalists would withdraw support and if they satisfied the latter, then Muslim leaders would be estranged. In the event, no further concessions were forthcoming and Jinnah withdrew his support to the report and went ahead to propose his famous `Fourteen Points' which were basally a reiteration of his objections to the Nehru Report.

\begin{center}*\end{center}

\paragraph*{}


Young and radical nationalists led by Jawaharlal Nehru had their own, very different, objections to the Nehru Report. They were dissatisfied with its declaration of Dominion Status on the lines of the self-governing dominions as the basis of the future constitution of India. Their slogan was Complete Independence.' And it was in December 1928, at the annual session of the Congress at Calcutta, that the battle was joined. Jawaharlal Nehru, Subhas Bose and Satyamurthi, backed by a large number of delegates, pressed for the acceptance of `Purna Swaraj' or complete independence as the goal of the Congress. Gandhiji, Motilal Nehru and many other older leaders felt that the national consensus achieved with such great difficulty on Dominion Status should not be abandoned in such haste and a period of two years be given to the Government for accepting this. Under pressure, the grace of period for the Government was reduced to a year and, more important, the Congress decided that if the Government did not accept a constitution based on Dominion Status by the end of the year the Congress would not only adopt complete independence as its goal, but it would also launch a civil disobedience movement to attain that goal. A resolution embodying this proposal won over the majority of the delegates, and further amendments seeking immediate adoption of complete independence were defeated.

\begin{center}*\end{center}

\paragraph*{}


If civil disobedience was to be launched after the end of `the present year of probation and grace,' as Gandhiji called it, then preparations had to begin in right earnest. Gandhiji cancelled his plans for a European tour, and explained in the issue of Young India dated 31 January, 1929: `I feel that I would be guilty of desertion if I now went away to Europe... The voice within me tells me that I must not only hold myself in readiness to do what comes my way, but I must even think out and suggest means of working out what to me is a great programme. Above all I must prepare myself for next year's struggle, whatever shape it may take.' 

Gandhiji had of course been preparing the people for the future struggle in multifarious ways. For one, since his release from jail in 1924 on medical grounds, he had been travelling incessantly through the country. By the beginning of 1929, he had already toured Kathiawad, Central Provinces, Bengal, Malabar, Travancore, Bihar, United Provinces, Kutch, Assam, Maharashtra, Karnataka, Tamil Nadu, and Orissa, many of them not once but twice. In 1929, in his sixtieth year, he began a tour of Sind, then proceeded via Delhi to Calcutta, then on to Burma, and back to Calcutta. in April, he began a six-week tour of Andhra Pradesh in which he visited 319 villages. In June, he was in Almora in the hills of U.P., and in September he covered the 

U.P. plains. The end of the year saw him in Lahore for the annual Congress session. He had also planned a visit to Kohat in the North-West Frontier Province, but was refused permission by the Government. 

The significance of these mass contact tours was expressed by Gandhiji in these words: `I travel because I fancy that the masses want to meet me. I certainly want to meet them. I deliver my simple message to them in few words and they and I are satisfied. It penetrates the mass mind slowly but surely.' While in his pre-1929 tours Gandhiji's emphasis had been on the constructive programme — khadi, Hindu-Muslim unity, and the removal of untouchability — he now began to prepare the people for direct political action. In Sind, for example, he told the youth to prepare for `the fiery ordeal,' and it was at his instance that the Congress Working Committee constituted a Foreign Cloth Boycott Committee to promote an aggressive programme of boycott and public burning of foreign cloth, in Calcutta, on 4 March, 1929, Gandhiji took the lead in initiating the campaign of public burning of foreign cloth by lighting a bonfire in a public park before a crowd of thousands. The Government issued warrants for his arrest, but allowed him to go to Burma on his scheduled tour and face trial on his return. His arrest sparked off bonfires of foreign cloth all over the country. And when he returned to face trial, another wave of bonfires was lit to defy the Government. Gandhiji warned the people that while they must carry on all manner of preparations for civil disobedience, they must remember that civil disobedience had not yet begun, and that they must as yet remain within the law as far as possible. 

Apart from the preparations which the Congress carried on at various levels, there were a number of other developments that kept political excitement in 1929 at fever-pitch. On 20 March, 1929, in a major swoop, the Government arrested thirty-one labour leaders, most of them Communists, and marched them off to Meerut, in U.P., for trial. Their arrest was condemned by all sections of the national movement including Gandhiji and the Congress. Youth organizations organized protest demonstrations. On 8 April, 1Q29, Bhagat Singh and Batukeswar Dutt of the Hindustan Socialist Republican Army (HSRA) threw harmless bombs in the Central Legislative Assembly and were arrested. In jail, the members of the HSRA went on a prolonged hunger strike demanding better treatment for political prisoners, and in September the death of one of them. Jatin Das on the 64th day of the hunger strike led to some of the biggest demonstrations the country had ever witnessed. 

Meanwhile, in May 1929, a Labour Government headed by Ramsay MacDonald took power in Britain and Lord Irwin, the Viceroy, was called to London for consultations. The sequel was an announcement on 31 October: `I am authorized on behalf of His Majesty's Government to state clearly that in their judgement it is implicit in the Declaration of 1917 that the natural issue of India's progress as there contemplated, is the attainment of dominion status.' He also promised a Round Table Conference as soon as the Simon Commission submitted its report. Two days later, a conference of major national leaders met and issued what came to be known as the Delhi manifesto, in which they demanded that it should be made clear that the purpose of the Round Table Conference was not to discuss when Dominion Status should be granted, but to formulate a scheme for its implementation. A debate in the House of Lords on 5 November, 1929 on this question had already raised serious doubts about British intentions; and, finally, on 23 December Irwin himself told Gandhiji and the others that he was in no position to give the assurance they demanded. The stage of negotiations was over and the stage of confrontation was about to begin.

\begin{center}*\end{center}

\paragraph*{}


The honour of hosting what was, perhaps, the most memorable of the Congress annual sessions went to Lahore, the capital city of Punjab, and the honour of declaring `Puma Swaraj' as the only honourable goal Indians could strive for went to the man who had done more than any other to popularize the idea — Jawaharlal Nehru. It was Gandhiji again who was the decisive voice in investing Jawaharlal Nehru with the office of President in what was to be a critical year of mass struggle. Only three out of eighteen Provincial Congress Committees had wanted Jawaharlal, but recognizing the appositeness of the occasion, and the upsurge of the youth who had made such a glorious success of the Simon Boycott, Gandhiji insisted and as usual got his way. The critics he countered by an assurance: `Some fear in this transference of power from the old to the young, the doom of the Congress. I do not... ``He is rash and impetuous,'' say some. This quality is an additional qualification, at the present moment. And if he has the dash and the rashness of a warrior, he has also the prudence of a statesman... He is undoubtedly an extremist thinking far ahead of his surroundings. But he is humble and practical enough not to force the pace to the breaking point.' He added: `Older men have had their innings. The battle of the future has to be fought by younger men and women. And it is but meet that they are led by one of themselves ... Responsibility will mellow and sober the youth, and prepare them for the burden they must discharge. Pandit Jawaharlal has everything to recommend him. He has for years discharged with singular ability and devotion the office of secretary of the Congress. By his bravery, determination, application, integrity and grit, he has captivated the imagination of the youth of the land. He has come in touch with labour and the peasantry. His close acquaintance with European politics is a great asset in enabling him to assess ours.' 

To those who argued that he should himself assume the office because of the delicate nature of the negotiations that would have to be carried out with other parties and the Government, especially on the Hindu-Muslim question, he said: `So long as I retain the affection and the confidence of our people, there is not the slightest danger of my not being able without holding office to make the fullest use of such powers as I may possess. God has enabled me to affect the life of the country since 1920 without the necessity of holding office.' And to the youth he said: `They may take the election of Jawaharlal Nehru as a tribute to their service... (and as) proof of the trust the nation reposes in its youth Let them prove worthy of the trust.'' 

Jawaharlal Nehru's Presidential Address was a stirring call to action: `We have now an open conspiracy to free this country from foreign rule and you, comrades, and all our countrymen and countrywomen are invited to join it.'' Nehru also made it known that in his view liberation did not mean only throwing off the foreign yoke: `I must frankly confess that I am a socialist and a republican, and am no believer in kings and princes, or in the order which produces the modern kings of industry, who have greater power over the lives and fortunes of men than even the kings of old, and whose methods are as predatory as those of the old feudal aristocracy.'' He also spelt out the methods of struggle: `Any great movement for liberation today must necessarily be a mass movement, and mass movements must essentially be peaceful, except in times of organized revolt... And if the principal movement is a peaceful one, contemporaneous attempts at sporadic violence can only distract attention and weaken it.'' 

On the banks of the river Ravi, at midnight on 31 December 1929, the tricolour flag of Indian independence was unfurled amidst cheers and jubilation. Amidst the excitement, there was also a grim resolve, for the year to follow was to be one of hard struggle.

\begin{center}*\end{center}

\paragraph*{}


The first task that the Congress set itself and the Indian people in the New Year was that of organizing all over the country, on 26 January, public meetings at which the Independence Pledge would be read out and collectively affirmed. This programme was a huge success, and in villages and towns, at small meetings and large ones, the pledge was read out in the local language and the national flag was hoisted. The text of the pledge bears quoting in full': `We believe that it is the inalienable right of the Indian people, as of any other people, to have freedom and to enjoy the fruits of their toil and have the necessities of life, so that they may have full opportunities of growth. We believe also that if any government deprives a people of these rights and oppresses them, the people have a further right to alter it or to abolish it. The British Government in India has not only deprived the Indian people of their freedom but has based itself on the exploitation of the masses, and has ruined India economically, politically, culturally and spiritually. We believe, therefore, that India must sever the British connection and attain Poorna Swaraj or Complete Independence. `India has been ruined economically. The revenue derived from our people is out of all proportion to our income. Our average income is seven p1cc, less than two pence, per day, and of the heavy taxes we pay, twenty per cent are raised from the land revenue derived from the peasantry and three per cent from the salt tax, which falls most heavily on the poor. `Village industries, such as hand-spinning, have been destroyed, leaving the peasantry idle for at least four months in the year, and dulling their intellect for want of handicrafts, and nothing has been substituted, as in other countries, for the crafts thus destroyed. `Customs and currency have been so manipulated as to heap further burdens on the peasantry. The British manufactured goods constitute the bulk of our imports. Customs duties betray clear partiality for British manufacturers, and revenue from them is used not to lessen the burden on the masses, but for sustaining a highly extravagant administration. Still more arbitrary has been the manipulation of the exchange ratio which has resulted in millions being drained away from the country. `Politically, India's status has never been so reduced, as under the British regime. No reforms have given real political power to the people. The tallest of us have to bend before foreign authority. The rights of free expression of opinion and free association have been denied to us, and many of our countrymen are compelled to live in exile abroad and they cannot return to their homes. All administrative talent is killed, and the masses have to be satisfied with petty village offices and clerkships. `Culturally, the system of education has torn us from our moorings, our training has made us hug the very chains that bind us. 

Spiritually, compulsory disarmament has made us unmanly, and the presence of an alien army of occupation, employed with deadly effect to crush in us the spirit of resistance, has made us think that we cannot look after ourselves or put up a defence against foreign aggression, or defend our homes and families from the attacks of thieves, robbers, and miscreants. `We hold it to be a crime against man and God to submit any longer to a rule that has caused this four-fold disaster to our country. We recognize, however, that the most effective way of gaining our freedom is not through violence. We will prepare ourselves, by withdrawing, so far as we can, all voluntary association from the British Government, and will prepare for civil disobedience including non-payment of taxes. We are convinced that if we can but withdraw our voluntary help, stop payment of taxes without doing violence, even under provocation, the end of this inhuman rule is assured. We, therefore, hereby solemnly resolve to carry out the Congress instructions issued from time to time for the purpose of establishing Poorna Swaraj.'
