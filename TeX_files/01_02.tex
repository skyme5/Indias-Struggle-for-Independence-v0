
\chapter{Civil Rebellions and Tribal Uprisings}

The Revolt of 1857 was the most dramatic instance of traditional India's struggle against foreign rule. But it was no sudden occurrence. It was the culmination of a century long tradition of fierce popular resistance to British domination.

The establishment of British power in India was a prolonged process of piecemeal conquest and consolidation and the colonialization of the economy and society. This process produced discontent, resentment and resistance at every stage. This popular resistance took three broad forms: civil rebellions, tribal uprisings and peasant movements. We will discuss the first two in this chapter.

\begin{center}*\end{center}

\paragraph*{}
The series of civil rebellions, which run like a thread through the first 100 years of British rule, were often led by deposed rajas and nawabs or their descendants, uprooted and impoverished zamindars, landlords and poligars (landed military magnates in South India), and ex-retainers and officials of the conquered Indian states. The backbone of the rebellions, their mass base and striking power came from the rack-rented peasants, ruined artisans and demobilized soldiers.

These sudden, localized revolts often took place because of local grievances although for short periods they acquired a broad sweep, involving armed bands of a few hundreds to several thousands. The major cause of all these civil rebellions taken as a whole was the rapid changes the British introduced in the economy, administration and land revenue system. These changes led to the disruption of the agrarian society, causing prolonged and widespread suffering among its constituents Above all, the colonial policy of intensifying demands for land revenue and extracting as large an amount as possible produced a veritable upheaval in Indian villages. In Bengal, for example, in less than thirty years land revenue collection was raised to nearly double the amount collected under the Mughals. The pattern was repeated in other us of the country as British rule spread. And aggravating the unhappiness of the farmers was the fact that not even a part of the enhanced revenue was spent on the development of agriculture or the welfare of the cultivator.

Thousands of zamindars and poligars lost control over their land and its revenues either due to the extinction of their rights by the colonial state or by the forced sale of their rights over land because of their inability to meet the exorbitant land revenue demanded. The proud zamindars and poligars resented this loss even more when they were displaced by rank outsiders --- government officials and the new men of money --- merchants and moneylenders. Thus they, as also the old chiefs, who had lost their principalities, had personal scores to settle with the new rulers.

Peasants and artisans, as we have seen earlier, had their own reasons to rise up in arms and side with the traditional elite. Increasing demands for land revenue were forcing large numbers of peasants into growing indebtedness or into selling their lands. The new landlords, bereft of any traditional paternalism towards their tenants, pushed up rents to ruinous heights and evicted them in the case of non-payment. The economic decline of the peasantry was reflected in twelve major and numerous minor famines from 1770 to 1857.

The new courts and legal system gave a further fillip to the dis-possessors of land and encouraged the rich to oppress the poor. Flogging, torture and jailing of the cultivators for arrears of rent or land revenue or interest on debt were quite common. The ordinary people were also hard hit by the prevalence of corruption at the lower levels of the police, judiciary and general administration. The petty officials enriched themselves freely at the cost of the poor. The police looted, oppressed and tortured the common people at will. William Edwards, a British official, wrote in 1859 that the police were `a scourge to the people' and that `their oppression and exactions form one of the chief grounds of dissatisfaction with our government.'

The ruin of Indian handicraft industries, as a result of the imposition of free trade in India and levy of discriminatory tariffs against Indian goods in Britain, pauperized millions of artisans. The misery of the artisans was further compounded by the disappearance of their traditional patrons and buyers, the princes, chieftains, and zamindars.

The scholarly and priestly classes were also active in inciting hatred and rebellion against foreign rule. The traditional rulers and ruling elite had financially supported scholars, religious preachers, priests, pandits and maulvis and men of arts and literature. With the coming of the British and the ruin of the traditional landed and bureaucratic elite, this patronage came to an end, and all those who had depended on it were impoverished.

Another major cause of the rebellions was the very foreign character of British rule. Like any other people, the Indian people too felt humiliated at being under a foreigner's heel. This feeling of hurt pride inspired efforts to expel the foreigner from their lands.

The civil rebellions began as British rule was established in Bengal and Bihar, arid they occurred in area after area as it was incorporated into colonial rule. There was hardly a year without armed opposition or a decade without a major armed rebellion in one part of the country or the other. From 1763 to 1856, there were more than forty major rebellions apart from hundreds of minor ones.

Displaced peasants and demobilized soldiers of Bengal led by religious monks and dispossessed zamindars were the first to rise up in the Sanyasi rebellion, made famous by Bankim Chandra Chatterjee in his novel Anand Math, that lasted from 1763 to 1800. It was followed by the Chuar uprising which covered five districts of Bengal and Bihar from 1766 to 1772 and then, again, from 1795 to 1816. Other major rebellions in Eastern India were those of Rangpur and Dinajpur, 1783; Bishnupur and Birbhum, 1799; Orissa zamindars, 1804--17; and Sambalpur, 1827--40.

In South India, the Raja of Vizianagram revolted in 1794, the poligars of Tamil Nadu during the 1790's, of Malabar and coastal Andhra during the first decade of the 19th century, of Parlekamedi during 1813--14. Dewan Velu Thampi of Travancore organized a heroic revolt in 1805. The Mysore peasants too revolted in 1830--31. There were major uprisings in Visakhapatnam from 1830--34, Ganjam in 1835 and Kurnool in 1846--47.

In Western India, the chiefs of Saurashtra rebelled repeatedly from 1816 to 1832. The Kolis of Gujarat did the same during 1824--28, 1839 and 1849. Maharashtra was in a perpetual state of revolt after the final defeat of the Peshwa. Prominent were the Bhil uprisings, 1818--31; the Kittur uprising, led by Chinnava, 1824; the Satara uprising, 1841; and the revolt of the Gadkaris. 1844.

Northern India was no less turbulent. The present states of Western U.P. and Haryana rose up in arms in 1824. Other major rebellions were those of Bilaspur, 1805; the taluqdars of Aligarh, 1814--17; the Bundelas of Jabalpur, 1842; and Khandesh, 1852. The second Punjab War in 1848--49 was also in the nature of a popular revolt by the people and the army.

These almost continuous rebellions were massive in their totality, but were wholly local in their spread and isolated from each other. They were the result of local causes and grievances, and were also localized in their effects. They often bore the same character not because they represented national or common efforts but because they represented common conditions though separated in time and space.

Socially, economically and politically, the semi-feudal leaders of these rebellions were backward looking and traditional in outlook. They still lived in the old world, blissfully unaware and oblivious of the modern world which had knocked down the defences of their society. Their resistance represented no societal alternative. It was centuries-old in form and ideological and cultural content. Its basic objective was to restore earlier forms of rule and social relations. Such backward looking and scattered, sporadic and disunited uprisings were incapable of fending off or overthrowing foreign rule. The British succeeded in pacifying the rebel areas one by one. They also gave concessions to the less fiery rebel chiefs and zamindars in the form of reinstatement, the restoration of their estates and reduction in revenue assessments so long as they agreed to live peacefully under alien authority. The more recalcitrant ones were physically wiped out. Velu Thampi was, for example, publicly hanged even after he was dead.

The suppression of the civil rebellions was a major reason why the Revolt of 1857 did not spread to South India and most of Eastern and Western India. The historical significance of these civil uprisings lies in that they established strong and valuable local traditions of resistance to British rule. The Indian people were to draw inspiration from these traditions in the later nationalist struggle for freedom.

The tribal people, spread over a large part of India, organized hundreds of militant outbreaks and insurrections during the 19th century. These uprisings were marked by immense courage and sacrifice on their part and brutal suppression and veritable butchery on the part of the rulers. The tribals had cause to be upset for a variety of reasons. The colonial administration ended their relative isolation and brought them fully within the ambit of colonialism. It recognized the tribal chiefs as zamindars and introduced a new system of land revenue and taxation of tribal products. It encouraged the influx of Christian missionaries into the tribal areas. Above all, it introduced a large number of moneylenders, traders arid revenue farmers as middlemen among the tribals. These middlemen were the chief instruments for bringing the tribal people within the vortex of the colonial economy and exploitation. The middlemen were outsiders who increasingly took possession of tribal lands and ensnared the tribals in a web of debt. In time, the tribal people increasingly lost their lands and were reduced to the position of agricultural laborers, share-croppers and rack-rented tenants on the land they had earlier brought under cultivation and held on a communal basis. Colonialism also transformed their relationship with the forest. They had depended on the forest for food, fuel and cattle-feed. They practiced shifting cultivation (jhum, podu, etc.), taking recourse to fresh forest lands when their existing lands showed signs of exhaustion. The colonial government changed all this. It usurped the forest lands and placed restrictions on access to forest products, forest lands and village common lands. It refused to let cultivation shift to new areas.

Oppression and extortion by policemen and other petty officials further aggravated distress among the tribals. The revenue farmers and government agents also intensified and expanded the system of begar --- making the tribals perform unpaid labor.

All this differed in intensity from region to region, but the complete disruption of the old agrarian order of the tribal communities provided the common factor for all the tribal uprisings. These uprisings were broad-based, involving thousands of tribals, often the entire population of a region.

The colonial intrusion and the triumvirate of trader, moneylender and revenue farmer in sum disrupted the tribal identity to a lesser or greater degree. In fact, ethnic ties were a basic feature of the tribal rebellions. The rebels saw themselves not as a discreet class but as having a tribal identity.

At this level the solidarity shown was of a very high order. Fellow tribals were never attacked unless they had collaborated with the enemy.

At the same time, not all outsiders were attacked as enemies. Often there was no violence against the non-tribal poor, who worked in tribal villages in supportive economic roles, or who had social relations with the tribals such as telis, gwalas, lohars, carpenters, potters, weavers, washermen, barbers, drummers, and bonded labourers and domestic servants of the outsiders. They were not only spared, but were seen as allies. In many cases, the rural poor formed a part of the rebellious tribal bands.

The rebellions normally began at the point where the tribals felt so oppressed that they felt they had no alternative but to fight. This often took the form of spontaneous attacks on outsiders, looting their property and expelling them from their villages. This led to clashes with the colonial authorities. When this happened, the tribals began to move towards armed resistance and elementary organization. Often, religious and charismatic leaders --- messiahs emerged at this stage and promised divine intervention and an end to their suffering at the hands of the outsiders, and asked their fellow tribals to rise and rebel against foreign authority. Most of these leaders claimed to derive their authority from God. They also often claimed that they possessed magical powers, for example, the power to make the enemies' bullets ineffective. Filled with hope and confidence, the tribal masses tended to follow these leaders to the very end.

The warfare between the tribal rebels and the British armed forces was totally unequal. On one side were drilled regiments armed with the latest weapons and on the other were men and women fighting in roving bands armed with primitive weapons such as stones, axes, spears and bows and arrows, believing in the magical powers of their commanders. The tribals died in lakhs in this unequal warfare.

\begin{center}*\end{center}

\paragraph*{}
Among the numerous tribal revolts, the Santhal hool or uprising was the most massive. The Santhals, who live in the area between Bhagalpur and Rajmahal, known as Daman-i-koh, rose in revolt; made a determined attempt to expel the outsiders --- the dikus --- and proclaimed the complete `annihilation' of the alien regime. The social conditions which drove them to insurrection were described by a contemporary in the Calcutta Review as follows: `Zamindars, the police, the revenue and court alas have exercised a combined system of extortions, oppressive exactions, forcible dispossession of property, abuse and personal violence and a variety of petty tyrannies upon the timid and yielding Santhals. Usurious interest on loans of money ranging from 50 to 500 per cent; false measures at the haul and the market; willful and uncharitable trespass by the rich by means of their untethered cattle, tattoos, ponies and even elephants, on the growing crops of the poorer race; and, such like illegalities have been prevalent.'

The Santhals considered the dikus and government servants morally corrupt being given to beggary, stealing, lying and drunkenness.

By 1854, the tribal heads, the majhis and parganites, had begun to meet and discuss the possibility of revolting. Stray cases of the robbing of zamindars and moneylenders began to occur. The tribal leaders called an assembly of nearly 6000 Santhals, representing 400 villages, at Bhaganidihi on 1855-06-30. It was decided to raise the banner of revolt, get rid of the outsiders and their colonial masters once and for all, the usher in Salyug, `The Reign of Truth,' and `True Justice.'

The Santhals believed that their actions had the blessings of God. Sido and Kanhu, the principal rebel leaders, claimed that Thakur (God) had communicated with them and told them to take up arms and fight for independence. Sido told the authorities in a proclamation: `The Thacoor has ordered me saying that the country is not Sahibs ... The Thacoor himself will fight. Therefore, you Sahibs and Soldiers (will) fight the Thacoor himself.'

The leaders mobilized the Santhal men and women by organizing huge processions through the villages accompanied by drummers and other musicians. The leaders rode at the front on horses and elephants and in palkis. Soon nearly 60,000 Santhals had been mobilized. Forming bands of 1,500 to 2,000, but rallying in many thousands at the call of drums on particular occasions, they attacked the mahajans and zamindars and their houses, police stations, railway construction sites, the dak (post) carriers --- in fact all the symbols of exploitation and colonial power.

The Santhal insurrection was helped by a large number of non-tribal and poor dikus. Gwalas (milkmen) and others helped the rebels with provisions and services; lohars (blacksmiths) accompanied the rebel bands, keeping their weapons in good shape.

Once the Government realized the scale of the rebellion, it organized a major military campaign against the rebels. It mobilized tens of regiments under the command of a major­general, declared Martial Law in the affected areas and offered rewards of up-to Rs. 10,000 for the capture of various leaders.

The rebellion was crushed ruthlessly. More than 15,000 Santhals were killed while tens of villages were destroyed. Sido was betrayed and captured and killed in 1855-08-00 while Kanhu was arrested by accident at the tail-end of the rebellion in 1866-02-00. And `the Rajmahal Hills were drenched with the blood of the fighting Santhal peasantry.' One typical instance of the heroism of Santhal rebels has been narrated by L.S. S. O'Malley: `They showed the most reckless courage never knowing when they were beaten and refusing to surrender. On one occasion, forty-five Santhals took refuge in a mud hut which they held against the Sepoy's. Volley after volley was fired into it… Each time the Santhals replied with a discharge of arrows. At last, when their fire ceased, the Sepoys entered the hut and found only one old man was left alive. A Sepoy called on him to surrender, whereupon the old man rushed upon him and cut him down with his battle axe.''

\begin{center}*\end{center}

\paragraph*{}
I shall describe briefly three other major tribal rebellions. The Kols of Chhotanagpur rebelled from 1820 to 1837. Thousands of them were massacred before British authority could be re-imposed. The hill tribesmen of Rampa in coastal Andhra revolted in 1879-04-00, against the depredations of the government-supported mansabdar and the new restrictive forest regulations. The authorities had to mobilize regiments of infantry, a squadron of cavalry and two companies of sappers and miners before the rebels, numbering several thousands, could be defeated by the end of 1880.

The rebellion (ulgulan) of the Munda tribesmen, led by Birsa Munda, occurred during 1899--19. For over thirty years the Munda sardars had been struggling against the destruction of their system of common land holdings by the intrusion of jagirdar, thikadar (revenue farmers) and merchant moneylenders.

Birsa, born in a poor share-cropper household in 1874, had a vision of God in 1895. He declared himself to be a divine messenger, possessing miraculous healing powers. Thousands gathered around him seeing in him a Messiah with a new religious message. Under the influence of the religious movement soon acquired an agrarian and political Birsa began to move from village to village, organizing rallies and mobilizing his followers on religious and political grounds. On Christmas Eve, 1899, Birsa proclaimed a rebellion to establish Munda rule in the land and encouraged `the killing of thikadars and jagirdars and Rajas and Hakims (rulers) and Christians.' Saiyug would be established in place of the present-day Kalyug. He declared that `there was going to be a fight with the dikus, the ground would be as red as the red flag with their blood.' The non-tribal poor were not to be attacked.

To bring about liberation, Birsa gathered a force of 6,000 Mundas armed with swords, spears, battle-axes, and bows and arrows. He was, however, captured in the beginning of 1900-02-00, and he died in jail in June. The rebellion had failed. But Birsa entered the realms of legend.
