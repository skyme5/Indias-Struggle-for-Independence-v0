\chapter{From Karachi to Wardha: The Years from 1932--34}\label{chapter:CH23}

The Congress met at Karachi on 29 March 1931 to endorse the Gandhi-Irwin or Delhi Pact. Bhagat Singh, Sukhdev and Rajguru had been executed six days earlier. Even though Gandhiji had made every attempt to save their lives, there was anger among the people, especially the youth, as to why he had not refused to sign the Pact on this question. All along Gandhiji's route to Karachi he was greeted with black flag demonstrations. The Congress passed a resolution drafted by Gandhiji by which it, `while dissociating itself from and disapproving of political violence in any shape or form,' admired `the bravery and sacrifice' of the three martyrs.' The Congress endorsed the Delhi Pact and reiterated the goal of Poorna Swaraj. 

The Karachi session became memorable for its resolution on Fundamental Rights and the National Economic Programme. Even though the Congress had from its inception fought for the economic interests, civil liberties and political rights of the people, this was the first time that the Congress defined what Swaraj would mean for the masses. It also declared that, `in order to end the exploitation of the masses, political freedom must include real economic freedom of the starving millions.' The resolution guaranteed the basic civil rights of free speech, free press, free assembly, and freedom of association; equality before the law irrespective of caste, creed or sex; neutrality of the state in regard to all religions; elections on the basis of universal adult franchise; and free and compulsory primary education. It promised substantial reduction in rent and revenue, exemption from rent in case of uneconomic holdings, and relief of agricultural indebtedness and control of usury; better conditions for workers including a living wage, limited hours of work and protection of women workers; the right to organize and form unions to workers and peasants; and state ownership or control of key industries, mines and means of transport. It also maintained that `the culture, language and script of the minorities and of the different linguistic areas shall be protected.' The Karachi resolution was to remain in essence the basic political and economic programme of the Congress in later years.

\begin{center}*\end{center}

\paragraph*{}

Gandhiji sailed for London on 29 August 1931 to attend the Second Round Table Conference. Nothing much was expected from the Conference for the imperialist political and financial forces, which ultimately controlled the British Government in London, were opposed to any political or economic concessions being given to India which could lead to its independence from their control. Winston Churchill, leader of the virulent right-wing, had strongly objected to the British Government negotiating on terms of equality with the `seditious fakir' and demanded strong government in India. The Conservative Daily Mail declared that `Without India, the British Commonwealth would fall to pieces. Commercially, economically, politically and geographically it is our greatest imperial asset. To imperil our hold on it would be the worst treason any Briton could commit.' In India, Irwin was replaced by Willingdon as the Viceroy. In Britain, after December 1931, the Laborite Ramsay MacDonald headed a Conservative- dominated Cabinet with the weak and reactionary Samuel Hoare as the Secretary of State for India. Apart from a few able individuals, the overwhelming majority of Indian delegates to the Round Table Conference (RTC), hand-picked by the Government, were loyalists, communalists, careerists, and place-hunters, big landlords and representatives of the princes. They were used by the Government to claim that the Congress did not represent the interests of all Indians vis-a-vis imperialism, and to neutralize Gandhiji and all his efforts to confront the imperialist rulers with the basic question of freedom. 

The great Gujarati poet, Meghani, in a famous poem gave expression to the nationalist misgivings regarding the RTC. Addressing Gandhiji on the eve of his departure for London, he sang in the first line: `Chchello Katoro Jerno Aa: Pi Jayo Bapu!' (Even this last cup of poison, you must drink, Bapu!) Gandhiji himself said: `When I think of the prospects in London, when I know that all is not well in India ... there is nothing wanting to fill me with utter despair ... There is every chance of my returning empty-handed'. That is exactly what happened in London. The British Government refused to concede the basic Indian demand for freedom. Gandhiji came back at the end of December 1931 to a changed political situation. 

The higher British officials in India had drawn their own lessons from the political impact of the Delhi Pact which had raised the political prestige of the Congress and the political morale of the people and undermined and lowered British prestige. They, as well as the new Viceroy, believed that the Government had made a major error in negotiating and signing a truce with the Congress, as if between two equal powers. They were now determined to reverse it all. No pact, no truce, no Gandhi-Viceroy meetings, no `quarter for the enemy' became the watchwords of Government policy. 

The British policy was now dominated by three major considerations: (a) Gandhiji must not be permitted to build up the tempo for a massive and protracted mass movement, as he had done in 1919, 1920-1 and 1930. (b) The Government functionaries --- village officials, police and higher bureaucrats --- and the loyalists --- `our friends' --- must not feel disheartened that Gandhiji was being `resurrected as a rival authority to the Government of India,' and that the Government was losing the will to rule. As the Home Member, H.G. Haig, put it: `We can, in my view, do without the goodwill of the Congress, and in fact I do not believe for a moment that we shall ever have it, but we cannot afford to do without the confidence of those who have supported us in the long struggle against the Congress.' (c) In particular, the nationalist movement must not be permitted to gather force and consolidate itself in rural areas, as it was doing all over India, especially in U.P., Gujarat, Andhra, Bihar, Bengal and NWFP. 

While Gandhiji was in London, the Government of India prepared, in secret, plain for the coming showdown with the nationalist forces. It decided to launch `a hard and immediate blow' against any revival of the movement and to arrest Gandhiji at the very outset. It drafted a series of ordinances which would usher in virtual martial law, though under civilian control. 

The shape of things to come had been overshadowed by what happened in U.P., NWFP and Bengal during the truce period, hi U.P. the Congress was leading a campaign for reduction of rent, remission of arrears of rent and prevention of eviction of tenants for non-payment of rents. By the first week of December, the Congress had launched a no-rent, no-revenue campaign in five districts. The Government's response was to arrest Jawaharlal on 26 December when he was going to Bombay to meet Gandhiji. In the North-Western Frontier Province, the Government continued its severe repression against the non-violent Khudai Khidmatgars (servants of God), also known as Red Shirts because of the colour of their shirts, and the peasants they led against the Government's policy of extracting revenue through cruel methods and torture. On 24 December, their leader, Abdul Ghaffar Khan, was arrested and Peshawar district was occupied by the army. In Bengal, the Government was ruling through draconian ordinances and detaining thousands of political workers in the name of fighting terrorism. In September, the police fired upon political prisoners in Hijli jail, killing two. 

Gandhiji landed in Bombay on 28 December. The Congress Working Committee met the next day and decided to resume civil disobedience. On the 31st, Gandhiji asked the Viceroy for a meeting, offering to suspend the decision on civil disobedience till such a meeting. The Viceroy refused to see Gandhiji --- the first of many such refusals during the next five years. On 4 January 1932, the Government launched its pre-emptive strike against the national movement by arresting Gandhiji, promulgating ordinances which gave the authorities unlimited power --- thus initiating what a historian has described as `Civil Martial Law.' Civil liberties no longer existed and the authorities could seize people and property at will. Within a week, leading Congressmen all over the country were behind bars. 

The Indian people responded with anger. Even though the Congress entered the battle rather unprepared, the popular response was massive. In the first four months, over 80,000 Satyagrahis, most of them urban and rural poor, were jailed, while lakhs took to the picketing of shops selling liquor and foreign cloth. Illegal gatherings, non-violent demonstrations, celebrations of various national days, and other forms of defiance of the ordinances were the rule of the day. 

The non-violent movement was met by relentless repression. The Congress and its allied organizations were declared illegal and their offices and funds seized. Nearly all the Gandhi Ashrams were occupied by the police. Peaceful picketers, Satyagrahis and processionists were lathi-charged, beaten and often awarded rigorous imprisonment and heavy fines, which were realized by selling their lands and property at throw away prices. Prisoners in jail were barbarously treated. Whipping as punishment became frequent. The no-tax campaigns in different parts of rural India were treated with great severity. Lands, houses, cattle, agricultural implements, and other property were freely confiscated. The police indulged in naked terror and committed innumerable atrocities. At Ras, a village in Gujarat, the non-tax paying peasants were stripped naked, publicly whipped and given electric shocks. The wrath of the Government fell with particular harshness on women. Conditions in jails were made extraordinarily severe with the idea of scaring away women from the Satyagraha. The freedom of the Press to report or comment on the movement, or even to print pictures of national leaders or Satyagrahis, was curtailed. Within the first six months of 1932 action was taken against 109 journalists and ninety-eight printing presses. Nationalist literature --- poems, stories and novels --- was banned on a large scale. 

The people fought back. But Gandhiji and other leaders had no. time to build up the tempo of the movement and it could not be sustained for long. The movement was effectively crushed within a few months. In August 1932, the number of those convicted came down to 3,047 and by August 1933 only 4,500 Satyagrahis were in jail. However, the movement continued to linger till early April 1934 when the inevitable decision to withdraw it was taken by Gandhiji. 

Political activists despaired at the turn the movement had taken. What have we achieved, many asked? Even a buoyant and active person like Jawaharlal gave voice to this sense of despair accentuated by his separation from his sick wife --- by copying a verse in his jail diary in June 1935: `Sad winds where your voice was; Tears, tears where my heart was; and ever with me, Child, ever with me, Silence where hope was.'7 Earlier, when Gandhiji had withdrawn the movement, Jawaharlal had felt `with a stab of pain' that his long association with Gandhiji was about to come to an end. Subhas Chandra Bose and Vithalbhai Patel had been much more critical of Gandhiji's leadership. In a strong statement from Europe they had said in 1933 that `Mr. Gandhi as a political leader has failed' and called for `a radical reorganization of the Congress on a new principle with a new method, for which a new leader is essential.' The enemies of Indian nationalism gloated over the frustration among the nationalists --- and grossly misread it. Willingdon declared in early 1933: `The Congress is in a definitely less favourable position than in 1930, and has lost its hold on the public.'' But Willingdon and company had completely failed to understand the nature and strategy of the Indian national movement --- it was basically a struggle for the minds of men and women. Seen in this light, if the colonial policy of negotiations by Irwin had failed earlier, so had the policy of ruthless suppression by Willingdon. People had been cowed down by superior force; they had not lost faith in the Congress. Though the movement from 1930 to 1934 had not achieved independence and had been temporarily crushed, the Indian people had been further transformed. The will to fight had been further strengthened; faith in British rule had been completely shattered. H.N. Brailsford, Laborite journalist, wrote, assessing the results of the nationalists' most recent struggle, that the Indians `had freed their own minds, they had won independence in their hearts.' 

And, as we have seen earlier, this hiatus in the movement too was primarily to rest and regroup. Withdrawal of the movement did not mean defeat or loss of mass support; it only meant, as Dr. Ansari put it, `having fought long enough we prepare to rest,' to fight another day a bigger battle with greater and better organized force.' Symbolic of the real outcome, the real impact of the civil disobedience, was the heroes' welcome given to prisoners on their release in 1934. And this became evident to all when the Congress captured a majority in six out of eleven provinces in the elections in 1937 despite the restricted nature of the franchise. 

Alone among his contemporaries, Gandhiji understood the true nature and outcome of the Civil Disobedience Movement. To Nehru, he wrote in September 1933: `I have no sense of defeat in me and the hope in me that this country of ours is fast marching towards its goal is burning as bright as it did in 1920.'' He reiterated this view to a group of Congress leaders in April 1934: `I feel no despondency in me. .. I am not feeling helpless ... The nation has got energy of which you have no conception but I have.'' He had, of course, an advantage over most other leaders. While they needed a movement to sustain their sense of political activism, he had always available the alternative of constructive work.

\begin{center}*\end{center}

\paragraph*{}

The British policy of `Divide and Rule' found another expression in the announcement of the Communal Award in August 1932. The Award allotted to each minority a number of seats in the legislatures to be elected on the basis of a separate electorate that is Muslims would be elected only by Muslims and Sikhs only by Sikhs, and so on. Muslims, Sikhs and Christians had already been treated as minorities. The Award declared the Depressed Classes (Scheduled Castes of today) also to be a minority community entitled to separate electorate and thus separated them from the rest of the Hindus. 

The Congress was opposed to a separate electorate for Muslims, Sikhs and `Christians as it encouraged the communal notion that they formed separate groups or communities having interests different from the general body of Indians. The inevitable result was to divide the Indian people and prevent the growth of a common national consciousness. But the idea of a separate electorate for Muslims had been accepted by the Congress as far back as 1916 as a part of the compromise with the Muslim League. Hence, the Congress took the position that though it was opposed to separate electorates, it was not in favour of changing the Award without the consent of the minorities. Consequently, though strongly disagreeing with the Communal Award, it decided neither to accept it nor to reject it. 

But the effort to separate the Depressed Classes from the rest of Hindus by treating them as separate political entities was vehemently opposed by all the nationalists. Gandhiji, in Yeravada jail at the time, in particular, reacted very strongly.' He saw the Award as an attack on Indian unity and nationalism, harmful to both Hinduism and the Depressed Classes, for it provided no answers to the socially degraded position of the latter. Once the Depressed Classes were treated as a separate community, the question of abolishing untouchability would not arise, and the work of Hindu social reform in this respect would come to a halt. 

Gandhiji argued that whatever harm separate electorates might do to Muslims or Sikhs, it did not affect the fact that they would remain Muslims or Sikhs. But while reformers like himself were working for the total eradication of untouchability, separate electorates would ensure that `untouchables remain untouchables in perpetuity.' What was needed was not the protection of the so-called interests of the Depressed Classes in terms of seats in the legislatures or jobs but the `root arid branch' eradication of untouchability. 

Gandhiji demanded that the representatives of the Depressed Classes should be elected by the general electorate under a wide, if possible universal, common franchise. At the same time he did not object to the demand for a larger number of the reserved seats for the Depressed Classes. He went on a fast unto death on 20 September 1932 to enforce his demand. In a statement to the Press, he said: `My life, I count of no consequence. One hundred lives given for this noble cause would, in my opinion, be poor penance done by Hindus for the atrocious wrongs they have heaped upon helpless men and women of their own faith.' 

While many political Indians saw the fast as a diversion from the ongoing political movement, all were deeply concerned and emotionally shaken. Mass meetings took place almost everywhere. The 20th of September was observed as a day of fasting and prayer. Temples, wells, etc., were thrown open to the Depressed Classes all over the country. Rabindranath Tagore sent a telegraphic message to Gandhiji: `It is worth sacrificing precious life for the sake of India's unity and her social integrity ... Our sorrowing hearts will follow your sublime penance with reverence and love.' Political leaders of different political persuasions, including Madan Mohan Malaviya, M.C. Rajah and 

B.R. Ambedkar, now became active. In the end they succeeded in hammering out an agreement, known as the Poona Pact, according to which the idea of separate electorates for the Depressed Classes was abandoned but the seats reserved for them in the provincial legislatures were increased from seventy- one in the Award to 147 and in the Central Legislature to eighteen per cent of the total.

\begin{center}*\end{center}

\paragraph*{}

Regarding the Poona agreement, Gandhiji declared after breaking his fast: `I would like to assure my Harijan friends ... that they may hold my life as a hostage for its due fulfilment.' He now set out to redeem his pledge. First from jail and then from outside, for nearly two years he gave up all other pre-occupations and earned on a whirlwind campaign against untouchability. After his release from prison, he had shifted to Satyagraha Ashram at Wardha after abandoning Sabarmati Ashram at Ahmedabad for he had vowed in 1930 not to return to Sabarmati till Swaraj was won. Starting from Wardha on 7 November 1933 and until 29 July 1934, for nearly nine months, he conducted an intensive `Harijan tour' of the country travelling over 20,000 kilometres by train, car, bullock cart, and on foot. collecting money for the recently founded Harijan Sewak Sangh, propagating the removal of untouchability in all its forms and practices, and urging social workers to leave all and go to the villages for the social, economic, cultural and political uplift of the Harijans --- his name for the Depressed Classes. 

In the course of his Harijan campaign, Gandhiji undertook two major fasts on 8 May and 16 August 1933 to convince his followers of the importance of the issue and the seriousness of his effort. `They must either remove untouchability or remove me from their midst.' He justified these fasts as answers to his `inner voice,' which, he said, could also be described as `dictates of reason.' These fasts created consternation in the ranks of the nationalists, throwing many of them into an emotional crisis. The fast of 8 May 1933 was opposed even by Kasturba, his wife. As the hour of the fast approached, Miraben sent a telegram: `Ba wishes me to say she is greatly shocked. Feels the decision very wrong but you have not listened to any others and so will not hear her. She sends her heartfelt prayers.' Gandhiji's reply was characteristic: `Tell Ba her father imposed on her a companion whose weight would have killed any other woman. I treasure her love. She must remain courageous to the end.' 

Throughout Gandhiji's Harijan campaign, he was attacked by orthodox and social reactionaries. They met him with black flag demonstrations and disrupted his meetings. They brought out scurrilous and inflammatory leaflets against him, putting fantastic utterances in his mouth. They accused him of attacking Hinduism. They publicly burnt his portraits. On 25 June 1934, at Poona, a bomb was thrown on a car believed to be carrying Gandhiji, injuring its seven occupants. The protesters offered the Government full support against the Congress and the Civil Disobedience Movement if it would not support the anti-untouchability campaign. The Government obliged by defeating the Temple Entry Bill in the Legislative Assembly in August 1934. Throughout his fast, Harijan work and Harijan tour, Gandhiji stressed on certain themes. One was the degree of oppression practised on the Harijans; in fact, day after day he put forward a damning indictment of Hindu society: `Socially they are lepers. Economically they are worse. Religiously they are denied entrance to places we miscall houses of God. They are denied the use, on the same terms as Hindus, of public roads, public schools, public hospitals, public wells, public taps, public parks and the like ... They are relegated for their residence to the worst quarters of cities and villages where they get no social services.' A second theme was that of the `root and branch removal of untouchability.' Symbolic or rather the entering wedge in this respect was to be the throwing open of all temples to Harijans. 

Gandhiji's entire campaign was based on the grounds of humanism and reason. But he also argued that untouchability, as practised at present, had no sanction in the Hindu Shastras. But even if this was not so, the Harijan worker should not feel daunted. Truth could -not be confined within the covers of a book. The Shastras should be ignored if they went against human dignity. 

A major running theme in Gandhiji's writings and speeches was the need for caste Hindus to do `penance' and `make reparations ... for the untold hardships to which we have subjected them (the Harijans) for centuries.' For this reason, he was not hostile to Dr. Ambedkar and other Harijans who criticized and distrusted him. `They have every right to distrust me,' he wrote. `Do I not belong to the Hindu section miscalled superior class or caste Hindus, who have ground down to powder the so called untouchables?' At the same time, he repeatedly warned caste Hindus that if this atonement was not made, Hinduism would perish: `Hinduism dies if untouchability lives, and untouchability has to die if Hinduism is to live.' (This strong theme of `penance' largely explains why caste Hindus born and brought up in pre-1947 India so readily accepted large scale reservations in jobs, enrolment in professional colleges and so on for the Scheduled Castes and Scheduled Tribes after independence). 

Gandhiji was not in favour of mixing up the issue of the removal of untouchability with the issues of inter-dining and inter-marriage. Restriction on the latter should certainly go, for `dining and marriage restrictions stunt Hindu society.' But they were also practised by caste Hindus among themselves as also the Harijans among themselves. The present All-India campaign, he said, had to be directed against the disabilities which were specific to the Harijans. Similarly, he distinguished between the abolition of caste system and the abolition of untouchability. He disagreed with Dr. Ambedkar when the latter asserted that `the outcaste is a by-product of the caste system. There will be outcastes as long as there are castes. And nothing can emancipate the outcaste except the destruction of the caste system. On the contrary, Gandhiji said that whatever the `limitations and defects' of the Vamashram, `there is nothing sinful about it, as there is about untouchability.' He believed that purged of untouchability, itself a product of `the distinction of high and low' and not of the caste system, this system could function in a manner that would make each caste `complementary of the other and none inferior or superior to any other.' In any case, he said, both the believers and the critics of the Varna system should join hands in fighting untouchability, for opposition to the latter was common to both. 

Gandhiji also stressed the positive impact that the struggles against untouchability would have on the communal and other questions. Non- Hindus were treated by Hindus as untouchables `in some way or the other,' especially in matters of food and drink, and non-Hindus certainly took note of this fact. Hence, `if untouchability is removed, it must result in bringing all Indians together.' Increasingly, he also began to point out that untouchability was only one form of the distinctions that society made between man and man; it was a product of the grading of society into high and low. To attack untouchability was to oppose `this high-and-lowness.' That is why `the phase we are now dealing with does not exhaust all the possibilities of struggle.' In keeping with his basic philosophy of non-violence, and being basically a 19th century liberal and believer in rational discussion, Gandhiji was opposed to exercising compulsion even on the orthodox supporters of untouchability, whom he described as the Sanatanists. Even they had to be tolerated and converted and won over by persuasion, `by appealing to their reason and their hearts.' His fasts, he said, were not directed against his opponents or meant to coerce them into opening temples and wells etc.; they were directed towards friends and followers to goad them and inspire them to redouble their anti-untouchability work. 

Gandhiji's Harijan campaign included a programme of internal reform by Harijans: promotion of education, cleanliness and hygiene, giving up the eating of carrion and beef, giving up liquor and the abolition of untouchability among themselves. But it did not include a militant struggle by the Harijans themselves through Satyagraha, breaking of caste taboos, mass demonstrations, picketing, and other forms of protests. At the same time, he was aware that his Harijan movement `must cause daily increasing awakening among the Harijans' and that in time `whether the savarna Hindus like it or not, the Harijans would make good their position.' 

Gandhiji repeatedly stressed that the Harijan movement was not a political movement but a movement to purify Hinduism and Hindu society. But he was also aware that the movement `will produce great political consequences,' just as untouchability poisoned `our entire social and political fabric.' In fact, not only did Harijan work, along with other items of constructive work, enable the Congress cadre to keep busy in its non-mass movement phases, it also gradually carried the message of nationalism to the Harijans, who also happened to be agricultural labourers in most parts of the country, leading to their increasing participation in the national as well as peasant movements.
