
\chapter{Introduction}

The Indian national movement was undoubtedly one of the biggest mass movements modern Society has ever seen, It was a movement which galvanized millions of People of all classes and ideologies into political action and brought to its knees a mighty colonial empire. Consequently, along with the British, French, Russian, Chine, Cuban and Vietnam revolutions, it is of great relevance to those wishing to alter the existing political and social structure.

Various aspects of the Indian national movement, especially Gandhian political strategy, are particularly relevant to these movements in societies that broadly function within the confines of the rule of law, and are characterized by a democratic and basically civil libertarian polity. But it is also relevant to other societies. We know for a fact that even Lech Walesa consciously tried to incorporate elements of Gandhian strategy in the Solidarity Movement in Poland.

The Indian national movement, in fact, provides the only actual historical example of a semi-democratic or democratic type of political structure being successfully replaced or transformed. It is the only movement where the broadly Gramscian theoretical perspective of position was successfully practiced a war in a single historical moment of revolution, but through prolonged popular struggle on a moral, political and ideological level; where reserves of counter-hegemony were built up over the years through progressive stages; where the phases of struggle alternated with `passive' phases.

The Indian national movement is also an example of how the constitutional space offered by the existing structure could be used without getting co-opted by it. It did not completely reject this space; as such rejection in democratic societies entails heavy costs in terms of hegemonic influence and often leads to isolation but entered it and used it effectively in combination with non-constitutional struggle to overthrow the existing structure.

The Indian national movement is perhaps one of the best examples of the creation of an extremely wide movement with a common aim in which diverse political and ideological currents could exist and work and simultaneously continue to contend for overall ideological political hegemony over it. While an intense debate on all basic Issues was allowed, the diversity and tension did not weaken the cohesion and striking power of the movement; on the contrary, this diversity and atmosphere of freedom and debate became a major source of its strength.

Today, over forty years after independence, we are still close enough to the freedom struggle to feel its warmth and yet far enough to be able to analyze it coolly, and with the advantage of hindsight. Analyze it we must, for our past, present and future are inextricably linked to it. Men and women in every age and society make their own history, but they do not make it in a historical vacuum, de novo. Their efforts, however innovative, at finding solutions to their problems in the present and charting out their future, are guided and circumscribed, molded and conditioned, by their respective histories, their inherited economic, political and ideological structures. To make myself clearer, the path that India has followed since 1947 has deep roots in the struggle for independence. The political and ideological features, which have had a decisive impact on post-independence development, are largely a legacy of the freedom struggle. It is a legacy that belongs to all the Indian people, regardless of which party or group they belong to now, for the `party' which led this struggle from 1885 to 1947 was not then a party but a movement all political trends from the Right to the Left were incorporated in it.

\begin{center}*\end{center}

\paragraph*{}
What are the outstanding features of the freedom struggle? A major aspect is the values and modern ideas on which the movement itself was based and the broad socio-economic and political vision of its leadership (this vision was that of a democratic, civil libertarian and secular India, based on a self-reliant, egalitarian social order and an independent foreign policy). The movement popularized democratic ideas and institutions in India. The nationalists fought for the introduction of a representative government on the basis of popular elections and demanded that elections be based on adult franchise. The Indian National Congress was organized on a democratic basis and in the form of a parliament. It not only permitted but encouraged the free expression of opinion within the party and the movement; some of the most important decisions in its history were taken after heated debates and on the basis of open voting.

From the beginning, the nationalists fought against attacks by the State on the freedoms of the Press, expression, and association, and made the struggle for these freedoms an integral part of the national movement. During their brief spell in power, from 1937--39, the Congress ministries greatly extended the scope of civil liberties. The defense of civil liberties was not narrowly conceived in terms of one political group but was extended to include the defense of other groups whose views were politically and ideologically different. The Moderates defended Tilak, the Extremist, and non-violent Congressmen passionately defended revolutionary terrorists and communists alike during their trials. In 1928, the Public Safety Bill and Trade Disputes Bill was opposed not only by Motilal Nehru but also by conservatives like Madan Mohan Malaviya and M.R. Jayakar. It was this strong civil libertarian and democratic tradition of the national movement which was reflected in the Constitution of independent India.

The freedom struggle was also a struggle for economic development. In time an economic ideology developed which was to dominate the views of independent India. The national movement accepted, with near unanimity, the need to develop India on the basis of industrialization which in turn was to be independent of foreign capital and was to rely on the indigenous capital goods sector. A crucial role was assigned to the public sector and, in the 1930's, there was a commitment to economic planning.

From the initial stages, the movement adopted a pro-poor orientation which was strengthened with the advent of Gandhi and the rise of the leftists who struggled to make the movement adopt a socialist outlook. The movement also increasingly moved towards a programme of radical agrarian reform. However, socialism did not, at any stage, become the official goal of the Indian National Congress thought there was a great deal of debate around it within the national movement and the Indian National Congress during the 1930s and 1940s. For various reasons, despite the existence of a powerful leftist trend within the nationalist mainstream, the dominant vision within the Congress did not transcend the parameters of a capitalist conception of society.

The national movement was, from its early days, fully committed to secularism. Its leadership fought hard to inculcate secular values among the people and opposed the growth of communalism. And, despite the partition of India and the accompanying communal holocaust, it did succeed in enshrining secularism in the Constitution of free India.

It was never inward looking. Since the days of Raja Rammohan Roy, Indian leaders had developed a broad international outlook. Over the years, they evolved a policy of opposition to imperialism on a world-wide scale and solidarity with anti-colonial movements in other parts of the world. They established the principle that Indians should hate British imperialism but not the British people. Consequently, they were supported by a large number of English men, women and political groups. They maintained close links with the progressive, anti-colonial and anti-capitalist forces of the world. A non-racist, anti-imperialist outlook, which continues to characterize Indian foreign policy, was thus part of the legacy of the anti-imperialist struggle.

\begin{center}*\end{center}

\paragraph*{}
This volume has been written within a broad framework that the authors, their colleagues, and students have evolved and are in the process of evolving through ongoing research on and study of the Indian national movement. We have in the preparation of this volume extensively used existing published and unpublished monographs, archival material, private papers, and newspapers. Our understanding also owes a great deal to our recorded interviews with over 1,500 men and women who participated in the movement from 1918 onwards. However, references to these sources have, for the ease of the reader and due to constraints of space, been kept to the minimum and, in fact, have been confined mostly to citations of quoted statements and to works readily available in a good library.

For the same reason, though the Indian national movement has so far been viewed from a wide variety of historiographic perspectives ranging from the hard-core imperialist to the Marxist, and through various stereotypes and shibboleths about it exist, we have generally avoided entering into a debate with those whose positions and analyses differ from our own --- except occasionally, as in the case of CHAPTER \ref{chapter:CH04}, on the origin of the Indian National Congress, which counters the hoary perennial theory of the Congress being founded as a safety valve. In all fairness to the reader, we have only briefly delineated the basic contours of major historiographical trends, indicated our differences with them, and outlined the alternative framework within which this volume has been written.

\begin{center}*\end{center}

\paragraph*{}
We differ widely from the imperialist approach which first emerged in the official pronouncements of the Viceroys, Lords Dufferin, Curzon and Minto, and the Secretary of State, George Hamilton. It was first cogently put forward by V. Chirol, the Rowlatt (Sedition) Committee Report, Verney Lovett, and the Montague-Chelmsford Report. It was theorized, for the first time, by Bruce T. McCully, an American scholar, in 1940. Its liberal version was adopted by Reginald Coupland and, after 1947, by Percival Spear, while its conservative version was refurbished and developed at length by Anil Seal and J.A. Gallagher and their students and followers after 1968. Since the liberal version is no longer fashionable in academic circles, we will ignore it here due to the shortage of space.

The conservative colonial administrators and the imperialist school of historians, popularly known as the Cambridge School, deny the existence of colonialism as an economic, political, social and cultural structure in India. Colonialism is seen by them primarily as foreign rule. They either do not see or vehemently deny that the economic, social, cultural and political development of India required the overthrow of colonialism. Thus, their analysis of the national movement is based on the denial of the basic contradiction between the interests of the Indian people and of British colonialism and the causative role this contradiction played in the rise of the national movement. Consequently, they implicitly or explicitly deny that the Indian national movement represented the Indian side of this contradiction or that it was anti-imperialist that is, it opposed British imperialism in India. They see the Indian struggle against imperialism as a mock battle (`mimic warfare'), ``a Dassehra duel between two hollow statues locked in motiveless and simulated combat.'' The denial of the central contradiction vitiates the entire approach of these scholars through their meticulous research does help others to use it within a different framework.

The imperialist writers deny that India was in the process of becoming a nation and believe that what is called India, in fact, consisted of religions, castes, communities, and interests. Thus, the grouping of Indian politics around the concept of an Indian nation or an Indian people or social classes is not recognized by them. There were instead, they said, pre-existing Hindu-Muslim, Brahmin, Non-Brahmin, Aryan, Bhadralok (cultured people) and other similar identities. They say that these prescriptive groups based on caste and religion are the real basis of political organization and, as such, caste and religion-based politics are primary and nationalism a mere cover. As Seal puts it: `What from a distance appear as their political strivings were often, on close examination, their efforts to conserve or improve the position of their own prescriptive groups.'(This also makes Indian nationalism, says Seal, different from the nationalism of China, Japan, the Muslim countries and Africa).

If the Indian national movement did not express the interests of the Indian people vis-a-vis imperialism, then whose interests did it represent? Once again the main lines of the answer and argument were worked out by the late 19th century and early 20th century officials and imperialist spokesmen. The national movement, assert the writers of the imperialist school, was not a people's movement but a product of the needs and interests of the elite groups who used it to serve either their own narrow interests or the interests of their prescriptive groups. Thus, the elite groups and their needs and interests provide the origin as well as the driving force of the idea, ideology, and movement of nationalism. These groups were sometimes formed around religious or caste identities and sometimes through political connections built around patronage. But, in each case, these groups had a narrow, selfish interest in opposing British rule or each other. Nationalism, then, is seen primarily as a mere ideology which these elite groups used to legitimize their narrow ambitions and to mobilize public support. The national movement was merely an instrument used by the elite groups to mobilize the masses and to satisfy their own interests.

Gallagher, Seal and their students have added to this viewpoint. While Dufferin, Curzon, Chirol, Lovett, McCully, and B.B. Misra talked of the frustrated educated middle classes using nationalism to fight the `benevolent Raj', Seal develops a parallel view, as found in Chirol and the Rowlett Committee Report, that the national movement represented the struggle of one Indian elite group against another for British favors. As he puts it: `It is misleading to view these native mobilizations as directed chiefly against foreign over-lordship. Much attention has been paid to the apparent conflicts between imperialism and nationalism; it would be at least equally profitable to study their real partnership'. The main British contribution to the rise and growth of the national movement, then, was that British rule sharpened mutual jealousies and struggles among Indians and created new fields and institutions for their mutual rivalry.

Seal, Gallagher and their students also extended the basis on which the elite groups were formed. They followed and added to the viewpoint of the British historian Lewis Namier and contended that these groups were formed on the basis of patron-client relationships. They theorize that, as the British extended administrative, economic and political power to the localities and provinces, local potentates started organizing politics by acquiring clients and patrons whose interests they served, and who in turn served their interests. Indian politics began to be formed through the links of this patron-client chain. Gradually, bigger leaders emerged who undertook to act as brokers to link together the politics of the local potentates, and eventually, because British rule encompassed the whole of India, all-India brokers emerged. To operate successfully, these all-India brokers needed province level brokers at the lower levels and needed to involve clients in the national movement. The second level leaders are also described as sub-contractors. Seal says the chief political brokers were Mahatma Gandhi, Jawaharlal Nehru, and Sardar Vallabhbhai Patel. And according to these historians, the people themselves, those whose fortunes were affected by all this power brokering, came in only in 1918. After that, we are told, their existential grievances such as war, inflation, disease, drought or depression --- which had nothing to do with colonialism --- were cleverly used to bamboozle them into participating in this factional struggle of the potentates.

Thus, this school of historians treats the Indian national movement as a cloak for the struggle for power between various sections of the Indian elite, and between them and the foreign elite, thus effectively denying its existence and legitimacy as a movement of the Indian people for the overthrow of imperialism and for the establishment of an independent nation-state. Categories of nation, class, mobilization, ideology, etc., which are generally used by historians to analyze national movements and revolutionary processes in Europe, Asia and Africa are usually missing from their treatment of the Indian national movement. This view not only denies the existence of colonial exploitation and underdevelopment, and the central contradiction, but also any idealism on the part of those who sacrificed their lives for the anti-imperialist cause. As S. Gopal has put it: `Namier was accused of taking the mind out of politics; this School has gone further and taken not only the mind but decency, character integrity and selfless commitment out of the Indian national movement'. Moreover, it denies any intelligent or active role in the mass of workers, peasant lower middle class, and women in the anti-imperialist Struggle. They are treated as child-people or dumb creatures who had no perception of their needs and interests. One wonders why the colonial rulers did not succeed in mobilizing them behind their own politics!

\begin{center}*\end{center}

\paragraph*{}
A few historians have of late initiated a new trend, described by its proponents as subaltern, which dismisses all previous historical Writing, including that based on a Marxist perspective, as elite historiography, and claims to replace this old, `bunkered' historiography with what it claims is a new people's or subaltern approach.

For them, the basic contradiction in Indian society in the colonial epoch was between the elite, both Indian and foreign, on the one hand, and the subaltern groups, on the other, and not between Colonialism and the Indian people. They believe that the Indian people were never united in a common anti-imperialist struggle, that there was no such entity as the Indian national movement. Instead, they assert that there were two distinct movements or streams, the real anti-imperialist stream of the subalterns and the bogus national movement of the elite. The elite stream, led by the `official' leadership of the Indian National Congress, was little more than a cloak for the struggle for power among the elite. The subaltern school's characterization of the national movement bears a disturbing resemblance to the imperialist and neo-imperialist characterization of the national movement, the only difference being that, while neo-imperialist historiography does not split the movement but characterizes the entire national movement in this fashion, `subaltern' historiography first divides the movement into two and then accepts the neo-imperialist characterization for the elite' Stream. This approach is also characterized by a generally ahistorical glorification of oil forms of popular militancy and consciousness and an equally ahistorical contempt for all forms of initiative and activity the intelligentsia organized Party leadership and other `elites'.

Consequently, it too denies the legitimacy of the actual, historical anti-colonial struggle that the Indian people waged. The new school, which promised to write a history based on the people's own consciousness, is yet to tap new sources that may be more reflective of popular perceptions; its `new' writing continues to be based on the same old `elite' sources.

\begin{center}*\end{center}

\paragraph*{}
The other major approach is nationalist historiography. In the colonial period, this school was represented by political activists such as Lajpat Rai, A.C. Mazumdar, R.G. Pradhan, Pattabhi Sitaramayya, Surendranath Banerjea, C.F. Andrews, and Girija Mukerji. More recently, B.R.Nanda, Bisheshwar Prasad, and Amles Tripathi have made distinguished contributions within the framework of this approach. The nationalist historians, especially the more recent ones, show an awareness of the exploitative character of colonialism, but on the whole, they feel that the national movement was the result of the spread and realization of the idea or spirit of nationalism or liberty. They also take full cognizance of the process of India becoming a nation and see the national movement as a movement of the people.

Their major weakness, however, is that they tend to ignore or, at least, underplay the inner contradictions of Indian society both in terms of class and caste. They tend to ignore the fact that while the national movement represented the interests of the people or nation as a whole (that is, of all classes vis-a-vis colonialism) it only did so from a particular class perspective, and that, consequently, there was a constant struggle between different social, ideological perspectives for hegemony over the movement. They also usually take up the position adopted by the right wing of the national movement and equate it with the movement as a whole. Their treatment of the strategic and ideological dimensions of the movement is also inadequate.

\begin{center}*\end{center}

\paragraph*{}
The Marxist school emerged on the scene later. Its foundations, so far as the study of the national movement is concerned, were laid by R. Palme Dutt and A.R. Desai; but several others have developed it over the years. Unlike the imperialist school, the Marxist historians clearly see the primary contradiction as well as the process of the nation-in-the-making and unlike the nationalists, they also take full note of the inner contradictions of Indian society. However, many of them and R. Palme Dutt, in particular, are not able to fully integrate their treatment of the Primary anti-imperialist contradiction and the secondary' inner contradictions and tend to counterpose the anti-imperialist struggle to the class or social struggle. They also tend to see the movement as a structured bourgeois movement, if not the bourgeoisie's movement, and miss it's open-ended and all class character. They see the bourgeoisie as playing the dominant role in the movement --- they tend to equate or conflate the national leadership, with the bourgeoisie or capitalist class. They also Interpret the class character of the movement in terms of its forms of Struggle (i.e., in its nonviolent character) and in the fact that it made strategic retreats and compromises. A few take an even narrower view. They suggest that access to financial resources determined the ability to influence the Course and direction of nationalist politics. Many of the Marxist writers also do not do an actual detailed historical investigation of the strategy, programme, ideology extent and forms of mass mobilization, and strategic and tactical maneuvers of the national movement.

\begin{center}*\end{center}

\paragraph*{}
Our own approach, while remaining, we believe, within the broad Marxist tradition, tries to locate the issues --- of the nature of the contradictions in colonial India; the relationship between the primary and the secondary contradictions, the class character of the movement; the relationship between the bourgeois and other social classes and the Indian National Congress and its leadership i.e., the relationship between class and party; the relationship between forms of struggle (including non-violence) and class character ideology, strategy and mass character of the movement and so on in a framework which differs in many respects from the existing approaches including the classical Marxist approach of Palme Dutt and A.R.Desai. The broad contours of that framework are outlined below.

\begin{center}*\end{center}

\paragraph*{}
In our view, India's Freedom Struggle was basically the result of a fundamental contradiction between the interests of the Indian people and that of British colonialism From the beginning itself, India's national leaders grasped this contradiction They were able to see that India was regressing economically and undergoing a process of underdevelopment. In time they were able to evolve a scientific analysis of colonialism. In fact, they were the first in the 19th century to develop an economic critique of colonialism and lay bare its complex structure. They were also able to see the distinction between colonial policy and the imperatives of the colonial structure. Taking the social experience of the Indian people as colonized subjects and recognizing the common interests of the Indian people vis-a-vis colonialism, the national leaders gradually evolved a clear-cut anti-colonial ideology on which they based the national movement. This anti-colonial ideology and critique of colonialism were disseminated during the mass phase of the movement.

The national movement also played a pivotal role in the historical process through which the Indian people got formed into a nation or a people. National leaders from Dadabhai Naoroji, Surendranath Banerjee, and Lokmanya Tilak to Gandhiji and Nehru accepted that India was not yet a fully structured nation but a nation-in-the-making and that one of the major objectives and functions of the movement was to promote the growing unity of the Indian people through a common struggle against colonialism. In other words, the national movement was seen both as a product of the process of the nation-in-the-making and as an active agent of the process. This process of the nation-in-the-making was never counter-posed to the diverse regional, linguistic and ethnic identities in India. On the contrary, the emergence of a national identity and the flowering of other narrower identities were seen as processes deriving strength from each other. The pre-nationalist resistance to colonial rule failed to understand the twin phenomena of colonialism and the nation­in-the-making. In fact, these phenomena were not visible, or available to be grasped, on the surface. They had to be grasped through hard analysis. This analysis and political consciousness based on it were then taken to the people by intellectuals who played a significant role in arousing the inherent, instinctive, nascent, anti-colonial consciousness of the masses.

\begin{center}*\end{center}

\paragraph*{}
As explained in CHAPTER \ref{chapter:CH38}, the Indian national movement had certain specific though untheorized, the strategy of struggle within which various phases and forms of struggle were integrated, especially after 1918. This strategy was formed by the waging of a hegemonic struggle for the mi and hearts of the Indian people. The purpose was to destroy the two basic constituents of colonial hegemony or the belief system through which the British secured the acquiescence of the Indian people in their rule: that British rule was benevolent or for the good of the Indians and that it was invincible or incapable of being overthrown. Replying to the latter aspect, Jawaharlal Nehru wrote in The Discovery of India: `The essence of his (Gandhi's) teaching was fearlessness ... not merely bodily courage but the absence of fear from the mind ... But the dominant impulse In India under British rule was that of fear, pervasive, oppressing, strangling fear; fear of the army, the police, the widespread secret service; fear of the official class; fear of laws meant to suppress and of prison; fear of the landlord's agents: fear of the money­lender; fear of unemployment and starvation, which were always on the threshold. It was against this all-pervading fear that Gandhiji's quiet and determined voice was raised: Be not afraid.' Relying basically on Gramsci we have used the concept of hegemony in an amended form since the exercise of hegemony in a colonial society both by the colonial rulers and the opposing anti-imperialist forces occurs in a context different from an Independent Capitalist Society. The concept of hegemony, as used by us, means an exercise of leadership as opposed to pure domination. More specifically it relates to the capacity as also the strategy, through which the rulers or dominant classes or leadership of popular movements organize consent among the ruled or the followers and exercise moral and ideological, leadership over them. According to Gramsci, in the case of class hegemony, the hegemonic class is able to make compromises with a number of allied classes by taking up their causes and interests and thus emerges as the representative of the current Interests of the entire society, It unifies these allies under its own leadership through `a web of institutions, social relations and ideas'. The Gramscian concept of hegemony is of course opposed to an economist notion of movements and ideologies which constitute primarily on immediate class interests in politics and ideology and tend to make a direct correlation between the two and sometimes even to derive the latter from the former.

\begin{center}*\end{center}

\paragraph*{}
And how was nationalist hegemony to be evolved? In the case of a popular anti-imperialist movement, we believe, the leadership, acting within a particular ideological framework, exercises hegemony by taking up the anti-colonial interests of the entire colonized people and by unifying them by adjusting the class interests of the different classes, strata, and groups constituting the colonized people. The struggle for ideological hegemony within a national movement pertains to changing the relative balance of advantages flowing from such adjustment and not to the question of adjustment itself. In the colonial situation, the anti-imperialist struggle was primary and the social --- class and caste --- struggles were secondary, and, therefore, struggles within Indian society were to be initiated and then compromised rather than carried to an extreme, with all mutually hostile classes and castes making concessions.

Further, the nationalist strategy alternated between phases of massive mass struggle which broke existing laws and phases of intense political-agitational work within the legal framework. The strategy accepted those mass movements by their very nature had ups and downs, troughs and peaks, for it was not possible for the vast mass of people to engage continuously in a Long-drawn-out extra-legal struggle that involved considerable sacrifice. This strategy also assumed freedom struggle advancing through stages, though the country was not to advance to freedom till the threshold of the last stage was crossed. Constructive work --- organized around the promotion of khadi, national education, Hindu-Muslim unity, the boycott of foreign cloth and liquor, the social upliftment of the Harijans (low caste `untouchables') and tribal people and the struggle against untouchability --- formed an important part of the nationalist strategy, especially during its constitutional phases. This strategy also involved participation in the colonial constitutional structure without falling prey to it or without getting co-opted by it.

And what was the role of non-violence? It was not, we believe, a mere dogma of Gandhiji nor was it dictated by the interests of the propertied classes. It was an essential part of a movement whose strategy involved the waging of a hegemonic struggle based on a mass movement which mobilized the people to the widest possible extent.

The nationalist strategy of a war of position, of hegemonic struggle, was also linked to the semi-hegemonic or legal authoritarian character of the colonial state which functioned through the rule of law, a rule-bound bureaucracy and a relatively independent judiciary while simultaneously enacting and enforcing extremely repressive laws and which extended a certain amount of civil liberties in normal times and curtailed them in periods of mass struggle. It also constantly offered constitutional and economic concessions though it always retained the basics of state power in its own hands.

Seen from this point of view, the peaceful and negotiated nature of the transfer of power in 1947 was no accident, nor was it the result of a compromise by a tired leadership, but was the result of the character and strategy of the Indian national movement, the culmination of a war of position where the British recognized that the Indian people were no longer willing to be ruled by them and the Indian part of the colonial apparatus could no longer be trusted to enforce a rule which the people did not want. The British recognized that they had lost the battle of hegemony or war of position and decided to retreat rather than make a futile attempt to rule such a vast country by a threat of a sword that was already breaking in their hands.

Seen in this strategic perspective, the various negotiations and agreements between the rulers and the nationalist leadership, the retreat of the movement in 1922 and 1934, the compromise involved in the Gandhi- Irwin Pact and the working of constitutional reforms after 1922 and in 1937 also have to be evaluated differently from that done by writers such as R. Palme Dutt. This we have done in the chapters dealing with these issues.

\begin{center}*\end{center}

\paragraph*{}
The Indian national movement was a popular, multi-class movement. It was not a movement led or controlled by the bourgeoisie, nor did the bourgeoisie exercise exclusive influence over it. Moreover, its multi-class, popular, and open-ended character meant that it was open to the alternative hegemony of socialist ideas.

The national movement did, in fact, undergo a constant ideological transformation. In the late 1920s and l930s, Jawaharlal Nehru, Subhas Bose, the Communists, the Congress Socialists, and other Left-minded socialist groups and individuals made an intense effort to give the movement arid the National Congress a socialistic direction. One aspect of this was the effort to organize the peasants in Kisan sabhas, the workers in trade unions and the youth in youth leagues and student unions. The other was the effort to give the entire national movement a socialist ideological orientation, to make it adopt a socialist vision of free India. This effort did achieve a certain success and socialist ideas spread widely and rapidly. Almost all young intellectuals of the 1930s and 1940s belonged to some shade of pink or red. Kisan sabhas and trade unions also tended to shift to the Left. Also important in this respect was the constant development of Gandhiji's ideas in a radical direction. But, when freedom came, the Left had not yet succeeded, for various reasons, in establishing the hegemony of socialist ideas over the national movement and the dominant vision within the movement remained that of bourgeois development. Thus, we suggest, the basic weakness of the movement was located in its ideological structure.

\begin{center}*\end{center}

\paragraph*{}
The Indian National Congress, is a movement and not just a party, included within its fold, individuals and groups which subscribed to widely divergent political and ideological perspectives. Communists, Socialists, and Royists worked within the Congress as did constitutionalists like Satyamurthy and K.M. Munshi. At the same time, the national movement showed a remarkable capacity to remain united despite diversity. A lesson was learned from the disastrous split of 1907 and the Moderates and Extremists, constitutionalists and non-constitutionalists and leftists and rightists did not split the Indian National Congress thereafter, even in the gravest crises.

There were, of course, many other streams flowing into the swelling river of India's freedom struggle. The Indian National Congress was the mainstream but not the only stream. We have discussed many of these streams in this volume: the pre-Congress peasant and tribal movements, the Revolutionary Terrorists, the Ghadar and Home Rule Movements, the Akali and Temple Reform movements of the 1920s, the struggle in the legislatures and in the Press, the peasant and working-class struggles, the rise of the Left inside and outside the Congress, the state people's movements, the politics of the capitalist class, the Indian National Army, the RIN Revolt, etc. We have, as a matter of fact, devoted nearly half of this volume to political movements which formally happened outside the Congress. But we do not treat these `non-Congress' movements as `parallel' streams, as some have maintained, Though they were outside the Congress, most of them were not really separate from it. They cannot be artificially counterposed to the movement led by the Congress, which, with all its positive and negative features, was the actual anti-imperialist movement of the Indian people incorporating their historical energies and genius, as in the case with any genuine mass movement.

In fact, nearly all these movements established a complex relations with the Congress mainstream and at no stage became alternatives to the Congress. They all became an integral part of the Indian national movement. The only ones which may be said to have formed part of an alternative stream of politics were the communal and casteist movements which were not nationalist or anti-imperialist but in fact betrayed loyalist pro-colonial tendencies.

\begin{center}*\end{center}

\paragraph*{}
In time, the Indian National Movement developed into one of the greatest mass movements in world history. It derived its entire strength, especially after 1918, from the militancy and self-sacrificing spirit of the masses. Satyagraha as a form of struggle was based on the active participation of the people and on the sympathy and support of the non-participating millions. Several Satyagraha campaigns --- apart from innumerable mass agitational campaigns --- were waged between 1919 and 1942. Millions of men and women were mobilized in myriad ways; they sustained the movement by their grit and determination. Starting out as a movement of the nationalist intelligentsia, the national movement succeeded in mobilizing the youth, women, the urban petty bourgeoisie, the urban and rural poor, urban and rural artisans, peasants, workers, merchants, capitalists, and a large number of small landlords.

The movement in its various forms and phases took modem politics to the people. It did not, in the main, appeal to their pre­modem consciousness based on religion, caste and locality or loyalty to the traditional rulers or chieftains. It did not mobilize people ideologically around religion, caste or region. It fought for no benefits on that basis. People did not join it as Brahmins, or Patidars, or Marathas; or Harijans. It made no appeal to religious or caste identities, though in some cases caste structure was used in villages to enforce discipline in a movement whose motivation and demands had nothing to do with caste.

Even while relying on the popular consciousness, experience, perception of oppression and the needed remedies, on notions of good rule or utopia the movement did not merely reflect the existing consciousness but also made every effort to radically transform it in the course of the struggle. Consequently, it created space for as well as got integrated with other modern, liberationist movements --- movements of women, youth, peasants, workers, Harijans and other lower castes. For example, the social and religious reform movements which developed during the 19th century as part of the defense against colonialization of Indian culture merged with the national movement. Most of them became a part of the broad spectrum of the national movement in the 20th century. But, in the end, the national movement had to surrender in part before communalism. We have tried to examine, at some length, the rise and growth of communalism and the reasons for the partial failure of the national movement to counter its challenge. The national movement also failed to undertake a cultural revolution despite some advances in the social position of women and lower castes. Moreover, it was unable to take the `cultural defense' of the late 19th century's social and religious reforms back to the rationalist critical phase of the early 19th century. It also could not fully integrate the cultural struggle with the political struggle despite Gandhiji's efforts in that direction.

The national movement was based on an immense faith in the capacity of the Indian people to make sacrifices. At the same time, it recognized the limits on this capacity and did not make demands based on unrealistic and romantic notions. After all, while a cadre-based movement can base itself on exceptional individuals capable of making uncommon sacrifices, a mass movement, even while having exceptional individuals as leaders, has to rely on the masses with all their normal strengths and weaknesses. It is these common people who hail to perform uncommon tasks. `The nation has got the energy of which you have no conception but I have,' Gandhiji told K.F. Nariman in 1934. At the same time, he said, a leader should not `put an undue strain on the energy.'

As a mass movement, the Indian national movement was able to tap the diverse energies, talents, and capacities of a large variety of people. It had a place for all --- old and young, rich and poor, women and men, the intellectuals and the masses. People participated in it in varied ways: from jail-going Satyagraha and picketing to participation in public meetings and demonstrations, from going on hartals and strikes to cheering the jathas of Congress volunteers from the sidelines, from voting for nationalist candidates in municipal, district, provincial and central elections to participating in constructive programmes, from becoming 4-anna (25 paise) members of the Congress to wearing khadi and a Gandhi cap, from contributing funds to the Congress to feeding and giving shelter to Congress agitators from distributing and reading the Young India and the Harijan or illegal Patrikas (bulletins) to staging and attending nationalist dramas and poetry festivals, and from writing and reading nationalist novels, poems and stones to walking and singing in the prabhat pheries (parties making rounds of a town or part of it).

The movement and the process of mass mobilization were also an expression of the immense creativity of the Indian people. They were able to give a full play to their innovativeness and initiative.

The movement did not lack exceptional individuals, both among leaders and followers. It produced thousands of martyrs. But as heroic were those who worked for years, day after day, in an unexciting humdrum fashion, forsaking their homes and Careers, and losing their lands and very livelihood --- whose families were often short of daily bread and whose children went without adequate education or health care.
