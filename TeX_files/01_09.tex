\chapter{Propaganda in the Legislature}
\begin{multicols}{2}

Legislative Councils in India had no real official power till 1920. Yet, work done in them by the nationalists helped the growth of the national movement.

\begin{center}*\end{center}

\paragraph*{}
The Indian Councils Act of 1861 enlarged the Governor-General's Executive Council for the purpose of making laws. The Governor-General could now add from six to twelve members to the Executive Council. At least half of these nominations had to be non-officials, Indian or British. This council came to be known as the Imperial Legislative Council. It possessed no powers at all. It could not discuss the budget or a financial measure or any other important bill without the previous approval of the Government. It could not discuss the actions of the administration. It could not, therefore, be seen as some kind of parliament, even of the most elementary kind. As if to underline this fact, the Council met, on an average, for only twenty-five days in a year till 1892.

The Government of India remained, as before 1858, an alien despot. Nor was this accidental. While moving the Indian Councils Bill of 1861, the Secretary of State for India, Charles Wood, said: `All experience reaches us that where a dominant race rules another, the mildest form of Government is despotism.' A year later he wrote to Elgin, the Viceroy, ``that the only government suitable for such a state of things as exists in India a despotism controlled from home.'' This `despotism controlled from home' was to remain the fundamental feature of the Government of India till 1947-08-15.

What was the role of Indian members in this Legislative Council? The Government had decided to add them in order to represent Indian views, for many British officials and statesmen had come to believe that one reason for the Revolt of 1857 was that Indian views were not known to the rulers. But, in practice, the Council did not serve even this purpose. Indian members were few in number --- in thirty years, from 1862 to 1892, only forty-five Indians were nominated to it. Moreover, the Government invariably chose rulers of princely states or their employees, big zamindars, big merchants or retired high government officials as Indian members. Only a handful of political figures and independent intellectuals such as Syed Ahmed Khan (1878--82), Kristodas Pal (1883), V.N. Mandlik (1884--87), K.L. Nulkar (1890--91) and Rash Behari Ghosh (1892) were nominated. The overwhelming majority of Indian nominees did not represent the Indian people or emerging nationalist opinion. It was, therefore, not surprising that they completely toed the official line. There is the interesting story of Raja Dig Vijay Singh of Balarampur --- nominated twice to the Council --- who did not know a word of English. When asked by a relative how he voted one way or the other, he replied that he kept looking at the Viceroy and when the Viceroy raised his hand he did so too and when he lowered it he did the same!

The voting record of Indian nominees on the Council was poor. When the Vernacular Press Bill came up before the Council, only one Indian member, Maharaja Jotendra Mohan Tagore, the leader of the zamindari-dominated British Indian Association was present. He voted for it. In 1885, the two spokesmen of the zamindars in the Council helped emasculate the pro-tenant character of the Bengal Tenancy Bill at a time when nationalist leaders like Surendranath Banerjee were agitating to make it more pro-tenant. In 1882, Jotendra Mohan Tagore and Durga Charan Laha, the representative of Calcutta's big merchants, opposed the reduction of the salt tax and recommended the reduction of the licence tax on merchants and professionals instead. The nationalists were demanding the opposite. In 1888, Peary Mohan Mukherjea and Dinshaw Petit, representatives of the big zamindars and big merchants respectively, supported the enhancement of the salt tax along with the non-official British members representing British business in India.

By this time nationalists were quite active in opposing the salt tax and reacted strongly to this support. In the newspapers and from the Congress platform they described Mukherjea and Petit as `gilded shams' and magnificient non-entities. They cited their voting behavior as proof of the nationalist contention that the existing Legislative Councils were unrepresentative of Indian opinion. Madan Mohan Malaviya said at the National Congress session of 1890: `We would much rather that there were no non-official members at all on the Councils than that there should be members who are not in the least in touch with people and who ... betray a cruel want of sympathy with them'. Describing Mukherjea and petit as `these big honourable gentlemen, enjoying private incomes and drawing huge salaries,' he asked rhetorically: `Do you think, gentlemen, such members would be appointed to the Council if the people were allowed any voice in their selection?' The audience shouted `No, no, never.'

However, despite the early nationalists believing that India should eventually become self-governing, they moved very cautiously in putting forward political demands regarding the structure of the state, for they were afraid of the Government declaring their activities seditious and disloyal and suppressing them. Till 1892, their demand was limited to the expansion and reform of the Legislative Councils. They demanded wider participation in them by a larger number of elected Indian members as also wider powers for the Councils and an increase in the powers of the members to `discuss and deal with' the budget and to question and criticize the day-to-day administration.

\begin{center}*\end{center}

\paragraph*{}

The nationalist agitation forced the Government to make some changes in legislative functioning by the Indian Councils Act of 1892. The number of additional members of the Imperial and Provincial Legislative Councils was increased from the previous six to ten to ten to sixteen. A few of these members could be elected indirectly through municipal committees, district boards, etc., but the official majority remained. The members were given the right to discuss the annual budget but they could neither vote on it nor move a motion to amend it. They could also ask questions but were not allowed to put supplementary questions or to discuss the answers. The `reformed' Imperial Legislative Council met, during its tenure till 1909, on an average for only thirteen days in a year, and the number of unofficial Indian members present was only five out of twenty- four!

The nationalists were totally dissatisfied with the Act of 1892. They saw in it a mockery of their demands. The Councils were still impotent; despotism still ruled. They now demanded a majority for non-official elected members with the right to vote on the budget and, thus, to the public purse. They raised the slogan `no taxation without representation.' Gradually, they raised their demands. Many leaders --- for example Dadabhai Naoroji in 1904, G.K. Gokhale in 1905 and Lokamanya Tilak in 1906 began to put forward the demand for self government the model of the self-governing colonies of Canada and Australia.

\begin{center}*\end{center}

\paragraph*{}

Lord Dufferin, who had prepared the outline of the Act of 1892, and other British statesmen and administrators, had seen in the Legislative Council a device to incorporate the more vocal Indian political leaders into the colonial political structure where they could, in a manner of Speaking let off their political steam. They knew that the members of the Councils enjoyed no real powers; they could only make wordy speeches and indulge in empty rhetorics, and the bureaucracy could afford to pay no attention to them.

But the British policy makers had reckoned without the political capacities of the Indian leaders who soon transformed the powerless and impotent councils, designed as mere machines for the endorsement of government policies, and measures and as toys to appease the emerging political leadership, into forums for ventilating popular grievances, mercilessly exposing the defects and shortcomings of the bureaucratic administration, criticizing and opposing almost every government policy and proposal, and raising basic economic issues, especially relating to public finance. They submitted the acts and policies of the Government to a ruthless examination regarding both their intention and their method and consequence. Far from being absorbed by the Councils, the nationalist members used them to enhance their own political stature in the county and to build a national movement. The safety valve was transformed into a major channel for nationalist propaganda. By sheer courage, debating skill, fearless criticism, deep knowledge and careful marshalling of data they kept up a constant campaign against the Government in the Councils undermining its political and moral influence and generating a powerful anti-imperialist sentiment.

Their speeches began to be reported at length in the newspapers and widespread public interest developed in the legislative proceedings.

The new Councils attracted some of the most prominent nationalist leaders. Surendranath Banerjee, Kalicharan Banerjee, Ananda Mohan Bose, Lal Mohan Ghosh, W.C. Bonnerji and Rash Beha Ghosh from Bengal, Ananda Charlu, C. Sankan Nair and Vijayaraghavachariar from Madras, Madan Mohan Malaviya, Ayodhyanath and Bishambar Nath from U.P., B.G. Tilak, Pherozeshah Mehta, R.M. Sayani, Chimanlal Setalvad, N.G. Chandravarkar and G.K. Gokhale from Bombay, and G.M. Chitnavis from Central Provinces were some of served as members of the Provincial or Central Legislative Councils from 1893 to 1909.

The two men who were most responsible for putting the Council to good use and introducing a new spirit in them were Pherozeshah Mehta and Gopal Krishna Gokhale. Both men were political Moderates. Both became famous for being fearlessly independent and the bete noir of British officialdom in India.

\begin{center}*\end{center}

\paragraph{Pherozeshah Mehta} Born in 1845 in Bombay, came under Dadabhai Naoroji's influence while studying law in London during the 1860s. He was one of the founders of the Bombay Presidency Association as also the Indian National Congress. From about the middle of the 1890s till his death in 1915 he was a dominant figure in the Indian National Congress and was often accused of exercising autocratic authority over it. He was a powerful debater and his speeches were marked by boldness, lucidity, incisiveness, a ready wit and quick repartee, and a certain literary quality.

Mehta's first major intervention in the Imperial Legislative Council came in January 1895 on a Bill for the amendment of the Police Act of 1861 which enhanced the power of the local authorities to quarter a punitive police force in an area and to recover its cost from selected sections of the inhabitants of the area. Mehta pointed out that the measure was an attempt to convict and punish individuals without a judicial trial under the garb of preserving law and order. He argued: `I cannot conceive of legislation more empirical, more retrograde, more open to abuse, or more demoralizing. It is impossible not to see that it is a piece of that empirical legislation so dear to the heart of executive officers, which will not and cannot recognize the scientific fact that the punishment and suppression of crime without injuring or oppressing innocence must be controlled by judicial procedure.' Casting doubts on the capacity and impartiality of the executive officers entrusted with the task of enforcing the Act, Mehta said: `It would be idle to believe that they can be free from the biases, prejudices, and defects of their class and position.' Nobody would today consider this language and these remarks very strong or censorious. But they were like a bomb thrown into the ranks of a civil service which considered itself above such criticism. How dare a mere `native' lay his sacrilegious hands on its fair name and reputation and that too in the portals of the Legislative Council? James Westland, the Finance Member, rose in the house and protested against `the new spirit' which Mehta `had introduced into the Council.' He had moreover uttered `calumnies' against and `arraigned' as a class as biased, prejudiced, utterly incapable of doing the commonest justice ... a most distinguished service, which had `contributed to the framing and consolidation of the Empire.' His remarks had gravely detracted `from the reputation which this Council has justly acquired for the dignity, the calmness and the consideration which characterize its deliberations.' In other words, Mehta was accused of changing the role and character of the colonial legislatures.

The Indian reaction was the very opposite. Pherozeshah Mehta won the instant approval of political Indians, even of his political opponents like Tilak, who readily accepted Westland's description that `a new spirit' had entered the legislatures. People were accustomed to such criticism coming from the platform or the Press but that the `dignified' Council halls could reverberate with such sharp and fearless criticism was a novel experience. The Tribune of Lahore commented: `The voice that has been so long shut out from the Council Chamber --- the voice of the people has been admitted through the open door of election ... Mr. Mehta speaks as the representative of the people ... Sir James Westland's protest is the outcry of the bureaucrat rapped over the knuckles in his own stronghold.' The bureaucracy was to smart under the whiplash of Mehta's rapier-like wit almost every time he spoke in the Council. We may give a few more examples of the forensic skill with which he regaled the Indians and helped destroy the moral influence and prestige of the British Indian Government and its holier-than-thou bureaucracy. The educated Indians and higher education were major bugbears of the imperialist administrators then as they are of the imperialist schools of historians today. Looking for ways and means of Cutting down higher education because it was producing `discontended and seditious babus,' the Government hit upon the expedient of counterposing to expenditure on primary education of the masses that on the college education of the elites.

Pointing to the real motives behind this move to check the spread of higher education, Mehta remarked: It is very well to talk of ``raising the subject to the pedestal of the rule?' but when the subject begins to press close at your heels, human nature is after all weak, and the personal experience is so intensely disagreeable that the temptation to kick back is almost irresistible.'' And so, most of the bureaucrats looked upon `every Indian college (as) a nursery for hatching broods of vipers; the less, therefore, the better.'

In another speech, commenting on the official desire to transfer public funds from higher to primary education, he said he was reminded of `the amiable and well-meaning father of a somewhat numerous family, addicted unfortunately to slipping off a little too often of an evening to the house over the way, who, when the mother appealed to him to do something for the education of the grown-up boys, begged of her with tears in his eyes to consider if her request was not unreasonable, when there was not even enough food and clothes for the younger children. The poor woman could not gainsay the fact, with the hungry eyes staring before her; but she could not help bitterly reflecting that the children could have food and clothes, and education to boot, if the kindly father could be induced to be good enough to spend a little less on drink and cards. Similarly, gentlemen, when we are reminded of the crying wants Of the poor masses for sanitation and pure water and medical relief and primary education, might we not respectfully venture to submit that there would be funds, and to spare, for all these things, and higher education too, if the enormous and growing resources of the country were not ruthlessly squandered on a variety of whims and luxuries, on costly residences and Sumptuous furniture, on summer trips to the hills, on little holiday excursions to the frontiers, but above and beyond all, on the lavish and insatiable humours of an irresponsible military policy, enforced by the very men whose view and opinions of its necessity cannot but accommodate themselves to their own interests and ambitions.''

The officials were fond of blaming the Indian peasant's poverty and indebtedness on his propensity to spend recklessly on marriages and festivals. In 1901, a Bill was brought in the Bombay Legislative to take away the peasant's right of ownership of land to prevent him from bartering it away because of his thriftlessness. Denying this charge and opposing the bill, Mehta defended the right of the peasant to have some joy, colour, and moments of brightness in his life. In the case of average Indian peasant, he said, `a few new earthenware a few wild flowers, the village tom-tom, a stomach-full meal, bad arecanut and betel leaves and a few stalks of cheap tobacco, and in some cases a few cheap tawdry trinkets, exhaust the joys of a festive occasion in the life of a household which has known only an unbroken period of unshrinking labour from morn to sunset.' And when the Government insisted on using its official majority to push through the Bill, Mehta along. With Gokhale, G.K. Parekh, Balachandra Krishna and D.A. Khare took the unprecedented step of organizing the first walk-out in India's legislative history. Once again officialdom was furious with Mehta. The Times of India, then British-owned even suggested that these members should be made to resign their seats!

Criticizing the Government's excise policy for encouraging drinking in the name of curbing it, he remarked in 1898 that the excise department ``seems to follow the example of the preacher who said that though he was bound to teach good principles, he was by no `means bound to practice them.'''

Pherozeshah Mehta retired from the Imperial Legislative Council in 1901 due to bad health. He got elected in his place thirty-five-year-old Gokhale, who had already made his mark as the Secretary of the Poona Sarvajanik Sabha and the editor of the Sudharak. In 1897, as a witness in London before the Royal Commission on Expenditure in India, Gokhale had outshone veterans like Surendranath Banerjee, D.E. Wacha, G. Subramaniya Iyer and Dadabhai Naoroji. Gokhale was to prove a more than worthy successor to Mehta.

\begin{center}*\end{center}

\paragraph{Gopal Krishna Gokhale} was an outstanding intellectual who had been carefully trained in Indian economics by Justice Ranade and G.V. Josh'. He was no orator. He did not use strong and forceful language as Tilak, Dadabhai Naoroji and R.C. Dun did. Nor did he take recourse, as Mehta did, to humour, irony and courteous, sarcasm. As a speaker he was gentle, reasonable, courteous, non-flamboyant and lucid. He relied primarily upon detailed knowledge and the careful data. Consequently, while his speeches did not entertain or hurt, they gradually took hold of the listeners or readers attention by their sheer intellectual power.

Gokhale was to gain great fame for his budget speeches which used to be reported extensively by the newspapers and whose readers would wait eagerly for their morning copy. He was to transform the Legislative Council into an open university for imparting political education to the people.

His very first budget speech on 1902-03-26, established him as the greatest parliamentarian that India has produced. The Finance Member, Edward Law, had just presented a budget with a seven-crore-rupees surplus for which he had received with great pride the congratulations, of the house. At this point Gokhale rose to speak. He could not, he said, `conscientiously join in the congratulations' because of the huge surplus. On the contrary, the surplus budget `illustrated the utter absence of a due correspondence between the Condition of the country and the condition of the finances of the country.' In fact, this surplus coming in times of serious depression and suffering, constituted `a wrong to the community'. The keynote of his speech was the poverty of the people. He examined the problem in all its aspects and came to the conclusion that the material condition of the mass of the people was `steadily deteriorating' and that the phenomenon was `the saddest in the whole range of the economic history of the world'. He then set out to analyze the budget in detail. He showed how land revenue and the salt tax had been going up even in times of drought and famine. He asked for the reduction of these two taxes and for raising the minimum level of income liable to income tax to Rs. 1,000 so that the lower middle classes would not be harassed. He condemned the large expenditure on the army and territorial expansion beyond Indian frontiers and demanded greater expenditure on education and industry instead. The management of Indian finances, he said, revealed that Indian interests were invariably subordinated to foreign interests. He linked the poor state of Indian finances and the poverty of the people with the colonial status of the Indian economy and polity. And he did all this by citing at length from the Government's own blue books.

Gokhale's first budget speech had `an electrifying effect' upon the people. As his biographer, B.R. Nanda, has put it: `Like Byron, he could have said that he woke up one fine morning and found himself famous''. He won instant praise even from his severest critics and was applauded by the entire nationalist Press. It was felt that he had raised Indian pride many notches higher. The Amrita Bazar Patrika, which had missed no opportunity in the past to berate and belittle him, gave unstinted expression to this pride: `We had ever entertained the ambition of seeing some Indian member openly and fearlessly criticizing the Financial Statement of the Government. But this ambition was never satisfied. When members had ability, they had not the requisite courage. When they had the requisite courage, they had not the ability ... For the first time in the annals of British rule in India, a native of India has not only succeeded in exposing the fallacies which underlie these Government statements, but has ventured to do it in an uncompromising manner.'' All this well-deserved acclaim did not go to Gokhale's head. He remained unassuming and modest as before. To G.V. Joshi (leading economist and one of his gurus), he wrote: ``Of course it is your speech more than mine and I almost feel I am practicing a fraud on the public in that I let all the credit for it come to me.''

In the next ten years, Gokhale was to bring this `mixture of courage, tenacity and ability' to bear upon every annual budget and all legislation, highlighting in the process the misery and poverty of the peasants, the drain of wealth from India, the Government neglect of industrial development, the taxation of the poor, the lack of welfare measures such as primary education and health and medical facilities, the official efforts to suppress the freedom of the Press and other civil liberties, the enslavement of Indian laborers in British colonies, the moral dwarfing of Indians, the under-development of the Indian economy and the complete neglect and subordination of Indian interests by the rulers.

Officials from the Viceroy downwards squirmed with impotent fury under his sharp and incisive indictments of their policies. In 1904, Edward Law, the Finance Member, cried out in exasperation: ``When he takes his seat at this Council table he unconsciously perhaps adopts the role and demeanour of the habitual mourner, and his sad wails and lamentations at the delinquencies of Government are as piteous as long practice and training can make them.'' Such was the fear Gokhale's budget speeches aroused among officials that in 1910, Lord Minto, the Viceroy, asked the Secretary of State to appoint R.W. Carlyle as Revenue Member because he had come to know privately of `an intended attack' in which Gokhale is interested on the whole of our revenue system and it is important that we should be well prepared to meet it.

Gokhale was to be repaid in plenty by the love and recognition of his own people. Proud of his legislative achievement they were to confer him the title of `the leader of the opposition'. Gandhiji was to declare him his political guru. And Tilak, his lifelong political opponent, said at his funeral: ``This diamond of India, this jewel of Maharashtra, this prince of workers, is taking eternal rest on the funeral ground. Look at him and try to emulate him.''
\end{multicols}{2}
