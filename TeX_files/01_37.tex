\chapter{Freedom and Partition}
\begin{multicols}{2}

The contradictory nature of the reality of 15 August 1947 continues to intrigue historians and torment people on both sides of the border to this day. A hard-earned, prized freedom was won after long, glorious years of struggle but a bloody, tragic Partition rent asunder the fabric of the emerging free nation. Two questions arise. Why did the British finally quit? Why was Partition accepted by the Congress? 

The imperialist answer is that independence was simply the fulfilment of Britain's self-appointed mission to assist the Indian people to self- government. Partition was the unfortunate consequence of the age old Hindu-Muslim rift, of the two communities' failure to agree on how and to whom power was to be transferred. The radical view is that independence was finally wrested by the mass actions of 1946--47 in which many Communists participated, often as leaders. But the bourgeois leaders of the Congress, frightened by the revolutionary upsurge struck a deal with the imperialist power by which power was transferred to them and the nation paid the price of Partition. 

These visions of noble design or revolutionary intent frustrated by traditional religious conflict or worldly profit, attractive as they may seem, blur, rather than illumine, the sombre reality. In fact, the Independence-Partition duality reflects the success-failure dichotomy of the anti-imperialist movement led by the Congress. The Congress had a two-fold task: structuring diverse classes, communities, groups and regions into a nation and securing independence from the British rulers for this emerging nation. While the Congress succeeded in building up nationalist consciousness sufficient to exert pressure on the British to quit India, it could not complete the task of welding the nation and particularly failed to integrate the Muslims into this nation. It is this contradiction --- the success and failure of the national movement --- which is reflected in the other contradiction --- independence, but with it Partition.

\begin{center}*\end{center}

\paragraph*{}

The success of the nationalist forces in the struggle for hegemony over Indian society was fairly evident by the end of the War. The British rulers had won the war against Hitler, but lost the one in India. The space occupied by the national movement was far larger than that over which the Raj cast its shadow. Hitherto unpoliticized areas and apolitical groups had fallen in line with the rest of the country in the agitation over the INA trials. As seen in the previous chapter, men in the armed forces and bureaucracy openly attended meetings, contributed money, voted for the Congress and let it be known that they were doing so. The militancy of the politicized sections was evident in the heroic actions of 1942 and in the fearlessness with which students and others expressed their 3Olidarity with INA and RIN men. The success of the nationalist movement could be plotted on a graph of swelling crowds, wide reach, and deep intensity of nationalist sentiment and the nationalist fervour of the people. 

A corresponding graph could also be drawn of the demoralization of the British officials and the changing loyalties of Indian officials and loyalists, which would tell the same story of nationalist success, but differently. In this tale, nationalism would not come across as a force, whose overwhelming presence left no place for the British. Rather, it would show the concrete way in which the national movement eroded imperialist hegemony, gnawed at the pillars of the colonial structure and reduced British political strategy to a mess of contradictions.' 

An important point to be noted is that British rule was maintained in part on the basis of the consent or at least acquiescence of many sections of the Indian people. The social base of the colonial regime was among the zamindars and upper classes etc., the `loyalists' who received the main share of British favours and offices. These were the Indians who manned the administration, supported government policy and worked the reforms the British reluctantly and belatedly introduced. The British also secured the consent of the people to their rule by successfully getting them to believe in British justice and fairplay, accept the British officer as the mai-baap of his people, and appreciate the prevalence of Pax Brittanica. Few genuinely believed in `Angrezi Raj ki Barkaten', but it sufficed for the British if people were impressed by the aura of stolidity the Raj exuded and concluded that its foundations were unshakable. The Raj to a large extent ran on prestige and the embodiment of this prestige was the district officer who belonged to the Indian Civil Service (ICS), the `heaven-born service' much vaunted as `the steel frame of the Raj.' 

When the loyalists began to jump overboard, when prestige was rocked, when the district officer and secretariat official left the helm, it became clear that the ship was sinking, and sinking fast. It was the result of years of ravage wrought from two quarters --- the rot within and the battering without. 

Paucity of European recruits to the ICS, combined with a policy of Indianization (partly conceded in response to popular demand), ended British domination of the ICS as early as the First World War. By 1939 British and Indian members had achieved parity. Overall recruitment was first cut in order to maintain this balance, and later stopped in 1943. Between 1940 and 1946, the total number of ICS officials fell from 1201 to 939, that of British ICS officials from 587 to 429 and Indian ICS officials from 614 to 510. By 1946, only 19 British ICS officials were available in Bengal for 65 posts.2 Besides, the men coming in were no longer Oxbridge graduates from aristocratic families whose fathers and uncles were `old India hands' and who believed m the destiny of the British nation to govern the `child-people' of India. They were increasingly grammar school and polytechnic boys for whom serving the Raj was a career, not a mission. The War had compounded the problem. By 1945, war-weariness was acute and long absences from home were telling on morale. Economic worries had set in because of inflation. Many were due to retire, others were expected to seek premature retirement. It was a vastly-depleted, war-weary bureaucracy, battered by the 1942 movement that remained. 

However, much more than manpower shortage, it was the coming to the fore of contradictions in the British strategy of countering nationalism that debilitated the ICS and the Raj. The British had relied over the years on a twin policy of conciliation and repression to contain the growing national movement. But after the Cripps Offer of 1942, there was little left to be offered as a concession except transfer of power --- full freedom itself. But the strategy of the national movement, of a multi-faceted struggle combining non-violent mass movement with working Constitutional reforms proved to be more than a match for them. When non-violent movements were met with repression, the naked force behind the government stood exposed, whereas if government did not clamp down on `sedition,' or effected a truce (as in 1931 when the Gandhi-Irwin Pact was signed) or conceded provincial autonomy under the Government of India Act 1935, it was seen to be too weak to wield control and its authority and prestige were undermined. On the other hand, the brutal repression of the 1942 movement offended the sensibilities of both liberals and loyalists. So did the government's refusal to release Gandhi, even when he seemed close to death during his 21 day fast in February-March 1943, and its decision to go ahead with the INA trials despite fervent appeals from liberals and loyalists to abandon them. The friends of the British were upset when the Government appeared to be placating its enemies --- as in 1945--46, when it was believed that the Government was wooing the Congress into a settlement and into joining the government. The powerlessness of those in authority dismayed loyalists. Officials stood by, while the violence of Congress speeches rent the air. This shook the faith of the loyalists in the might of the `Raj.' 

If the loyalists' crisis was one of faith, the services' dilemma was that of action. Action could be decisive only if policy was clear-cut --- repression or conciliation --- not both. The policy mix could not but create problems when the same set of officials had to implement both poles of policy. This dilemma first arose in the mid-1930s when officials were worried by the prospect of popular ministries as the Congressmen they repressed during the Civil Disobedience Movement were likely to become their political masters in the provincial Ministries. This prospect soon became a reality in eight provinces. 

Constitutionalism wrecked services morale as effectively as the mass movement before it, though this is seldom realized. If fear of authority was exorcised by mass non-violent action, confidence was gained because of `Congress Raj.' People could not fail to notice that the British Chief Secretary in Madras took to wearing khadi or that the Revenue Secretary in Bombay, on tour with the Revenue Minister, Morarji Desai, would scurry across the railway platform from his first-class compartment to the latter's third-class carriage so that the Honourable Minister my not be kept waiting. Among Indian officials disloyalty was not evident, but where loyalty to the Raj was paraded earlier, `it was the done thing to parade one's patriotism and, if possible, a third cousin twice removed who had been to jail in the civil disobedience movement.'' 

But most importantly, the likelihood of Congress returning to power became a consideration with officials when dealing with subsequent Congress agitations. There was no refusal to carry out orders, but in some places this consideration resulted in half-hearted action against the individual disobedience movement in 

U.P. in 1940 and even against the 1942 rebels in East UP and Bihar. But action was generally harsh in 1942 and this was to create concrete entanglements between repression and conciliation at the end of the War when Congressmen were released and provincial Ministries were again on the cards. Morale of officials nosedived when Congressmen's demands for enquiries and calls for revenge were not proceeded against on the ground that some latitude had to be allowed during electioneering. The previous Viceroy, Linlithgow, had pledged that there would be no enquiries, but the services had little faith in the Government's ability to withstand Congress pressure. The then Viceroy, Wavell, confessed that enquiries were the most difficult issue posed by the formation of provincial Ministries. 

By the end of the War, the portents were clear to those officials and policy-makers who understood the dynamics of power and authority. The demand for leniency to [NA men from within the army and the revolt in a section of the RJN further conveyed to the far-sighted officials, as much as a full-scale mutiny would to others more brashly confident, that the storm brewing this time may prove irrepressible. The structure was still intact, but it was feared that the services and armed forces may not be reliable if Congress started a mass movement of the 1942 type after the elections, which provincial Ministries would aid, not control. The Viceroy summed up the prospect: `We could still probably suppress such a revoke' but `we have nothing to put in its place and should be driven to an almost entirely official rule, for which the necessary numbers of efficient officials do not exist.' 

Once it was recognized that British rule could not survive on the old basis for long, a graceful withdrawal from India, to be effected after a settlement had been reached on the modalities of transfer of power and the nature of the post-imperial relationship between Britain and India, became the overarching aim of British policy-makers.' The British Government was clear that a settlement was a must both for good future relations and to bury the ghost of a mass movement. Since failure could not be afforded, the concessions had to be such as would largely meet Congress demands. With the Congress demand being that the British quit India, the Cabinet Mission went out to India in March 1946 to negotiate the setting up of a national government and to set into motion a machinery for transfer of power. It was not an empty gesture like the Cripps Mission in 1942 --- the Cabinet Mission was prepared for a long stay. 

The situation seemed ripe for a settlement as the imperialist rulers were cognisant of the necessity of a settlement and the nationalist leaders were willing to negotiate with them. But rivers of blood were to flow before Indian independence, tacitly accepted in early 1946, became a reality in mid 1947. By early 1946 the imperialism nationalism conflict, being resolved in principle, receded from the spotlight. The stage was then taken over by the warring conceptions of the post-imperial order held by the British, the Congress and the Muslim League. 

The Congress demand was for transfer of power to one centre, with minorities' demands being worked out in a framework ranging from autonomy to Muslim provinces to self- determination on secession from the Indian Union --- but after the British left. The British bid was for a united India, friendly with Britain and an active partner in Commonwealth defence. It was believed that a divided India would lack depth in defence, frustrate joint defence plans and be a blot on Britain's diplomacy. Pakistan was not seen by Britain as her natural future ally, as the Government's policy of fostering the League ever since its inception in 1906 and the alignment today between Pakistan and the Western imperialist bloc may suggest. 

British policy in 1946 clearly reflected this preference for a united India, in sharp contrast to earlier declarations. Attlee's 15 March 1946 statement that a `minority will not be allowed to place a veto on the progress of the majority' was a far cry from Wavell's allowing Jinnah to wreck the Simla Conference in June- July 1945 by his insistence on nominating all Muslims. The Cabinet Mission was convinced that Pakistan was not viable and that the minorities' autonomy must somehow be safeguarded within the framework of a united India. The Mission Plan conceived three sections, A --- comprising Madras, Bombay, Uttar Pradesh, Bihar, C.P. and Orissa; B --- consisting of Punjab, NWFP and Sind; and C --- of Bengal and Assam --- which would meet separately to decide on group constitutions. There would be a common centre controlling defence, foreign affairs and communications. After the first general elections a province could come out of a group. After ten years a province could call for a reconsideration of the group or union constitution. Congress wanted that a province need not wait till the first elections to leave a group, it should have the option not to join it in the first place. It had Congress- ruled provinces of Assam and NWFP (which were in Sections C and B respectively) in mind when it raised this question. The League wanted provinces to have the right to question the union constitution now, not wait for ten years. There was obviously a problem in that the Mission Plan was ambivalent on whether grouping was compulsory or optional. It declared that grouping was optional but sections were compulsory. This was a contradiction, which rather than removing, the Mission deliberately quibbled about in the hope of somehow reconciling the irreconcilable. 

The Congress and League interpreted the Mission Plan in their own way, both seeing it as a confirmation of their stand. Thus, Patel maintained that the Mission's Plan was against Pakistan, that the League's veto was gone and that one Constituent Assembly was envisaged. The League announced its acceptance of the Plan on 6 June in so far as the basis of Pakistan was implied in the Mission's plan by virtue of the compulsory grouping. Nehru asserted the Congress working Committee's particular interpretation of the plan in his speech to the AICC on 7 July 1946: `We are not bound by a single thing except that we have decided to go into the Constituent Assembly.' The implication was that the Assembly was sovereign and would decide rules of procedure. Jinnah seized the opportunity provided by Nehru's speech to withdraw the League's acceptance of the Mission Plan on 29th July, 1946. 

The dilemma before the Government was whether to go ahead and form the Interim Government with the Congress or await League agreement to the plan. Wavell, who had opted for the second course at the Simla Conference a year earlier, preferred to do the same again. But His Majesty's Government, especially the Secretary of State, argued that it was vital to get Congress cooperation. Thus, the Interim Government was formed on 2nd September 1946 with Congress members alone with Nehru as de facto head. This was against the League's insistence that all settlements be acceptable to it. The British in 1946, in keeping with their strategic interests in the post-independence Indian subcontinent, took up a stance different from their earlier posture of encouraging communal forces and denying the legitimacy of nationalism and the representative nature of the Congress. Continuance of rule had demanded one stance, withdrawal and post-imperial links dictated a contrary posture. 

However, Jinnah had no intention of allowing the British to break with their past. His thinly veiled threat to Attlee that he should `avoid compelling the Muslims to shed their blood ... (by a) surrender to the Congress had already been sent out and the weapon of Direct Action forged. Jinnah had become `answerable to the wider electorate of the streets.'' With the battle cry, Lekar rahenge Pakistan, Larke lenge Pakistan. Muslim communal groups provoked communal frenzy in Calcutta 16 August 1946. Hindu communal groups retaliated in equal measure and the cost was 5000 lives lost. The British authorities were worried that they had lost control over the `Frankenstein monster' they had helped to create but felt it was too late to tame it. They were frightened into appeasing the League by Jinnah's ability to unleash civil war. Wavell quickly brought the League into the Interim Government on 26 October 1946 though it had not accepted either the short or long term provisions of the Cabinet Mission Plan and had not given up its policy of Direct Action. The Secretary of State argued that without the League's presence in the Government civil war would have been inevitable. Jinnah had succeeded in keeping the British in his grip. 

The Congress demand that the British get the League to modify its attitude in the Interim Government or quit was voiced almost from the tine the League members were sworn in. Except Liaqat Ali Khan, all the League nominees were second-raters, indicating that what was at stake was power, not responsibility to run the country. Jinnah had realized that it was fatal to leave the administration in Congress hands and had sought a foothold in the Government to fight for Pakistan. For him, the Interim Government was the continuation of civil war by other means. League ministers questioned actions taken by Congress members, including appointments made, and refused to attend the informal meetings which Nehru had devised as a means of arriving at decisions without reference to Wavell. Their disruptionist tactics convinced Congress leaders of the futility of the Interim Government as an exercise in Congress-League cooperation But they held on till 5th February 1947 when nine members of the Interim Government wrote to the Viceroy demanding that the League members resign. The League's demand for the dissolution of the Constituent Assembly that had met for the first time on 9th December 1946 had proved to be the last straw. Earlier it had refused to join the constituent Assembly despite assurances from His Majesty's Government in their 6th December 1946 statement that the League's interpretation of grouping was the correct one. A direct bid for Pakistan, rather than through the Mission Plan, seemed to be the card Jinnah now sought to play. 

This developing crisis was temporarily defused by the statement made by Attlee in Parliament on 20 February, 1947, The date for British withdrawal from India was fixed as 30 June 1948 and the appointment of a new Viceroy, Lord Mountbatten, was announced. The hope was that the date would shock the parties into agreement on the main question and avert the constitutional crisis that threatened. Besides, Indians would be finally convinced that the British were sincere about conceding independence, however, both these hopes were introduced into the terminal date notion after it had been accepted. The basic reason why the Attlee Government accepted the need for a final date was because they could not deny the truth of Wavell's assessment that an irreversible decline of Government authority had taken place. They could dismiss the Viceroy, on the ground that he was pessimistic, which they did in the most discourteous manner possible. The news was common gossip in New Delhi before Wavell was even informed of it. But they could not dismiss the truth of what he said. So the 20 February statement was really an acceptance of the dismissed Viceroy, Wavell's reading of the Indian situation. 

The anticipation of freedom from imperial rule lifted the gloom that had set in with continuous internal wrangling. The statement was enthusiastically received in Congress circles as a final proof of British sincerity to quit. Partition of the country was implied in the proviso that if the Constituent Assembly was not fully representative (i.e. if Muslim majority provinces did not join) power would be transferred to more than one central Government. But even this was acceptable to the Congress as it meant that the existing Assembly could go ahead and frame a constitution for the areas represented in it. It offered a way out of the existing deadlock, in which the League not only refused to join the Constituent Assembly but demanded that it be dissolved. Nehru appealed to Liaqat All Khan: `The British are fading out of the picture and the burden of this decision must rest on all of us here. It seems desirable that we should face this question squarely and not speak to each other from a distance.' There seemed some chance of fulfilment of Attlee's hopes that the date would force the two political parties in India to come together.' 

This was an illusory hope, for Jinnah was more convinced than ever that he only had to bide his time in order to reach his goal. This is precisely what Conservative members of Parliament had warned would happen, in the contentious debate that following the 20th February statement. Godfrey Nicolson had said of Cripps' speech --- `if ever there was a speech which was a direct invitation to the Muslim League to stick their toes in and hold out for Pakistan that was one.'' The Punjab Governor, Evan Jenkins was equally emphatic --- `the statement will be regarded as the prelude to the final communal showdown,' with everyone out to `seize as much power as they can --- if necessary by force.'' Jenkins' prophecy took immediate shape with the League launching civil disobedience in Punjab and bringing down the Unionist Akali- Congress coalition ministry led by Khizr Hayat Khan. Wavell wrote in his diary on 13th March 1941 -- `Khizr's resignation was prompted largely by the statement of February 20.' 

This was the situation in which Mountbatten came to India as Viceroy. He was the last Viceroy and charged with the task of winding up the Raj by 30th June 1948. Mountbatten has claimed to have introduced the time limit into the 20 February settlement: `I made the great point about it. I had thought of the time and I had great difficulty in bringing him (Attlee) upto it. . I think the time limit was fundamental. I believe if I'd gone out without a time limit, I'd still be there.'' This is so obviously untrue that it should need no refutation, but Lapierre and Collins in Freedom at Midnight and others have passed off as history Mountbatten's self-proclamations of determining history single-handedly. The idea of a fixed date was originally Wavell's, 31 March 1948 being the date by which he expected a stage of responsibility without power to set in. Attlee thought mid-1948 should be the date aimed at. Mountbatten insisted it be a calendar date and got 30th June 1948. 

Mountbatten's claim of having plenipotentiary powers, such that he need make no reference back to London, is equally misleading. It is true that he had more independence than the Viceroys preceding him and his views were given due consideration by the Labour Government. Yet he referred back to London at each stage of the evolution of his Plan, sent his aide Ismay to London and finally went himself to get Attlee and his Cabinet to agree to the 3rd June Plan. 

Mountbatten had a clear cut directive from His Majesty's Government, he did not write his own ticket, as he has claimed. He was directed to explore the options of unity and division till October, 1947 after which he was to advise His Majesty's Government on the form transfer of power should take. Here again he soon discovered that he had little real choice. The broad contours of the scenario that was to emerge were discernible even before he came out. Mountbatten found out within two months of his arrival that more flogging would not push the Cabinet Mission Plan forward. It was a dead horse. Jinnah was obdurate that the Muslims would settle for nothing less than a sovereign state. Mountbatten found himself unable to move Jinnah from this stand: `He gave the impression that he was not listening. He was impossible to argue with ... He was, whatever was said, intent on his Pakistan.'' 

The British could keep India united only if they gave up their role as mediators trying to effect a solution Indians had agreed upon. Unity needed positive intervention in its favour, including putting down communal elements with a firm hand. This they chose not to do. Attlee wrote later --- `We would have preferred a United India. We couldn't get it, though we tried hard.'' They in fact took the easy way out. A serious attempt at retaining unity would involve identifying with the forces that wanted a unified India and countering those who opposed it. Rather than doing that, they preferred to woo both sides into friendly collaboration with Britain on strategic and defence issues. The British preference for a united Indian subcontinent that would be a strong ally in Commonwealth defence was modified to two dominions, both of which would be Britain's allies and together serve the purpose a united India was expected to do. The poser now was, how was friendship of both India and Pakistan to be secured? Mountbatten's formula was to divide India but retain maximum unity. The country would be partitioned but so would Punjab and Bengal, so that the limited Pakistan that emerged would meet both the Congress and League's positions to some extent. The League's position on Pakistan was conceded to the extent that it would be created, but the Congress position on unity would be taken into account to make Pakistan as small as possible. Since Congress were asked to concede their main point i.e. a unified India, all their other points would be met. Whether it was ruling out independence for the princes or unity for Bengal or Hyderabad's joining up with Pakistan instead of India, Mountbatten firmly supported Congress on these issues. He got His Majesty's Government to agree to his argument that Congress goodwill was vital if India was to remain in the commonwealth. 

The Mountbatten Plan, as the 3rd June, 1947 Plan came to be known, sought to effect an early transfer of power on the basis of Dominion Status to two successor states, India and Pakistan. Congress was willing to accept Dominion Status for a while because it felt it must assume full power immediately and meet boldly the explosive situation in the country. As Nehru put it, Murder stalks the streets and the most amazing cruelties are indulged in by both the individual and the mob.'' Besides Dominion Status gave breathing time to the new administration as British officers and civil service officials could stay on for a while and let Indians settle in easier into their new positions of authority. For Britain, Dominion Status offered a chance of keeping India in the Commonwealth, even if temporarily, a prize not to be spurned. Though Jinnah offered to bring Pakistan into the Commonwealth, a greater store was laid by India's membership of the Commonwealth, as India's economic strength and defence potential were deemed sounder and Britain had a greater value of trade and Investment there. 

The rationale for the early date for transfer of power, 15th August 1947 as securing Congress agreement to Dominion Status. The additional benefit was that the British could escape responsibility for the rapidly deteriorating communal situation. As it is, some officials were more than happy to pack their bags and leave the Indians to stew in their own juice. As Patel said to the Viceroy, the situation was one where you won't govern yourself, and you won't let us govern.'' Mountbatten was to defend his advancing the date to 15th August, 1947 on the ground that things would have blown up under their feet had they not got out when they did. Ismay, the Viceroy's Chief of Staff, felt that August, 1947 was too late, rather than too early. From the British point of view, a hasty retreat was perhaps the most suitable action. That does not make it the inevitable option, as Mountbatten and Ismay would have us believe. Despite the steady erosion of government authority, the situation of responsibility without power was still a prospect rather than a reality. In the short term the British could assert their authority, but did not care to, as Kripalani, then Congress President, pertinently pointed out to Mountbatten.' Moreover, the situation, rather than warranting withdrawal of authority, cried out for someone to wield it. 

If abdication of responsibility was callous, the speed with which it was done made it worse. The seventy-two day timetable, 3rd June to 15th August 1947, for both transfer of power and division of the country, was to prove disastrous. Senior officials in India like the Punjab Governor, Jenkins and the Commander-in-Chief, Auchinleck, felt that peaceful division could take a few years at the very least. As it happened, the Partition Council had to divide assets, down to typewriters and printing presses, in a few weeks. There were no transitional institutional structures within which the knotty problems spilling over from division could be tackled. Mountbatten had hoped to be common Governor-General of India and Pakistan and provide the necessary link but this was not to be as Jinnah wanted the position himself. Hence even the joint defence machinery set up failed to last beyond December 1947 by which time Kashmir had already been the scene of a military conflict rather than a political settlement. 

The Punjab massacres that accompanied Partition were the final indictment of Mountbatten. His loyal aide, Ismay, wrote to his wife on 16 September 1947: `Our mission was so very nearly a success: it is sad that it has ended up such a grim and total failure.''9 The early date, 15th August 1947, and the delay in announcing the Boundary Commission Award, both Mountbatten's decisions, compounded the tragedy that took place. A senior army official, Brigadier Bristow, posted in Punjab in 1947, was of the view that the Punjab tragedy would not have occurred had partition been deferred for a year or so. Lockhart, Commander-in-Chief of the Indian Army from 15 August to 31 December 1947, endorsed this view: `Had officials in every grade in the civil services, and all the personnel of the armed services, been in position in their respective new countries before Independence Day, it seems there would have been a better chance of preventing widespread disorder.' The Boundary Commission Award was ready by 12th August, 1947 but Mountbatten decided to make it public after Independence Day, so that the responsibility would not fall on the British. Independence Day in Punjab and Bengal saw strange scenes. Flags of both India and Pakistan were flown in villages between Lahore and Amritsar as people of both communities believed that they were on the right side of the border. The morrow after freedom was to find them aliens in their own homes, exiled by executive fiat. 

Why and how did the Congress come to accept Partition? That the League should assertively demand it and get its Shylockian pound of flesh, or that the British should concede it, being unable to get out of the web of their own making. seems explicable. But why the Congress wedded to a belief in one Indian nation, accepted the division of the country, remains a question difficult to answer. Why did Nehru and Patel advocate acceptance of the 3rd June Plan and the Congress Working Committee and AICC pass a resolution in favour of it? Most surprising of all, why did Gandhi acquiesce? Nehru and Patel's acceptance of Partition has been popularly interpreted as stemming from their lust for quick and easy power, which made them betray the people. Gandhiji's counsels are believed to have been ignored and it is argued that he felt betrayed by his disciples and even wished to end his life, but heroically fought communal frenzy single-handedly `a one man boundary force,' as Mountbatten called him. It is forgotten that Nehru, Patel and Gandhiji in 1947 were only accepting what had become inevitable because of the long-term failure of the Congress to draw in the Muslim masses into the national movement and stem the surging waves of Muslim communalism, which, especially since 1937, had been beating with increasing fury. This failure was revealed with stark clarity by the 1946 elections in which the League won 90 per cent Muslim seats. Though the war against Jinnah was lost by early 1946, defeat was conceded only after the final battle was mercilessly aged an the streets of Calcutta and Rawalpindi and the village lanes of Noakhali and Bihar. The Congress leaders felt by June 1947 that only an immediate transfer of power could forestall the spread of Direct Action and communal disturbances. The virtual collapse of the Interim Government 4150 made Pakistan appear to be an unavoidable reality. Patel argued in the AICC meeting on 14th June, 1947 that we have to face up to the fact that Pakistan was functioning in Punjab, Bengal and in the Interim Government. Nehru was dismayed at the turning of the Interim Government into an arena of struggle. Ministers wrangled, met separately to reach decisions and Liaquat Ali Khan as Finance Member hamstrung the functioning of the other ministries. In the face of the Interim Government's powerlessness to check Governors from abetting the League and the Bengal provincial Ministry's inaction and even complicity in riots, Nehru wondered whether there was any Point in continuing in the Interim Government while people were being butchered. Immediate transfer of power would at least mean the setting up of a government which could exercise the control it was now expected to wield, but was powerless to exercise. 

There was an additional consideration in accepting immediate transfer of power to two dominions. The prospect of balkanisation was ruled out as the provinces and princes were not given the option to be independent --- the latter were, in fact, much to their chagrin, cajoled and coerced into joining one or the other dominion. This was no mean achievement. Princely states standing out would have meant a graver blow to Indian unity than Pakistan was. 

The acceptance of Partition in 1947 was, thus, only the final act of a process of step by step concession to the League's intransigent championing of a sovereign Muslim state. Autonomy of Muslim majority provinces was accepted in 1942 at the time of the Cripps Mission. Gandhiji went a step further and accepted the right of self-determination of Muslim majority provinces in his talks with Jinnah in 1944. In June 1946, Congress conceded the possibility of Muslim majority provinces (which formed Group B and C of the Cabinet Mission Plan) setting up a separate Constituent Assembly, but opposed compulsory grouping and upheld the right of NWFP and Assam not to join their groups if they so wished. But by the end of the year, Nehru said he would accept the ruling of the Federal Court on whether grouping was compulsory or optional. The Congress accepted without demur the clarification by the British Cabinet in December, 1946 that grouping was compulsory. Congress officially referred to Partition in early March 1947 when a resolution was passed in the Congress Working Committee that Punjab (and by implication Bengal) must be partitioned if the country was divided. The final act of surrender to the League's demands was in June 1947 when Congress ended up accepting Partition under the 3rd June Plan. 

The brave words of the leaders contrasted starkly with the tragic retreat of the Congress. While loudly asserting the sovereignty of the Constituent Assembly, the Congress quietly accepted compulsory grouping and abandoned NWFP to Pakistan. Similarly the Congress leaders finally accepted Partition most of all because they could not stop communal riots, but their words were all about not surrendering to the blackmail of violence. Nehru wrote to Wavell on 22nd August 1946: `We are not going to shake hands with murder or allow it to determine the country's policy.' 

What was involved here was a refusal to accept the reality that the logic of their past failure could not be reversed by their present words or action. This was hardly surprising at the time for hardly anybody had either anticipated the quick pace of the unfolding tragedy or was prepared to accept it as irrevocable. It is a fact that millions of people on both sides of the new border refused to accept the finality of Partition long after it was announced, and that is one major reason why the transfer of population became such a frenzied, last-minute affair. Wishful thinking, clinging to fond hopes and a certain lack of appreciation of the dynamics of communal feeling characterized the Congress stand, especially Nehru's. The right of secession was conceded by the Congress as it was believed that `the Muslims would not exercise it but rather use it to shed their fears.' It was not realised that what was in evidence in the mid-1940s was not the communalism of the 1920s or even 1930s when minority fears were being assiduously fanned, but an assertive `Muslim nation,' led by an obdurate leader, determined to have a separate state by any means. The result was that each concession of the Congress, rather than cutting the ground from under the communalists' feet, consolidated their position further as success drew more Muslims towards them. Jinnah pitched his claim high, seeing that Congress was yielding. Hindu communalism got a chance to grow by vaunting itself as the true protector of Hindu interests, which, it alleged, the Congress was sacrificing at the altar of unity. 

Another unreal hope was that once the British left, differences would be patched up and a free India built by both Hindus and Muslims. This belief underestimated the autonomy of communalism by this time --- it was no longer merely propped up by the British, in fact it had thrown away that crutch and was assertively independent, defying even the British. Yet another fond hope was that Partition was temporary --- it had became unavoidable because of the present psyche of Hindus and Muslims but was reversible once communal passions subsided and sanity returned. Gandhiji often told people that Pakistan could not exist for long if people refused to accept Partition in their hearts. Nehru wrote to Cariappa: `But of one thing I am convinced that ultimately there will be a united and strong India. We have often to go through the valley of the shadow before we reach the sun-lit mountain tops.' 

The most unreal belief, given what actually happened was the one that Partition would be peaceful. No riots were anticipated. No transfers of population planned, as it was assumed that once Pakistan was conceded, what was there to fight over? Nehru continued to believe as always in the goodness of his people, despite the spate of riots which plagued India from August 1946 onwards. The hope was that madness would be exorcised by a clean surgical cut. But the body was so diseased, the instruments used infected, that the operation proved to be terribly botchy. Worse horrors were to accompany Partition than those that preceded it. 

What about Gandhiji? Gandhiji's unhappiness and helplessness have often being pointed out. His inaction has been explained in terms of his forced isolation from the Congress decision making councils and his inability to condemn his disciples, Nehru and Patel, for having succumbed to the lust for power, as they had followed him faithfully for many years. at great personal sacrifice. 

In our view, the root of Gandhiji's helplessness was neither Jinnah's intransigence nor his disciples' alleged lust for power. but the communalisation of his people. At his prayer meeting on 4th June 1947 he explained that Congress accepted Partition because the people wanted it: `The demand has been granted because you asked for it. The Congress never asked for it ... But the Congress can feel the pulse of the people. It realized that the Khalsa as also the Hindus desired it.' It was the Hindus' and Sikhs' desire for Partition that rendered him ineffective, blind, impotent. The Muslims already considered him their enemy. What was a mass leader without masses who would follow his call? How could he base a movement to fight communalism on a communalised people? He could defy the leaders' counsels, as he had done in 1942, when he saw clearly that the moment was right for a struggle. But he could not `create a situation,' as he honestly told N.K. Bose, who asked him to do so. His special ability, in his own words, only lay in being able to instinctively feel what is stirring in the hearts of the masses' and `giving a shape to what was already there.' In 1947, there were no `forces of good' which Gandhiji could `seize upon' to `build up a programme' - --- `Toy I see no sign of such a healthy feeling. And, therefore, I shall have to wait until the time comes.' But, political developments did not wait till a `blind man groping in the dark all alone' found a way to the light. The Mountbatten Plan confronted him and Gandhiji saw the inevitability of Partition in the ugly gashes left by riots on the country's face and in the rigor mortis the Interim Government had fallen into. He walked bravely into the AICC meeting on 14 June, 1947 and asked Congressmen to accept Partition as an unavoidable necessity in the given circumstances, but to tight it in the long run by not accepting it in their hearts. He did not accept it in his heart and kept alive, like Nehru, his faith in his people. He chose to plough a lonely furrow, walking barefoot through the villages of Noakhali, bringing confidence h his presence to the Muslims in Bihar and preventing riots by persuasion and threats of a fast in Calcutta. Ekla Cholo had long been his favourite song --- `if no one heeds your call, walk alone, walk alone.' He did just that. 15th August 1947, dawned revealing the dual reality of independence and Partition. As always, between the two of them, Gandhiji and Nehru mirrored the feelings of the Indian people. Gandhiji prayed in Calcutta for an end to the carnage taking place. His close follower, Mridula Sarabhai, sat consoling a homeless, abducted 15-year-old girl in a room somewhere in Bombay. Gandhiji's prayers were reflective of the goings on in the dark, the murders, abductions and rapes. Nehru's eyes were on the light on the horizon, the new dawn, the birth of a free India. `At the stroke of the midnight hour when the world sleeps India shall awake to light and freedom.' His poetic words, `Long years ago, we made a tryst with destiny,' reminded the people that their angry bewilderment today was not the only truth. There was a greater truth --- that of a glorious struggle, hard-fought and hard- won, in which many fell martyrs and countless others made sacrifices, dreaming of the day India would be free. That day had come. The people of India saw that too, and on 15 August --- despite the sorrow in their hearts for the division of their land danced in the streets with abandon and joy.
\end{multicols}