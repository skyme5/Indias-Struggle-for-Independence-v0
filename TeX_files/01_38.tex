\chapter{The Long Term Strategy of the National Movement}\label{chapter:CH38}
\begin{multicols}{2}

A very basic aspect of the long-term dynamics of the Indian national movement was the strategy it adopted in its prolonged struggle against colonial rule. The capacity of a people to struggle depends not only on the fact of exploitation and domination and on its comprehension by the people but also on the strategy and tactics on which their struggle is based.

The existing writings on the subject have failed to deal with, or even discuss, the strategy adopted by the national movement. It appears as if the movement was a mere conglomeration of different struggles or, in the case of its Gandhian phase, certain principles such as non-violence and certain forms of struggle such as satyagraha, picketing, etc., but without an overall strategy. One reason for this failure in the existing writings on the subject is the largely untheorized character of the nationalist strategy. Unlike the leaders of the Russian and Chinese Revolutions, the leaders of the Indian national movement were not theoretically inclined and did not write books and articles putting forth their political strategy in an explicit form. But, in fact, the various phases of the struggle, phases of constitutional activity, constructive work, basic political decisions, forms of struggle, non-violence, Satyagraha, etc., cannot be properly understood or historically evaluated unless they are seen as integral parts of a basic strategy.

Large elements of the nationalist strategy were evolved during the Moderate and the Extremist phases of the movement; it was structured and came to fruition during the Gandhian phase of the movement and in Gandhiji's political practice. Historians and other social scientists, as also contemporary commentators, have tended to concentrate on Gandhiji's philosophy of life. But, in fact, his philosophy of life had only a limited impact on the people. It was as a political leader and through his political strategy and tactics of struggle that he moved millions into political action.

At the very outset, it is to be noted that the nationalist strategy was based on the specific nature and character of British rule and the colonial state. While fully grasping the exploitative and dominational character of colonial rule, Indian leaders also realized that the colonial state was semi-hegemonic and semi- authoritarian in character. It was not like Hitler's Germany or Czarist Russia, or Chiang Kai-shek's China, or Batista's Cuba. Its character could, perhaps, be best described as legal authoritarianism.

The colonial state was established by force and force remained its ultimate sanction. Naked force was often used to suppress peaceful movements. But it was not based just on force. It was also based on the creation of certain civil institutions, such as elected assemblies, local government institutions, courts, and schools and colleges, and, above all, on the rule of law. It provided a certain amount of civil liberties in non- movement periods. Moreover, often, even while suppressing popular opposition, it observed certain rules of law and codes of administration. In other words it was semi-democratic, semi- authoritarian.

The semi-hegemonic character of the colonial state arose from the fact that it relied very heavily for the acquiescence of the Indian people in their rule on two notions carefully inculcated over a long period of time. One as the notion that the foreign rulers were benevolent and just, that they were the Mai-Baap of the people, that they were economically and socially and culturally developing or modernizing' India. The second notion was that the colonial rulers were invincible, that it was futile to oppose them, that the Indian people were too weak and disunited to oppose them successfully, that they would crush all opposition except to the extent they themselves permitted it, that all opposition had, therefore, to proceed along constitutional lines. The colonial rulers also offered constitutional, economic and other concessions to popular movements and did not rely on their repression alone; they followed a policy of the carrot and the stick It was in the context of and in opposition to this semi- hegemonic, semi-authoritarian colonial state that the national movement gradually evolved its strategy and tactics.

The basic strategic perspective of the national movement was to wage a long-drawn out hegemonic struggle, or, in Gramscian terms, a war of position. By hegemonic struggle, we mean a struggle for the minds and hearts of men and women so that the nationalist influence would continuously grow among the people through different channels and through the different phases and stages of the national movement. The movement alternated between phases of extra-legal or law-breaking mass movements and phases of functioning within the four walls of the law. But both phases were geared to expanding the influence of the national movement among the people. The basic strategy of the national movement was, moreover, not a strategy of gradual reform. It was a strategy of active struggle with the objective of wresting power from the colonial rulers.

The effectiveness and validity of the nationalist strategy lay in the active participation of the masses in the movement. The masses had, therefore, to be politicized and activized. The political passivity of the masses, especially in the villages, consciously inculcated and nurtured by the colonial authorities, was a basic factor in the stability of colonial rule. A major objective of the movements of the Gandhian era was to bring the masses into active politics and political action. As Gandhiji repeatedly declared, people `can have Swaraj for the asking' when they `have attained the power to take it.'

The second objective of the nationalist strategy was to erode the hegemony or ideological influence of the colonial rulers inch by inch and in every area of life. Since the British did not rule primarily by force but by' a carefully organized belief system or ideology, it was necessary to undermine and overthrow this belief system. The battle then had to be one of ideas. The objective was to have more and more people adopt nationalist ideas and ideology. A major objective of the hegemonic colonial ideology was to hide the face of the real enemy --- colonialism --- that is, to hide the primary contradiction between the interests of the Indian people and colonialism. The basic task of the counter hegemonic nationalist movement was to expose the face of the colonial enemy and the primary contradiction to the light of day. Hence the most important element of nationalist strategy was its ideological-political work.

Above all, this meant the undermining of the twin notions of the benevolence and invincibility of British rule. The process of undermining the first, i.e., the notion of benevolence, and creating an intellectual framework for it was initiated and performed brilliantly by Dadabhai Naoroji, Justice Ranade, R.C. Dutt and other Moderates. This framework was carried to the lower middle classes by the Extremists and to the masses during the Gandhian era. The sturdily independent newspapers of the late 19th century, the work in the legislative councils by leaders like Pherozeshah Mehta and G.K. Gokhale, the bold propaganda of Lokamanya Tilak, Aurobindo Ghose and other Extremists, and the death-defying deeds of the Revolutionary Terrorists frontally challenged the notion of the invincibility of the colonial state. But it was the law-breaking mass movements of the post- 1918 period which basically performed the task among the mass of the Indian people. The basic objective of these movements was to destroy the notion that British rule could not be challenged, to create among the people fearlessness and courage and the capacity to fight and make sacrifices, and to inculcate the notion that no people could be ruled without their consent.

A third objective of the Congress strategy was to undermine the hold of the colonial state on the members of its own state apparatuses --- members of the civil services, the police and the armed forces --- and to win them over to the nationalist cause or at least to weaken their loyalty and obedience to the colonial regime. The nationalist movement was, in fact, quite successful in this task. Gradually, the behaviour of the police and jail officials underwent a qualitative change. A large number of officials of all types actively helped the 1942 movement at great personal risk, As we have seen earlier, the virtual disappearance of loyalty among the police, army and bureaucracy after 1945 and the consequent disarray of the British administrative structure were major reasons for the British decision to finally quit India.

The national movement, from the beginning, made efforts to weaken the hegemony of colonial ideology among the British people and public opinion. There was a basic continuity in this respect from the work of the British Committee of the National Congress during the 1890s using the services of William Digby, William Wedderburn, and others to the work of the India League in which persons like V.K. Krishna Menon and Fenner Brockway were active.

This as well as efforts to win the support of non-Congress leaders and public opinion within India also aided the achievement of a fourth objective of the nationalist strategy: to constantly expand the semi- democratic political space, and to prevent the colonial authorities from limiting the existing space, within which legal activities and peaceful mass struggles could be organized.

The second major aspect of nationalist strategy was the long-drawn out character of the hegemonic struggle. Under this strategy, which may be described as Struggle-Truce-Struggle or S-T-S', a phase of vigorous extra-legal mass movement and open confrontation with colonial authority was followed by a phase during which direct confrontation was withdrawn, and political concessions, if any, wrested from the colonial regime were worked and shown to be inadequate. During this latter, more `passive,' phase, intense political and ideological work was carried on among the masses within the existing legal and constitutional framework, and forces were gathered for another mass movement at a higher level. The culmination of this strategy of S-T-S' came with a call for `Quit India' and the achievement of independence. Both phases of the movement were utilized, each in its own way, to undermine colonial hegemony, to recruit and train nationalist workers and to build up the people's capacity to struggle.

The entire political process of S-T-S' was an upward spiralling one. This strategy also assumed advance through stages. Each stage represented an advance over the previous one. At the same time, it was realized that the task of national liberation was incomplete till state power was transferred. Even an advanced stage of constitutional reforms did not mean that freedom had been partially transferred. Freedom was a whole; till it was fully won, it was not won at all. Any other view would tend to make Indians `partners' of colonialism during the `reform' phases of the movement, and the national movement would tend to be co-opted by the colonial state. The Indian nationalists avoided this trap by treating the non-mass movement phases also as phases of political, anti-colonial struggle. The working of the reforms was not equated with the working of the colonial system. A basic feature of the nationalist strategy was to move from stage to stage without getting co-opted by the colonial regime which was opposed and struggled against at each stage. Only the form of struggle changed. In the extra-legal mass movement phases, laws were broken and civil disobedience was practised: in the non-mass movement or `passive' phases, there was mass agitation. intense ideological work, including extensive tours by leaders, organization of public meetings on an extensive scale, and the organization of workers, peasants and students and youth and their struggles, mostly by the left-wing, during the late 1920s and the 1930s. Thus, both types of phases were seen as political phases of the anti-imperialist struggle equally rich in anti-imperialist content, and parts of the same anti-imperialist strategy. So the political struggle was perpetual only its forms underwent change. As Gandhiji put it, `suspension of civil disobedience does not mean suspension of war. The latter can only end when India has a Constitution of her own making.'

A basic question regarding the S-T-S' strategy is: why did there have to be two types of phases in the national movement? Why should a phase of non-mass movement or war of position' inevitably follow a phase of extra-legal mass struggle or `war of movement' in Gramscian terms? Why could the national movement not take the form of one continuous mass struggle till freedom was won? Would this not have brought freedom much earlier? The nationalist strategy, under Gandhiji's leadership, was based on the assumptions that by its very nature a mass movement could not be carried on or sustained indefinitely or for a prolonged period, that a mass movement must ebb sooner or later, that mass movements had to be short lived, and that periods of rest and consolidation, of `breathing time,' must intervene so that the movement could consolidate, recuperate and gather strength for the next round of struggle.

This was so because the masses on whom the movement was based invariably got exhausted after some time. Their capacity to confront the state or to face state repression imprisonment, brutal lathi-charges, heavy fines, confiscation of houses, land and other property --- or to make sacrifices was not unlimited. The national leadership made continuous efforts to increase the people's capacity to sacrifice and face colonial repression through ideological work. Simultaneously, it recognized the limits of their capacity to suffer, and therefore did not overstrain this capacity over much. It also based its tactics on the fact that the colonial state was not yet, at least till 1945, in disarray, that its state apparatuses were still loyal to it, that it was till 1945 a strong state, and that it had, consequently, a considerable capacity to crush a movement, as it did in 1932--33 and 1942.

The strategic perspective that there should be two types of phases of the national movement was also based on the perception that though a mass movement needed a `standing army' or `steel frame' of whole time political workers, it could not be based only on them. Its real striking power could come only from the masses. The national movement produced thousands of these whole time workers who devoted their entire lives to the freedom struggle. They spent their entire lives in jails, or Ashrams, or khadi bhandars, or trade union and kisan sabha offices. But while they played a crucial role -in organizing and mobilizing the masses, the movement had to be based on the masses. Consequently, recourse to a mass movement that confronted the colonial state and then its shift to a phase of non- confrontation were an inherent part of a strategy of political struggle that was based on the masses. The Gandhian strategy was thus based on a specific understanding of the limits to which both the people and the Government could go.

Once it was realized that the S-T-S' strategy of the mass movement required the launching of a massive mass movement as well as shifting it to a non-mass movement phase, the decision to shift from one phase to the other became a purely tactical one and not a matter of principle. The question then was: When was the decision to make the shift to be made in keeping with the reality on the ground? In two of the rare instances when Gandhiji theorized his political practice, he gave an inkling of how he perceived the role of leadership in this context. He wrote in 1938: `A wise general does not wait till he is actually routed: he withdraws in time in an orderly manner from a position which he knows he would not be able to hold,. And again in 1939: An able general always gives battle in his own time on the ground of his choice. He always retains the initiative in these respects and never allows it to pass into the hands of the enemy. In a Satyagraha campaign the mode of fight and the choice of tactics, e.g., whether to advance or retreat, offer civil resistance or organize non-violent strength through constructive work and purely' selfless humanitarian service, are determined according to the exigencies of the situation.'

In other words, the very important question of the timing of starting or withdrawing a movement was decided by Gandhiji and the national leadership on the basis of their perception of the strength or weakness of the movement the staying power of the masses and the political and administrative reserves of the Government. Similarly, the question was not whether negotiations with the Government should or should not be held. The question was --- - when one negotiated, how did one choose the right psychological moment to negotiate, how did one actually negotiate, what did one negotiate about, what would the outcome of the negotiations he, and what would the terms on which a truce was signed be, if there was a truce. As the AICC resolution on Congress Policy, adopted on 22 September 1945, stated: `The method of negotiation and conciliation which is the key note of peaceful policy can never he abandoned by the Congress, no matter how grave may be the provocation, any more than can that of non-cooperation, complete or modified. Hence, the guiding maxim of the Congress must remain: negotiation and settlement when possible and non-cooperation and direct action when necessary.'

Constructive work played an important role in Gandhian (and even pre Gandhian) strategy. It was primarily organized around the promotion of khadi, spinning and village industries, national education and, Hindu Muslim unity, the struggle against untouchability and the social upliftment of the Harijans. and the boycott of foreign cloth and liquor. Constructive work was symbolized by hundreds of Ashrams which came up all over the country, almost entirely in the villages.

Constructive work was basic to a war of position. It played a crucial role during the `passive' or non-mass movement phase in filling the political space left vacant by the withdrawal of civil disobedience. It solved a basic problem that a mass movement faces --- - the sustenance of a sense of activism in the non-mass movement phases of the struggle. Constructive work had also the advantage of involving a large number of people. Parliamentary and intellectual work could be done by relatively few, constructive work could involve millions. Moreover, not all could go to jail. But constructive work was within the reach of all.

The hard core of constructive workers also provided a large cadre for the Civil Disobedience Movement. They were Gandhiji's steel-frame or standing army.

Constitutional reforms and legislative councils formed a basic element of the complex colonial strategy to meet the challenge of Indian nationalism. The Indians had to evolve an equally complex approach towards legislatures. Complexity also arose from the fact that, on the one hand, the constitutional structure and constitutional reforms represented instruments of colonial domination and of colonial efforts to co-opt and derail the national movement; while, on the other hand, they represented the fruits of the anti-colonial struggle of the Indian people, a measure of the changing balance of forces and the widening of the democratic space in which, the national movement could operate. The colonial authorities hoped that constitutional work would weaken the nationalist urge to take to mass politics, promote dissensions and splits within the nationalist ranks on the basis of constitutionalist vs. non-constitutionalist and Right vs. Left.

In opposing the colonial strategy, the national leaders had to follow the logic of the constitutional reforms as well as the logic of their own strategy. Once colonialism was forced to yield a political space the space had to he occupied so that political- ideological struggle against colonialism could be waged from it. The reforms had to be worked; the question was in what manner. The answer, found after a great deal of experimentation and debate within the nationalist ranks, was to work the reforms but in a way that would upset imperialist calculations and advance the nationalist cause. In fact, the dominant sections of the national leadership from 1880 onwards looked upon the councils in the wider perspective of undermining colonial hegemony. Work in the legislative councils, municipal bodies, and, after 1937, through popular ministries was also used to promote reforms so as to give relief to the hard-pressed people, to build up confidence among the people in their capacity to govern themselves and to acquire prestige for the Congress and the national movement. For a people who had been for long deprived of political power, and subjected to the colonial ideology that they were incapable of exercising political power or challenging the colonial rulers, the strong speeches of a Pherozeshah Mehta, or a O.K. Gokhale, or a

C.R. Das, or a Motilal Nehru in the legislative councils, the defeats of the Government in the legislatures during the 1920s, the wielding of elements of state power in the 1930s by the Congress ministries, and the nationalist exercise of municipal power in numerous cities, towns and districts, provided a boost to their sense of self-worth and self-confidence.

The nationalist strategy vis-a-vis legislative councils and constitutional reforms did register considerable success. Work in the councils did fill the political void at a time when the national movement was recouping its strength. And those working in the legislatures and municipal bodies did, on the whole, avoid getting co-opted or absorbed by the colonial state. They also successfully exposed the hollowness of colonial reforms and showed that India was, despite these reforms, being ruled from Britain in British interests and with the aid of `lawless laws' whenever the rulers found it in their interests to do so.

The National Congress also successfully avoided a split once the lessons of the Surat split of 1907 had been learnt. All this was possible because Congressmen after 1919 were as a whole committed to mass politics and not to constitutional politics. Whenever the mass upsurge came, Congressmen abandoned the legislatures and plunged into the mass movement. They' saw legislatures not as instruments of the gradual reform of the colonial structure but as arenas for the struggle against, or rather the struggle for the overthrow of, the colonial state. For Gandhiji non-violence was a matter of principle. But for most of his contemporaries in the Congress --- C.R. Das, Motilal Nehru, Jawaharlal Nehru, Maulana Azad, Sardar Patel, Acharya Narendra Dev, and so on --- it was a matter of policy. As policy and as a form of political action and behaviour, it was an essential component of the overall strategy of the National Congress. In fact, non-violence was in some essential ways integral to the nature of the Indian national movement as a hegemonic movement based on wide mass mobilization It was because of this hegemonic and mass character of the national movement that non-violence became one of its basic elements.

The adoption of non-violent forms of struggle enabled the participation of the mass of the people who could not have participated in a similar manner in a movement that adopted violent forms. This was particularly true of women's participation. Women would have found it difficult to join an armed struggle in large numbers. But when it came to undergoing suffering, facing lathi-charges, picketing for hours on end in the summer or the winter, women were probably stronger than men. Non-violence as a form of struggle and political behaviour was also linked to the semi- hegemonic, semi authoritarian character of the colonial state and the democratic character of the polity in Britain.

Non-violence meant above all fighting on the terrain of moral force. Non-violent mass movements placed the colonial authorities in the wrong and exposed the underpinning of colonial state power in brute force, when the authorities used armed force against peaceful Satyagrahis. In fact, a non-violent mass movement put the rulers on the horns of a dilemma. If they hesitated to suppress it because it was peaceful, they lost an important part of their hegemony, because the civil resisters did break existing colonial laws; not to take action against them amounted to the abdication of administrative authority and a confession of the lack of strength to rule. If they suppressed the movement by use of force, they still lost, for it was morally difficult to justify the suppression of a peaceful movement and non-violent law-breakers through the use of force. They were in a no-win situation. The national movement had, on the other hand, a winning strategy: a semi-democratic rule had no answer to a mass movement that was non-violent and had massive popular support. In practice, the colonial authorities constantly vacillated between the two choices, usually plumping in the end for suppression. By taking recourse to suppression of a non-violent movement, they had to suffer constant erosion of hegemony by exposing the basic underpinnings of colonial rule in force and coercion. Consequently, the hegemony of colonial rule or its moral basis was destroyed bit by bit.

The adoption of non-violence was also linked to the fact that a disarmed people had hardly any other alternative. The colonial state had, through an elaborate system, completely disarmed the Indian people since 1858 and made it difficult, indeed nearly impossible, for them to obtain arms or training in their use. The leaders of the national movement understood from the beginning that Indians did not possess the material resources necessary to wage an armed struggle against the strong colonial state. In non-violent mass struggle, on the other hand, it was moral strength and the force of massive and mobilized public opinion that counted. And here the disarmed Indian people were not at a disadvantage. In other words, in a war of position, the non-violence of a mass movement was a way of becoming equal in political resources to the armed colonial state.

Basic here was also the understanding that the disarmed Indian people would not be able to withstand massive government repression, and that the use of violence would provide justification to the Government for launching a massive attack on the popular movement. Such heavy repression it was believed, would demoralize the people and lead to political passivity.

Two further remarks may be made in this context. First, the question whether a mass movement could assume a violent form or as suggested by Jawaharlal Nehru and Bhagat Singh in short but pregnant statements, do mass movements in which millions participate as distinguished from cadre-based movements --- have to be, by their very nature, non-violent Second, in India's case, non-violent struggle was as revolutionary in character as an armed struggle in other contexts: a part of a revolutionary strategy of hegemonic struggle of a Gramscian war of position --- for changes in the structure of state and society.

Once the basic character and objectives of the nationalist strategy are grasped, once it is realised that both phases of the national movement were geared to the twin tasks of winning the hearts and minds of the Indian people and making them active participants in the movement and makers of their own history, the successes and failures of the different phases of the movement and of its basic strategy have to be evaluated in a new manner. The criterion of Success or failure here is the extent to which the colonial hegemony over the Indian people was undermined and the people were politicized and prepared for struggle. Judged in this light, we would see that these objectives were progressively achieved through successive waves of mass movements alternating with phases of truce. Even when the mass movements were suppressed (1932, 1942), withdrawn (1922), ignored and suppressed (1940--41) or ended in compromise (1930--31) and were apparently defeated in terms of their stated objectives of winning freedom; in terms of hegemony, these movements were great successes, and marked leaps in mass political consciousness.

The strategic practice of the Indian national movement, especially during its leadership by Gandhiji, has a certain significance in world history comparable to that of the British, French, Russian, Chinese, Cuban and Vietnamese revolutions. India is the only actual historical example of a semi-democratic or democratic type of state structure being replaced or transformed, of the broadly Gramscian theoretical perspective of a war of position being successfully practised. The study of its experience can yield many insights into the processes of historical change and state transformation, both in the past and the present, both to the historian and the political activist.

It is the one concrete example of a long-drawn out hegemonic struggle in which state power is not seized in a single historical moment of revolution but through a prolonged political process, in which the main terrain of popular struggle is the `national-popular,' that is, the moral, political and ideological on a national or societal plane, in which the reserves of counter- hegemony are patiently built up over the years, in which mass movements are occasional but politics is perpetual, in which the struggle for state power goes through stages, each stage marking a step forward over the previous one, in which masses play an active part and do not depend upon a `standing army' of cadres and yet the cadres play a critical role, in which the movement goes through the inevitable `passive' phases but the popular political morale is not only kept up but enhanced. The problems of popular mobilization, of waging national- popular and hegemonic struggle or a war of position in societies functioning within the confines of the rule of law and a democratic and basically civil libertarian polity have something in common, with the problems and circumstances of the Indian national movement. It is unquestionable that the study of the rich experience of the Indian national movement and in particular of Gandhian political strategy and style of leadership, as distinguished from Gandhian philosophy, has a certain significance for the revolutionary, that is, basic transformation of democratic, hegemonic states and societies.
\end{multicols}{2}
