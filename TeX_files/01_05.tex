\chapter{Foundation of the Indian National Congress: The Reality}
\begin{multicols}{2}

In the last CHAPTER we began the story of the foundation of the Indian National Congress. We could not, however, make much headway because the cobwebs had to be cleared, the myth of the safety-valve had to be laid to rest, the mystery of the `missing volumes' had to be solved, and Hume's mahatmas had to be sent back to their resting place in Tibet. In this CHAPTER we resume the more serious part of the story of the emergence of the Indian National Congress as the apex nationalist organization that was to guide the destiny of the Indian national movement till the attainment of independence.

The foundation of the Indian National Congress in 1885 was not a sudden event, or a historical accident. It was the culmination of a process of political awakening that had its beginnings in the 1860's and 1870's and took a major leap forward in the late 1870's and early 1880's. The year 1885 marked a turning point in this process, for that was the year the political Indians, the modem intellectuals interested in politics, who no longer saw themselves as spokesmen of narrow group interests, but as representatives of national interest vis-a-vis foreign rule, as a `national party,' saw their efforts bear fruit. The all-India nationalist body that they brought into being was to be the platform, the organizer, the headquarters, the symbol of the new national spirit and politics.

British officialdom, too, was not slow in reading the new messages that were being conveyed through the nationalist political activity leading to the founding of the Congress, and watched them with suspicion, and a sense of foreboding. As this political activity gathered force, the prospect of disloyalty, sedition and Irish-type agitations began to haunt the Government.

The official suspicion was not merely the over-anxious response of an administration that had not yet recovered from the mutiny complex, but was in fact, well-founded. On the surface, the nationalist Indian demands of those years --- no reduction of import duties on textile import no expansion in Afghanistan or Burma, the right to bear arms, freedom of the Press, reduction of military expenditure, higher expenditure on famine relief, Indianization of the civil services, the right of Indians to join the semi-military volunteer corps, the right of Indian judges to try Europeans in criminal cases, the appeal to British voters to vote for a party which would listen to Indians --- look rather mild, especially when considered separately. But these were demands which a colonial regime could not easily concede, for that would undermine its hegemony over the colonial people. It is true that any criticism or demand no matter how innocuous its appearance but which cannot be accommodated by a system is in the long-run subversive of the system.

The new political thrust in the years between 1875 and 1885 was the creation of the younger, more radical nationalist intellectuals most of whom entered politics during this period. They established new associations, having found that the older associations were too narrowly conceived in terms of their programmes and political activity as well as social bases. For example, the British Indian Association of Bengal had increasingly identified itself with the interests of the zamindars and, thus, gradually lost its anti-British edge. The Bombay Association and Madras Native Association had become reactionary and moribund. And so the younger nationalists of Bengal, led by Surendranath Banerjee and Anand Mohan Bose, founded the Indian Association in 1876. Younger men of Madras --- M. Viraraghavachariar, G. Subramaniya Iyer, P. Ananda Charlu and others --- formed the Madras Mahajan Sabha in 1884. In Bombay, the more militant intellectuals like K.T. Telang and Pherozeshah Mehta broke away from older leaders like Dadabhai Framji and Dinshaw Petit on political grounds and formed the Bombay Presidency Association in 1885. Among the older associations only the Poona Sarvajanik Sabha carried on as before. But, then, it was already in the hands of nationalist intellectuals.

A sign of new political life in the country was the coming into existence during these years of nearly all the major nationalist newspapers which were to dominate the Indian scene till 1918 --- The Hindu, Tribune, Bengalee, Mahraua and Kesari. The one exception was the Amrita Bazar Patrika which was already edited by new and younger men. It became an English language newspaper only in 1878.

By 1885, the formation of an all-India political organization had become an objective necessity, and the necessity was being recognized by nationalists all over the country. Many recent scholars have furnished detailed information on the many moves that were made in that direction from 1877. These moves acquired a greater sense of urgency especially from 1883 and there was intense political activity. The Indian Mirror of Calcutta was carrying on a continuous campaign on the question. The Indian Association had already in December 1883 organized an All-India National Conference and given a call for another one in December 1885. Surendranath Banerjee, who was involved in the All-India National Conference, could not for that reason attend the founding session of the National Congress in 1885.

Meanwhile, the Indians had gained experience, as well as confidence, from the large number of agitations they had organized in the preceding ten years. Since 1875, there had been a continuous campaign around cotton import duties which Indians wanted to stay in the interests of the Indian textile industry. A massive campaign had been organized during 1877--88 around the demand for the Indianization of Government services. The Indians had opposed the Afghan adventure of Lord Lytton and then compelled the British Government to contribute towards the cost of the Second Afghan War. The Indian Press had waged a major campaign against the efforts of the Government to control it through the Vernacular Press Act. The Indians had also opposed the effort to disarm them through the Arms Act. In 1881--82 they had organized a protest against the Plantation Labour and the Inland Emigration Act which condemned plantation labourers to serfdom. A major agitation was organized during 1883 in favour of the Ilbert Bill which would enable Indian magistrates to try Europeans. This Bill was successfully thwarted by the Europeans. The Indians had been quick to draw the political lesson. Their efforts had failed because they had not been coordinated on an all-India basis. On the other hand, the Europeans had acted in a concerted manner. Again in 1883-07-00, a massive all-India effort was made to raise a National Fund which would be used to promote political agitation in India as well as England. In 1885, Indians fought for the right to join the volunteer corps restricted to Europeans, and then organized an appeal to British voters to vote for those candidates who were friendly towards India. Several Indians were sent to Britain to put the Indian case before British voters through public speeches, and other means.

\begin{center}*\end{center}

\paragraph*{}

It thus, becomes clear that the foundation of the Congress was the natural culmination of the political work of the previous years. By 1885, a stage had been reached in the political development of India when certain basic tasks or objectives had to be laid down and struggled for. Moreover these objectives were correlated and could only be fulfilled by the coming together of political workers in a single organization formed on an all-India basis. The men who met in Bombay on 28 December 1885, were inspired by these objectives and hoped to initiate the process of achieving them. The success or failure and the future character of the Congress would be determined not by who founded it but by the extent to which these objectives were achieved in the initial years.

\begin{center}*\end{center}

\paragraph*{}

India had just entered the process of becoming a nation or a people. The first major objective of the founders of the Indian national movement was to promote this process, to weld Indians into a nation, to create an Indian people. It was common for colonial administrators and ideologues to assert that Indians could not be united or freed because they were not a nation or a people but a geographical expression, a mere congeries of hundreds of diverse races and creeds. The Indians did not deny this but asserted that they were now becoming a nation. India was as Lokmanya Tilak, Surendranath Banerjee and many others were fond of saying --- a nation-in-the-making. The Congress leaders recognized that objective historical forces were bringing the Indian people together. But they also realized that the people had to become subjectively aware of the objective process and that for this it was necessarily to promote the feeling of national unity and nationalism among them.

Above all, India being a nation-in-the-making its nationhood could not be taken for granted. It had to be constantly developed and consolidated. The promotion of national unity was a major objective of the Congress and later its major achievement. For example, P. Ananda Charlu in his presidential address to the Congress in 1891 described it ```as a mighty nationalizer' and said that this was its most `glorious role'. Among the three basic aims and objectives of the Congress laid down by its first President'',

W.C. Bannerji, was that of `the fuller development and Foundation of the Indian National Congress: The Reality consolidation of those sentiments of national unity.' The Russian traveller, I.P. Minayeff wrote in his diary that, when travelling with Bannerji, he asked, `what practical results did the Congress leaders expect from the Congress,' Bonnerji replied: `Growth of national feeling and unity of Indians.' Similarly commenting on the first Congress session, the Indu Prakash of Bombay wrote: `It was the beginning of a new life ... it will greatly help in creating a national feeling and binding together distant people by common sympathy and common ends.'

The making of India into a nation was to be a prolonged historical process. Moreover, the Congress leaders realized that the diversity of India was such that special efforts unknown to other parts of the world would have to be made and national unity carefully nurtured. In an effort to reach all regions, it was decided to rotate the Congress session among different parts of the country. The President was to belong to a region other than where the Congress session was being held.

To reach out to the followers of all religions and to remove the fears of the minorities a rule was made at the 1888 session that no resolution was to be passed to which an overwhelming majority of Hindu or Muslim delegates objected. In 1889, a minority clause was adopted in the resolution demanding reform of legislative councils. According to the clause, wherever Parsis, Christians, Muslims or Hindus were a minority their number elected to the Councils would not be less than their proportion in the Population. The reason given by the mover of the resolution was that India was not yet a homogenous country and political methods here had, therefore, to differ from those in Europe. The early national leaders were also determined to build a secular nation, the Congress itself being intensely secular.

\begin{center}*\end{center}

\paragraph*{}

The second major objective of the early Congress was to create a common political platform or programme around which political workers in different parts of the country could gather and Conduct their political activities, educating and mobilizing people on an all-India basis. This was to be accomplished by taking up those grievances and fighting for those rights which Indians had in common in relation to the rulers.

For the same reason the Congress was not to take up questions of social reform. At its second session, the President of the Congress, Dadabhai Naoroji, laid down this rule and said that `A National Congress must confine itself to questions in which the entire nation has a direct participation.' Congress was, therefore, not the right place to discuss social reforms. `We are met together,' he said, `as a political body to represent to our rulers our political aspirations.' Modern politics --- the politics of popular participation, agitation mobilization --- was new to India. The notion that politics was not the preserve of the few but the domain of everyone was not yet familiar to the people. No modern political movement was possible till people realized this. And, then, on the basis of this realization, an informed and determined political opinion had to be created. The arousal, training, organization and consolidation of public opinion was seen as a major task by the Congress leaders. All initial activity of the early nationalism was geared towards this end.

The first step was seen to be the politicization and unification of the opinion of the educated, and then of other sections. The primary objective was to go beyond the redressal of immediate grievances and organize sustained political activity along the lines of the Anti-Corn Law League (formed in Britain by Cobden and Bright in 1838 to secure reform of Corn Laws). The leaders as well as the people also had to gain confidence in their own capacity to organize political opposition to the most powerful state of the day.

All this was no easy task. A prolonged period of politicization would be needed. Many later writers and critics have concentrated on the methods of political struggle of the early nationalist leaders, on their petitions, prayers and memorials. It is, of course, true that they did not organize mass movements and mass struggles. But the critics have missed out the most important part of their activity --- that all of it led to politics, to the politicization of the people. Justice Ranade, who was known as a political sage, had, in his usual perceptive manner, seen this as early as 1891 When the young and impatient twenty-six-year-old Gokhale expressed disappointment when the Government sent a two line reply to a carefully and laboriously prepared memorial by the Poona Sarvajanik Sabha, Ranade reassured him: ``You don't realize our place in the history of our country. These memorials are nominally addressed to Government, in reality they are addressed to the people, so that they may learn how to think in these matters. This work must be done for many years, without expecting any other result, because politics of this kind is altogether new in this land.''

\begin{center}*\end{center}

\paragraph*{}

As part of the basic objective of giving birth to a national movement, it was necessary to create a common all-India national-political leadership, that is, to construct what Antonio Gramsci, the famous Italian Marxist, calls the headquarters of a movement. Nations and people become capable of meaningful and effective political action only when they are organized. They become a people or `historical subjects' only when they are organized as such. The first step in a national movement is taken when the `carriers' of national feeling or national identity begin to organize the people. But to be able to do so successfully, these `carriers' or leaders must themselves be unified; they must share a collective identification, that is, they must come to know each other and share and evolve a common outlook, perspective, sense of purpose, as also common feelings. According to the circular which, in March 1885, informed political workers of the coming Congress session, the Congress was intended `to enable all the most earnest labourers in the cause of national progress to become personally known to each other.', W.C. Bonnerjee, as the first Congress President, reiterated that one of the Congress objectives was the `eradication, by direct friendly personal intercourse, of all possible race, creed, or provincial prejudices amongst all lovers of our country,' and `the promotion of personal intimacy and friendship amongst all the more earnest workers in our country's cause in (all) parts of the Empire.'

In other words, the founders of the Congress understood that the first requirement of a national movement was a national leadership. The social- ideological complexion that this leadership would acquire was a question that was different from the main objective of the creation of a national movement. This complexion would depend on a host of factors: the role of different social classes, ideological influences, outcomes of ideological struggles, and so on.

The early nationalist leaders saw the internalization and indigenization of political democracy as one of their main objectives. They based their politics on the doctrine of the sovereignty of the people, or, as Dadabhai Naoroji put it, on `the new lesson that Kings are made for the people, not peoples for their Kings.'

From the beginning, the Congress was organized in the form of a Parliament. In fact, the word Congress was borrowed from North American history to connote an assembly of the people. The proceedings of the Congress sessions were conducted democratically, issues being decided through debate and discussion and occasionally through voting. It was, in fact, the Congress, and not the bureaucratic and authoritarian colonial state, as some writers wrongly argue, which indigenized, popularized and rooted parliamentary democracy in India.

Similarly, the early national leaders made maintenance of civil liberties and their extension an integral part of the national movement. They fought against every infringement of the freedom of the Press and speech and opposed every attempt to curtail them. They struggled for separation of the judicial and executive powers and fought against racial discrimination.

\begin{center}*\end{center}

\paragraph*{}

It was necessary to evolve an understanding of colonialism and then a nationalist ideology based on this understanding. In this respect, the early nationalist leaders were simultaneously learners and teachers. No ready-made anti-colonial understanding or ideology was available to them in the 1870's and 1880's. They had to develop their own anti-colonial ideology on the basis of a concrete study of the reality and of their own practice.

There could have been no national struggle without an ideological struggle clarifying the concept of we as a nation against colonialism as an enemy They had to find answers to many questions. For example, is Britain ruling India for India's benefit? Are the interests of the rulers and the ruled in harmony, or does a basic contradiction exist between the two? Is the contradiction of the Indian people with British bureaucrats in India, or with the British Government, or with the system of colonialism as such? Are the Indian people capable of fighting the mighty British empire? And how is the fight to be waged?

In finding answers to these and other questions many mistakes were made. For example, the early nationalists failed to understand, at least till the beginning of the 20th century, the character of the colonial state. But, then, some mistakes are an inevitable part of any serious effort to grapple with reality. In a way, despite mistakes and setbacks, it was perhaps no misfortune that no ready-made, cut and dried, symmetrical formulae were available to them. Such formulae are often lifeless and, therefore, poor guides to action.

True, the early national leaders did not organize mass movements against the British. But they did carry out an ideological struggle against them. It should not be forgotten that nationalist or anti-imperialist struggle is a struggle about colonialism before it becomes a struggle against colonialism. And the founding fathers of the Congress carried out this `struggle about colonialism' in a brilliant fashion.

\begin{center}*\end{center}

\paragraph*{}

From the beginning, the Congress was conceived not as a party but as a movement. Except for agreement on the very broad objectives discussed earlier, it did not require any particular political or ideological commitment from its activists. It also did not try to limit its following to any social class or group. As a movement, it incorporated different political trends, ideologies and social classes and groups so long as the commitment to democratic and secular nationalism was there. From the outset, the Congress included in the ranks of its leadership persons with diverse political thinking, widely disparate levels of political militancy and varying economic approaches.

To sum up: The basic objectives of the early nationalist leaders were to lay the foundations of a secular and democratic national movement, to politicize and politically educate the people, to form the headquarters of the movement, that is, to form an all-India leadership group, and to develop and propagate an anti-colonial nationalist ideology.

History will judge the extent of the success or failure of the early national movement not by an abstract, ahistorical standard but by the extent to which it was able to attain the basic objectives it had laid down for itself. By this standard, its achievements were quite substantial and that is why it grew from humble beginnings in the 1880's into the most spectacular of popular mass movements in the 20th century. Historians are not likely to disagree with the assessment of its work in the early phase by two of its major leaders. Referring to the preparatory nature of the Congress work from 1885 to 1905, Dadabhai Naoroji wrote to D.E. Wacha in January 1905: `The very discontent and impatience it (the Congress) has evoked against itself as slow and non-progressive among the rising generation are among its best results or fruit. It is its own evolution and progress….(the task is) to evolve the required revolution --- whether it would be peaceful or violent. The character of the revolution will depend upon the wisdom or unwisdom of the British Government and action of the British people.'

And this is how G.K. Gokhale evaluated this period in 1907: ``Let us not forget that we are at a stage of the country's progress when our achievements are bound to be small, and our disappointments frequent and trying. That is the place which it has pleased Providence to assign to us in this struggle, and our responsibility is ended when we have done the work which belongs to that place. It will, no doubt, be given to our countrymen of future generations to serve India by their successes; we, of the present generation, must be content to serve her mainly by our failures. For, hard though it be, out of those failures the strength will come which in the end will accomplish great tasks.''

\begin{center}*\end{center}

\paragraph*{}

As for the question of the role of A.O. Hume, if the founders of the Congress were such capable and patriotic men of high character, why did they need Hume to act as the chief organizer of the Congress? It is undoubtedly true that Hume impressed --- and, quite rightly --- all his liberal and democratic contemporaries, including Lala Lajpat Rai, as a man of high ideals with whom it was no dishonor to cooperate. But the real answer lies in the conditions of the time. Considering the size of the Indian subcontinent, there were very few political persons in the early 1880's and the tradition of open opposition to the rulers was not yet firmly entrenched.

Courageous and committed persons like Dadabhai Naoroji, Justice Ranade, Pherozeshah Mehta, G. Subramaniya Iyer and Surendranath Banerjee (one year later) cooperated with Hume because they did not want to arouse official hostility at such an early stage of their work. They assumed that the rulers would be less suspicious and less likely to attack a potentially subversive organization if its chief organizer was a retired British civil servant. Gokhale, with his characteristic modesty and political wisdom, gazed this explicitly in 1913: `No Indian could have started the Indian National Congress ... if an Indian had come forward to start such a movement embracing all India, the officials in India would not have allowed the movement to come into existence. If the founder of the congress had not been a great Englishman and a distinguished ex-official, such was the distrust of political agitation in those days that the authorities would have at once found some way or the other to suppress the movement.

In other words, if Hume and other English liberals hoped to use the Congress as a safety-valve, the Congress leaders hoped to use Hume as a lightning conductor. And as later developments show, it was the Congress leaders whose hopes were fulfilled.
\end{multicols}