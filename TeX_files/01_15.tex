
\chapter{The Non-cooperation Movement — 1920-22}



The last year of the second decade of the twentieth century found India highly discontented. With much cause, The Rowlatt Act, the Jallianwala Bagh massacre and martial law in Punjab had belied all the generous wartime promises of the British. The Montague-Chelmsford Reforms announced towards the end of 1919, with their ill-considered scheme of dyarchy satisfied few. The Indian Muslims were incensed when they discovered that their loyalty had been purchased during the War by assurances of generous treatment of Turkey after the War — a promise British statesman had no intention of fulfilling. The Muslims regarded the Caliph of Turkey as their spiritual head and were naturally upset when they found that he would retain no control over the holy places it was his duty as Caliph to protect. Even those who were willing to treat the happenings at Jallianwala Bagh and other places in Punjab as aberrations, that would soon be `corrected', were disillusioned when they discovered that the Hunter Committee appointed by the Government to enquire into the Punjab disturbances was an eye wash and that the House of Lords had voted in favour of General Dyer's action and that the British public had demonstrated its support by helping the Morning Post collect 30,000 pounds for General Dyer. 

By the end of the first quarter of 1920, all the excuses in favour of the British Government were fast running out. The Khilafat leaders were told quite clearly that they should not expect anything more and the Treaty of Sevres signed with Turkey in May 1920 made it amply clear that the dismemberment of the Turkish Empire was complete. Gandhiji, who had been in close touch with the Khilafat leaders for quite some time, and was a special invitee to the Khilafat Conference in November 1919, had all along been very sympathetic to their cause, especially because he felt the British had committed a breach of faith by making promises that they had no intention of keeping. In February 1920, he suggested to the Khilafat Committee that it adopt a programme of non-violent non-cooperation to protest the Government's behavior. On 9 June 1920, the Khilafat Committee at Allahabad unanimously accepted the suggestion of non-cooperation and asked Gandhiji to lead the movement. 

Meanwhile, the Congress was becoming skeptical of any possibility of political advance through constitutional means. It was disgusted with the Hunter Committee Report especially since it was appraised of brutalities in Punjab by its own enquiry committee. In the circumstances, it agreed to consider non-cooperation. The AICC met in May 1920 and decided to convene a special session in September to enable the Congress to decide on its course of action. 

It was apparent they had to work out something soon for it was clear that the people were chafing for action. Large numbers of them, who had been awakened to political consciousness by the incessant propaganda efforts that the nationalist leadership had been making for the previous four decades or more, were thoroughly outraged by what they perceived as insults by the British government. To swallow these insults appeared dishonourable and cowardly. Also many sections of Indian society suffered considerable economic distress. In the towns, the workers and artisans, the lower middle class and the middle class had been hit by high prices, and shortage of food and essential commodities. The rural poor and peasants were in addition victims of widespread drought and epidemics.

\begin{center}*\end{center}

\paragraph*{}


The movement was launched formally on 1 August 192O, after the expiry of the notice that Gandhiji had given to the Viceroy in his letter of 22 June. in which he had asserted the right recognized `from time immemorial of the subject to refuse to assist a ruler who misrules.' Lokamanya Tilak passed away in the early hours of 1 August, and the day of mourning and of launching of the movement merged as people all over the country observed hartal and took out processions. Many kept a fast and offered prayers. 

The Congress met in September at Calcutta and accepted non-cooperation as its own. The main opposition, led by C.R. 

Das, was to the boycott of legislative councils, elections to which were to be held very soon. But even those who disagreed with the idea of boycott accepted the Congress discipline and withdrew from the elections. The voters, too, largely stayed away. 

By December, when the Congress met for its annual session at Nagpur, the opposition had melted away; the elections were over and, therefore, the boycott of councils was a non-issue, and it was CR. Das who moved the main resolution on non-cooperation. The programme of non-cooperation included within its ambit the surrender of titles and honours, boycott of government affiliated schools and colleges, law courts, foreign cloth, and could be extended to include resignation from government service and mass civil disobedience including the non-payment of taxes. National schools and colleges were to be set up, panchayats were to be established for settling disputes, hand-spinning and weaving was to be encouraged and people were asked to maintain Hindu- Muslim unity, give up untouchability and observe strict non-violence. Gandhiji promised that if the programme was fully implemented, Swaraj would be ushered in within a year. The Nagpur session, thus, committed the Congress to a programme of extra-constitutional mass action. Many groups of revolutionary terrorists, especially in Bengal, also pledged support to the movement. 

To enable the Congress to fulfil its new commitment, significant changes were introduced in its creed as well as in its organizational structure. The goal of the Congress was changed from the attainment of self-government by constitutional and legal means to the attainment of Swaraj by peaceful and legitimate means. The new constitution of the Congress, the handiwork of Gandhiji, introduced other important changes. 

The Congress was now to have a Working Committee of fifteen members to look after its day-to-day affairs. This proposal, when first made by Tilak in 1916, had been shot down by the Moderate opposition. Gandhiji, too, knew that the Congress could not guide a sustained movement unless it had a compact body that worked round the year. Provincial Congress Committees were now to be organized on a linguistic basis, so that they could keep in touch with the people by using the local language. The Congress organization was to reach down to the village and the mohalla level by the formation of village and mohalla or ward committees. The membership fee was reduced to four annas per year to enable the poor to become members. Mass involvement would also enable the Congress to have a regular source of income. In other ways, too, the organization structure was both streamlined and democratized. The Congress was to use Hindi as far as possible.

\begin{center}*\end{center}

\paragraph*{}


The adoption of the Non-Cooperation Movement (initiated earlier by the Khilafat Conference) by the Congress gave it a new energy and, from January 1921, it began to register considerable success all over the country. Gandhiji, along with the Ali brothers (who were the foremost Khilafat leaders), undertook a nation-wide tour/during which he addressed hundreds of meetings and met a large number of political workers. In the first month itself, thousands of students (90,000 according to one estimate) left schools and colleges and joined more than 800 national schools and colleges that had sprung up all over the country. The educational boycott was particularly successful in Bengal, where the students in Calcutta triggered off a province-wide strike to force the managements of their institutions to disaffiliate themselves from the Government. C.R. Das played a major role in promoting the movement and Subhas Bose became the principal of the National Congress in Calcutta. The Swadeshi spirit was revived with new vigour, this time as part of a nation-wide struggle. Punjab, too, responded to the educational boycott and was second only to Bengal, Lala Lajpat Rai playing a leading part here despite his initial reservations about this item of the programme. Others areas that were active were Bombay, U.P., Bihar, Orissa and Assam, Madras remained lukewarm. 

The boycott of law courts by lawyers was not as successful as the educational boycott, but it was very dramatic and spectacular. Many leading lawyers of the country, like C.R. Das, Motilal Nehru, M.R. Jayakar, Saifuddin Kitchlew, Vallabhbhai Patel, C. Rajagopalachari, T. Prakasam and Asaf Ali gave up lucrative practices, and their sacrifice became a source of inspiration for many. In numbers again Bengal led, followed by Andhra Pradesh, U.P., Karnataka and Punjab. 

But, perhaps, the most successful item of the programme was the boycott of foreign cloth. Volunteers would go from house to house collecting clothes made of foreign cloth, and the entire community would collect to light a bonfire of the goods. Prabhudas Gandhi, who accompanied Mahatma Gandhi on his nation-wide tour in the first part of 1921, recalls how at small way-side stations where their train would stop for a few minutes. Gandhiji would persuade the crowd, assembled to greet him, to at least discard their head dress on the spot. Immediately, a pile of caps, dupattas, and turbans would form and as the train moved out they would see the flames leaping upwards.2 Picketing of shops selling foreign cloth was also a major form of the boycott. The value of imports of foreign cloth fell from Rs. 102 crore in 1920-21 to Rs. 57 crore in 1921-22. Another feature of the movement which acquired great popularity in many parts of the country, even though it was not part of the original plan, was the picketing of toddy shops. Government revenues showed considerable decline on this count and the Government was forced to actually carry on propaganda to bring home to the people the healthy effects of a good drink. 

The Government of Bihar and Orissa even compiled and circulated a list of all the great men in history (which included Moses, Alexander, Julius Caesar, Napoleon, Shakespeare, Gladstone, Tennyson and Bismarck) who enjoyed their liquor. 

The AICC, at its session at Vijayawada in March 1921, directed that for the next three months Congressmen should concentrate on collection of funds, enrolment of members and distribution of charkhas. As a result, a vigorous membership drive was launched and though the target of one crore members was not achieved, Congress membership reached a figure roughly of 50 lakhs. The Tilak Swaraj Fund was oversubscribed, exceeding the target of rupees one crore. Charkhas were popularized on a wide scale and khadi became the uniform of the national movement. There was a complaint at a students meeting Gandhiji addressed in Madurai that khadi was too costly. Gandhiji retorted that the answer lay in wearing less clothes and, from that day, discarded his dhoti and kurta in favour of a 1ango For the rest of his life, he remained a `half-naked fakir.' 

In July 1921, a new challenge was thrown to the Government. Mohammed Ali, at the All India Khilafat Conference held at Karachi on 8 July, declared that it was `religiously unlawful for the Muslims to continue in the British Army' and asked that this be conveyed to every Muslim in the Army. As a result, Mohammed Ali, along with other leaders, was immediately arrested. In protest, the speech was repeated at innumerable meetings all over the country. On 4 October, forty-seven leading Congressmen, including Gandhiji, issued a manifesto repeating whatever Mohammed Ali had said and added that every civilian and member of the armed forces should sever connections with the repressive Government. The next day, the Congress Working Committee passed a similar resolution, and on 16 October, Congress committees all over the country held meetings at which the same resolution was adopted. The Government was forced to ignore the whole incident, and accept the blow to its prestige. 

The next dramatic event was the visit of the Prince of Wales which began on 17 November, 1921. The day the Prince landed in Bombay was observed as a day of hartal all over the country. In Bombay, Gandhiji himself addressed a mammoth meeting in the compound of the Elphinstone Mill owned by the nationalist Umar Shobhani, and lighted a huge bonfire of foreign cloth. Unfortunately, however, clashes occurred between those who had gone to attend the welcome function and the crowd returning from Gandhiji's meeting. Riots followed, in which Parsis, Christians, Anglo-Indians became special targets of attack as identifiable loyalists. There was police firing, and the three-day turmoil resulted in fifty-nine dead. Peace returned only after Gandhiji had been on fast for three days. The whole sequence of events left Gandhiji profoundly disturbed and worried about the likelihood of recurrence of violence once mass civil disobedience was sanctioned. 

The Prince of Wales was greeted with empty streets and downed shutters wherever he went. Emboldened by their successful defiance of the Government, non-cooperators became more and more aggressive. The Congress Volunteer Corps emerged as a powerful parallel police, and the sight of its members marching in formation and dressed in uniform was hardly one that warmed the Government's heart. The Congress had already granted permission to the PCCs to sanction mass civil disobedience wherever they thought the people were ready and in some areas, such as Midnapur district in Bengal, which had started a movement against Union Board Taxes and Chirala- Pirala and Pedanandipadu taluqa in Guntur district of Andhra, no-tax movements were already in the offing.' 

The Non-Cooperation Movement had other indirect effects as well. In the Avadh area of U.P., where kisan sabhas and a kisan movement had been gathering strength since 1918, Non-cooperation propaganda, carried on among others by Jawaharlal Nehru, helped to fan the already existing ferment, and soon it became difficult to distinguish between a Non cooperation meeting and a kisan meeting.' In Malabar in Kerala, Non cooperation and Khilafat propaganda helped to arouse the Muslims tenants against their landlords, but the movement here, unfortunately, at tunes took on a communal colour.' 

In Assam, labourers on tea plantations went on strike. When the fleeing workers were fired upon, there were strikes on the steamer service, and on the Assam-Bengal Railway as well. 

J.M. Sengupta, the Bengali nationalist leader, played a leading role in these developments. In Midnapur, a cultivators' strike against a White zamindari company was led by a Calcutta medical student. Defiance of forest laws became popular in Andhra. Peasants and tribals in some of the Rajasthan states began movements for securing better conditions of life. In Punjab, the Akali Movement for `Test1ng control of the gurudwaras from the corrupt mahants (priests) was a part of the general movement of Non-cooperation, and the Akalis observed strict non-violence in the face of tremendous repression? The examples could be multiplied, but the point is that the spirit of unrest and defiance of authority engendered by the Non-Cooperation Movement contributed to the rise of many local movements in different parts of the country, movements which did not often adhere strictly either to the programme of the Non-Cooperation Movement or even to the policy of non-violence.

\begin{center}*\end{center}

\paragraph*{}


In this situation, it was hardly surprising that the Government came to the conclusion that its earlier policy had not met with success and that the time to strike had arrived. In September 1920, at the beginning of the movement, the Government had thought it best to leave it alone as repression would only make martyrs of the nationalists and fan the spirit of revolt. In May 1921, it had tried, through the Gandhi-Reading talks, to persuade Gandhiji to ask the Ali brothers to withdraw from their speeches those passage that contained suggestions of violence; this was an attempt to drive a wedge between the Khilafat leaders and Gandhiji, but it failed. By December, the Government felt that things were really going too far and announced a change of policy by declaring the Volunteer Corps illegal and arresting all those who claimed to be its members. 

C.R. Das was among the first be arrested, followed by his wife Basanti Debi, whose arrest so incensed the youth of Bengal that thousands came forward to court arrest. In the next two months, over 30,000 people were arrested from all over the country, and soon only Gandhiji out of the top leadership remained out of jail. In mid-December, there was an abortive attempt at negotiations, initiated by Malaviya, but the conditions offered were such that it meant sacrificing the Khilafat leaders, a course that Gandhiji would not accept. In any case, the Home Government had already decided against a settlement and ordered the Viceroy, Lord Reading, to withdraw from the negotiations. Repression continued, public meetings and assemblies were banned, newspapers gagged, and midnight raids on Congress and Khilafat offices became common. 

Gandhiji had been under considerable pressure from the Congress rank and file as well as the leadership to start the phase of mass civil disobedience. The Ahmedabad session of the Congress in December 1921 had appointed him the sole authority on the issue. The Government showed no signs of relenting and had ignored both the appeal of the All- Parties Conference held in mid-January 1922 as well as Gandhiji's letter to the Viceroy announcing that, unless the Government lifted the ban on civil liberties and released political prisoners, he would be forced to go ahead with mass civil disobedience. The Viceroy was unmoved and, left with no choice, Gandhiji announced that mass civil disobedience would begin in Bardoli taluqa of Surat district, and that all other parts of the country should cooperate by maintaining total discipline and quiet so that the entire attention of the movement could be concentrated on Bardoli. But Bardoli was destined to Wait for another six years before it could launch a no-tax movement. Its fate was decided by the action of members of a Congress and Khilafat procession in Chauri- Chaura in Gorakhpur district of U.P. on 5 February 1922. Irritated by the behavior of some policemen, a section of the crowd attacked them. The police opened fire. At this, the entire procession attacked the police and when the latter hid inside the police station, set fire to the building. Policemen who tried to escape were hacked to pieces and thrown into the fire. In all twenty-two policemen were done to dead. On hearing of the incident, Gandhiji decided to withdraw the movement. He also persuaded the Congress Working Committee to ratify his decision and thus, on 12 February 1922, the Non-Cooperation Movement came to an end. 

Gandhiji's, decision to withdraw the movement in response to the violence at Chauri Chaura raised a Controversy whose heat can still be felt in staid academic seminars and sober volumes of history. Motilal Nehru, C.R. Das, Jawaharlal Nehru, Subhas Bose, and many others have recorded their utter bewilderment on hearing the news. They could not understand why the whole country had to pay the price for the crazy behavior of some people in a remote village. Many in the country thought that the Mahatma had failed miserably as a leader and that his days of glory were over. 

Many later commentators, following, the tradition established by R. Palme Dutt in India Today, have continued to condemn the decision taken by Gandhiji, and seen in it proof of the Mahatma's concern for the propertied classes of Indian society. Their argument is that Gandhiji did not withdraw the movement simply because of his belief in the necessity of non-violence. He withdrew it because the action at Chauri Chaura was a symbol and an indication of the growing militancy of the Indian masses, of their growing radicalization, of their willingness to launch an attack on the status quo of property relations. Frightened by this radical possibility and by the prospect of the movement going out of his hands and into the hids of radical forces, and in order to protect the interests of landlords and capitalists who would inevitably be at the receiving end of this violence, Gandhiji cried halt to the movement. They have found supportive proof in the resolution of the Congress Working Committee of 12 February 1922 popularly known as the Bardoli resolution which while announcing the withdrawal, asked the peasants to pay taxes and tenants to pay rents. This, they say, was the real though hidden motive behind the historic decision of February 1922. 

It seems, however, that Gandhiji's critics have been less than fair to him. First, the argument that violence in a remote village could not be a sufficient cause for the decision is in itself a weak one. Gandhiji had repeatedly warned that he did not even want any non-violent movement in y other part of the country while he was conducting mass civil disobedience in Bardoli, and in fact had asked the Andhra PCC to withdraw the permission that it had granted to some of the District Congress Committees to start civil disobedience. One obvious reason for this was that, in such a situation of mass ferment and activity, the movement might easily take a violent turn, either due to its own volatile nature or because of provocation by the authorities concerned (as had actually happened in Bombay in November 1921 and later in Chauri Chaura); also if violence occurred anywhere it could easily be made the excuse by the Government to launch a massive attack on the movement as a whole. The Government could always cite the actual violence in one part as proof of the likelihood of violence in another part of the country, and thus justify its repression. This would upset the whole strategy of non violent civil disobedience which was based on the principle that the forces of repression would always stand exposed since they would be using armed force against peaceful civil resisters. It was, therefore, not enough to assert that there was no connection between Chauri Chaura and Bardoli. 

It is entirely possible that in Gandhiji's assessment the chances of his being allowed to conduct a mass civil disobedience campaign in Bardoli had receded further after Chauri Chaura. The Government would have had excuse to remove him and other activists from the scene and use force to cow down the people. Mass civil disobedience would be defeated even before it was given a fair trail. By taking the onus of withdrawal on himself and on the Working Committee, Gandhiji was protecting the movement from likely repression, and the people from demoralization. True, the withdrawal itself led to considerable demoralization, especially of the active political workers, but it is likely that the repression and crushing of the movement (as happened in 1932) would have led to even greater demoralization. Perhaps, in the long run, it was better to have felt that, if only Gandhiji had not withdrawn the movement, it would have surged forward, than to see it crushed and come to the conclusion that it was not possible for a mass movement to succeed in the face of government repression. It is necessary to remember that, after all, the Non Cooperation Movement was the first attempt at an all-India mass struggle against the British, and a serious reverse at this elementary stage could have led to a prolonged period of demoralization and passivity. 

The other argument that the real motive for withdrawal was the fear of the growth of radical forces and that Chauri Chaura was proof of the' emergence of precisely such a radical sentiment is on even thinner ground. The crowd at Chauri Chaura had not demonstrated any intention of attacking landlords or overturning the structure of property relations, they were merely angered by the overbearing behavior of policemen and vented their wrath by attacking them. Peasant unrest in most of Avadh and Malabar had died out long before this time, and the Eka movement that was on in some of the rural areas of Avadh showed no signs of wanting to abolish the zamindari system; it only wanted zamindars to stop `illegal' cesses and arbitrary rent enhancements. In fact, one of the items of the oath that was taken by peasants who joined the Eka movement was that they would `pay rent regularly at Kharif and Rabi.'' The no-tax movement m Guntur was very much within the framework of the Non-Cooperation Movement; it was directed against the government and remained totally peaceful. Moreover, it was already on the decline before February 1922. It is difficult to discern where the threat from radical tendencies is actually located. 

That the Bardoli resolution which announced the withdrawal also contained clauses which asked peasants to pay up taxes and tenants to pay up rents, and assured zamindars that the Congress had no intention of depriving them of their rights, is also no proof of hidden motives. The Congress had at no stage during the movement sanctioned non-payment of rent or questioned the rights of zamindars; the resolution was merely a reiteration of its position on this issue. Non-payment of taxes was obviously to cease if the movement as a whole was being withdrawn. 

There are also some indications that Gandhiji's decision may have been prompted by the fact that in many parts of the country, by the second half of 1921, the movement had shown clear signs of being on the ebb. Students had started drifting back to schools and colleges, lawyers and litigants to law courts, the commercial classes showed signs of weariness and worry at the accumulating stocks of foreign cloth, attendance at meetings and rallies had dwindled, both in the urban and rural areas. This does not mean that in some pockets, like Bardoli in Gujarat or Guntur in Andhra, where intensive political work had been done, the masses were not ready to carry on the struggle. But the mass enthusiasm that was evident all over the country in the first part of 1921 had, perhaps, receded. The cadre and the active political workers were willing to carry on the fight, but a mass movement of such a nature required the active participation of the masses, and not only of the highly motivated among them. However, at the present stage of research, it is not possible to argue this position with great force; we only wish to urge the possibility that this too was among the factors that led to the decision to withdraw. 

Gandhiji's critics often fail to recognize that mass movements have an inherent tendency to ebb after reaching a certain height, that the capacity of the masses to withstand repression, endure suffering and make sacrifices is not unlimited, that a time comes when breathing space is required to consolidate, recuperate, and gather strength for the next round of struggle, and that, therefore, withdrawal or a shift to a phase of non-confrontation is an inherent part of a strategy of political action that is based on the masses. Withdrawal is not tantamount to betrayal; it is an inevitable part of the strategy itself. 

Of course, whether or not the withdrawal was made at the correct time can always be a matter open to debate. But perhaps Gandhiji had enough reasons to believe that the moment he chose was the right one. The movement had already gone on for over a year, the Government was in no mood for negotiations, and Chauri Chaura presented an opportunity to retreat with honour, before the internal weaknesses of the movement became apparent enough to force a surrender or make the retreat look like a rout.

\begin{center}*\end{center}

\paragraph*{}


Gandhiji had promised Swaraj within a year if his programme was adopted. But the year was long over, the movement was withdrawn, and there was no sign of Swaraj or even of any tangible concessions. Had it all been in vain? Was the movement a failure? One could hardly answer in the affirmative. The Non- Cooperation Movement had in fact succeeded on many counts. It certainly demonstrated that it commanded the support and sympathy of vast sections of the Indian people. After Non-cooperation, the charge of representing a `microscopic minority,' made by the Viceroy, Dufferin, in 1888,' could never again be hurled at the Indian National Congress. Its reach among many sections of Indian peasants, workers, artisans, shopkeepers, traders, professionals, white-collar employees, had been demonstrated. The spatial spread of the movement was also nation-wide. Some areas were more active than others, but there were few that showed no signs of activity at all. 

The capacity of the `poor dumb millions' of India to take part in modem nationalist politics was also demonstrated. By their courage, sacrifice, and fortitude in the face of adversity and repression, they dispelled the notion that the desire for national freedom was the preserve of the educated and the rich and showed that it was an elemental urge common to all members of a subject nation. They may not as yet have fully comprehended all its implications, understood all the arguments put forth in its favour or observed all the discipline that the movement demanded for its successful conduct. This was, after all, for many of them, first contact with the modem world of nationalist politics and the modern ideology of nationalism. This was the first time that nationalists from the towns, students from schools and colleges or even the educated and politically aware in the villages had made a serious attempt to bring the ideology and the movement into their midst. Its success was bound to be limited, the weaknesses many. There were vast sections of the masses that even then remained outside the ambit of the new awakening. But this was only the beginning and more serious and consistent efforts were yet in the offing. But the change was striking. The tremendous participation of Muslims in the movement, and the maintenance of communal unity, despite the Malabar developments, was in itself no mean achievement. There is hardly any doubt that it was Muslim participation that gave the movement its truly mass character in many areas, at some places two-thirds of those arrested were Muslims. And it was, indeed, unfortunate that this most positive feature of the movement was not to be repeated in later years once communalism began to take its toll. The fraternization that was witnessed between Hindus and Muslims, with Gandhiji and other Congress leaders speaking from mosques, Gandhiji being allowed to address meetings of Muslim women in which he was the only male who was not blind-folded, all these began to look like romantic dreams in later years.

\begin{center}*\end{center}

\paragraph*{}


The retreat that was ordered on 12 February, 1922 was only a temporary one. The battle was over, but the war would continue. To the challenge thrown by Montague and Birkenhead that `India would not challenge with success the most determined people in the world, who would once again answer the challenge with all the vigour and determination at its command,' Gandhiji, in an article written in Young India on 23 February 1922 after the withdrawal of the movement, replied: `It is high time that the British people were made to realize that the fight that was commenced in 1920 is a fight to the finish, whether it lasts one month or one year or many months or many years and whether the representatives of Britain re enact all the indescribable orgies of the Mutiny days with redoubled force or whether they do not.''
