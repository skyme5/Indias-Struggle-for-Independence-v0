
\chapter{The Rise of the Left Wing}



A powerful left-wing group developed in India in the late 1920s and 1930s contributing to the radicalization of the national movement. The goal of political independence acquired a clearer and sharper social and economic content. The stream of national struggle for independence and the stream of the struggle for social and economic emancipation of the suppressed and the exploited began to come together. Socialist ideas acquired roots in the Indian soil; and socialism became the accepted creed of Indian youth whose urges came to be symbolized by Jawaharlal Nehru and Subhas Chandra Bose. Gradually there emerged two powerful parties of the Left, the Communist Party of India (CPI) and the Congress Socialist Party (CSP).

\begin{center}*\end{center}

\paragraph*{}


Seminal in this respect was the impact of the Russian Revolution. On 7 November 1917, the Bolshevik (Communist) party, led by V.I. Lenin, overthrew the despotic Czarist regime and declared the formation of the first socialist state. The new Soviet regime electrified the colonial world by unilaterally renouncing its imperialist rights in China and other parts of Asia. Another lesson was driven home: If the common people — the workers and peasants and the intelligentsia — could unite and overthrow the mighty Czarist empire arid establish a social order where there was no exploitation of one human being by another, then the Indian people battling against British imperialism could also do so. Socialist doctrines, especially Marxism, the guiding theory of the Bolshevik Party, acquired a sudden attraction, especially for the people of Asia. Bipin Chandra Pal, the famous Extremist leader, wrote in 1919: `Today after the downfall of German militarism, after the destruction of the autocracy of the Czar, there has grown up all over the world a new power, the power of the people determined to rescue their legitimate rights the right to live freely and happily without being exploited and victimized by the wealthier and the so-called higher classes.' Socialist ideas now began to spread rapidly especially because many young persons who had participated actively in the Non- Cooperation Movement were unhappy with its outcome and were dissatisfied with Gandhian policies and ideas as well as the alternative Swarajist programme. Several socialist and communist groups came into existence all over the country. In Bombay, S.A. Dange published a pamphlet Gandhi and Lenin and started the first socialist weekly, The Socialist; in Bengal, Muzaffar Ahmed brought out Navayug and later founded the Langal in cooperation with the poet NazruI Islam; in Punjab, Ghulam Hussain and others published Inquilab; and in Madras, 

M. Singaravelu founded the Labour-Kisan Gazette. 

Student and youth associations were organized all over the country from 1927 onwards. Hundreds of youth conferences were organized all over the country during 1928 and 1929 with speakers advocating radical solutions for the political, economic and social ills from which the country was suffering. Jawaharlal Nehru and Subhas Bose toured the country attacking imperialism, capitalism, and landlordism and preaching the ideology of socialism. The Revolutionary Terrorists led by Chandrasekhar Azad and Bhagat Singh also turned to socialism. Trade union and peasant movements grew rapidly throughout the 1920s. Socialist ideas became even more popular during the 1930s as the world was engulfed by the great economic depression. Unemployment soared all over the capitalist world. The world depression brought the capitalist system into disrepute and drew attention towards Marxism and socialism. Within the Congress the left-wing tendency found reflection in the election of Jawaharlal Nehru as president for 1936 and 1937 and of Subhas Bose for 1938 and 1939 and in the formation of the Congress Socialist Party.

\begin{center}*\end{center}

\paragraph*{}


It was above all Jawaharlal Nehru who imparted a socialist vision to the national movement and who became the symbol of socialism and socialist ideas in India after 1929. The notion that freedom could not be defined only m political terms but must have a socioeconomic content began increasingly to be associated with his name. 

Nehru became the president of the historic Lahore Congress of 1929 at a youthful forty. He was elected to the post again in 1936 and 1937. As president of the Congress and as the most popular leader of the national movement after Gandhiji, Nehru repeatedly toured the country, travelling thousands of miles and addressing millions of people. In his books (Autobiography and Glimpses of World History), articles and speeches, Nehru propagated the ideas of socialism and declared that political freedom would become meaningful only if it led to the economic emancipation of the masses; it had to, therefore, be followed by the establishment of a socialist society, Nehru thus moulded a whole generation of young nationalists and helped them accept a socialist orientation. 

Nehru developed an interest in economic questions when he came in touch with the peasant movement in eastern U.P. in 1920-21. He then used his enforced leisure in jail, during 1922-23, to read widely on the history of the Russian and other revolutions. In 1927, he attended the international Congress against Colonial Oppression and imperialism, held at Brussels, and came into contact with communists and anti-colonial fighters from all over the world. By now he had begun to accept Marxism in its broad contours. The same year he visited the Soviet Union and was deeply impressed by the new socialist society. On his return he published a book on the Soviet Union on whose title page he wrote Wordsworth's famous lines on French Revolution: `Bliss was it in that drawn to be alive, but to be young was very heaven.' Jawaharlal returned to India, in the words of his biographer S. Gopal, `a self-conscious revolutionary radical.' 

In 1928, Jawaharlal joined hands with Subhas to organize the Independence for India League to fight for complete independence and `a socialist revision of the economic structure of society.' At the Lahore session of the Congress in 1929, Nehru proclaimed: `I am a socialist and a republican, and am no believer in kings and princes, or in the order which produces the modem kings of industry, who have a greater power over the lives and fortunes of men than even the kings of old, and whose methods are as predatory as those of the old feudal aristocracy.' India, he said, would have to adopt a full `socialist programme' if she was `to end her poverty and inequality.' It was also not possible for the Congress to hold the balance between capital and labour and landlord and tenant, for the existing balance was `terribly weighted' in favour of the capitalists and landlords. 

Nehru's commitment to socialism found a clearer and sharper expression during 1933-36. Answering the question Whither India' in October 1933, he wrote: `Surely to the great human goal of social and economic equality, to the ending of all exploitation of nation by nation and class by class.' And in December 1933 he wrote: `The true civic ideal is the socialist ideal, the communist ideal.' He put his commitment to socialism in clear, unequivocal and passionate words in his presidential address to the Lucknow Congress in April 1936: `I am convinced that the only key to the solution of the world's problems and of India's problems lies in socialism, and when I use this world I do so not in a vague humanitarian way but in the scientific, economic sense... I see no way of ending the poverty, the vast unemployment, the degradation, and the subjection of the Indian people except through socialism. That involves vast and revolutionary changes in our political and social structure. That means the ending of private property, except in a restricted sense, and the replacement of the present profit system by a higher ideal of cooperative service. During these years, Nehru also emphasized the role of class analysis and class struggle. 

Nehru developed a complex relationship with Gandhiji during this period. He criticized Gandhiji for refusing to recognize the conflict of classes, for preaching harmony among the exploiters and the exploited, and for putting forward the theories of trusteeship by, and conversion of, the capitalists and landlords. In fact, Nehru devoted a whole CHAPTER in his Autobiography to gently combating some of the basic aspects of Gandhian ideology. At the same time, he fully appreciated the radical role that Gandhiji had played and was playing in Indian society. Defending Gandhiji against his left-wing critics, Jawaharlal contended in an article written in January 1936 that `Gandhi has played a revolutionary role in India of the greatest importance because he knew how to make the most of the objective conditions and could reach the heart of the masses; while groups with a more advanced ideology functioned largely in the air.' Moreover, Gandhiji's actions and teachings had `inevitably raised mass consciousness tremendously and made social issues vital. And his insistence on the raising of the masses at the cost, wherever necessary, of vested interests has given a strong orientation to the national movement in favour of the masses.' Nehru's advice to other Leftists in 1939 regarding the approach to be adopted towards Gandhiji and the Congress has been well summed up by Mohit Sen: Nehru believed that `the overwhelming bulk of the Congress was composed of amorphous centrists, that Gandhiji not only represented them but was also essential for any genuinely widespread mass movement, that on no account should the Left be at loggerheads with him or the centrists, but their strategy should rather be to pull the centre to the left — possibilities for which existed, especially as far as Gandhiji was concerned.' But Nehru's commitment to socialism was given within a framework that recognized the primacy of the political, anti- imperialist struggle so long as India was ruled by the foreigner. In fact the task was to bring the two commitments together without undermining the latter. Thus, he told the Socialists in 1936 that the two basic urges that moved him were `nationalism and political freedom as represented by the Congress and social freedom as represented by socialism'; and that `to continue these two outlooks and make them an organic whole is the problem of the Indian socialist.' 

Nehru, therefore, did not favour the creation of an organization independent of or separate from the Congress or making a break with Gandhiji and the right-wing of the Congress. The task was to influence and transform the Congress as a whole in a socialist direction. And this could be best achieved by working under its banner and bringing its workers and peasants to play a greater role in its organization. And in no case, he felt, should the Left become a mere sect apart from the mainstream of the national movement.

\begin{center}*\end{center}

\paragraph*{}


Attracted by the Soviet Union and its revolutionary commitment, a large number of Indian revolutionaries and exiles abroad made their way there. The most well-known and the tallest of them was M.N. Roy, who along with Lenin, helped evolve the Communist International's policy towards the colonies. Seven such Indians, headed by Roy, met at Tashkent in October 1920 and set up a Communist Party of India. Independently of this effort, as we have seen, a number of left-wing and communist groups and organizations had begun to come into existence in India after 1920. Most of these groups came together at Kanpur in December 1925 and founded an all-India organization under the name the Communist Party of India (CPI). After some time, 

S.V. Ghate emerged as the general secretary of the party. The CPI called upon all its members to enroll themselves as members of the Congress, form a strong left-wing in all its organs, cooperate with all other radical nationalists, and make an effort to transform the Congress into a more radical mass-based organization. 

The main form of political work by the early Communists was to organize peasants' and workers' parties and work through them. The first such organization was the Labour-Swaraj Party of the Indian National Congress organized by Muzaffar Ahmed, Qazi Nazrul Islam, Hemanta Kumar Sarkar, and others in Bengal in November 1925. In late 1926, a Congress Labour Party was formed in Bombay and a Kirti-Kisan Party in Punjab. A Labour Kisan Party of Hindustan had been functioning in Madras since 1923. By 1928 all of these provincial organizations had been renamed the Workers' and Peasants' Party (WPP) and knit into an All India party, whose units were also set up in Rajasthan, UP and Delhi. All Communists were members of this party. The basic objective of the WPPs was to work within the Congress to give it a more radical orientation and make it `the party of the people' and independently organize workers and peasants in class organizations, to enable first the achievement of complete independence and ultimately of socialism. The WPPs grew rapidly and within a short period the communist influence in the Congress began to grow rapidly, especially in Bombay. Moreover, Jawaharlal Nehru and other radical Congressmen welcomed the WPPs' efforts to radicalize the Congress. Along with Jawaharlal and Subhas Bose, the youth leagues and other Left forces, the WPPs played an important role in creating a strong left-wing within the Congress and in giving the Indian national movement a leftward direction. The WPPs also made rapid progress on the trade union front and played a decisive role in the resurgence of working class struggles during 1927-29 as also in enabling in Communists to gain a strong position in the working class. 

The rapid growth of communist and WPP influence over the national movement was, however, checked and virtually wiped out during 1929 and after by two developments. One was the severe repression to which Communists were subjected by the Government. Already in 1922-24, Communists trying to enter India from the Soviet Union had been tried in a series of conspiracy cases at Peshawar and sentenced to long periods of imprisonment. In 1924, the Government had tried to cripple the nascent communist movement by trying S.A. Dange, Muzaffar Ahmed, Nalini Gupa and Shaukat Usmani in the Kanpur Bolshevik Conspiracy Case. All four were sentenced to four years of imprisonment. 

By 1929, the Government was deeply worried about the rapidly growing communist influence in the national and trade union movements. It decided to strike hard. In a sudden swoop, in March 1929, it arrested thirty-two radical political and trade union activists, including three British Communists — Philip Spratt, Ben Bradley and Lester Hutchinson — who had come to India to help organize the trade union movement. The basic aim of the Government was to behead the trade union movement and to isolate the Communists from the national movement. The thirty-two accused were put up for trial at Meerut. The Meerut Conspiracy Case was soon to become a cause celebre. The defence of the prisoners was to be taken up by many nationalists including Jawaharlal Nehru, M.A. Ansari and M.C. Chagla. Gandhiji visited the Meerut prisoners in jail to show his solidarity with them and t0 seek their cooperation in the coming struggle. Speeches of defence made in the court by the prisoners were carried by all the nationalist newspapers thus familiarizing lakhs of people for the first time with communist ideas. The Government design to isolate the Communists from the mainstream of the national movement, not only miscarried but had the very opposite consequence. It did, however, succeed in one respect. The growing working class movement was deprived of its leadership. At this early stage, it was not easy to replace it with a new leadership. 

As if the Government blow was not enough, the Communists inflicted a more deadly blow on themselves by taking a sudden lurch towards what is described in leftist terminology as sectarian politics or `leftist deviation'. 

Guided by the resolutions of the Sixth Congress of the Communist International, the Communists broke their connection with the National Congress and declared it to be a class party of the bourgeoisie. Moreover, the Congress and the bourgeoisie it supposedly represented were declared to have become supporters of imperialism. Congress plans to organize a mass movement around the slogan of Poorna Swaraj were seen as sham efforts to gain influence over the masses by bourgeois leaders who were working for a compromise with British imperialism. Congress left leaders, such as Nehru and Bose, were described as `agents of the bourgeoisie within the national movement who were out to `bamboozle the mass of workers' and keep the masses under bourgeois influence. The Communists were now out to `expose' all talk of non-violent struggle and advance the slogan of armed struggle against imperialism, in 1931, the Gandhi-Irwin Pact was described as a proof of the Congress betrayal of nationalism. 

Finally, the Workers' and Peasants' Party was also dissolved on the ground that it was unadvisable to form a two-class (workers' and peasants') party for it was likely to fall prey to petty bourgeois influences. The Communists were to concentrate, instead, on the formation of an `illegal, independent and centralized' communist party. The result of this sudden shift in the Communists' political position was their isolation from the national movement at the very moment when it was gearing up for its greatest mass struggle and conditions were ripe for massive growth in the influence of the Left over it. Further, the Communists split into several splinter groups. The Government took further advantage of this situation and, in 1934, declared the CPI illegal. 

The Communist movement was, however, saved from disaster because, on the one hand, many of the Communists refused to stand apart from the Civil Disobedience Movement (CDM) and participated actively in it, and, on the other hand, socialist and communist ideas continued to spread in the country. Consequently, many young persons who participated in the CDM or in Revolutionary Terrorist organizations were attracted by socialism, Marxism and the Soviet Union, and joined the CPI after 1934. 

The situation underwent a radical change in 1935 when the Communist Party was reorganized under the leadership of P.C. Joshi. Faced with the threat of fascism the Seventh Congress of the Communist International, meeting at Moscow in August 1935, radically changed its earlier position and advocated the formation of a united front with socialists and other anti-fascists in the capitalist Countries and with bourgeois-led nationalist movements in colonial countries. The Indian Communists were to once again participate in the activities of the mainstream of the national movement led by the National Congress. The theoretical and political basis for the change in communist politics in India was laid in early 1936 by a document popularly known as the Dun-Bradley Thesis. According to this thesis, the National Congress could play `a great part and a foremost part in the work of realizing the anti-imperialist people's front.' 

The Communist Party now began to call upon its members to join the Congress and enrol the masses under their influence to the Congress. In 1938, it went further and accepted that the Congress was `the central mass political organization of the Indian people ranged against imperialism.'' And, in 1939, P.C. Joshi wrote in the party weekly, National Front, that the greatest class struggle today is our national struggle' of which Congress was the `main organ.''2 At the same time, the party remained committed to the objective of bringing the national movement under the hegemony of the working class, that is, the Communist Party. Communists now worked hard inside the Congress. Many occupied official positions inside the Congress district and provincial committees; nearly twenty were members of the All- India Congress Committee. During 1936-42, they built up powerful peasant movements in Kerala, Andhra, Bengal and Punjab. What is more important, they once again recovered their popular image of being the most militant of anti-imperialists.

\begin{center}*\end{center}

\paragraph*{}


The move towards the formation of a socialist party was made in the jails during 1930-31 and 1932-34 by a group of young Congressmen who were disenchanted with Gandhian strategy and leadership and attracted by socialist ideology. Many of them were active in the youth movement of the late 1920s. In the jails they studied and discussed Marxian and other socialist ideas. Attracted by Marxism, communism and Soviet Union, they did not find themselves in agreement with the prevalent political line of the CPI. Many of them were groping towards an alternative. Ultimately they came together and formed the Congress Socialist Party (CSP) at Bombay in October 1934 under the leadership of Jayaprakash Narayan, Acharya Narendra Dev and Minoo Masani. From the beginning, all the Congress socialists were agreed upon four basic propositions: that the primary struggle in India was the national struggle for freedom and that nationalism w..s a necessary stage on the way to socialism; that socialists must work inside the National Congress because it was the primary body leading the national struggle and, as Acharya Narendra Dev put it in 1934, It would be a suicidal policy for us to cut ourselves 3ff from the national movement that the Congress undoubtedly represents; that they must give the Congress and the national movement a socialist direction; and that to achieve this objective they must organize the workers and peasants in their class organizations, wage struggles for their economic demands and make them the social base of the national struggle.'' 

The CSP from the beginning assigned itself the task of both transforming the Congress and of strengthening it. The task of transforming the Congress was understood in two senses. One was the ideological sense. Congressmen were to be gradually persuaded to adopt a socialist vision of independent India and a more radical pro-labour and pro-peasant stand on current economic issues. This ideological and programmatic transformation was, however, to be seen not as an event but as a process. As Jayaprakash Narayan repeatedly told his followers in 1934: `We are placing before the Congress a programme and we want the Congress to accept it. If the Congress does not accept it, we do not say we are going out of the Congress. If today we fail, tomorrow we will try and if tomorrow we fail, we will try again.'' 

The transformation of the Congress was also seen in an organizational sense, that is, in terms of changes in its leadership at the top. Initially, the task was interpreted as the displacement of the existing leadership, which was declared to be incapable of developing the struggle of the masses to a higher level. The CSP was to develop as the nucleus of the alternative socialist leadership of the Congress. As the Meerut Thesis of the CSP put it in 1935, the task was to `wean the anti-imperialist elements in the Congress away from its present bourgeois leadership and to bring them under the leadership of revolutionary socialism.'' 

This perspective was, however, soon found to be unrealistic and was abandoned in favour of a `composite' leadership in which socialists would be taken into the leadership at all levels. The notion of alternate Left leadership of the Congress and the national movement came up for realization twice at Tripuri in 1939 and at Ramgarh in 1940. But when it came to splitting the Congress on a Left-Right basis and giving the Congress an executive left-wing leadership, the CSP (as also the CPI) shied away. Its leadership (as also CPI's) realized that such an effort would not only weaken the national movement but isolate the Left from the mainstream, that the Indian people could be mobilized into a movement only under Gandhiji's leadership and that, in fact, there was at the time no alternative to Gandhiji's leadership. However, unlike Jawaharlal Nehru, the leadership of the CSP, as also of other Left groups and parties, was not able to fully theorize or internalize this understanding and so it went back again and again to the notion of alternative leadership. 

The CSP was, however, firmly well grounded in the reality of the Indian situation. Therefore, it never carried its opposition to the existing leadership of the Congress to breaking point. Whenever it came to the crunch, it gave up its theoretical position and adopted a realistic approach close to that of Jawaharlal Nehru's. This earned it the condemnation of the other left-wing groups and parties — for example, in 1939, they were chastised for their refusal to support Subhas Bose in his confrontation with Gandhiji and the Right wing of the Congress. At such moments, the socialists defended themselves and revealed flashes of an empiricist understanding of Indian reality. Jayaprakash Narayan, for example, said in 1939 after Tripuri: `We Socialists do not want to create factions in the Congress nor do we desire to displace the old leadership of the Congress and to establish rival leadership. We are only concerned with the policy and programme of the Congress. We only want to influence the Congress decisions. Whatever our differences with the old leaders, we do not want to quarrel with them. We all want to march shoulder to shoulder in our common fight against imperialism.'' 

From the beginning the CSP leaders were divided into three broad ideological currents: the Marxian, the Fabian and the current influenced by Gandhiji. This would not have been a major weakness — in fact it might have been a source of strength — for a broad socialist party which was a movement. But the CSP was already a part, and a cadre-based party at that, within a movement that was the National Congress. Moreover, the Marxism of the 1930s was incapable of accepting as legitimate such diversity of political currents on the Left. The result was a confusion which plagued the CSP till the very end. The party's basic ideological differences were papered over for a long time because of the personal bonds of friendship and a sense of comradeship among most of the founding leaders of the party, the acceptance of Acharya Narendra Dev and Jayaprakash Narayan as its senior leaders, and its commitment to nationalism and socialism.

\begin{center}*\end{center}

\paragraph*{}


Despite the ideological diversity among the leaders, the CSP as a whole accepted a basic identification of socialism with Marxism. Jayaprakash Narayan, for example, observed in his book Why Socialism? that `today more than ever before it is possible to say that there is only one type, one theory of Socialism — Marxism.'' Gradually, however as Gandhiji's politics began to be more positively evaluated, large doses of Gandhian and liberal democratic thought were to become basic elements of the CSP leadership's thinking. 

Several other groups and currents developed on the Left in the I 930s. M.N. Roy came back to India in 1930 and organized a strong group of Royists who underwent several political and ideological transformations over the years. Subhas Bose and his left-wing followers founded the Forward Bloc in 1939 after Bose was compelled to resign from the Presidentship of the Congress. The Hindustan Socialist Republican Association, the Revolutionary Socialist Party, and various Trotskyist groups also functioned during the 193Os. There were also certain prestigious left-wing individuals, such as Swami Sahajanand Saraswati, Professor N.G. Ranga, and Indulal Yagnik, who worked outside the framework of any organized left-wing party. 

The CPI, the CSP and Jawaharlal Nehru, Subhas Bose and other Left groups and leaders all shared a common political programme which enabled them, despite ideological and organizational differences, to work together after 1935 and make socialism a strong current in Indian politics. The basic features of this programme were: consistent and militant anti-imperialism, anti-landlordism, the organization of workers and peasants in trade unions and kisan sabhas, the acceptance of a socialist vision of independent India and of the socialist programme of the economic and social transformation of society, and an anti-fascist, anti-colonial and anti-war foreign policy. 

Despite the fact that the Left cadres were among the most courageous, militant and sacrificing of freedom fighters, the Left failed in the basic task it had taken upon itself— to establish the hegemony of socialist ideas and parties over the national movement. It also failed to make good the promise it held out in the l930s. This is, in fact, a major enigma for the historian. 

Several explanations for this complex phenomenon suggest themselves. The Left invariably fought the dominant Congress leadership on wrong issues and, when it came to the crunch, was either forced to trail behind that leadership or was isolated from the national movement. Unlike the Congress right-wing, the Left failed to show ideological and tactical flexibility. It sought to oppose the right-wing with simplistic formulae and radical rhetoric. It fought the right-wing on slippery and wrong grounds. It chose to tight not on questions of ideology but on methods of struggle and on tactics. For example, its most serious charge against the Congress right-wing was that it wanted to compromise with imperialism, that it was frightened of mass struggle, that its anti-imperialism was not wholehearted because of bourgeois influence over it. The right-wing had little difficulty in disposing of such charges. The people rightly believed it and not the Left. Three important occasions may be cited as examples. In 1936-37, the Left fought the Right within the Congress on the issue of elections and office acceptance which was seen as a compromise with imperialism. In 1939-42, the tight was waged on the issue of the initiation of a mass movement, when Gandhiji's reluctance was seen as an aspect of his soft attitude towards imperialism and as the missing of a golden opportunity And, in 1945-47, the Left confronted the dominant Congress leadership, including Jawaharlal Nehru and Maulana Azad, on the question of negotiations for the transfer of power, which were seen as British imperialism's last ditch effort to prolong their domination and the tired Congress leadership's hunger for power or even betrayal. 

The Left also failed to make a deep study of Indian reality. With the exception of Jawaharlal Nehru, the Left saw the dominant Congress leadership as bourgeois its policy of negotiations as working towards a compromise with imperialism any resort to constitutional work as a step towards the `abandonment of the struggle for independence'. It took recourse to a simplistic model of analysing Indian social classes and their political behaviour. It saw all efforts to guide the national movement in a disciplined manner as imposing restrictions on the movement. It constantly counterposed armed struggle to non-violence as a superior form and method of struggle, rather than concentrating on the nature of mass involvement and mobilization and ideology. It was Convinced that the masses were ever ready for struggles in any form if only the leaders were willing to initiate them. It constantly overestimated its support among the people. Above all, the Left failed to grasp the Gandhian strategy of struggle. 

A major weakness of the Left was the failure of the different Left parties, groups and individuals to work unitedly except for short periods. All efforts at forging a united front of left-wing elements ended in frustration. Their doctrinal disputes and differences were too many and too passionately held, and the temperamental differences among the leaders overpowering. Nehru and Bose could not work together for long and bickered publicly in 1939. Nehru and the Socialists could not coordinate their politics. Bose and Socialists drifted apart after 1939. The CSP and the Communists made herculean efforts to work together from 1935 to 1940: The CSP opened its doors to Communists and Royists in 1935 so that the illegal Communist Party could have legal avenues for political work. But the Socialists and Communists soon drifted apart and became sworn enemies. The inevitable result was a long-term schism between the Socialists who suffered from an anti-Communist phobia and Communists who saw every Socialist leader as a potential bourgeois or (after 1947) American agent.

\begin{center}*\end{center}

\paragraph*{}


The Left did succeed in making a basic impact on Indian society and politics. The organization of workers and peasants, discussed elsewhere, was one of its greatest achievements. Equally important was its impact on the Congress. Organizationally, the Left was able to command influence over nearly one-third of the votes in the All-India Congress Committee on important issues. Nehru and Bose were elected Congress presidents from 1936 to 1939. Nehru was able to nominate three prominent Socialists, Acharya Narendra Dev, Jayaprakash Narayan and Achyut Patwardhan, to his Working Committee. In 1939, Subhas Bose, as a candidate of the Left, was able to defeat Pattabhi Sitaramayya in the presidential election by a majority of 1580 to 1377. 

Politically and ideologically, the Congress as a whole was given a strong Left orientation. As Nehru put it, Indian nationalism had been powerfully pushed `towards vital social changes, and today it hovers, somewhat undecided, on the brink of a new social ideology.'' The Congress, including its right-wing, accepted that the poverty and misery of the Indian people was the result not only of colonial domination but also of the internal socio-economic structure of Indian society which had, therefore, to be drastically transformed. The impact of the Left on the national movement was reflected in the resolution on Fundamental Rights and Economic Policy passed by the Karachi session of the Congress in 1931, the resolutions on economic policy passed at the Faizpur session in 1936, the Election Manifesto of the Congress in 1936, the setting up of a National Planning Committee in 1938, and the increasing shift of Gandhiji towards radical positions on economic and class issues.

\begin{center}*\end{center}

\paragraph*{}
The foundation of the All-India Students' Federation and the Progressive Writers' Association and the convening of the first All- India States' People's Conference in 1936 were some of the other major achievements of the Left The Left was also very active in the All-India Women's Conference. Above all, two major parties of the Left, the Communist Party and the Congress Socialist Party, had been formed, and were being built up.

\begin{center}*\end{center}

\paragraph*{}
Discussed in Chapters \ref{chapter:CH23}, \ref{chapter:CH25} and \ref{chapter:CH39}.