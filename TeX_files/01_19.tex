\chapter[The Years of Stagnation]{The Years of Stagnation: Swarajists, No Chargers and Gandhiji}



The withdrawal of the Non-Cooperation Movement in February 1922 was followed by the arrest of Gandhiji in March and his conviction and imprisonment for six years for the crime of spreading disaffection against the Government. The result was the spread of disintegration, disorganization and demoralization in the nationalist ranks. There arose the danger of the movement lapsing into passivity. Many began to question the wisdom of the total Gandhian strategy. Others started looking for ways out of the impasse.

A new line of political activity, which would keep up the spirit of resistance to colonial rule, was now advocated by C.R. Das and Motilal Nehru. They suggested that the nationalists should end the boycott of the legislative councils, enter them, expose them as `sham parliaments' and as `a mask which the bureaucracy has put on,' and obstruct `every work of the council.' This, they argued, would not be giving up non-cooperation but continuing it in a more effective form by extending it to the councils themselves. It would be opening a new front in the battle.

C.R. Das as the President of the Congress and Motilal as its Secretary put forward this programme of `either mending or ending' the councils at the Gaya session of the Congress in December 1922. Another section of the Congress, headed by Vallabhbhai Patel, Rajendra Prasad and C. Rajagopalachari, opposed the new proposal which was consequently defeated by 1748 to 890 votes. Das and Motilal resigned from their respective offices in the Congress and on 1 January 1923 announced the formation of the Congress-Khilafat Swaraj Party better known later as the Swaraj Party. Das was the President and Motilal one of the Secretaries of the new party. The adherents of the council- entry programme came to be popularly known as `pro-changers' and those still advocating boycott of the councils as `no— changers.' The Swaraj Party accepted the Congress programme in its entirety except in one respect — it would take part in elections due later in the `ear It declared that it would present the national demand for self- government in the councils and in case of its rejection its elected members would adopt `a policy of uniform, continuous and consistent obstruction within the councils, with a view to make the Government through the councils impossible.' The councils would, thus, be wrecked from within by creating deadlocks on every measure that came before them.

Both Das (born in 1870) and Motilal (born in 1861) were highly successful lawyers who had once been Moderates but had accepted the politics of boycott and non-cooperation in 1920. They had given up their legal practice, joined the movement as whole time workers and donated to the nation their magnificent houses in Calcutta and Allahabad respectively. They were great admirers of Gandhiji but were also his political equals. Both were brilliant and effective parliamentarians. One deeply religious and the other a virtual agnostic, both were secular to the core. Different in many ways, they complemented each other and formed a legendary political combination. Das was imaginative and emotional and a great orator with the capacity to influence and conciliate friends and foes. Motilal was firm, coolly analytical, and a great organizer and disciplinarian. They had such absolute trust and confidence in each other that each could use the other's name for any statement without prior consultation.

The no-changers, whose effective head was Gandhiji even though he was in jail, argued for the continuation of the full programme of boycott and non-cooperation, effective working of the constructive programme and quiet preparations for the resumption of the suspended civil disobedience.

\begin{center}*\end{center}

\paragraph*{}


The pro-changers and the no-changers were soon engaged in a fierce controversy. There was, of course, a lot of common ground between the two Both agreed that civil disobedience was not possible immediately and that no mass movement could be carried on indefinitely or for a prolonged period. Hence, breathing time was needed and a temporary retreat from the active phase of the movement was on the agenda. Both also accepted that there was need to rest and to reinvigorate the anti-imperialist forces, overcome demoralization, intensify politicization, widen political participation and mobilization, strengthen organization, arid keep up the recruitment, training and morale of the cadre. In fact, the national movement was facing the basic problem that any mass movement has to face: how were they to carry on political work in the movements' non- active phases?

It was in the answer to this last question that the two sides differed. The Swarajists said that work in the councils was necessary to fill in the temporary political void. This would keep up the morale of the politicized Indians, fill the empty newspaper spaces, and enthuse the people. Electioneering and speeches in the councils would provide fresh avenues for political agitation and propaganda.

Even without Congressmen, said the Swarajists, the councils would continue to function and, perhaps, a large number of people would participate in voting. This would lead to the weakening of the hold of the Congress. Moreover, non- Congressmen would capture positions of vantage and use them to weaken the Congress. Why should such vantage points in a revolutionary fight be left in the hands of the enemy?' By joining the councils and obstructing their work. Congressmen would prevent undesirable elements from doing mischief or the Government from getting some form of legitimacy for their laws. In other words, the Swarajists claimed that they would transform the legislatures into arenas of political struggle and that their intention was not to use them, as the Liberals desired, as organs for the gradual transformation of the colonial state, but to use them as the ground on which the struggle for the overthrow of the colonial state was to be carried out.

The no-changers opposed council-entry mainly on the ground that parliamentary work would lead to the neglect of constructive and other work among the masses, the loss of revolutionary zeal and political corruption. The legislators who would go into the councils with the aim of wrecking them would gradually give up the politics of obstruction. get sucked into the imperial constitutional framework, and start cooperating with the Government on petty reforms and piecemeal legislation Constructive work among the masses, on the other hand, would prepare them for the next round of civil disobedience.

As the pro-changer no-changer clash developed, the atmosphere of dismay in nationalist ranks began to thicken, and they began to be haunted by the fear of the repetition of the disastrous split of 1907. Pressure began to develop on the leaders to put a check on their public bickerings.

Both groups of leaders began to pull back from the brink and move `wards mutual accommodation. This trend was helped by several factors. First, the need for unity was felt very strongly by all the Congressmen. Secondly, not only the no-changers but also the Swarajists realized that however useful parliamentary work might be, the real sanctions which would compel the Government to accept national demands would be forged only by a mass movement outside the legislatures — and this would need unity. Lastly, both groups of leaders fully accepted the essentiality of Gandhiji's leadership.

Consequently, in a special session of the Congress held at Delhi in September 1923, the Congress suspended all propaganda against council entry and permitted Congressmen to stand as candidates and exercise their franchise in forthcoming elections.

\begin{center}*\end{center}

\paragraph*{}


Gandhiji was released from jail on 5 February 1924 on health grounds. He was completely opposed to council-entry as also to the obstruction of work in the councils which he believed was inconsistent with non-violent non-cooperation. Once again a split in the Congress loomed on the horizon. The Government very much hoped for and banked on such a split. When releasing the Mahatma, the Bombay Government had suggested that he `would denounce the Swarajists for their defection from the pure principle of non-cooperation, and thus considerably reduce in legislatures their power for harm.'2 Similarly, Reading, the Viceroy, told the Secretary of State for India, on 6 June 1924: `The probability of a split between Swarajists and Gandhiji is increasing ... Moonje, (The Swarajist leader from the Central Provinces) adds that the Swarajists are now driven to concentrating all their energy on breaking Gandhiji's hold on the Congress.'

But Gandhiji did not oblige. Step by step, he moved towards an accommodation with the Swarajists. In fact, his approach towards the Swarajists at this stage brings out some of the basic features of his political style, especially when dealing with co­ workers with whom he differed, and is therefore, worth discussing, however briefly.

Gandhiji's starting point was the fact that even when opposing the Swarajist leaders he had full trust in their bonafides. He described their as `the most valued and respected leaders' and as persons who `have made great sacrifices in the cause of the country and who yield to no one in their love of freedom of the motherland'4 Moreover, he and Das and Motilal Nehru throughout maintained warm personal relations based on mutual respect and regard. Immediately after his release, Gandhiji refused to publicly comment on council-entry till he had discussions with the Swarajist leaders. Even after meeting them, while he continued to believe in the futility and even harmful character of the Swarajists' programme, he remained convinced that public opposition to the `settled fact' of council-entry would be counterproductive.

The courageous and uncompromising manner in which the Swarajists had functioned in the councils convinced Gandhiji that, however politically wrong, they were certainly not becoming a limb of imperial administration. To the contrary, he noted, `they have shown determination, grit, discipline and cohesion and have not feared to carry their policy to the point of defiance. Once assume the desirability of entering Councils and it must be admitted that they have introduced a new spirit into the Indian Legislatures.''

Gandhiji was also pained by the bickerings in the worst of taste among the proponents of the two schools. As he wrote in April 1924: `Even the ``changers'' and the ``no-changers'' have flung mud against one another. Each has claimed the monopoly of truth and, with an ignorant certainty of conviction, sworn at the other for his helpless stupidity.' He was very keen to end such mud-slinging.

In any case, felt Gandhiji, council entry had already occurred and now to withdraw would be `disastrous' and would be `misunderstood' by the Government and the people `as a rout and weakness.'' This would further embolden the Government in its autocratic behaviour and repressive policy and add to the state of political depression among the people.

The last straw came when the Government launched a full attack on civil liberties and the Swarajists in Bengal in the name of fighting terrorism. It promulgated an ordinance on 25 October 1924 under which it conducted raids on Congress offices and house searches and arrested a large number of revolutionary terrorists and Swarajists and other Congressmen including Subhas Chandra Bose and two Swarajist members of the Bengal legislature, Anil Baran Roy and S.C. Mitra.

Perceiving a direct threat to the national movement, Gandhiji's first reaction was anger. He wrote in Young India on 31 October: `The Rowlatt Act is dead but the spirit that prompted it is like an evergreen. So long as the interest of Englishmen is antagonistic to that of Indians, so long must there be anarchic crime or the dread of it and an edition of the Rowlatt Act in answer.' As an answer to the Government's offensive against the Swarajists, he decided to show his solidarity with the Swarajists by `surrendering' before them. As he wrote in Young India: `I would have been false to the country if I had not stood by the Swaraj Party in the hour of its need... I must stand by it even though I do not believe in the efficacy of Council-entry or even some of the methods of conducting Council Warfare And again `Though an uncompromising No-changer. I must not only tolerate their attitude and work with them, but I must even strengthen them wherever I can.''

On 6 November 1924, Gandhiji brought the strife between the Swarajists and no-changers to an end, by signing a joint statement with Das and Motilal that the Swarajist Party would carry on work in the legislatures on behalf of the Congress and as an integral part of the Congress. This decision was endorsed in December at the Belgaum session of the Congress over which Gandhiji presided. He also gave the Swarajists a majority of seats on his Working Committee.

\begin{center}*\end{center}

\paragraph*{}


Elections to the legislative councils were held in November 1923. The Swarajist manifesto, released on 14 October, took up a strong anti- imperialist position: `The guiding motive of the British in governing India is to secure the selfish interests of their own country and the so-called ref onus arc a mere blind to further the said interests under the pretence of granting responsible government to India, the real object being to continue the exploitation of the unlimited resources of the country by keeping Indians permanently in a subservient position to Britain.'' It promised that the Swarajists would wreck the sham reforms from within the councils. Even though the Swarajists got only a few weeks to prepare for the elections and the franchise was extremely narrow -— only about 6.2 million or less than three per cent had the right to vote — they managed to do quite well. They won forty-two out of 101 elected seats in the Central Legislative Assembly they got a clear majority in the Central Provinces; they were the largest party in Bengal; and they fared quite well in Bombay and U.P., though not in Madras and Punjab because of strong casteist and communal currents.

In the Central Legislative Assembly, the Swarajists succeeded in building a common political front with the Independents led by M.A. Jinnah, the Liberals, and individuals such as Madan Mohan Malaviya. They built similar coalitions in most of the provinces. And they set out to inflict defeat after defeat on the Government.

The legislatures, reformed in 1919, had a `semblance' of power without any real authority. Though they had a majority of elected members, the executive at the centre or in the provinces was outside their control, being responsible only to the British Government at home. Moreover, the Viceroy or the Governor could certify any legislation, including a budgetary grant, if it was rejected in the legislature. The Swarajists forced the Government to certify legislation repeatedly at the centre as well as in many of the provinces, thus exposing the true character of the reformed councils. In March 1925, they succeeded in electing Vithalbhai Patel, a leading Swarajist, as the President of the Central Legislative Assembly.

Though intervening on every issue and often outvoting the Government, the Swarajists took up at the centre three major sets of problems on which they delivered powerful speeches which were fully reported in the Press and followed avidly every morning by the readers. One was the problem of constitutional advance leading to self-Government; second of civil liberties, release of political prisoners, and repeal of repressive laws; and third of the development of indigenous industries. In the very first session, Motilal Nehru put forward the national demand for the framing of a new constitution, which would transfer real power to India. This demand was passed by 64 votes to 48. It was reiterated and passed in September 1925 by 72 votes to 45. The Government had also to face humiliation when its demands for budgetary grants under different heads were repeatedly voted out. On one such occasion, Vithalbhai Patel told the Government: `We want you to carry on the administration of this country by veto and by certification. We want you to treat the Government of India Act as a scrap of paper which I am sure it has proved to be.'

Similarly, the Government was defeated several times on the question of the repeal of repressive laws and regulations and release of political prisoners. Replying to the official criticism of the revolutionary terrorists, C.S. Ranga Iyer said that the Government officials were themselves `criminals of the worst sort, assassins of the deepest dye, men who are murdering the liberties of a liberty-loving race.'' Lala Lajpat Rai said: `Revolutions and revolutionary movements are only natural ... there can be no progress in the world without revolutions and revolutionary movements.'' CR. Das was no less critical of the Government's repressive policy. He told the Bengal Provincial Conference: `Repression is a process in the consolidation of arbitrary power — and I condemn the violence of the Government for repression is the most violent form of violence —just as I condemn violence as a method of winning political liberty.''

The Swarajist activity in the legislatures was spectacular by any standards. It inspired the politicized persons and kept their political interest alive. People were thrilled every time the all- powerful foreign bureaucracy was humbled in the councils. Simultaneously, during 1923-24, Congressmen captured a large number of municipalities and other local bodies. Das became the Mayor of Calcutta (with Subhas Bose as his Chief Executive Officer), and Vithalbhai Patel. the President of Bombay Corporation, Vallabhbhai Patel of Ahmedabad Municipality, Rajendra Prasad of Patna Municipality, and Jawaharlal Nehru of Allahabad Municipality. The no-changers actively joined in these ventures since they believed that local bodies could be used to promote the constructive programme.

Despite their circumscribed powers, many of the municipalities and district boards, headed by a galaxy of leaders, set out to raise, however little, the quality of life of the people. They did excellent work in the fields of education, sanitation, health, ariti-untouchability, and khadi promotion, won the admiration of friend and foe, and quite often aroused popular enthusiasm.

The Swarajists suffered a major loss when C.R. Das died on 16 June 1925. Even more serious were a few other political developments. In the absence of a mass movement, communalism raised its ugly head and the political frustrations of the people began to find expression in communal riots. Actively encouraged by the colonial authorities, the communalists of all hues found a fertile field for their activities.

Its preoccupation with parliamentary politics also started telling on the internal cohesion of the Swaraj Party. For one, the limits of politics of obstruction were soon reached. Having repeatedly outvoted the Government and forced it to certify its legislation, there was no way of going further inside the legislatures and escalating the politics of confrontation. This could be done only by a mass movement outside. But the Swarajists lacked any policy of coordinating their militant work in the legislatures with mass political work outside. In fact, they relied almost wholly on newspaper reporting.

The Swarajists also could not carry their coalition partners for ever and in every respect, for the latter did not believe in the Swarajists' tactic of `uniform, continuous and consistent obstruction.' The logic of coalition politics soon began to pull back the Swarajists from militant obstructionism. Some of the Swarajist legislators could also not resist the pulls of parliamentary perquisites and positions of status and patronage.

The Government's policy of creating dissension among the nationalists by trying to separate the Swarajists from the Liberals, militant Swarajists from the more moderate Swarajists, and Hindus from Muslims began to bear fruit. In Bengal, the majority in the Swaraj Party failed to support the tenants' cause against the zamindars and, thereby, lost the support of its pro- tenant, mostly Muslim, members. Nor could the Swaraj Party avoid the intrusion of communal discord in its own ranks.

Very soon, a group of Responsivists arose in the party who wanted to work the reforms and to hold office wherever possible. The Responsivists joined the Government in the Central Provinces. Their ranks were soon swelled by N.C. Kelkar, M.R. Jayakar and other leaders. Lajpat Rai and Madan Mohan Malaviya too separated themselves from the Swaraj Party on Responsivist as well as communal grounds.

To prevent further dissolution and disintegration of the party, the spread of parliamentary `corruption,' and further weakening of the moral fibre of its members, the main leadership of the party reiterated its faith in mass civil disobedience and decided to withdraw from the legislatures in March 1926. Gandhiji, too, had resumed his critique of council-entry. He wrote to Srinivasa Iyengar in April 1926: The more I study the Councils' work, the effect of the entry into the Councils upon public life, its repercussions upon the Hindu-Muslim question, the more convinced I become not only of the futility but the inadvisability of Council-entry.''

\begin{center}*\end{center}

\paragraph*{}
The Swaraj Party went into the elections held in November 1926 as a party in disarray — a much weaker and demoralized force. It had to face the Government and loyalist elements and its own dissenters on the one side and the resurgent Hindu and Muslim communalists on the other. A virulent communal and unscrupulous campaign was waged against the Swarajists. Motilal Nehru was, for example, accused of sacrificing Hindu interests, of favouring cow-slaughter, and of eating beef. The Muslim communalists were no less active in branding the Swarajists as anti- Muslim. The result was a severe weakening of the Swaraj Party. It succeeded in winning forty seats at the centre and half the seats in Madras but was severely mauled in all other provinces, especially in U.P., C.P., and Punjab. Moreover, both Hindu and Muslim communalists increased their representation in the councils. The Swarajists also could not form a nationalist coalition in the legislatures as they had done in 1923.

Once again the Swarajists passed a series of adjournment motions and defeated the Government on a number of bills. Noteworthy was the defeat of the Government on the Public Safety Bill in 1928. Frightened by the spread of socialist and communist ideas and influence and believing that the crucial role in this respect was being played by British and other foreign agitators sent to India by the Communist International, the Government proposed to acquire the power to deport `undesirable' and `subversive' foreigners. Nationalists of all colours, from the moderates to the militants, united in opposing the Bill. Lala Lajpat Rai said, `Capitalism is only another name for Imperialism ... We are in no danger from Bolshevism or Communism. The greatest danger we are in, is from the capitalists and exploiters.'17 Motilal Nehru narrated his experiences in the Soviet Union and condemned anti-Soviet propaganda. He described the Public Safety Bill as `a direct attack on Indian nationalism, on the Indian National Congress' and as `the Slavery of India, Bill No. 1.' T. Prakasam said that the Bill's main aim was to prevent the spread of nationalism among workers and peasants.' Diwan Chaman Lall, then a firebrand protege of Motilal, declared: `If you are trying to preach against socialism, if you are demanding powers to suppress socialism, you will have to walk over our dead bodies before you can get that power.' Even the two spokesmen of the capitalist class, Purshottamdas Thakurdas and G.D. Birla, firmly opposed the Bill.

In March 1929, having failed to get the Bill passed, the Government arrested thirty-one leading communists, trade unionists and other leftwing leaders and put them on trial at Meerut. This led to strong criticism of the Government by the nationalists. Describing the arrests as presaging a period of terrorism,' Gandhi said that the motive behind these prosecutions is not to kill Communism, it is to strike terror.' He added: `Evidently it (the Government) believes in a periodical exhibition of its capacity (supersede all law and to discover to a trembling India the red claws which usually remain under cover.' The Swarajists finally walked out of the legislatures in 1930 as a result of the Lahore Congress resolution and the beginning of civil disobedience.

Their great achievement lay in their filling the political void at a time when the national movement was recouping its strength. And this they did without getting co-opted by the colonial regime. As Motilal Nehru wrote to his son: `We have stood firm.' While some in their ranks fell by the wayside as was inevitable in the parliamentary framework, the overwhelming majority proved their mettle and stood their ground. They worked in the legislatures in an orderly disciplined manner and withdrew from them whenever the call came. Above all, they showed that it was possible to use the legislatures in a creative manner even as they promoted the politics of self-reliant anti-imperialism. They also successfully exposed the hollowness of the Reform Act of 1919 and showed the people that India was being ruled by ` lawIess laws.

\begin{center}*\end{center}

\paragraph*{}


In the meantime, the no-changers carried on laborious, quiet, undemonstrative, grass-roots constructive work around the promotion of khadi and spinning, national education and Hindu-Muslim unity, the struggle against untouchability and the boycott of foreign cloth. This work was symbolized by hundreds of ashrams that came up all over the country where political cadres got practical training in khadi work and work among the lower castes and tribal people. For example, there was the Vedchi

Ashram in Bardoli taluqa, Gujarat, where Chimanlal Mehta, Jugatram Dave and Chimanlal Bhatt devoted their entire lives to the spread of education among the adivasis or kaliparaj; or the work done by Ravishankar Maharaj among the lower caste Baralyas of Kheda district.

In fact, Gandhian constructive work was multi-faceted in its content. It brought some much-needed relief to the poor, it promoted the process of the nation-in-the-making; and it made the urban-based and upper caste cadres familiar with the conditions of villages and lower castes. It provided Congress political workers or cadres Continuous and effective work in the passive phases of the national movement, helped build their bonds with those sections of the masses who were hitherto untouched by politics, and developed their organizing capacity and self-reliance. It filled the rural masses with a new hope and increased Congress influence among them.

Without the uplift of the lower castes and Adivasis there could be no united struggle against colonialism. The boycott of foreign cloth was a stroke of genius which demonstrated to rulers and the world the Indian people's determination to be free. National schools and colleges trained young men in an alternative, non-colonial ideological framework. A large number of young men and women who dropped out in 1920-21 went back to the officially recognized educational institutions but many often became whole time cadres of the movement.

As a whole, constructive work was a major channel for the recruitment of the soldiers of freedom and their political training — as also for the choosing and testing of their `officers' and leaders. Constructive workers were to act as the steelframe of the nationalist movement in its active Satyagraha phase. It was, therefore, not accidental that khadi bhandar workers, students and teachers of national schools and colleges, and Gandhian ashrams' inmates served as the backbone of the civil disobedience movements both as organizers and as active Satyagrahis.

The years 1922-27 were a period of contradictory developments. While the Swarajists and Gandhian constructive workers were quite active in their own separate ways, there simultaneously prevailed virulent factionalism and indiscipline in both the camps. By 1927, on the whole, an atmosphere of apathy and frustration had begun to prevail. Gandhiji wrote in May 1927: `My only hope therefore lies in prayer and answer to prayer.'

But underneath, after years of rest and recoupment, the forces of nationalism were again getting ready to enter a period of active struggle. This became evident in the rise of youth power and the national response to the Simon Commission.
