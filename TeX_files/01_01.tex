\chapter{The First Major Challenge: The Revolt of 1857}

It was the morning of 1857-05-11. The city of Delhi had not yet woken up when a band of Sepoys from Meerut, who had defied and killed the European officers the previous day, crossed the Jamuna, set the toll house on fire and marched to the Red Fort. They entered the Red Fort through the Raj Ghat gate, followed by an excited crowd, to appeal to Bahadur Shah II, the Moghul Emperor — a pensioner of the British East India Company, who possessed nothing but the name of the mighty Mughals — to become their leader, thus, give legitimacy to their cause. Bahadur Shah vacillated as he was neither sure of the intentions of the sepoys nor of his own ability to play an effective role. He was however persuaded, if not coerced, to give in and was proclaimed the Shahenshah-e-Hindustan. The sepoys, then, set out to capture and control the imperial city of Delhi. Simon Fraser, the Political Agent, and several other Englishmen were killed; the public offices were either occupied or destroyed. The Revolt of an unsuccessful but heroic effort to eliminate foreign rule had begun. The capture of Delhi and the proclamation of Bahadur Shah as the Emperor of Hindustan gave a positive political meaning to the revolt and provided a rallying point for the rebels by recalling the past glory of the imperial city.

The Revolt at Meerut and the capture of Delhi was the precursor to a widespread mutiny by the sepoys and rebellion almost all over North India, as well as Central and Western India. South India remained quiet and Punjab and Bengal were only marginally affected. Almost half the Company's sepoy strength of 2,32,224 opted out of their loyalty to their regimental colors and overcame the ideology of the army, meticulously constructed over a period of time through training and discipline.

Even before the Meerut incident, there were rumblings of resentment in various cantonments. The 19th Native Infantry at Berhampur which refused to use the newly introduced Enfield Rifle was disbanded in 1857-03-00. A young sepoy of the 34th Native Infantry, Mangal Pande, went a step further and fired at the Sergeant Major of his regiment. He was overpowered and executed and his regiment too was disbanded. The 7th Oudh regiment which defied its officers met with a similar fate.

Within a month of the capture of Delhi, the Revolt spread to different parts of the country: Kanpur, Lucknow, Benares, Allahabad, Bareilly, Jagdishpur, and Jhansi. The rebel activity was marked by intense anti-British feelings and the administration was invariably toppled. In the absence of any leaders from their own ranks, the insurgents turned to the traditional leaders of Indian society — the territorial aristocrats and feudal chiefs who had suffered at the hands of the British.

At Kanpur, the natural choice was Nana Saheb, the adopted son of the last Peshwa, Baji Rao II. He had refused the family title and, banished from Poona, was living near Kanpur. Begum Hazrat Mahal took over the reigns where popular sympathy was overwhelmingly in favor of the deposed Nawab. Her son, Birjis Qadir, was proclaimed the Nawab and a regular administration was organized with important offices shared equally by Muslims and Hindus.

At Barielly, Khan Bahadur, a descendant of the former ruler of Rohilkhand was placed in command. Living on a pension granted by the British, he was not too enthusiastic about this and had, in fact, warned the Commissioner of the impending mutiny. Yet, once the Revolt broke out, he assumed the administration, organized an army of 40,000 soldiers and offered stiff resistance to the British.

\begin{center}*\end{center}

\paragraph*{}
In Bihar, the Revolt was led by Kunwar Singh, the zamindar of Jagdishpur, a 70-year-old man on the brink of bankruptcy. He nursed a grudge against the British. He had been deprived of his estates by them and his repeated appeals to be entrusted with their management again fell on deaf ears. Even though he had not planned an uprising, he unhesitatingly joined the sepoys when they reached Arrah from Dinapore.

The most outstanding leader of the Revolt was Rani Lakshmibai, who assumed the leadership of the sepoys at Jhansi. Lord Dalhousie, the Governor-General, had refused to allow her adopted son to succeed to the throne after her husband died and had annexed the state by the application of the Doctrine of Lapse. The Rani Laxmibai had tried everything to reverse the decision. She even offered to keep Jhansi `safe' for the British if they would grant her wishes. When it was clear nothing was working she joined the sepoys and, in time, became one of the most formidable enemies the British had to contend with.

The Revolt was not confined to these major centers. It had embraced almost every cantonment in the Bengal and a few in Bombay. Only the Madras army remained totally loyal. Why did the sepoys revolt? It was considered prestigious to be in the service of the Company; it provided economic stability. Why, then, did the sepoys choose to forego these advantages for the sake of an uncertain future? A proclamation issued at Delhi indicates the immediate cause: `it is well known that in these days all the English have entertained these evil designs — first, to destroy the religion of the whole Hindustani Army, and then to make the people by compulsion Christians. Therefore, we, solely on account of our religion, have combined with the people, and have not spared alive one infidel, and have re-established the Delhi dynasty on these terms'.

It is certainly true that the conditions of service in the Company's army and cantonments increasingly came into conflict with the religious beliefs and prejudices of the sepoys, who were predominantly drawn from the upper caste Hindus of the North-Western Provinces and Oudh. Initially, the administration sought to accommodate the sepoys' demands: facilities were provided to them to live according to the dictates of their caste and religion. But, with the extension of the Army's operation not only to various parts of India but also to countries outside, it was not possible to do so anymore. Moreover, caste distinctions and segregation within a regiment were not conducive to the cohesiveness of a fighting unit. To begin with, the administration thought of an easy way out: discourage the recruitment of Brahmins; this apparently did not succeed and, by the middle of the nineteenth century, the upper castes predominated in the Bengal Army, for instance.

The unhappiness of the sepoys first surfaced in 1824 when the 47th Regiment at Barrackpur was ordered to go to Burma. To the religious Hindu, crossing the sea meant a loss of caste. The sepoys, therefore, refused to comply. The regiment was disbanded and those who led the opposition were hanged. The religious sensibilities of the sepoys who participated in the Afghan War were more seriously affected. During the arduous and disastrous campaigns, the fleeing sepoys were forced to eat and drink whatever came their way. When they returned to India, those at home correctly sensed that they could not have observed caste stipulations and therefore, were hesitant to welcome them back into the biradiri (caste fraternity). Sitaram who had gone to Afghanistan found himself outcast not only in his village but even in his own barracks. The Prestige of being in the pay of the Company was not enough to hold his Position in society; religion and caste proved to be more powerful.

\begin{center}*\end{center}

\paragraph*{}
The rumors about the Government's secret designs to promote conversions to Christianity further exasperated the sepoys. The official-missionary nexus gave credence to the rumor. In some cantonments, missionaries were permitted to preach openly and their diatribe against other religions angered the sepoys. The reports about the mixing of bone dust in atta and the introduction of the Enfield rifle enhanced the sepoys' growing disaffection with the Government. The cartridges of the new rifle had to be bitten off before loading and the grease was reportedly made of beef and pig fat. The army administration did nothing to allay these fears, and the sepoys felt their religion was in real danger.

The sepoys' discontent was not limited to religion alone. They were equally unhappy with their emoluments. A sepoy in the infantry got seven rupees a month. A sawar in the cavalry was paid Rs. 27, out of which he had to pay for his own uniform, food and the upkeep of his mount, and he was ultimately left with only a rupee or two. What was more galling was the sense of deprivation compared to his British counterparts. He was made to feel a subordinate at every step and was discriminated against racially and in matters of promotion and privileges. `Though he might give the signs of a military genius of Hyder,' wrote T.R. Holmes, `he knew that he could never attain the pay of an English subaltern and that the rank to which he might attain, after 30 years of faithful service, would not protect him from the insolent dictation of an ensign fresh from England.'' The discontent of the sepoys was not limited to matters military; they felt the general disenchantment with and opposition to British rule. The sepoy, in fact, was a peasant in uniform,' whose consciousness was not divorced from that of the rural population. A military officer had warned Dalhousie about the possible consequences of his policies: `Your army is derived from the peasantry of the country who have rights and if those rights are infringed upon, you will no longer have to depend on the fidelity of the army ... If you infringe the institutions of the people of India, that army will sympathize with them; for they are part of the population, and in every infringement you may make upon the rights of the individuals, you infringe upon the rights of men who are either themselves in the army or upon their sons, their fathers or their relations.'

\begin{center}*\end{center}

\paragraph*{}
Almost every agricultural family in Oudh had a representative in the army; there were 75,000 men from Oudh. Whatever happened there was of immediate concern to the sepoy. The new land revenue system introduced after the annexation and the confiscation of lands attached to charitable institutions affected his well-being. That accounted for the 14,000 petitions received from the sepoys about the hardships of the revenue system. A proclamation issued by the Delhi rebels clearly reflected the sepoy's awareness of the misery brought about by British rule. The mutiny in itself, therefore, was a revolt against the British and, thus, a political act. What imparted this character to the mutiny was the sepoy's identity of interests with the general population.

The Revolt of the sepoys was accompanied by a rebellion of the civil population, particularly in the North-Western Provinces and Oudh, the two areas from which the sepoys of the Bengal army were recruited. Except in Muzzafarnagar and Saharanpur, civil rebellion followed the Revolt of the sepoys. The action of the sepoys released the rural population from fear of the state and the control exercised by the administration. Their accumulated grievances found immediate expression and they rose en masse to give vent to their opposition to British rule. Government buildings were destroyed, the ``treasury was plundered, the magazine was sacked, barracks and courthouses were burnt and prison gates were flung open.'' The civil rebellion had a broad social base, embracing all sections of society — the territorial magnates, peasants, artisans, religious mendicants and priests, civil servants, shopkeepers, and boatmen. The Revolt of the sepoys, thus, resulted in a popular uprising.

\begin{center}*\end{center}

\paragraph*{}
The reason for this mass upsurge has to be sought in the nature of British rule which adversely affected the interests of almost all sections of society Under the burden of excessive taxes the peasantry became progressively indebted and impoverished. The only interest of the Company was the realization of maximum revenue with minimum effort.

Consequently, settlements were hurriedly undertaken, often without any regard for the resources of the land. For instance, in the district of Bareilly in 1812, the settlement was completed in the record time of ten months with a dramatic increase of Rs. 14.73,188 over the earlier settlement. Delighted by this increase, the Government congratulated the officers for their `zeal, ability, and indefatigable labor.' It did not occur to the authorities that such a sharp and sudden increase would have disastrous consequences on the cultivators. Naturally, the revenue could not be collected without coercion and torture: in Rohilkhand there were as many as 2,37,388 coercive collections during 1848-56. Whatever the conditions, the Government was keen on collecting revenue. Even in very adverse circumstances, remissions were rarely granted. A collector, who repeatedly reported his inability to realize revenue from an estate, as the only grass was grown there, was told that grass was a very good product and it should be sold for collecting revenue!

The traditional landed aristocracy suffered no less. In Oudh, which was a storm center of the Revolt, the taluqdars lost all their power and privileges. About 21,000 taluqdars whose estates were confiscated suddenly found themselves without a source of income, `unable to work, ashamed to beg, condemned to penury.' These dispossessed taluqdars smarting under the humiliation heaped on them, seized the opportunity presented by the Sepoy Revolt to oppose the British and regain what they had lost.

\begin{center}*\end{center}

\paragraph*{}
British rule also meant misery to the artisans and handicraftsmen. The annexation of Indian states by the Company cut off their major source of patronage. Added to this, British policy discouraged Indian handicrafts and promoted British goods. The highly skilled Indian craftsmen were deprived of their source of income and were forced to look for alternate sources of employment that hardly existed, as the destruction of Indian handicrafts was not accompanied by the development of modem industries.

The reforming zeal of British officials under the influence of utilitarianism had aroused considerable suspicion, resentment, and opposition. The orthodox Hindus and Muslims feared that through social legislation the British were trying to destroy their religion and culture. Moreover, they believed that legislation was undertaken to aid the missionaries in their quest for evangelization. The orthodox and the religious, therefore, arrayed against the British. Several proclamations of the rebels expressed this cultural concern in no uncertain terms.

The coalition of the Revolt of the sepoys and that of the civil population made the 1857 movement an unprecedented popular upsurge. Was it an organized and methodically planned Revolt or a spontaneous insurrection? In the absence of any reliable account left behind by the rebels, it is difficult to be certain. The attitude and activities of the leaders hardly suggest any planning or conspiracy on their part and if at all it existed it was at an embryonic stage.

When the sepoys arrived from Meerut, Bahadur Shah seems to have been taken by surprise and promptly conveyed the news to the Lt.Governor at Agra. So did Rani Lakshmibhai of Jhansi who took quite some time before openly joining the rebels. Whether Nana Saheb and Maulvi Ahmad Shah of Faizabad had established links with various cantonments and were instrumental in instigating Revolt is yet to be proved beyond doubt. Similarly, the message conveyed by the circulation of chappatis and lotus flowers is also uncertain. The only positive factor is that within a month of the Meerut incident the Revolt became quite widespread.

\begin{center}*\end{center}

\paragraph*{}
Even if there was no planning and organization before the revolt, it was important that it was done, once it started. Immediately after the capture of Delhi, a letter was addressed to the rulers of all the neighboring states and of Rajasthan soliciting their support and inviting them to participate. In Delhi, a court of administrators was established which was responsible for all matters of state. The court consisted of ten members, six from the army and four from the civilian departments. All decisions were taken by a majority vote. The court conducted the affairs of the state in the name of the Emperor. `The Government at Delhi,' wrote a British official, `seems to have been a sort of constitutional Milocracy. The king was king and honored as such, like a constitutional monarch; but instead of a Parliament, he had a council of soldiers, in whom power rested, and of whom he was no degree a military commander.' In other centers, also attempts were made to bring about an organization.

Bahadur Shah was recognized as the Emperor by all rebel leaders Coins were struck and orders were issued in his name. At Bareilly, Khan Bahadur Khan conducted the administration in the name of the Mughal Emperor. It is also significant that the first impulse of the rebels was always to proceed to Delhi whether they were at Meerut, Kanpur or Jhansi. The need to create an organization and a political institution to preserve the gains was certainly felt. But in the face of the British counter-offensive, there was no chance to build on these early nebulous ideas.

For more than a year, the rebels carried on their struggle against heavy odds. They had no source of arms and ammunition; what they had captured from the British arsenals could not carry them far. They `were often forced to fight with swords and pikes against an enemy supplied with the most modern weapons. They had no quick system of communication at their command and, hence, no coordination was possible. Consequently, they were unaware of the strength and weaknesses of their compatriots and as a result could not come to each other's rescue in times of distress. Everyone was left to play a lonely hand.

\begin{center}*\end{center}

\paragraph*{}
Although the rebels received the sympathy of the people, the country as a whole was not behind them. The merchants, intelligentsia and Indian rulers not only kept aloof but actively supported the British. Meetings were organized in Calcutta and Bombay by them to pray for the success of the British. Despite the Doctrine of Lapse, the Indian rulers who expected their future to be safer with the British liberally provided them with men and materials. Indeed, the sepoys might have made a better fight of it if they had received their support.

Almost half the Indian soldiers not only did not Revolt but fought against their own countrymen. The recapture of Delhi was affected by five columns consisting of 1700 British troops and 3200 Indians. The blowing up of Kashmere Gate was conducted by six British officers and NCOs and twenty-four Indians, of whom ten were Punjabis and fourteen were from Agra and Oudh.

Apart from some honorable exceptions like the Rani of Jhansi, Kunwar Singh, and Maulvi Ahmadullah, the rebels were poorly served by their leaders. Most of them failed to realize the significance of the Revolt and simply did not do enough. Bahadur Shah and Zeenat Mahal had no faith in the sepoys and negotiated with the British to secure their safety. Most of the taluqdars tried only to protect their own interests. Some of them, like Man Singh, changed sides several times depending on which side had the upper hand.

Apart from a commonly shared hatred for the alien rule, the rebels had no political perspective or a definite vision of the future. They were all prisoners of their own past, fighting primarily to regain their lost privileges. Unsurprisingly, they proved incapable of ushering in a new political order. John Lawrence rightly remarked that `had a single leader of ability arisen among them (the rebels) we must have been lost beyond redemption.'

That was not to be, yet the rebels showed exemplary courage, dedication, and commitment. Thousands of men courted death, fighting for a cause they held dear. Their heroism alone, however, could not stem the onslaught of a much superior British army. The first to fall was Delhi on 1857-09-20, after a prolonged battle. Bahadur Shah, who took refuge in Humayun's tomb, was captured, tried and deported to Burma. With that, the back of the Revolt was broken since Delhi was the only possible rallying point. The British military then dealt with the rebels in one center after another. The Rani of Jhansi died fighting on 1858-06-17. General Hugh Rose, who defeated her, paid high tribute to his enemy when he said that `here lay the woman who was the only man among the rebels.' Nana Saheb refused to give in and finally escaped to Nepal at the beginning of 1859, hoping to renew the struggle. Kunwar Singh, despite his old age, was too quick for the British troops and constantly kept them guessing till his death on 1858-05-09. Tantia Tope, who successfully carried on guerrilla warfare against the British until 1859-04-00, was betrayed by a zamindar, captured and put to `death' by the British.

Thus, came to an end the most formidable challenge the British Empire had to face in India. It is a matter of speculation as to what the course of history would have been had the rebels succeeded. Whether they would have put the clock back' and resurrected and reinforced a feudal order need not detain us here; although that was not necessarily the only option. Despite the sepoys' limitations and weaknesses, their effort to emancipate the country from foreign rule was a patriotic act and a Progressive step. If the importance of a historical event is not limited to its immediate achievements the Revolt of 1857 was not a pure historical tragedy. Even in failure, it served a grand purpose: a source of inspiration for the national liberation movement which later achieved what the Revolt could not.
