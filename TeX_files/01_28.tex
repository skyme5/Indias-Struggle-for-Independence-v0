\chapter{The Freedom Struggle in Princely India}
\begin{multicols}{2}

The variegated pattern of the British conquest of India, and the different stratagems through which the various parts of the country were brought under colonial rule, had resulted in two- fifths of the sub-continent being ruled by Indian princes. The areas ruled by the Princes included Indian States like Hyderabad, Mysore and Kashmir that were equal in size to many European countries, and numerous small States who counted their population in the thousands. The common feature was that all of them, big and small, recognized the paramountcy of the British Government. 

In return, the British guaranteed the Princes against any threat to their autocratic power, internal or external. Most of the princely States were run as unmitigated autocracies, with absolute power concentrated in the hands of the ruler or his favourites. The burden of the land tax was usually heavier than in British India and there was usually much less of the rule of law and civil liberties. The rulers had unrestrained power over the state revenues for personal use, and this often led to ostentatious living and waste Some of the more enlightened rulers and their ministers did make attempts, from time to time, to introduce reforms in the administration, the system of taxation and even granted powers to the people to participate in government But the vast majority of the States were bastions of economic, social, political and educational backwardness, for reasons not totally of their own making. 

Ultimately, it was the British Government that was responsible for the situation in which the Indian States found themselves in the twentieth century. As the national movement grew in strength, the Princes were increasingly called upon to play the role of `bulwarks of reaction.' Any sympathy with nationalism, such as that expressed by the Maharaja of Baroda, was looked upon with extreme disfavour. Many a potential reformer among the rulers was gradually drained of initiative by the constant surveillance and interference exercised by the British residents. There were honorable exceptions, however, and some States, like Baroda and Mysore, succeeded in promoting industrial and agricultural development, administrative and political reforms, and education to a considerable degree.

\begin{center}*\end{center}

\paragraph*{}

The advance of the national movement in British India, and the accompanying increase in political consciousness about democracy, responsible government and civil liberties had an inevitable impact on the people of the States. In the first and second decade of the twentieth century, runaway terrorists from British India seeking shelter in the States became agents of politicization. A much more powerful influence was exercised by the Non-Cooperation and Khilafat Movement launched in 1920; around this time and under its impact, numerous local organizations of the States' people came into existence. Some of the States in which praja mandals or States' People's Conferences were organized were Mysore, Hyderabad, Baroda, the Kathiawad States, the Deccan States, Jamnagar, Indore, and Nawanagar. This process came to a head in December 1927 with the convening of the All India States' People's Conference (AISPC) which was attended by 700 political workers from the States. The men chiefly responsible for this initiative were Baiwantrai Mehta, Manikial Kothari and G.R. Abhayankar. 

The policy of the Indian National Congress towards the Indian states had been first enunciated in 1920 at Nagpur when a resolution calling upon the Princes to grant full responsible government in their States had been passed. Simultaneously, however, the Congress, while allowing residents of the States to become members of the Congress, made it clear that they could not initiate political activity in the States in the name of Congress but only in their individual capacity or as members of the local political organizations. Given the great differences in the political conditions between British India and the States, and between the different States themselves, the general lack of civil liberties including freedom of association, the comparative political backwardness of the people, and the fact that the Indian States were legally independent entities, these were understandable restraints imposed in the interest of the movements in the States as ell as the movement in British India. The main emphasis was that people of the States should build up their own strength and demonstrate their capacity to struggle for their demands. Informal links between the congress and the various organisations of the people of the States, including the AISPC, always continued to be close. In 1927, the Congress reiterated as resolution of 1920, and in 1929. Jawaharlal Nehru, in his presidential address to the famous Lahore Congress, declared that `the Indian states cannot live apart from the rest of India ... the only people who have a right to determine the future of the states must be the people of those states') In later years, the Congress demanded that the Princes guarantee fundamental rights to their people. 

In the mid thirties, two associated developments brought about a distinct change in the situation in the Indian States. First, the Government of India Act of 1935 projected a scheme of federation in which the Indian States were to be brought into a direct constitutional relationship with British India and the States were to send representatives to the Federal Legislature. The catch was that these representatives would be nominees of the Princes and not democratically elected representatives of the people. They would number one-third of the total numbers of the Federal legislature and act as a solid conservative block that could be trusted to thwart nationalist pressures. The Indian National Congress and the AISPC and other organizations of the States' people clearly saw through this imperialist manoeuvre and demanded that the States be represented not by the Princes' nominees but by elected representatives of the people. This lent a great sense of urgency to the demand for responsible democratic government in the States. 

The second development was the assumption of office by Congress Ministries in the majority of the provinces in British India in 1937. The tact that the Congress was in power created a new sense of confidence and expectation in the people of the Indian States and acted as a spur to greater political activity. The Princes too had to reckon with a new political reality --- the Congress was no longer just a party in opposition but a party in power with a capacity to influence developments in contiguous Indian States. 

The years 1938--39, in fact, stand out as years of a new awakening in the Indian States and were witness to a large number of movements demanding responsible government and other reforms. Praja mandals mushroomed in many States that had earlier no such organizations. Major struggles broke out in Jaipur, Kashmir, Rajkot, Patiala, Hyderabad, Mysore, Travancore, and the Orissa States. 

These new developments brought about a significant change in Congress policy as well. Whereas, even in the Haripura session in 1938, the Congress had reiterated its policy that movements in the States should not be launched in the name of the Congress but should rely on their own independent strength and fight through local organizations, a few months later, on seeing the new spirit that was abroad among the people and their capacity to struggle. Gandhiji and the Congress changed their attitude on this question. The radicals and socialists in the Congress, as well as political workers in the States, had in any case been pressing for this change for quite some time. Explaining the shift in policy in an interview to the Times of India on 24 January, 1939, Gandhiji said: `The policy of non-intervention by the Congress was, in my opinion, a perfect piece of statesmanship when the people of the States were not awakened. That policy would be cowardice when there is all- round awakening among the people of the States and a determination to go through a long course of suffering for the vindication of their just rights ... The moment they became ready, the legal, constitutional and artificial boundary was destroyed.' 

Following upon this, the Congress at Tripuri in March 1939 passed a resolution enunciating its new policy: `The great awakening that is taking place among the people of the States may lead to a relaxation, or to a complete removal of the restraint which the Congress imposed upon itself, thus resulting in an ever increasing identification of the Congress with the States' peoples'.3 Also in 1939, the AISPC elected Jawaharlal Nehru as its President for the Ludhiana session, thus setting the seal on the fusion of the movements in Princely India and British India. 

The outbreak of the Second World War brought about a distinct change in the political atmosphere. Congress Ministries resigned, the Government armed itself with the Defence of India Rules, and in the States as well there was less tolerance of political activity. Things came to a head again in 1942 with the launching of the Quit India Movement. This time the Congress made no distinction between British India and the Indian States and the call for struggle was extended to the people of the States. The people of the States thus formally joined the struggle for Indian independence, and in addition to their demand for responsible government they asked the British to quit India and demanded that the States become integral parts of the Indian nation. 

The negotiations for transfer of power that ensued after the end of the War brought the problem of the States to the centre of the stage. It was, indeed, to the credit of the national leadership, especially Sardar Patel, that the extremely complex situation created by the lapse of British paramountcy which rendered the States legally independent --- was handled in a manner that defused the situation to a great degree. Most of the States succumbed to a combination of diplomatic pressure, arm- twisting, popular movements and their own realization that independence was not a realistic alternative, and signed the Instruments of Accession. But some of the States like Travancore, Junagadh, Kashmir and Hyderabad held out till the last minute. Finally, only Hyderabad held out and made a really serious bid for Independence. 

To illustrate the pattern of political activity in the Indian States, it is instructive to look more closely at the course of the movements in two representative States, Rajkot and Hyderabad --- one among the smallest and the other the largest, one made famous by Gandhiji's personal intervention and the other by its refusal to accede to the Indian Union in 1947, necessitating the use of armed forces to bring about its integration.

\begin{center}*\end{center}

\paragraph*{}

Rajkot, a small state with a population of roughly 75,000, situated in the Kathiawad peninsula, had an importance out of all proportion to its size and rank among the States of Western India because Rajkot city was the seat of the Western India State Agency from where the British Political Agent maintained his supervision of the numerous States of the area. Rajkot had enjoyed the good fortune of being ruled for twenty years till 1930 --- by Lakhajiraj, who had taken great care to promote the industrial, educational and political development of his state. Lakhajiraj encouraged popular participation in government by inaugurating in 1923 the Rajkot Praja Pratinidhi Sabha. This representative assembly consisted of ninety representatives elected on the basis of universal adult franchise, something quite unusual in those times. Though the Thakore Sahib, as the ruler was called, had full power to veto any suggestion, yet under Lakhajiraj this was the exception rather than the rule and popular participation was greatly legitimized under his aegis. 

Lakhajiraj had also encouraged nationalist political activity by giving permission to Mansukhlal Mehta and Amritlal Sheth to hold the first Kathiawad Political Conference in Rajkot in 1921 which was presided over by Vithalbhai Patel. He himself attended the Rajkot and Bhavnagar (1925) sessions of the Conference, donated land in Rajkot for the starting of a national school that became the centre of political activity' and, in defiance of the British Political Agent or Resident, wore khadi as a symbol of the national movement. He was extremely proud of Gandhiji and his achievements and often invited `the son of Rajkot' to the Durbar and would then make Gandhiji sit on the throne while he himself sat in the Durbar. He gave a public reception to Jawaharlal Nehru during his visit to the State. 

Lakhajiraj died in 1939 and his son Dharmendra Singhji, a complete contrast to the father, soon took charge of the State. The new Thakore was interested only in pleasure, and effective power fell into the hands of Dewan Virawala, who did nothing to stop the Thakore from frittering away the State's wealth, and finances reached such a pass that the State began to sell monopolies for the sale of matches, sugar, rice, and cinema licences to individual merchants. This immediately resulted in a rise in prices and enhanced the discontent that had already emerged over the Thakore's easy-going life-style and his disregard for popular participation in government as reflected in the lapse of the Pratinidhi Sabha as well as the increase in taxes. The ground for struggle had been prepared over several years of political work by political groups in Rajkot and Kathiawad. The first group had been led by Mansukhlal Mehta and Amritlal Sheth and later by Balwantrai Mehta. another by Phulchand Shah, a third by Vrajlal Shukia, and a fourth group consisted of Gandhian constructive workers who, after 1936, under the leadership of U.N. Dhebar, emerged as the leading group in the Rajkot struggle. 

The first struggle emerged under the leadership of Jethalal Joshi, a Gandhian worker, who organized the 800 labourers of the state-owned cotton mill into a labour union and led a twenty- one day strike in 1936 to secure better working conditions. The Durbar had been forced to concede the union's demands. This victory encouraged Joshi and Dhebar to convene, in March 1937, the first meeting of the Kathiawad Rajakiya Parishad to be held in eight years. The conference, attended by 15,000 people, demanded responsible government, reduction in taxes and state expenditure. 

There was no response from the Durbar and, on 15 August 1938, the Panshad workers organized a protest against gambling (the monopoly for which had been sold to a disreputable outfit called Carnival) at the Gokulashtmi Fair. According to a pre-arranged plan, the protesters were severely beaten with lathis first by the Agency police and then by the State police. This resulted in a complete hanal in Rajkot city, and a session of the Parishad was held on 5 September and presided over by Sardar Patel. In a meeting with Dewan Virawala, Patel, on behalf of the Parishad, demanded a committee to frame proposals for responsible government, a ne' election to the Pratinidhi Sabha, reduction of land revenue by fifteen percent, cancellation of all monopolies or /ijaras, and a limit on the ruler's claim on the State treasury. The Durbar, instead of conceding the demands, asked the Resident to appoint a British officer as Dewan to deal effectively with the situation, and Cadell took over on 12 September. Meanwhile, Virawala himself became Private Adviser to the Thakore, so that he could continue to operate from behind the scenes. 

The Satyagraha now assumed major proportions and included withhold of land revenue, defiance of monopoly rights, boycott of all goods produced by the State, including electricity and cloth. There was a run on the State Bank and strikes in the state cotton mill and by students. All sources of income of the state, including excise and custom duties, were sought to be blocked. 

Sardar Patel, though most of the tune not physically present in Rajkot, kept in regular touch with the Rajkot leaders by telephone every evening. Volunteers began to arrive from other parts of Kathiawad, from British Gujarat and Bombay. The movement demonstrated a remarkable degree of organization: a secret chain of command ensured that on the arrest of one leader another took charge and code numbers published in newspapers informed each Satyagrahi of his arrival date and arrangements in Rajkot. 

By the end of November, the British were clearly worried about the implications of a possible Congress victory in Rajkot. The Viceroy, Linlithgow, wired to the Secretary of State: `I have little doubt that if Congress were to win in the Rajkot case the movement would go right through Kathiawad, and that they would then extend their activities in other directions . . ` 

But the Durbar decided to ignore the Political Department's advice and go ahead with a settlement with Sardar Patel. The agreement that was reached on 26 December, 1938, provided for a limit on the Thakore's Privy Purse and the appointment of a committee of ten State subjects or officials to draw up a scheme of reforms designed to give the widest possible powers to the people. A separate letter to the Sardar by the Thakore contained the informal understanding that `seven members of the Committee ... are to be recommended by Sardar Vallabhbhai Patel and they are to be nominated by us'. All prisoners were released and the Satyagraha was withdrawn. 

But such open defiance by the Thakore could hardly be welcomed by the British government. Consultations involving the Resident, the Political Department, the Viceroy and the Secretary of State were immediately held and the Thakore was instructed not to accept the Sardar's list of members of the Committee, but to select another set with the help of the Resident. Accordingly, the list of names sent by Patel was rejected, the excuse being that it contained the names of only Brahmins and Banias, and did not give any representation to Rajputs, Muslims and the depressed classes. 

The breach of agreement by the State led to a resumption of the Satyagraha on 26 January 1939. Virawala answered with severe repression. As before, this soon led to a growing concern and sense of outrage among nationalists outside Rajkot. Kasturba, Gandhiji's wife, who had been brought up in Rajkot, was so moved by the state of affairs that she decided, in spite of her poor health and against everybody's advice, to go to Rajkot. On arrival, she and her companion Maniben Patel, the Sardar's daughter, were arrested and detained in a village sixteen miles from Rajkot. 

But Rajkot was destined for even more dramatic events. The Mahatma decided that he, too, must go to Rajkot. He had already made it clear that he considered the breach of a solemn agreement by the Thakore Sahib a serious affair and one that was the duty of every Satyagrahi to resist. He also felt that he had strong claims on Rajkot because of his family's close association with the State and the Thakore's family, and that this justified and prompted his personal intervention. In accordance with his wishes, mass Satyagraha was suspended to prepare the way for negotiations. But a number of discussions with the Resident, the Thakore and Dewan Virawala yielded no results and resulted in an ultimatum by Gandhiji that if, by 3 March, the Durbar did not agree to honour its agreement with the Sardar, he would go on a fast unto death. The Thakore, or rather Virawala, who was the real power behind the throne, stuck to his original position and left Gandhiji with no choice but to begin his fast. 

The fast was the signal for a nation-wide protest. Gandhiji's health was already poor and any prolonged fast was likely to be dangerous. There were hartals, an adjournment of the legislature and finally a threat that the Congress Ministries might resign. The Viceroy was bombarded with telegrams asking for his intervention. Gandhiji himself urged the Paramount Power to fulfil its responsibility to the people of the State by persuading the Thakore to honour his promise. On 7 March, the Viceroy suggested arbitration by the Chief Justice of India, Sir Maurice Gwyer, to decide whether in fact the Thakore had violated the agreement. This seemed a reasonable enough proposition, and Gandhiji broke his fast. 

The Chief Justice's award, announced on 3 April, 1939, vindicated the Sardar's position that the Durbar had agreed to accept seven of his nominees. The ball was now back in the Thakore's court. But there had been no change of heart in Rajkot. Virawala continued with his policy of propping up Rajput, Muslim and depressed classes' claims to representation and refused to accept any of the proposals made by Gandhiji to accommodate their representatives while maintaining a majority of the Sardar's and the Parishad's nominees. 

The situation soon began to take an ugly turn, with hostile demonstrations by Rajputs and Muslims during Gandhiji's prayer meetings, and Mohammed Au Jinnah's and Ambedkar's demand that the Muslims and depressed classes be given separate representation. The Durbar used all this to continue to refuse to honour the agreement in either its letter or spirit. The Paramount Power, too, would not intervene because it had nothing to gain and everything to lose from securing an outright Congress victory. Nor did it see its role as one of promoting responsible government in the States. 

At this point, Gandhiji, analyzing the reasons for his failure to achieve a `change of heart' in his opponents, came to the conclusion that the cause lay in his attempt to use the authority of the Paramount Power to coerce the Thakore into an agreement. This, for him, smacked of violence; non-violence should have meant that he should have directed his fast only at the Thakore and Virawala, arid relied only on the strength of his suffering to effect a `change of heart'. Therefore, he released the Thakore from the agreement, apologized to the Viceroy and the Chief Justice for wasting their time, and to his opponents, the Muslims and the Rajputs, and left Rajkot to return to British India. 

The Rajkot Satyagraha brought into clear focus the paradoxical situation that existed in the States and which made the task of resistance a very complex one. The rulers of the States were protected by the might of the British Government against any movements that aimed at reform and popular pressure on the British Government to induce reform could always be resisted by pleading the legal position of the autonomy of the States. This legal independence, however, was usually forgotten by the British when the States desired to follow a course that was unpalatable to the Paramount Power. It was, after all, the British Government that urged the Thakore to refuse to honour his agreement with the Sardar. But the legal separation of power and responsibility between the States and the British Government did provide a convenient excuse for resisting pressure, an excuse that did not exist in British India. This meant that movements of resistance in the States operated in conditions that were very different from those that provided the context for movements in British India. Perhaps, then, the Congress had not been far wrong when for years it had urged that the movements in Princely India and British India could not be merged. Its hesitation to take on the Indian States was based on a comprehension of the genuine difficulties in the situation, difficulties which were clearly shown up by the example of Rajkot. 

Despite the apparent failure of the Rajkot Satyagraha, it exercised a powerful politicizing influence on the people of the States, especially in Western India. It also demonstrated to the Princes that they survived only because the British were there to prop them up, and thus, the struggle of Rajkot, along with others of its time, facilitated the process of the integration of the States at the time of independence. Many a Prince who had seen for himself that the people were capable of resisting would hesitate in 1947 to resist the pressure for integration when it came. In the absence of these struggles, the whole process of integration would inevitably have been arduous and protracted. It is hardly a matter of surprise that the man who was responsible more than any other for effecting the integration in 1947--48 was the same Sardar who was a veteran of many struggles against the Princes.

\begin{center}*\end{center}

\paragraph*{}
Bu there was one State that refused to see the writing on the wall --- - Hyderabad. Hyderabad was the largest princely State in India both by virtue of its size and its population. The Nizam's dominions included three distinct linguistic areas: Marathi- speaking (twenty-eight percent), Kannadas peaking (twenty-two) and Telugu-speaking (fifty per cent). Osman Ali Khan, who became Nizam in 1911 and continued till 1948, ruled the State as a personalized autocracy. The sarf khas, the Nizam's own estate, which accounted for ten per cent of the total area of the State, went directly to meet the royal expenses. Another thirty per cent of the States' area was held as jagirs by various categories of the rural population and was heavily burdened by a whole gamut of illegal levies and exactions and forced labour or vethi. 

Particularly galling to the overwhelmingly Hindu population of the State was the cultural and religious suppression practised by the Nizam. Urdu was made the court language and all efforts were made to promote it, including the setting up of the Osmania University. Other languages of the State --- Telugu, Marathi and Kannada --- were neglected and even private efforts to promote education in these languages were obstructed. Muslims were given a disproportionately large share of the jobs in the administration, especially in its upper echelons. The Arya Samaj Movement that grew rapidly in the 1920s was actively suppressed and official permission had to be sought to set up a havan kund for Arya Samaj religious observances. The Nizam's administration increasingly tried to project Hyderabad as a Muslim state, and this process was accelerated after 1927 with the emergence of the Ittehad ul Muslimin, an organization that based itself on the notion of the Nizam as the `Royal Embodiment of Muslim Sovereignty in the Deccan.' 

It is in this context of political, economic, cultural and religious oppression that the growth of political consciousness and the course of the State's People's Movement in Hyderabad has to be understood. 

As in other parts of India, it was the Non-Cooperation and Khilafat Movement of 1920--22 that created the first stirrings of political activity. From various parts of the State, there were reports of charkhas being popularized national schools being set up, of propaganda against drink and untouchability, of badges containing pictures of Gandhiji and the All brothers being sold. Public meetings were not much in evidence, expect in connection with the Khilafat Movement, which could take on a more open form because the Nizam hesitated to come out openly against it. Public demonstration of Hindu-Muslim unity was very popular in the' years. 

This new awakening found expression in the subsequent years in the holding of a series of Hyderabad political conferences at different venues outside the State. The main discussion at these conferences cantered around the need for a system of responsible government and for elementary civil liberties that were lacking in the State. Oppressive practices like vethi or veth begar and exorbitant taxation, as well as the religious and cultural suppression of the people, were also condemned. 

Simultaneously, there began a process of regional cultural awakening, the lead being taken by the Telengana area. A cohesion to this effort was provided by the founding of the Andhra Jana Sangham which later grew into the Andhra Mahasabha. The emphasis initially was on the promotion of Telugu language and literature by setting up library associations, schools, journals and newspapers and promoting a research society. Even these activities came under attack from the State authorities, and schools, libraries and newspapers would be regularly shut down. The Mahasabha refrained from any direct political activity or stance till the 1940s. 

The Civil Disobedience Movement of 1930--32, in which many people from the State participated by going to the British areas, carried the process of politicization further. Hyderabad nationalists, especially many of the younger ones, spent time in jail with nationalists from British India and became part of the political trends that were sweeping the rest of the nation. A new impatience was imparted to their politics, and the pressure for a more vigorous politics became stronger. In 1937, the other two regions of the State also set up their own organizations --- the Maharashtra Parishad and the Kannada Parishad. And, in 1938, activists from all three regions came together and decided to found the Hyderabad State Congress as a state-wide body of the people of Hyderabad. This was not a branch of the Indian National Congress, despite its name, and despite the fact that its members had close contacts with the Congress. But even before the organization could be formally founded, the Nizam's government issued orders banning it, the ostensible ground being that it was a communal body of Hindus and that Muslims were not sufficiently represented in it. Negotiations with the Government bore no fruit, and the decision was taken to launch a Satyagraha. 

The leader of this Satyagraha was Swami Ramanand Tirtha, a Marathi-speaking nationalist who had given up his studies during the Non-Cooperation Movement, attended a national school and college, worked as a trade unionist in Bombay and Sholapur and finally moved to Mominabad in Hyderabad State where he ran a school on nationalist lines. A Gandhian in his life-style and a Nehruite in his ideology, Swamiji emerged in 1938 as the leader of the movement since the older and more established leaders were unwilling or unable to venture into this new type of politics of confrontation with the State. 

The Satyagraha started in October 1938 and the pattern adopted was that a group of five Satyagrahis headed by a popular leader and consisting of representatives of all the regions would defy the ban by proclaiming themselves as members of the State Congress. This was repeated thrice a week for two months and all the Satyagrahis were sent to jail. Huge crowds would collect to witness the Satyagraha and express solidarity with the movement. The two centres of the Satyagraha were Hyderabad city and Aurangabad city in the Marathwada area. 

Gandhiji himself took a keen personal interest in the developments, and regularly wrote to Sir Akbar Hydari, the Prime Minister, pressing him for better treatment of the Satyagrahis and for a change in the State's attitude. And it was at his instance that, after two months, in December j 1938, the Satyagraha was withdrawn. The reasons for this decision were to be primarily found in an accompanying development --- the Satyagraha launched by the Arya Samaj and the Hindu Civil Liberties Union at the same time as the State Congress Satyagraha. The Arya Samaj Satyagraha, which was attracting Satyagrahis from all over the country, was launched as a protest against the religious persecution of the Arya Samaj, and it had clearly religious objectives. It also tended to take on communal overtones. The State Congress and Gandhiji increasingly felt that in the popular mind their clearly secular Satyagraha with distinct political objectives were being confused with the religious-communal Satyagraha of the Arya Samaj and that it was, therefore, best to demarcate themselves from it by withdrawing their own Satyagraha. The authorities were in any case lumping the two together and seeking to project the State Congress as a Hindu communal organizahon. 

Simultaneously, there was the emergence of what came to be known as the Vande Mataram Movement. Students of colleges in Hyderabad city org arnz.cd a protest strike against the authorities' refusal to let them sing Vande Mataram in their hostel prayer rooms. This strike rapidly spread to other parts of the State and many of the students who were expelled from the Hyderabad colleges left the State and continued their studies in Nagpur University in the Congress-ruled Central Provinces where they were given shelter by a hospitable Vice-Chancellor This movement was extremely significant because it created a young and militant cadre that provided the activists as well as the leadership of the movement in later years. 

The State Congress, however continued to be banned, and the regional cultural organizations remained the main forums of activity. The Andhra Mahasabha was particularly active in this phase, and the majority of the younger newly-politicized cadre flocked to it. A significant development that occurred around the year 1940 was that Ravi Narayan Reddy, who had emerged as a major leader of the radicals in the Andhra Mahasabha and had participated in the State Congress Satyagraha along with B. Yella Reddy, was drawn towards the Communist Party. As a result, several of the younger cadres also came under Left and Communist influence, and these radical elements gradually increased in strength and pushed the Andhra Mahasabha towards more radical politics. The Mahasabha began to take an active interest in the problems of the peasants. 

The outbreak of the War provided an excuse to the government for avoiding any moves towards political and constitutional reforms. A symbolic protest against the continuing ban was again registered by Swami Ramanand Tirtha and six others personally selected by Gandhiji. They were arrested in September 1940 and kept in detention till December 1941. A resumption of the struggle was ruled out by Gandhiji since an All-India struggle was in the offing and now all struggles would be part of that. 

The Quit India Movement was launched in August 1942 and it was made clear that now there was no distinction to be made between the people of British India and the States: every Indian was to participate. The meeting of the AISPC was convened along with the AICC session at Bombay that announced the commencement of struggle. Gandhiji and Jawaharlal Nehru both addressed the AISPC Standing Committee, and Gandhiji himself explained the implications of the Quit India Movement and told the Committee that henceforth there would be one movement. The movement in the States was now to be not only for responsible government but for the independence of India and the integration of the States with British India. 

The Quit India Movement got a considerable response from Hyderabad, especially the youth. Though arrests of the main leaders, including Swamiji, prevented an organized movement from emerging, many people all over the State offered Satyagraha and many others were arrested. On 2 October 1942, a batch of women offered Satyagraha in Hyderabad city, and Sarojini Naidu was arrested earlier in the day. Slogans such as `Gandhi Ka Charkha Chalana Padega, Goron ko London Jana Padega' (Gandhiji's wheel will have to be spun, while the Whites will have to return to London) became popular. In a state where, till a few years ago even well-established leaders had to send their speeches to the Collector in advance and accept deletions made by him, the new atmosphere was hardly short of revolutionary. 

But the Quit India Movement also sealed the rift that had developed between the Communist and non-Communist radical nationalists after the Communist Party had adopted the slogan of People's War in December 1941. Communists were opposed to the Quit India Movement as it militated against their understanding that Britain must be supported in its anti-Fascist War. The young nationalists in Telengana coalesced around Jamalpuram Keshavrao but a large section went with Ravi Narayan Reddy to the Communists. The Communists were also facilitated by the removal of the ban on the CPI by the Nizam, in keeping with the policy of the Government of India that had removed the ban because of the CPI's pro- War stance. Therefore, while most of the nationalists were clamped in jail because of their support to the Quit India Movement, the Communists remained free to extend and consolidate their base among the people. This process reached a head in 1944 when a split occurred in the Andhra Mahasabha session at Bhongir, and the pro-nationalist as well as the liberal elements walked out and set up a separate organization. The Andhra Mahasabha now was completely led by the Communists and they soon launched a programme of mobilization and organization of the peasantry. The end of the War in 1945 brought about a change in the Peoples' War line, and the restraint on organizing struggles was removed. 

The years 1945--46, and especially the latter half of 1946, saw the growth of a powerful peasant struggle in various pockets in Nalgonda district, and to some extent in Warangal and Khammam. The main targets of attack were the forced grain levy, the practice of veth begar, illegal exactions and illegal seizures of land. Clashes took place initially between the landlords' goondas and the peasants led by the Sangham (as the Andhra Mahasabha was popularly known), and later between the armed forces of the State police and peasants armed with sticks and stones. The resistance was strong, but so was the repression, and by the end of 1946 the severity of the repression succeeded in pushing the movement into quietude. Thousands were arrested and beaten, many died, and the leaders languished in jails. Yet, the movement had succeeded in instilling into the oppressed and downtrodden peasants of Telengana a new confidence in their ability to resist. 

On 4 June 1947, the Viceroy, Mountbatten, announced at a press conference that the British would soon leave India for good on 15 August. On 12 June, the Nizam announced that on the lapse of British paramountcy he would become a sovereign monarch. The intention was clear: he would not accede to the Indian Union. The first open session of the Hyderabad State Congress which demanded accession to the Indian Union and grant of responsible government was held from 16 to 18 June. The State Congress, with the full support of the Indian National Congress, had also thwarted an attempt by the Nizam. a few months earlier, to foist an undemocratic constitution on the people. The boycott of the elections launched by them had received tremendous support. With this new confidence, they began to take a bold stand against the Nizam's moves. 

The decision to launch the final struggle was taken by the leaders of the State Congress in consultation with the national leaders in Delhi. As recorded by Swami Ramanand Tirtha in his Memoirs of Hyderabad Freedom Struggle: `That (the) final phase of the freedom struggle in Hyderabad would have to be a clash of arms with the Indian Union, was what we were more than ever convinced of. It would have to be preceded by a Satyagraha movement on a mass scale'. 

After the preliminary tasks of setting up the Committee of Action under the Chairmanship of D.G. Bindu (which would operate from outside the State to avoid arrest), the establishment of offices in Sholapur, Vijayawada, Gadag and a central office at Bombay, mobilization `f funds in which Jayaprakash Narayan played a critical role, the struggle was formally launched on 7 August which was to be celebrated as `Join Indian Union Day'. The response was terrific, and meetings to defy the bans were held in towns and villages all over the State. Workers and students went on strike, including 12,000 Hyderabadi workers in Bombay. Beatings and arrests were common. On 13 August, the Nizam banned the ceremonial hoisting of the national flag. Swamiji gave the call: `This order is a challenge to the people of Hyderabad and I hope they will accept it'. Swamiji and his colleagues were arrested in the early hours of 15 August, 1947, soon after the dawn of Indian Independence. But, despite tight security arrangements, 100 students rushed out of the Hyderabad Students' union office and hoisted the flag in Sultan Bazaar as scheduled. in subsequent days, the hoisting of the Indian national flag became the major form of defiance and ingenious methods were evolved. Trains decorated with national flags would steam into Hyderabad territory from neighbouring Indian territory. Students continued to play a leading role in the movement, and were soon joined by women in large numbers, prominent among them being Brij Rani and Yashoda Ben. 

As the movement gathered force and gained momentum, the Nizam and his dministration cracked down on it. But the most ominous development was the encouragement given to the storm troopers of the Ittihad ul Muslimin, the Razakars, by the State to act as a paramilitary force to attack the peoples' struggle. Razakars were issued arms and let loose on protesting crowds; they set up camps near rebellious villages and carried out armed raids. 

On 29 November 1947, the Nizam signed a Standstill Agreement with the Indian Government, but simultaneously the repression was intensified, and the Razakar menace became even more acute. Many thousands of people who could afford to do so fled the State and were housed in camps in neighbouring Indian territory. The people increasingly took to self-defence and protected themselves with whatever was available. In organizing the defence against the Razakars and attacks on Razakar camps, the Communists played a very important role, especially in the areas of Nalgonda, Warangal and Khammam that were their strongholds. Peasants were organized into dalams, given training in arms, and mobilized for the anti-Nizam struggle. In these areas, the movement also took an anti-landlord stance and many cruel landlords were attacked, some even killed, and illegally occupied land was returned to the original owners. Virtually all the big landlords had run away, and their land was distributed to and cultivated by those with small holdings or no land. 

The State Congress, too, organized armed resistance from camps on the State's borders. Raids were made on customs' outposts, police Stations and Razaicar camps. Outside the Communist strongh%olds in the Telengana areas, it was the State Congress that was the main vehicle for organizing popular resistance. Over 20,000 Satyagrahis were in jail and many more were participating in the movement outside. 

By September 1948, it became clear that all negotiations to make the Nizam accede to the Union had failed. On 13 September, 1948, the Indian Army moved in and on 18 September the Nizam surrendered. The process of the integration of the Indian Union was finally complete. The people welcomed the Indian Army as an army of liberation, an army that ended the oppression of the Nizam and the Razakars. Scenes of jubilation were evident all over, and the national flag was hoisted. The celebration was, however, marred by the decision of the Communists to refuse to lay down arms and continue the struggle against the Indian Union, but that is another long story that falls outside the scope of our present concerns.

\begin{center}*\end{center}

\paragraph*{}

The cases of Hyderabad, and that of Rajkot, are good examples of how methods of struggle evolved to suit the conditions in British India, such as non-violent mass civil disobedience or Satyagraha, did not have the same viability or effectiveness in the India States. The lack of civil liberties, and of representative institutions, meant that the political space for hegemonic politics was very small, even when compared to the conditions prevailing under the semi-hegemonic and semi- repressive colonial state in British India. The ultimate protection provided by the British enabled the rulers of the States to withstand popular pressure to a considerable degree, as happened in Rajkot. As a result, there was a much greater tendency in these States for the movements to resort to violent methods of agitation --- this happened not only in Hyderabad, but also in Travancore, Patiala, and the Orissa States among others. In Hyderabad, for example, even the State Congress ultimately resorted to violent methods of attack, and, in the final count, the Nizam could only be brought into line by the Indian Army. 

This also meant that those such as the Communists and other Left groups, who had less hesitation than the Congress in resorting to violent forms of struggle, were placed in a more favourable situation in these States and were able to grow as a political force in these areas. Here, too, the examples of Hyderabad, Travancore, Patiala and the Orissa States were quite striking. 

The differences between the political conditions in the States and British India also go a long way in explaining the hesitation of the Congress to merge the movements in the States with those in British India. The movement in British India adopted forms of struggle and a strategy that was specifically suited to the political context. Also, political sagacity dictated that the Princes should not be unnecessarily pushed into taking hard positions against Indian nationalism, at least till such time as this could be counter-balanced by the political weight of the people of the state.
\end{multicols}{2}
