\chapter[Ideological Dimension of Indian National Movement]{The Indian National Movement: Ideological Dimension}\label{chapter:CH39}

The Indian national movement was basically the product of the central contradiction between colonialism and the interests of the Indian people. The leadership of the movement gradually arrived at, and based itself on a clear, scientific and firm understanding of colonialism --- that the British were using their political control to subordinate the Indian economy and society to the needs of the British economy and society. It began to perceive that overall the country was regressing and underdeveloping. On this basis, it evolved an understanding of the Indian reality and gradually generated and formed a clear-cut anti-colonial ideology.

Already, by the end of the 19th century, the founding fathers of the national movement had worked out a clear understanding of all the three modes of colonial exploitation: through plunder, taxation and the employment of Englishmen in India, through free and unequal trade, and through the investment of British capital. They had also grasped that India's colonial relationship was not an accident of history or a result of political policy but sprang rather from the very character of British society and India's subordination to it. Their entire critique of colonialism got its focus in the theory of the drain of wealth from India --- the theory that a large part of India's capital and wealth were being transferred to Britain.

This understanding of the complex economic mechanism of modem imperialism was further advanced after 1918 under the impact of the anti-imperialist mass movements and the spread of Marxist ideas. The nationalist leadership also understood that the central contradiction could be resolved only through the transformation or overthrow of colonial economic relations. Moreover, at each stage of the movement's development, the leadership linked its analysis to the analysis of colonialism.

This anti-colonial world view was fully internalized by the lowermost cadres of the national movement. During the Gandhian era of mass politics, they disseminated this critique of colonialism among the common people in the urban as well as the rural areas. The twin themes of the drain of wealth and the use of India as a market for Britain's manufactured goods and the consequent destruction of the Indian handicraft industries formed the very pith and marrow of their agitation. This agitation undermined the foundations of colonial rule in the minds of the Indian people --- it destroyed the carefully inculcated colonial myth that the British ruled India for the benefit of Indians. that they were the Mai-Baap of the common people.

Thus, if the primary contradiction provided the material or structural basis of the national movement, its grasping through the anti-co1onial ideology provided its ideological basis. This opened the way to a firm and consistent anti-imperialist movement, which could follow highly flexible tactics precisely because of its rootedness in and adherence to the anti- colonial principle.

This strong anti-colonial basis of the movement was also very important because in any mass movement ideology plays a crucial role. In normal politics, passive support or opposition to, or voting for and against, a regime do not require very strong motivation. But active participation in a mass movement, involving immense sacrifice, cannot take place only on the basis of a sense of being poor or being exploited. It requires a strong. a very strong ideological commitment based on an understanding of the causes of the social condition. Therefore, it was the movement's scientific anti-colonial ideology which became the prime mover in its anti-imperialist struggle. Along with the anti- colonial world view, certain other ideological elements constituted the broad socio-economic-political vision of the Indian national movement. Broadly speaking, this vision was that of bourgeois or capitalist independent economic development and a secular, republican, democratic, civil libertarian political order, both the economic and political order to be based on principles of social equality. Interestingly, this vision was to remain unquestioned throughout the only controversy was confined to the capitalist character of the economic order, which was questioned in a serious manner after 1920.

The national movement was fully committed to parliamentary democracy and civil liberties. It provided the soil and climate in which these two could root themselves at a time when the colonial rulers were preaching that because of India's climate, the historical traditions of the Indian people and the nature of their religious and social institutions, democracy was not suited to India --- that Indians' must be ruled in an authoritarian and despotic manner. The British also increasingly tampered with and attacked the freedoms of speech and the Press.

Consequently, it was left to the national movement to fight for democracy and to internalize and indigenize it, that is to root it in the Indian soil. From the beginning it fought for the introduction of a representative form of government on the basis of popular elections. Tilak and other nationalists before 1920 and, then, Gandhiji and the Congress demanded the introduction of adult franchise so that all adult men and women could vote. From its inception, the Indian National Congress was organized along democratic lines. All its resolutions were publicly debated and then voted upon. The Congress permitted and encouraged minority opinion to freely express itself.

Some of the most important decisions in its history were taken after heated debates and on the basis of open voting. For example, the decision to start the Non-Cooperation Movement was taken in 1920 at Calcutta with 1886 voting for and 884 against Gandhiji's resolution. Similarly, at the Lahore Congress in 1929, a resolution sponsored by Gandhiji condemning the Revolutionary Terrorists' bomb attack on the Viceroy's train was passed by a narrow majority of 942 to 794. In 1942, thirteen Communist members of the AICC voted against the famous Quit India resolution. But instead of condemning these thirteen, Gandhiji, at the very beginning of his famous `Do or Die' speech, said: `I congratulate the thirteen friends who voted against the resolution, in doing so they had nothing to be ashamed of. For the last twenty years we have tried to learn not to lose courage even when we are in a hopeless minority and are laughed at. We have learned to hold on to our beliefs in the confidence that we are in the right. It behoves us to cultivate this courage of conviction, for it ennobles man and raises his moral stature. I was, therefore, glad to see that these friends had imbibed the principle which I have tried to follow for the last fifty years and more.''

The national movement was from the beginning zealous in defence of civil liberties. From the beginning, the nationalists fought against attack by the colonial authorities on the freedom of the Press, speech and association and other civil liberties. Lokamanya Tilak, for instance, often claimed that `liberty of the Press and liberty of speech give birth to a nation and nourish it.'

Gandhiji's commitment to civil liberties was also total. At the height of the Non-Cooperation Movement, he wrote in the Young India in January 1922: `Swaraj, the Khilafat, the Punjab occupy a subordinate place to the issue sprung upon the country by the Government. We must first make good the right of free speech and free association before we can make any further progress towards our goal ... We must defend these elementary rights with our lives.' In another article soon after, he went on to explain these rights: `Liberty of speech means that it is unassailed even when the speech hurts; liberty of the Press can be said to be truly respected only when the Press can comment in the severest terms upon and even misrepresent matters . Freedom of association is truly respected when assemblies of people can discuss even revolutionary projects.'3 One other quotation from Gandhiji on the subject is of great relevance: `Civil liberty consistent with the observance of non-violence is the first step towards Swaraj. It is the breath of political and social life. It is the foundation of freedom. There is no room there for dilution or compromise. It is the water of life.'

Jawaharlal Nehru was, perhaps the strongest champion of civil liberties He assigned as much importance to them as he did to economic equality and socialism. The resolution on fundamental rights, passed by the Karachi Congress in 1931 and drafted by him, guaranteed the rights of free expression of opinion through speech and the Press and the freedom of association. In August 1936, as a result of his efforts, the Indian Civil Liberties Union was formed on non-parts, non-sectarian lines to mobilize public opinion against all encroachments on civil liberties. He declared at this time: `If civil liberties are suppressed, a nation loses all vitality and becomes impotent for anything substantial.' And again in March 1940: `The freedom of the Press does not consist in our permitting such things as we like to appear. Even a tyrant is agreeable to this type of freedom. Civil liberty and freedom of the Press consist in our permitting what we do not like, in our putting up with criticisms of ourselves, in our allowing public expression of views which seem to us even to be injurious to our cause itself.'

Thus, over the years, the nationalist movement successfully created an ideology and culture of democracy and civil liberties based on respect for dissent, freedom of expression, the majority principle, and the right of minority opinions to exist and grow.

Secularism was from the beginning made a basic constituent of the nationalist ideology and a strong emphasis was laid on Hindu-Muslim unity. Although the national movement failed to eradicate communalism and prevent the partition of the country, this was due not to its deviance from a secular ideology but to weaknesses in its strategy for fighting communalism and its failure to fully grasp the socio-economic and ideological roots of communalism. The national movement also opposed caste oppression and after 1920 made abolition of untouchability a basic constituent of its programme and political work, though in this aspect, too, serious ideological flaws remained. In particular, a strong anti-caste ideology was not formed and propagated. The cause of women's liberation was also not taken up seriously.

The national movement fully recognized the multifaceted diversity of the Indian people. That India was not yet a developed or structured nation, but a nation-in-the-making, was accepted and made the basis of political and ideological work and agitation. It was fully grasped that common subjection to colonial rule provided the material and emotional basis for nation-making and that one of the functions of the movement was to structure the nation through a common struggle against colonialism. It was also seen that the political and ideological practices of the movement would play a crucial role in the process of nation-in-the-making. Furthermore, it was clearly understood that the objective of unifying the Indian people into a nation would have to be realized by taking into account regional, religious, caste, ethnic and linguistic differences. The cultural aspirations of the different linguistic groups were given full recognition. From 1921, the Congress organized its provincial or area committees along linguistic lines and not according to the British-created multi-lingual provinces.

The Indian national movement accepted from the beginning, and with near unanimity, the objective of a complete economic transformation of the country on the basis of modem industrial and agricultural development. From Justice Ranade onwards, the nationalists were agreed that industrialization was the only means of overcoming the poverty of the people. Gandhiji was to some extent an exception to this unanimous opinion, but not wholly so. Nor did he counterpose his opinion to that of the rest of the national leadership. Moreover, his stand on the use of machines and large-scale industry has been grossly distorted. He was opposed to machines only when they displaced the labour of the many or enriched the few at the expense of the many. On the other hand, he repeatedly said that he would prize every invention of science made for the benefit of all.' He repeatedly said that he was not opposed to modem large-scale industry so long as it augmented, and lightened the burden of, human labour and not displaced it. Moreover, he laid down another condition: All large-scale industry should be owned and controlled by the state and not by private capitalists.

The nationalists were fully committed to the larger goal of independent self-reliant economic development to be based on independence from foreign capital, the creation of an indigenous capital goods or machine- making sector and the foundation and development of independent science and technology. Ever since the 1840s, British economists and administrators had argued for the investment of foreign capital as the major instrument for the development of India. The Indian nationalists, from Dadabhai Naoroji and Tilak to Gandhiji and Nehru, disagreed vehemently. Foreign capital, they argued, did not develop a country but underdeveloped it. It suppressed indigenous capital and made its future growth difficult. It was also, the nationalists said, politically harmful because, sooner or later it began to wield an increasing and dominating influence over the administration.

Starting with Dadabhai Naoroji and Ranade, the nationalists visualized a crucial role for the public sector in the building of an independent and modem economy. In the l930s, Jawaharlal Nehru, Gandhiji, and the left-wing also argued for the public sector, especially in large-scale and key industries, as a means of preventing the concentration of wealth in a few hands. In the late 1930s, the objective of economic planning was also widely accepted. In 1938, the Congress, then under the presidentship of Subhas Chandra Bose, set up the National Planning Committee under the chairmanship of Nehru, to draw up a development plan for free India. During World War H, several other plans were devised, the most important being the Bombay Plan drawn up by the big three of the Indian capitalist world - --- J.R. D. Tata, G.D. Birla and Sri Ram. This plan too visualized far-reaching land reforms, a large public sector and massive public and private investment.

As brought out earlier, the world outlook of the national movement based on anti-colonialism, anti-Fascism, peace and national independence was a powerful element of its overall ideology. From its early days, the national movement adopted a pro- poor orientation. The entire economic agitation of the Moderates and their critique of colonialism was linked to the growing poverty of the masses. This orientation was immensely strengthened by the impact of the Russian Revolution of 1917, the coming of Gandhiji on the political stage and the growth of powerful left-wing parties and groups during the 1920s and I 930s. The movement adopted policies and a programme of reforms during various stages of the struggle that were quite radical by contemporary standards.

Compulsory primary education, the lowering of taxation on the poor and middle classes, the reduction of the salt tax, land revenue and rent, relief from indebtedness and the provision of cheap credit to peasants, the protection of tenants' rights, workers' right to a living wage and a shorter working day, higher wages for low-paid government servants, including policemen, the defence of the right of workers' and peasants' to organize themselves, the protection and promotion of village industries, the promotion of modern science and technical education, the eradication of the drink evil, the improvement of the social position of women including their right to work and education and to equal political rights, the initiation of legal and social measures for the abolition of untouchability, and the reform of the machinery of law and order were some of the major reforms demanded by the Indian national movement.

The basic pro-people or pro-poor orientation of the national movement and the notion that politics must be based on the people, who must be politicized, activized and brought into politics, also made it easier to give it a socialist orientation. But still, as pointed out earlier, the nationalist developmental perspective was confined within bourgeois parameters, that is, independent economic development was visualized within a capitalist framework. After 1919, when the national movement became a mass movement, Gandhiji evolved and propagated a different, non-capitalist, basically peasantist-artisanist outlook but his socio-economic programme and thought were not capable of challenging the basic hegemony of bourgeois ideology over the national movement.

It is true that the national movement, as an anti-colonial movement in a colony in which the primary contradiction pitted the entire society against colonialism, was a popular, people's movement it was a multiclass movement which represented the interests of the different classes and strata of Indian society. However, the Indian people, though unified against colonialism and in the anti-imperialist struggle, were at the same time divided into social classes which had their own contradictions with colonialism as well as with each other. Different classes and strata had different levels and degrees of contradiction with colonialism as also different extent and manner of participation in the anti-imperialist struggle. The result vas that the anti- colonial struggle could have several different class consequences. The final outcome of the struggle could see several different balances of class or political and ideological forces. This balance of forces would help decide in whose class interests would the primary contradiction get resolved as a result of the anti- imperialist struggle, that is, what sort of India would come into existence after freedom. In other words, freedom could result in a socialist or a capitalist societal order.

Beginning with the 1920s, a powerful socialist trend developed in the national movement. The bourgeois developmental perspective of the national movement was challenged in a serious manner by early Communist groups,

Jawaharlal Nehru, Subhas Chandra Bose, and a large number of socialist-minded groups and individuals. The struggle for the spread of socialist ideas was intensified in the 1930s when these were joined by the Congress Socialist Party, a reorganized Communist Party and the Royists. The Great Depression of the 1 930s in the capitalist world, the Russian Revolution and the success of the Soviet Five Year Plans, and the anti- fascist wave the world over during the 1930s made socialist ideas attractive. Most of the leaders of the youth movement of the late I 920s and a large number of Revolutionary Terrorists also made the turn to socialism. Throughout the 1920s, 1930s, and 1940s, the youthful nationalist cadres were increasingly turning to socialist ideas.

The left-wing tried to popularize the idea that constant class struggles were going on within India between peasants and landlords and workers and capitalists. It tried to organize these struggles through their class organizations --- kisan sabhas and trade unions. But above all, it struggled to transform the national movement in a leftward, socialist ideological direction, to impart to the movement' a ``ision of socialist India after independence.

Jawaharlal Nehru played a very important role in popularizing the vision of a socialist India both within the national movement and in the country at large. Nehru argued that political freedom must mean the economic emancipation of the masses. Throughout the 1930s, he pointed to the inadequacy of the existing nationalist ideology and the hegemony of bourgeois ideology over the national movement, and stressed the need to inculcate a new socialist or basically Marxist ideology, which would enable the people to study their social condition scientifically and to give the Congress a new socialist ideological orientation.

The 1930s were highly favourable to socialist ideas, and they spread widely and rapidly. But though the left-wing and socialist ideas grew in geometric proportions, they did not succeed in becoming the dominant ideological trend within the national movement. They did, however, succeed in becoming a basic constituent of the national movement and in constant shifting it leftward. The national movement continuously defined itself further and further in a radical direction in terms of the popular clement. Increasing, freedom was defined in socio-economic terms which went far beyond the mere absence of foreign rule. By the late 1930s, the Indian national movement was one of the most radical of the national liberation movements.

This radicalism found reflection in the Congress resolutions at Karachi, Lucknow and Faizpur (in 1931 and 1936), in the election manifestoes of 1936 and 1945--46 and in the economic and social reforms of the Congress Ministries from 1937--39. In fact, the Congress progressively evolved in a radical socio-economic-political direction and increasingly adopted most of the demands put forward by the Left though with a time lag of a few years. The politics of the Left and workers' and peasants' struggles, of course, played a crucial role in this evolution. One result was that even the Congress Right was not only firmly anti- imperialist but also committed to basic changes in political and economic power even though it was opposed to socialism. It remained bourgeois in outlook but with a reformist outlook.

This becomes evident when we study the evolution of the agrarian policy of the Congress, for after all the key question in India was that of the social condition of the peasant. The Congress had always fought for the peasant demands vis-a-vi the colonial state. But goaded by the left- wing and the peasant movements, the Congress accepted at Faizpur in 1936 a programme of substantial reduction in rent and revenue, abolition of feudal dues and forced labour, fixity of tenure and a living wage for agricultural labourers. The Congress Ministries passed legislation, which varied in its radical content from province to province, to protect tenants' rights and prevent expropriation by the moneylenders. Finally, in 1945, the Congress Working Committee accepted the policy of the abolition of landlordism and of land belonging to the tiller when it declared: `The reform of the land system involves the removal of intermediaries between the peasant and the state.''

A major ideological dimension of the national movement was the overall social outlook of Gandhiji and the Gandhians. Gandhiji did not accept a class analysis of society and the role of class struggle. He was also opposed to the use of violence even in defence of the interests of the poor. But his basic outlook was that of social transformation. He was committed to basic changes in the existing system of economic and political power. Moreover, he was constantly moving in a radical direction during the 1930s and 1940s. In 1933, he agreed with Nehru that `without a material revision of vested interests the condition of the masses can never be improved.' He was beginning to oppose private property and thus radicalize his theory of trusteeship. He repeatedly argued for the nationalization of large-scale industry. He condemned the exploitation of the masses inherent in capitalism and landlordism. He was highly critical of the socio-economic role played by the middle classes.

His emphasis on the removal of distinction and discrimination between physical and mental labour, his overall emphasis on social and economic equality and on the self-activity of the masses, his opposition to caste inequality and oppression, his active support to women's social liberation, and the general orientation of his thought and writing towards the exploited, the oppressed and the down-trodden tended in general to impart a radical ideological direction to the national movement.

The most remarkable development was Gandhiji's shift towards agrarian radicalism. In 1937, he said: `That the land today does not belong to the people is too true... (But) Land and all property is his who will work it. Unfortunately the workers are or have been kept ignorant of this simple fact.'' In 1942, he again declared that `the land belongs to those who will work on it and to no one else.' Similarly, in June 1942 Gandhiji told Louis Fischer in answer to his question: `What is your programme for the improvement of the lot of the peasantry?' that `the peasants would take the land. We would not have to tell them to take it. They would take it.' And when Fischer asked, `Would the landlords be compensated?' He replied: `No, that would be fiscally impossible.' Fischer asked: `Well, how do you actually see your impending civil disobedience movement?' Gandhiji replied: `In the villages, the peasants will stop paying taxes. They will make salt despite official prohibition... Their next step will be to seize the land.' `With violence?' asked Fischer. Gandhiji replied: `There may be violence, but then again the landlords may cooperate... They might cooperate by fleeing.' Fischer said that the landlords `might organize violent resistance.' Gandhiji's reply was. `There may be fifteen days of chaos, but I think we could soon bring that under Control.' Did this mean, asked Fischer, that there must be `confiscation without compensation?' Gandhiji replied: `Of course. It would be financially impossible for anybody to compensate the landlords.''

Thus the national movement based itself on a clear-Cut anti-colonial ideology and the vision of a civil libertarian, democratic, `secular and socially radical society. The Indian economy was to be developed along independent, self-reliant lines. It was this vision, combined. With anti-Colonial ideology and a pro-poor radical socio-economic orientation that enabled the national movement to base itself on the politically awakened and politically active people and to acquire the character of a popular people's movement.
