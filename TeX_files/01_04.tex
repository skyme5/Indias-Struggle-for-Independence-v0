\chapter{Foundation of the Congress: The Myth}\label{chapter:CH04}

Indian National Congress was founded in 1885-12-00 by seventy-two political workers. It was the first organized expression of Indian nationalism on an all-India scale. A.O. Hume, a retired English ICS officer, played an important role in its formation. But why was it founded by these seventy-two men and why at that time?

A powerful and long-lasting myth, the myth of `the safety valve,' has arisen around this question. Generations of students and political activists have been fed on this myth. But despite widespread popular belief, this myth has little basis in historical fact. The myth is that the Indian National Congress was started by A.O. Hume and others under the official direction, guidance and advice of no less a person than Lord Dufferin, the Viceroy, to provide a safe, mild, peaceful, and constitutional outlet or safety valve for the rising discontent among the masses, which was inevitably leading towards a popular and violent revolution. Consequently, the revolutionary potential was nipped in the bud. The core of the myth, that a violent revolution was on the cards at the time and was avoided only by the foundations of the Congress, is accepted by most writers; the liberals welcome it, the radicals use it to prove that the Congress has always been compromising if not loyalist vis-a-vis imperialism, the extreme right use it to show that the Congress has been anti-national from the beginning. All of them agree that the manner of its birth affected the basic character and future work of the Congress in a crucial manner.

In his Young India published in 1916, the Extremist leader Lala Lajpat Rai used the safety-valve theory to attack the Moderates in the Congress. Having discussed the theory at length and suggested that the Congress `was a product of Lord Dufferin's brain,' he argued that `the Congress was started more with the object of saving the British Empire from danger than with that of winning political liberty for India. The interests of the British Empire were primary and those of India only secondary.'

And he added: `No one can say that the Congress has not been true to that ideal.' His conclusion was: `So this is the genesis of the Congress, and this is sufficient to condemn it in the eyes of the advanced Nationalists.''

More than a quarter century later, R. Palme Dutt's authoritative work India Today made the myth of the safety-valve a staple of left-wing opinion. Emphasizing the myth, Dutt wrote that the Congress was brought into existence through direct Governmental initiative and guidance and through `a plan secretly pre-arranged with the Viceroy' so that it (the Government) could use it `as an intended weapon for safeguarding British rule against the rising forces of popular unrest and anti-British feeling.' It was `an attempt to defeat, or rather forestall, an impending revolution.' The Congress did, of course, in time become a nationalist body; `the national character began to overshadow the loyalist character.' It also became the vehicle of mass movements. But the `original sin' of the manner of its birth left a permanent mark on its politics. Its `two-fold character' as an institution which was created by the Government and yet became the organizer of the anti-imperialist movement `ran right through its history.' It both fought and collaborated with imperialism. It led the mass movements and when the masses moved towards the revolutionary path, it betrayed the movement to imperialism. The Congress, thus, had two strands: `On the one hand, the strand of cooperation with imperialism against the ``menace'' of the mass movement; on the other hand, the strand of leadership of the masses in the national struggle.' This duality of the Congress leadership from Gokhale to Gandhi, said Dutt, in fact reflected the two-fold and vacillating character of the Indian bourgeoisie itself; `at once in conflict with the British bourgeoisie and desiring to lead the Indian people, yet feeling that ``too rapid'' advance may end in destroying its privileges along with those of the imperialists.' The Congress had, thus, become an organ of opposition to real revolution, that is, a violent revolution. But this role did not date from Gandhiji; `this principle was implanted in it by imperialism at the outset as its intended official role.' The culmination of this dual role was its `final capitulation with the Mountbatten Settlement.'

Earlier, in 1939, M.S. Golwalkar, the RSS chief, had also found the safety-valve theory handy in attacking the Congress for its secularism and, therefore, anti-nationalism. In his pamphlet We Golwalkar complained that Hindu national consciousness had been destroyed by those claiming to be `nationalists' who had pushed the `notions of democracy' and the perverse notion that `our old invaders and foes', the Muslims, had something in common with Hindus. Consequently, `we have allowed our foes to be our friends and with our hands are undermining true nationality.' In fact, the tight in India was not between Indians and the British only. It was `a triangular fight.' Hindus were at war with Muslims on the one hand and with the British on the other. What had led Hindus to enter the path of `denationalization,' said Golwalkar, were the aims and policy laid down by Flume, Cotton, and Wedderburn in 1885; `the Congress they founded as a ``safety valve'' to ``seething nationalism,'' as a toy which would lull the awakening giant into slumber, an instrument to destroy National consciousness, has been, as far as they are concerned, a success.'

The liberal C.F. Andrews and Girija Mukherji fully accepted the safety-valve theory in their work, The Rise and Growth of the Congress in India published in 1938. They were happy with it because it had helped avoid `useless bloodshed.' Before as well as after 1947, tens of scholars and hundreds of popular writers have repeated some version of these points of view.

\begin{center}*\end{center}

\paragraph*{}
Historical proof of the safety-valve theory was provided by the seven volumes of secret reports which Flume claimed to have read at Simla in the summer of 1878 and which convinced him of the existence of `seething discontent' and a vast conspiracy among the lower classes to violently overthrow British rule.

Before we unravel the mystery of the seven volumes, let us briefly trace the history of its rise and growth. It was first mentioned in William Wedderburn's biography of A.O. Flume published in 1913. Wedderburn (ICS) found an undated memorandum in Hume's papers which dealt with the foundation of the Congress. He quoted at length from this document. To keep the mystery alive so that the reader may go along with the writer step by step towards its solution, I will withhold an account of Wedderburn's writing, initially giving only those paragraphs which were quoted by the subsequent writers. According to Lajpat Rai, despite the fact that Hume was `a lover of liberty and wanted political liberty for India under the aegis of the British crown,' he was above all `an English patriot.' Once he saw that British rule was threatened with `an impending calamity' he decided to create a safety valve for the discontent.

As decisive proof of this Lajpat Rai provided a long quotation from Hume's memorandum that Wedderburn had mentioned along with his own comments in his book. Since this passage is quoted or cited by all subsequent authors, it is necessary to reproduce it here at length. ``I was shown,'' wrote Hume, ``several large volumes containing a vast number of entries; English abstracts or translations longer or shorter — of vernacular reports or communications of one kind or another, all arranged according to districts (not identical with ours) The number of these entries was enormous; there were said, at the time to be communications from over 30,000 different reporters.'' He (Hume) mentions that he had the volumes in his possession only for a week... Many of the entries reported conversations between men of the lowest classes, ``all going to show that these poor men were pervaded with a sense of the hopelessness of the existing state of affairs; that they were convinced that they would starve and die, and that they wanted to do something, and stand by each other, and that something meant violence. a certain small number of the educated classes, at the time desperately, perhaps unreasonably, biller against the Government, would join the movement assume here and there the lead, give the outbreak cohesion, and direct it as a national revolt.'''

Very soon, the seven volumes, whose character, origin, etc., were left undefined in Lajpat Rai's quotation, started undergoing a metamorphosis. In 1933, in Gurmukh Nihal Singh's hands, they became `government reports.' Andrews and Mukherji transformed them into `several volumes of secret reports from the CID' which came into Hume's possession `in his official capacity.' The classical and most influential statement came from R. Palme Dutt. After quoting the passage quoted by Lajpat Rai from Wedderburn, Dutt wrote: `Hume in his official capacity had received possession of the voluminous secret police reports.''

Numerous other historians of the national movement including recent ones such as R.C. Majumdar and Tara Chand, were to accept this product of the creative imagination of these writers as historical fact.

So deeply rooted had become the belief in Hume's volumes as official documents that in the 1950s a large number of historians and would-be historians, including the present writer, devoted a great deal of time and energy searching for them in the National Archives. And when their search proved futile, they consoled themselves with the thought that the British had destroyed them before their departure in 1947. Yet only if the historians had applied a minimum of their historiographic sense to the question and looked at the professed evidence a bit more carefully, they would not have been taken for a ride. Three levels of historical evidence and logic were available to them even before the private papers of Ripon and Dufferin became available.

The first level pertains to the system under which the Government of India functioned in the 1870s. In 1878, Flume was Secretary to the Department of Revenue, Agriculture and Commerce. How could the Secretary of these departments get access to Home Department files or CID reports? Also he was then in Simla while Home Department files were kept in Delhi; they were not sent to Simla. And from where would 30,000 reporters come? The intelligence departments could not have employed more than a few hundred persons at the time! And, as Lajpat Rai noted, if Congress was founded out of the fear of an outbreak, why did Flume and British officialdom wait for seven long years?

If these volumes were not government documents, what were they? The clue was there in Wedderburn's book and it was easily available if a writer would go to the book itself and not rely on extracts from it reproduced by previous authors as nearly all the later writers seem to have done. This brings us to the second level of historical evidence already available in Wedderburn.

The passages quoted by Lajpat Rai, R. Palme Dutt and others are on pages 80-81 of Wedderburn's book. Two pages earlier, pages 78-80, and one page later, 82-83, Wedderburn tells the reader what these volumes were and who provided them to Hume. The heading of the section where the quoted passages occur is `Indian religious leaders.' In the very beginning of the section, Wedderburn writes that a warning of the threatened danger came to Flume `from a very special source that is, from the leaders among those devoted, in all parts of India, to a religious life.' Hume referred in his memorandum to the legions of secret quasi-religious orders, with literally their millions of members, which form so important a factor in the Indian problem.'' These religious sects and orders were headed by Gurus, ``men of the highest quality who . . have purged themselves from earthly desires, and fixed their desires on the highest good.'' And ``these religious leaders, through their Chelas or disciples, are hilly informed of all that goes on under the surface, and their influence is great in forming public opinion.'' It was with these Gurus, writes Wedderburn, `that Mr. Hume came in touch, towards the end of Lord Lytton's Viceroyalty.' These Gurus approached Hume because Hume was a keen student of Eastern religions, but also because they ``feared that the ominous `unrest' throughout the country… would lead to terrible outbreak'' and it was only men like Hume who had access to the Government who could help `avert a catastrophe.' ``This,'' wrote Hume, ``is how the case was put to me.'' With this background the passages on pages 80-81 become clearer.

In other words, the evidence of the seven volumes was shown to Hume by the Gurus who had been sent reports by thousands of Chelas. But why should Hume believe that these reports `must necessarily be true?' Because Chelas were persons of a special breed who did not belong to any particular sect or religion or rather belonged to all religions. Moreover they were `bound by vows and conditions, over and above those of ordinary initiates of low grade.' They were `all initiates in some of the many branches of the secret knowledge' and were `all bound by vows, they cannot practically break, to some farther advanced seeker than themselves.' The leaders were of `no sect and no religion, but of all sects and all religions.' But why did hardly anyone in India know of the existence of these myriads of Gurus and Chelas? Because, explained Hume, absolute secrecy was an essential feature in their lives. They had communicated with Hume only because they were anxious to avert calamity.

And, finally, we come to the third level of historiography, the level of profound belief and absolute fantasy. The full character of the Gurus and Chelas was still not revealed by Wedderburn, for he was sheltering the reputation of his old friend, as friendly biographers usually do. The impression given by him was that these Gurus and Chelas were ordinary mortal men. This was, however, not the case. Reconstructing the facts on the basis of some books of Theosophy and Madame Blavatsky and the private papers of the Viceroys Ripon and Dufferin, we discover that these Gurus were persons who, because of their practice of `peculiar Eastern religious thought,' were supposed to possess supernatural occult powers; they could communicate and direct from thousands of miles, enter any place go anywhere, sit anywhere unseen, and direct men's thoughts and opinions without their being aware of it.

\begin{center}*\end{center}

\paragraph*{}
In 1881, Hume came under the spell of Madame Blavatsky who claimed be in touch with these Gurus who were described by her as mahatmas. These mahatmas lived as part of a secret brotherhood in Tibet, but they could contact or `correspond' with persons anywhere in the world because of their occult powers. Blavatsky enabled Hume to get in touch with one of these mahatmas named `Koot Hoomi Lal Singh.' It is this invisible brotherhood that gathered secret information on Indian affairs through their Chelas. In a book published in 1880, A.P. Sinnet, editor of the Pioneer and another follower of Blavatsky, had quoted a letter from Koot Hoomi that these mahatmas had used their power in 1857 to control the Indian masses and saved the British Empire and that they would do the same in future.

Hume believed all this. He was keen to acquire these occult powers by which the Chelas could know all about the present and the future. He started a `correspondence' with the mahatmas in Tibet. By 1883 Hume had quarreled with Blavatsky, but his faith in the Gurus or mahatmas continued unabated. He also began to use his connection with the mahatmas to promote political objectives dear to his heart — attempting to reform Indian administration and make it more responsive to Indian opinion.

In December 1883, he wrote to Ripon: `I am associated with men, who though never seen by the masses ... are yet reverenced by them as Gods ... and who feel every pulse of public feeling.' He claimed a Superior knowledge `of the native mind' because `a body of men, mostly of Asiatic origin ... who possess facilities which no other man or body of men living do, for gauging the feelings of the natives... have seen fit... to give me their confidence to a certain limited extent.' In January 1884, he informed Ripon that even earlier, in 1848, he had been in contact with the brotherhood or association of his mystical advisers and that it was their intervention which had defeated the revolutions of 1848 in Europe and the `mutiny' of 1857. From distant Tibet they were now acting through him and others like him to help Ripon introduce reforms and avoid `the possibility of such a cataclysm recurring.' This association of mahatmas was also helping him, he told Ripon, to persuade the Queen to give a second term as Viceroy to Ripon and to `tranquilize the native press'.

Hume tried to play a similar role with Dufferin, but more hesitatingly, not sharing with him the information that his advisers were astral, occult figures so that even many historians have assumed that these advisers were his fellow Congress leaders! Only once did he lift the veil before Dufferin when the latter during 1887 angrily pressed him to reveal the source through which he claimed to have gained access to the Viceroy's secret letter to the Secretary of State. Pressed to the wall, Hume told him copies of the letter had been obtained by his friends through occult methods or `through the medium of supernatural photography.' And when Dufferin showed him the original letter, proving that the copy was false, Hume had no answer.'

Once earlier, too, Hume had indirectly tried to tell Dufferin that his advisers were not ordinary political leaders but `advanced initiates' and mahatmas; but he had done so in a guarded fashion. In a letter to Dufferin in November 1886, he said that he had been trying to persuade those who had shown him the volumes in Simla to also show them to Dufferin so that the Viceroy could get their veracity checked by his own sources. But, at present they say that this is impossible.' Nor would they agree to communicate with the Viceroy directly. `Most of them, I believe, could not. You have not done, and would not do, what is required to enable them to communicate with you directly after their fashion.' But there was hope. `My own special friend' who spent more than a month with Hume in Simla (in 1878), and who was often in India might agree to see the Viceroy. Hume suggested: `if ever a native gentleman comes to the Private Secretary and says that Mr. Hume said the Viceroy would like to see him, see him at once. You will not talk to him ten minutes without finding out that he is no ordinary man. You may never get the chance — goodness knows — they move in a mysterious way their wonders to

But Hume was worried that he could offer no visible or direct proof of his knowledge or connections. He told the Viceroy that he was `getting gradually very angry and disgusted' because he was not able to get `this vouching for directly.' None of the `advanced initiates' under whose advice and guidance' he was working would `publicly stand by me,' so that most Europeans in India `look upon me either as a lunatic or a liar.' And hence, he informed the Viceroy, while he had decided to continue the political work, he had decided to `drop all references to my friends.''

Thus, it turns out that the seven volumes which Hume saw were prepared by mahatmas and Gurus, and his friends and advisers were these occult figures and not Congressmen!

\begin{center}*\end{center}

\paragraph*{}
Further proof offered for the safety-valve theory was based on W.C. Bannerjee's statement in 1898 in Indian Politics that the Congress, `as it was originally started and as it has since been carried on, is in reality the work of the Marquis of Dufferin and Ava.' He stated that Flume had, in 1884, thought of bringing together leading political Indians once a year ``to discuss social matters'' and did not ``desire that politics should form part of their discussion.'' But Dufferin asked Flume to do the opposite and start a body to discuss politics so that the Government could keep itself informed of Indian opinion. Such a body could also perform `the functions which Her Majesty's Opposition did in England.'

Clearly, either W.C. Bannerjee's memory was failing or he was trying to protect the National Congress from the wrath of the late 19th century imperialist reaction, for contemporary evidence clearly indicated the opposite. All the discussions Hume had with Indian leaders regarding the holding of an annual conference referred to a political gathering. Almost the entire work of earlier associations like the Bombay Presidency Association, Poona Sarvajanik Sabha, Madras Mahajan Sabha and Indian Association was political. Since his retirement from the Indian Civil Service in 1882, Hume had been publicly urging Indians to take to politics. He had also been asking his Indian friends not to get divided on social questions.

When, in January 1885, his friend B.M. Malabari wrote some editorials in the Indian Spectator urging educated Indians to inaugurate a movement for social reform, Hume wrote a letter to the Indian Spectator criticizing Malabari's proposals, warning against the dangerous potential of such a move, and arguing that political reforms should take precedence over social reform.' Dufferin, on his part, in his St. Andrews' Day dinner speech in 1888, publicly criticized the Congress for pursuing politics to serve narrow interests rather than take to social reform which would benefit millions.', Earlier he had expressed the same sentiment in a private letter to the Secretary of State.

A perusal of Dufferin's private papers, thrown open to scholars in the late 1950s, should have put an end to the myth of Dufferin's sponsor of or support to the Congress. It was only after Hume had sent him a Copy of the letter to the Indian Spectator with a covering note deprecating Malabari's views on social reform that Dufferin expressed agreement with Hume and asked him to meet him. Definite confirmation of the fact that Hume never proposed a social gathering but rather a political one comes in Dufferin's letter to Lord Reay, Governor of Bombay, after his friendly meeting with Hume in May 1885: ``At his last interview he told me that he and his friends were going to assemble a political convention of delegates, as far as I understood, on the lines adopted by O'Connell previous to Catholic emancipation.''

Neither Dufferin and his fellow-liberal Governors of Bombay and Madras nor his conservative officials like Alfred and J.B. Lyall, D.M Wallace, A. Colvin and S.C. Bayley were sympathetic to the Congress. It was not only in 1888 that Dufferin attacked the Congress in a vicious manner by writing that he would consider `in what way the happy dispatch may be best applied to the Congress,' for `we cannot allow the Congress to continue to exist.'' In May 1885 itself, he had written to Reay asking him to be careful about Hume's Congress, telling him that it would be unwise to identify with either the reformers or the reactionaries. Reay in turn, in a letter in June 1885, referred with apprehension to the new political activists as `the National Party of India' and warned against Indian delegates, like Irish delegates, making their appearance on the British political scene. Earlier, in May, Reay had cautioned Dufferin that Hume was `the head-centre of an organization ... (which) has for its object to bring native opinion into a focus.'

In fact, from the end of May 1885, Dufferin had grown cool to Hume and began to keep him at an arm's length. From 1886 onwards he also began to attack the `Bengali Baboos and Mahratta Brahmins' for being `inspired by questionable motives' and for wanting to start Irish-type revolutionary agitations in India. And, during May-June 1886. he was describing Hume as `cleverish, a little cracked, excessively vain, and absolutely indifferent to truth,' his main fault being that he was `one of the chief stimulants of the Indian Home Rule movement. To conclude, it is high time that the safety-valve theory of the genesis of the Congress was confined to the care of the mahatmas from whom perhaps it originated!
