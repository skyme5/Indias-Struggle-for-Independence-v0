\cleardoublepage
\chapter{Peasant Movements and Nationalism in the 1920's}

Peasant discontent against established authority was a familiar feature of the nineteenth century. But in the twentieth century, the movements that emerged out of this discontent were marked by a new feature: they were deeply influenced by and in their turn had a marked impact on the ongoing struggle for national freedom. To illustrate the complex nature of this relationship, we will recount the story of three important peasant struggles that emerged in the second and third decade of the country: The Kisan Sabha and Eka movements in Avadh in U.P., the Mappila rebellion in Malabar and the Bardoli Satyagraha in Gujarat.

\begin{center}*\end{center}

\paragraph*{}


Following the annexation of Avadh in 1856, the second half of the nineteenth century had seen the strengthening of the hold of the taluqdars or big landlords over the agrarian society of the province. This had led to a situation in which exorbitant rents, illegal levies, renewal fees or nazrana, and arbitrary ejectments or bedakhli had made life miserable for the majority of the cultivators. The high price of food and other necessities that accompanied and followed World War I made the oppression all the more difficult to bear, and the tenants of Avadh were ripe for a message of resistance.

It was the more active members of the Home Rule League in

U.P. who initiated the process of the organization of the peasants of the province on modem lines into kisan sabhas. The U.P. Kisan Sabha was set up in February 1918 through the efforts of Gauri Shankar Misra and lndra Narain Dwivedi, and with the support of Madan Mohan Malaviya. The U.P. Kisan Sabha demonstrated considerable activity, and by June 1919 had established at least 450 branches in 173 tehsils of the province.

A consequence of this activity was that a large number of kisan delegates from U.P. attended the Delhi and Amritsar sessions of the Indian National Congress in December 1918 and 1919.

Towards the end of 1919, the first signs of grass-roots peasant activity were evident in the reports of a nai-dhobi band (a form of social boycott) on an estate in Pratapgarh district. By the summer of 1920, in the villages of taluqdari Avadh, kisan meetings called by village panchayats became frequent. The names of Thinguri Singh and Durgapal Singh were associated with this development. But soon another leader, who became famous by the name of Baba Ramchandra, emerged as the rallying point. Baba Ramchandra, a Brahmin from Maharashtra, was a wanderer who had left home at the age of thirteen, done a stint as an indentured labourer in Fiji and finally turned up in Faizabad in U.P. in 1909. Till 1920, he had wandered around as a sadhu, carrying a copy of Tulsidas' Ramavan on his back, from which he would often recite verses to rural audiences. In the middle of 1920, however, he emerged as a leader of the peasants of Avadh, and soon demonstrated considerable leadership and organizational capacities.

In June 1920, Baba Ramchandra led a few hundred tenants from the Jaunpur and Pratapgarh districts to Allahabad. There he met Gauri Shankar Misra and Jawaharlal Nehru and asked them to visit the villages to see for themselves the living conditions of the tenants. The result was that, between June and August, Jawaharlal Nehru made several visits to the rural areas and developed close contacts with the Kisan Sabha movement.

\begin{center}*\end{center}

\paragraph*{}


Meanwhile, the kisans found sympathy in Mehta, the Deputy Commissioner of Pratapgarh, who promised to investigate complaints forwarded to him. The Kisan Sabha at village Roor in Pratapgarh district became the centre of activity and about one lakh tenants were reported to have registered their complaints with this Sabha on the payment of one anna each. Gauri Shankar Mia was also very active in Pratapgarh during this period, and was in the process of working out an agreement with Mehta over some of the crucial tenant complaints such as bedakhli and nazrana.

But, in August 1920, Mehta went on leave and the taluqdars used the opportunity to strike at the growing kisan movement. They succeeded in getting Ramchandra and thirty-two kisans arrested on a trumped-up charge of theft on 28 August 1920. Incensed at this, 4,000 to 5,000 kisans collected at Pratapgarh to see their leaders in jail and were dispersed after a great deal of persuasion.

Ten days later, a rumour that Gandhiji was coming to secure the release of Baba Ramchandra brought ten to twenty thousand kisans to Pratapgarh, and this time they returned to their homes only after Baba Ramchandra gave them darshan from atop a tree in a sugar-cane field. By now, their numbers had swelled to sixty thousand. Mehta was called back from leave to deal with the situation and he quickly withdrew the case of theft and attempted to bring pressure on the landlords to change their ways This easy victory, however, gave a new confidence to the movement and it burgeoned forth.

Meanwhile, the Congress at Calcutta had chosen the path of non cooperation and many nationalists of U.P. had committed themselves to the new political path. But there were others, including Madan Mohan Malaviya, who preferred to stick to constitutional agitation. These differences were reflected in the

U.P. Kisan Sabha as well, and soon the Non-cooperators set up an alternative Oudh Kisan Sabha at Pratapgarh on 17 October 1920. This new body succeeded in integrating under its banner all the grassroots kisan sabhas that had emerged in the districts of Avadh in the past few months; through the efforts of Misra, Jawaharlal Nehru, Mata Badal Pande, Baba Ramchandra, Deo Narayan Pande and Kedar Nath, the new organization brought under its wing, by the end of October, over 330 kisan sabhas. The Oudh Kisan Sabha asked the kisans to refuse to till bedakhli land, not to offer hari and begar (forms of unpaid labour), to boycott those who did not accept these conditions and to solve their disputes through panchayats. The first big show of strength of the Sabba was the rally held at Ayodhya, near Faizabad town, on 20 and 21 December which was attended by roughly 100,000 peasants. At this rally, Baba Ramchandra turned up bound in ropes to symbolize the oppression of the kisans. A marked feature of the Kisan Sabha movement was that kisans belonging to the high as well as the low castes were to be found in its ranks.

In January 1921, however, the nature of the peasant activity underwent a marked change. The centres of activity were primarily the districts of Rae Bareli, Faizabad and, to a lesser extent, Sultanpur. The pattern of activity was the looting of bazaars, houses, granaries, and clashes with the police. A series of incidents, small and big, but similar in character. Some, such as the ones at Munshiganj and Karhaiya Bazaar in Rae Bareli, were sparked off by the arrests or rumours of arrest of leaders. The lead was often taken not by recognized Kisan Sabha activists, but by local figures-- sadhus, holy men, and disinherited ex- proprietors.

The Government, however, had little difficulty in suppressing these outbreaks of violence. Crowds were fired upon and dispersed, leaders and activists arrested, cases launched and, except for a couple of incidents in February and March, the movement was over by the end of January itself. In March, the Seditious Meetings Act was brought in to cover the affected districts and all political activity came to a standstill. Nationalists continued to defend the cases of the tenants in the courts, but could do little else. The Government, meanwhile, pushed through the Oudh Rent (Amendment) Act, and though it brought little relief to the tenants, it helped to rouse hopes and in its own way assisted in the decline of the movement.

\begin{center}*\end{center}

\paragraph*{}


Towards the end of the year, peasant discontent surfaced again in Avadh, hut this time the centres were the districts of Hardoi, Bahraich, and Sitapur in the northern part of the province. The initial thrust here was provided by Congress and Khilafat leaders and the movement grew under the name of the Eka or unity movement. The main grievances here related to the extraction of a rent that was generally fifty per cent higher than the recorded rent, the oppression of thekedars to whom the work of rent- collection was farmed out and the practice of share-rents.

The Eka meetings were marked by a religious ritual in which a hole that represented the river Ganges was dug in the ground and filled with water, a priest was brought in to preside and the assembled peasants `owed that they would pay only the recorded rent but pay it on time, would not leave when ejected, would refuse to do forced labour, would give no help to criminals and abide by the panchayat decisions. The Eka Movement, however, soon developed its own grass­ roots leadership in the form of Madari Pasi and other low-caste leaders who were no particularly inclined to accept the discipline of non-violence that the Congress and Khilafat leaders urged. As a result, the movement's contact with the nationalists diminished and it went its own way. However, unlike the earlier Kisan Sabha movement that was based almost solely on tenants, the Eka Movement included in its ranks many small zamindars who found themselves disenchanted with the Government because of its heavy land revenue demand. By March 1922, however, severe repression on the part of the authorities succeeded in bringing the Eka Movement to its end.

\begin{center}*\end{center}

\paragraph*{}


In August 1921, peasant discontent erupted in the Malabar district of Kerala. Here Mappila (Muslim) tenants rebelled. Their grievances related to lack of any security of tenure, renewal fees, high rents, and other oppressive landlord exactions. In the nineteenth century as well, there had been cases of Mappila resistance to landlord oppression but what erupted in 1921 was on a different scale together.

The impetus for resistance had first come from the Malabar District Congress Conference held at Manjeri in April 1920. This conference supported the tenants' cause and demanded legislation to regulate landlord- tenant relations. The change was significant because earlier the landlords had successfully prevented the Congress from committing itself to the tenants' cause. The Manjeri conference was followed by the formation of a tenants' association at Kozhikode, and soon tenants' associations were set up in other parts of the district.

Simultaneously, the Khilafat Movement was also extending its sweep. In fact, there was hardly any way one could distinguish between Khilafat and tenants' meetings, the leaders and the audience were the same, and the two movements were inextricably merged into one. The social base of the movement was primarily among the Mappila tenants, and Hindus were quite conspicuous by their absence, though the movement could count on a number of Hindu leaders. Disturbed by the growing popularity of the Khilafat-cum­ tenant agitation, which had received considerable impetus from the visits of Gandhiji, Shaukat Au, and Maulana Azad, the Government issued prohibitory notices on all Khilafat meetings on 5 February 1921. On 18 February, all the prominent Khilafat and Congress leaders, Yakub Hasan, U. Gopala Menon, P. Moideen Koya and K. Madhavan Nair, were arrested. This resulted in the leadership passing into the hands of the local Mappila leaders.

Angered by repression and encouraged by rumours that the British, weakened as a result of the World War, were no longer in a position to take strong military action, the Mappilas began to exhibit increasing signs of turbulence and defiance of authority. But the final break came only when the District Magistrate of Eranad taluq. E.F. Thomas, on 20 August 1921, accompanied by a contingent of police and troops, raided the mosque at Tirurangadi to arrest Ali Musaliar, a Khilafat leader and a highly respected priest. They found only three fairly insignificant Khilafat volunteers and arrested them. However the news that spread was that the famous Mambrath mosque, of which Au Musaliar was the priest, had been raided and destroyed by the British army. Soon Mappilas from Kottakkal, Tanur and Parappanagadi converged at Tirurangadi and their leaders met the British officers to secure the release of the arrested volunteers. The people were quiet and peaceful, but the police opened fire on the unarmed crowd and many were killed. A clash ensued, and Government offices were destroyed, records burnt and the treasury looted. The rebellion soon spread into the Eranad, Walluvanad and Ponnani taluqs, all Mappila strongholds.

In the first stage of the rebellion, the targets of attack were the unpopular jenmies (landlords), mostly Hindu, the symbols of Government authority's such as kutcheris (courts), police stations, treasuries and offices, and British planters. Lenient landlords and poor Hindus were rarely touched. Rebels would travel many miles through territory populated by Hindus and attack only the landlords and burn their records. Some of the rebel leaders, like Kunhammed Haji, took special care to see that Hindus were not molested or looted and even punished those among the rebels who attacked the Hindus. Kunhammed Haji also did not discriminate in favour of Muslims: he ordered the execution and punishment of a number of pro-government Mappilas as well. But once the British declared martial law and repression• began in earnest, the character of the rebellion underwent a definite change. Many Hindus were either pressurized into helping the authorities or voluntarily gave assistance and this helped to strengthen the already existing anti- Hindu sentiment among the poor illiterate Mappilas who in any case were motivated by a strong religious ideology. Forced conversions, attacks on and murders of Hindus increased as the sense of desperation mounted. What had been largely an anti-government and anti-landlord affair acquired strong communal overtones.

The Mappilas' recourse to violence had in any case driven a wedge between them and the Non-Cooperation Movement which was based on the principle of non-violence. The communalization of the rebellion completed the isolation of the Mappilas. British repression did the rest and by December 1921 all resistance had come to a stop. The toll was heavy indeed: 2,337 Mappilas had lost their lives. Unofficial estimates placed the number at above 10,000. A total of 45,404 rebels were captured or had surrendered. But the toll was in fact even heavier, though in a very different way. From then onwards, the militant Mappilas were so completely crushed and demoralized that till independence their participation in any form of politics was almost nil. They neither joined the national movement nor the peasant movement that was to grow in Kerala in later years under the Left leadership.

\begin{center}*\end{center}

\paragraph*{}


The peasant movements in U.P. and Malabar were thus closely linked with the politics at the national level. In UP., the impetus had come from the Home Rule Leagues and, later, from the Non-Cooperation and Khilafat movement. In Avadh, in the early months of 1921 when peasant activity was at its peak, it was difficult to distinguish between a Non cooperation meeting and a peasant rally. A similar situation arose in Malabar, where Khilafat and tenants' meetings merged into one. But in both places, the recourse to violence by the peasants created a distance between them and the national movement and led to appeals by the nationalist leaders to the peasants that they should not indulge in violence. Often, the national leaders, especially Gandhiji, also asked the peasants to desist from taking extreme action like stopping the payment of rent to landlords.

This divergence between the actions and perceptions of peasants and local leaders and the understanding of the national leaders had often been interpreted as a sign of the fear of the middle class or bourgeois leadership that the movement would go out of its own `safe' hands into that of supposedly more radical and militant leaders of the people. The call for restraint, both in the demands as well as in the methods used, is seen as proof of concern for the landlords and propertied classes of Indian society. It is possible, however, that the advice of the national leadership was prompted by the desire to protect the peasants from the consequences of violent revolt, consequences which did not remain hidden for long as both in U.P. and Malabar the Government launched heavy repression in order to crush the movements. Their advice that peasants should not push things too far with the landlords by refusing to pay rent could also stem from other considerations. The peasants themselves were not demanding abolition of rent or landlordism, they only wanted an end to ejectments, illegal levies, and exorbitant rents — demands which the national leadership supported. The recourse to extreme measures like refusal to pay rent was likely to push even the small landlords further into the lap of the government and destroy any chances of their maintaining a neutrality towards the on-going conflict between the government and the national movement.

\begin{center}*\end{center}

\paragraph*{}


The no-tax movement that was launched in Bardoli taluq of Surat district in Gujarat in 1928 was also in many ways a child of the Non-cooperation days.' Bardoli taluq had been selected in 1922 as the place from where Gandhiji would launch the civil disobedience campaign, but events in Chauri Chaura had changed all that and the campaign never took off. However, a marked change had taken place in the area because of the various preparations for the civil disobedience movement and the end result was that Bardoli had undergone a process of intense politicization and awareness of the political scene. The local leaders such as the brothers Kalyanji and Kunverji Mehta, and Dayalji Desai, had worked hard to spread the message of the Non-Cooperation Movement. These leaders, who had been working in the district as social reformers and political activists for at least a decade prior to Non-cooperation, had set up many national schools, persuaded students to leave government schools, carried out the boycott of foreign cloth and liquor, and had captured the Surat municipality.

After the withdrawal of the Non-Cooperation Movement, the Bardoli Congressmen had settled down to intense constructive work.

Stung by Gandhiji's rebuke in 1922 that they had done nothing for the upliftment of the low-caste untouchable and tribal inhabitants — who were known by the name of Kaliparaj (dark people) to distinguish them from the high caste or Ujaliparaj (fair people) and who formed sixty per cent of the population of the taluq — these men, who belonged to high castes started work among the Kaliparaj through a network of six ashrams that were spread out over the taluq. These ashrams, many of which survive to this day as living institutions working for the education of the tribals, did much to lift the taluq out of the demoralization that had followed the withdrawal of 1922. Kunverji Mehta and Keshavji Ganeshji learnt the tribal dialect, and developed a `Kaliparaj literature' with the assistance of the educated members of the Kaliparaj community, which contained poems and prose that aroused the Kaliparaj against the Hali system under which they laboured as hereditary labourers for upper-caste landowners, and exhorted them to abjure intoxicating drinks and high marriage expenses which led to financial ruin. Bhajan mandalis consisting of Kaliparaj and Ujaliparaj members were used to spread the message. Night schools were started to educate the Kaliparaj and in 1927 a school for the education of Kaliparaj children was set up in Bardoli town. Ashram workers had to often tce the hostility of upper-caste landowners who feared that all this would `spoil' their labour. Annual Kallparaj conferences were held in 1922 and, in 1927, Gandhiji, who presided over the annual conference, initiated an enquiry into the conditions of the Kaliparaj , who he also now renamed as Raniparaf or the inhabitants of the forest in preference to the derogatory term Kaliparaj or dark people. Many leading figures of Gujarat including Narhari Parikh and Jugatram Dave conducted the inquiry which turned into a severe indictment of the Hall system, exploitation by money lenders and sexual exploitation of women by upper-castes. As a result of this, the Congress had built up a considerable' base among the Kaliparaj, and could count on their support in the future.

Simultaneously, of course, the Ashram workers had continued to work among the landowning peasants as well, and had to an extent regained their influence among them. Therefore, when in January 1926 it became known that Jayakar, the officer charged with the duty of reassessment of the land revenue demand of the taluq, had recommended a thirty percent increase over the existing assessment, the Congress leaders were quick to protest against the increase and set up the Bardoli Inquiry Committee to go into the issue. Its report, published in July 1926, came to the conclusion that the increase was unjustified. This was followed by a campaign in the Press, the lead being taken by Young India and Navjivan edited by Gandhiji. The constitutionalist leaders of the area, including the members of the Legislative Council, also took up the issue. In July 1927, the Government reduced the enhancement to 21.97 per cent.

But the concessions were too meagre and came too late to satisfy anybody. The constitutionalist leaders now began to advise the peasants to resist by paying only the current amount and withholding the enhanced amount. The `Ashram' group, on the other hand, argued that the entire amount must be withheld if it was to have any effect on the Government. However, at this stage, the peasants seemed more inclined to heed the advice of the moderate leaders.

Gradually, however, as the limitations of the constitutional leadership became more apparent, and their unwillingness to lead even a movement based on the refusal of the enhanced amount was clear, the peasants began to move towards the `Ashram' group of Congress leaders. The latter, on their pan had in the meanwhile contacted Vallabhbhai Patel and were persuading him to take on the leadership of the movement A meeting of representatives of sixty villages at Bamni in Kadod division formally invited Vallabhbhai to lead the campaign. The local leaders also met Gandhiji and after having assured him that the peasants were fully aware of the implications of such a campaign, secured his approval.

Patel reached Bardoli on 4 February and immediately had a series of meetings with the representatives of the peasants and the constitutionalist leaders. At one such meeting, the moderate leaders frankly told the audience that their methods had failed and they should now try Vallabhbhai's methods. Vallabhbhai explained to the peasants the consequences of their proposed plan of action and advised them to give the matter a week's thought. He then returned to Ahmedabad and wrote a letter to the Governor of Bombay explaining the miscalculations in the settlement report and requesting him to appoint an independent enquiry; else, he wrote, he would have to advise the peasants to refuse to pay the Land revenue and suffer the consequences.

On 12 February, Patel returned to Bardoli and explained the situation, including the Government's curt reply, to the peasants' representatives, following this, a meeting of the occupants of Bardoli taluq passed a resolution advising all occupants of land to refuse payment of the revised assessment until the Government appointed an independent tribunal or accepted the current amount as full payment. Peasants were asked to take oaths in the name of Prabhu (the Hindu name for god) and Khuda (the Muslim name for god) that they would not pay the land revenue. The resolution was followed by the recitation of sacred texts from the Gita and the Koran and songs from Kabir, who symbolized Hindu-Muslim unity. The Satyagraha had begun.

Vallabhbhai Paid was ideally suited for leading the campaign. A veteran of the Kheda Satyagraha, the Nagpur Flag Satyagraha, and the Borsad Punitive Tax Satyagraha, he had emerged as a leader of Gujarat who was second only to Gandhiji. His capacities as an organizer, speaker, indefatigable campaigner, inspirer of ordinary men and women were already known, but it was the women of Bardoli who gave him the title of Sardar. The residents of Bardoli to this day recall the stirring effect of the Sardar's speeches which he delivered in an idiom and style that was close to the peasant's heart.

The Sardar divided the taluq into thirteen workers' camps or Chhavanis each under the charge of an experienced leader. One hundred political workers drawn from all over the province, assisted by 1,500 volunteers, many of whom were students, formed the army of the movement. A publications bureau that brought out the daily Bardoli Satyagraha Patrika was set up. This Patrika contained reports about the movement, speeches of the leaders, pictures of the jabti or confiscation proceedings and other news. An army of volunteers distributed this to the farthest corners of the taluq. The movement also had its own intelligence wing, whose job was to find out who the indecisive peasants were. The members of the intelligence wing would shadow them night and day to see that they did not pay their dues, secure information about Government moves, especially of the likelihood of jabti (confiscation) and then warn the villagers to lock up their houses or flee to neighbouring Baroda.

The main mobilization was done through extensive propaganda via meetings, speeches, pamphlets, and door to door persuasion. Special emphasis was placed on the mobilization of women and many women activists like Mithuben Petit, a Parsi lady from Bombay, Bhaktiba, the wife of Darbar Gopaldas, Maniben Patel, the Sardar' s daughter, Shardaben Shah and Sharda Mehta were recruited for the purpose. As a result, women often outnumbered men at the meetings and stood firm in their resolve not to submit to Government threats. Students were another special target and they were asked to persuade their families to remain thin.

Those who showed signs of weakness were brought into line by means of social pressure and threats of social boycott. Caste and village panchayats were used effectively for this purpose and those who opposed the movement had to face the prospect of being refused essential services from sweepers, barbers, washermen, agricultural labourers, and of being socially boycotted by their kinsmen and neighbours. These threats were usually sufficient to prevent any weakening. Government officials faced the worst of this form of pressure. They were refused supplies, services, transport and found it almost impossible to carry out their official duties. The work that the Congress leaders had done among the Kaliparaj people also paid dividends during this movement and the Government was totally unsuccessful in its attempts to use them against the upper caste peasants.

Sardar Patel and his colleagues also made constant efforts to see that they carried the constitutionalist and moderate leadership, as well as public opinion, with them on all important issues. The result of this was that very soon the Government found even its supporters and sympathizers, as well as impartial men, deserting its side. Many members of the Bombay Legislative Council like K.M. Munshi and Laiji Naranji, the representatives of the Indian Merchants Chamber, who were not hot-headed extremists, resigned their seats. By July 1928, the Viceroy, Lord Irwin, himself began to doubt the correctness of the Bombay Government's stand and put pressure on Governor Wilson to find a way out. Uncomfortable questions had started appearing in the British Parliament as well.

Public opinion in the country was getting more and more restive and anti-Government. Peasants in many parts of Bombay Presidency were threatening to agitate for revision of the revenue assessments in their areas. Workers in Bombay textile mills were on strike and there was a threat that Patel and the Bombay Communists would combine in bringing about a railway strike that would make movement of troops and supplies to Bardoli impossible. The Bombay Youth League and other organizations had mobilized the people of Bombay for huge public meetings and demonstrations. Punjab was offering to send jathas on foot to Bardoli. Gandhiji had shifted to Bardoli on 2 August, 1928, in order to take over the reins of the movement if Patel was arrested. All told, a retreat, if it could be covered up by a face saving device, seemed the best way out for the Government.

The face-saving device was provided by the Legislative Council members from Surat who wrote a letter to the Governor assuring him that his pre-condition for an enquiry would be satisfied. The letter contained no reference to what the pre­ condition was (though everyone knew that it was full payment of the enhanced rent) because an understanding had already been reached that the full enhanced rent would not be paid. Nobody took the Governor seriously when he declared that he had secured an `unconditional surrender.'' It was the Bardoli peasants who had won.

The enquiry, conducted by a judicial officer, Broomfield, and a revenue officer, Maxwell, came to the conclusion that the increase had been unjustified, and reduced the enhancement to 6.03 per cent. The New statesman of London summed up the whole affair on 5 May 1929: `The report of the Committee constitutes the worst rebuff which any local government in India has received for many years and may have far- reaching results... It would be difficult to find an incident quite comparable with this in the long and controversial annals of Indian Land Revenue. `

The relationship of Bardoli and other peasant struggles with the struggle for freedom can best be described in Gandhiji's pithy words: `Whatever the Bardoli struggle may be, it clearly is not a struggle for the direct attainment of Swaraj. That every such awakening, every such effort as that of Bardoli will bring Swaraj nearer and may bring it nearer even than any direct effort is undoubtedly true.'
