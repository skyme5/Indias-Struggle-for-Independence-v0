\cleardoublepage
\chapter{Civil Disobedience: 1930-31}

The Lahore Congress of I929 authorized the Working Committee to launch a programme civil disobedience including non-payment of taxes. It had also called upon all members of legislatures to resign their seats. In mid-February, 1930, the Working Committee, meeting at Sabarmati Ashram, invested Gandhiji with fill powers to launch the Civil Disobedience Movement at a time and place of his choice. The acknowledged expert on mass struggle was already `desperately in search of an effective formula.'' His ultimatum of 31 January to Lord Irwin, stating the minimum demands in the form of II points, had been ignored, and there was now only one way out: civil disobedience.

\begin{center}*\end{center}

\paragraph*{}


By the end of February, the formula began to emerge as Gandhiji began to talk about salt: `There is no article like salt outside water by taxing which the State can reach even the starving millions, the sick, the maimed and the utterly helpless. The tax constitutes therefore the most inhuman poll tax the ingenuity of man can devise.' On 2 March, he addressed his historic later to the Viceroy in which he first explained at great length why he regarded British rule as a curse: `It has impoverished the dumb millions by a system of progressive exploitation ... It has reduced us politically to serfdom. It has sapped the foundations of our culture ... it has degraded us spiritually.' He then informed the Viceroy of his plan of action, as he believed every true Satyagrahi must: `...on the 11th day of this month. I shall proceed with such co-workers of the Ashram as I can take, to disregard the provisions of the salt laws. It is, I know, open to you to frustrate my design by arresting me. I hope that there will be tens of thousands ready, in a disciplined manner, to take up the work after me, and, m the act of disobeying the Salt Act to lay themselves open to the penalties of a law that should never have disfigured the Statute-book.' 

The plan was brilliantly conceived though few realized its significance when it was first announced. Gandhiji, along with a band of seventy-eight members of the Sabarmati Ashram, among whom were men belonging to almost every region and religion of India, was to march from his headquarters in Ahmedabad through the villages of Gujarat for 240 miles. On reaching the coast at Dandi, he would break the salt laws by collecting salt from the beach. The deceptively innocuous move was to prove devastatingly effective. Even before the march began, thousands began to throng the Sabarmati Ashram in anticipation of the dramatic events that lay ahead. And Gandhiji painstakingly explained his plans, gave directions for future action, impressed on the people the necessity for non-violence, arid prepared them for the Government's response: `Wherever possible, civil disobedience of salt laws should be started ... Liquor and foreign- cloth shops can be picketed. We can refuse to pay taxes if we have the requisite strength. The lawyers can give up practice. The public can boycott the courts by refraining from litigation. Government servants can resign their posts ... I prescribe only one condition, viz., let our pledge of truth and non-violence as the only means for the attainment of Swaraj be faithfully kept.' 

Explaining the power of civil disobedience, he said: `Supposing ten persons from each of the 700,000 villages in India come forward to manufacture salt and to disobey the Salt Act, what do you think this Government can do? Even the worst autocrat you can imagine would not dare to blow regiments of peaceful civil resisters out of a cannon's mouth. If only you will bestir yourselves just a little, I assure you we should be able to tire this Government out in a very short time.' 

He also explained how non-violence enabled the widest participation of the people, and put the Government in an unenviable quandary. To a crowd who came to the ashram on 10 March, he said: `Though the battle is to begin in a couple of days, how is it that you can come here quite fearlessly? I do not think any one of you would be here if you had to face rifle-shots or bombs. But you have no fear of rifle-shots or bombs? Why? 

Supposing I had announced that I was going to launch a violent campaign (not necessarily with men aimed with rifles, but even with sticks or stones), do you think the Government would have left me free until now? Can you show me an example in history (be it in England, America or Russia) where the State has tolerated violent defiance of authority for a single day? But here you know that the Government is puzzled and perplexed.' 

And as Gandhiji began his march, staff in hand, at the head of his dedicated band, there was something in the image that deeply stirred the imagination of the people. News of his progress, of his speeches, of the teeming crowds that greeted and followed the marchers, of the long road lovingly strewn with leaves and festooned with banners and flags, of men and women quietly paying their homage by spinning yam on their charkas as Gandhiji passed, of the 300 village officials in Gujarat who resigned their posts in answer to his appeal, was carried day after day by newspapers to readers across the country and broadcast live by thousands of Congress workers to eager listeners. By the time Gandhiji reached Dandi, he had a whole nation, aroused and expectant, waiting restlessly for the final signal. On 6 April 1930, by picking up a handful of salt, Gandhiji inaugurated the Civil Disobedience Movement, a movement that was to remain unsurpassed in the history of the Indian national movement for the country-wide mass participation it unleashed.

\begin{center}*\end{center}

\paragraph*{}


While Gandhiji was marching to Dandi, Congress leaders and workers had been busy at various levels with the hard organizational task of enrolling volunteers and members, forming grass-roots Congress Committees, collecting funds, and touring villages and towns to spread the nationalist message. Preparations for launching the salt Satyagraha were made, sites chosen, volunteers prepared, and the logistics of battle worked out. 

Once the way was cleared by Gandhiji's ritual beginning at Dandi, the defiance of salt laws started all over the country. In Tamil Nadu, C. Rajagopalachari, led a salt march from Trichinopoly to Vedaranniyam on the Tanjore coast. By the time he was arrested on 30 April he had collected enough volunteers to keep the campaign going for quite some time in Malabar, K. Kelappan, the hero of the Vaikom Satyagraha, walked from Calicut to Payannur to break the salt law. A band of Satyagrahis walked all the way from Sylhet in Assam to Noakhali on the Bengal Coast to make salt. In Andhra, a number of sibirams (military style camps) were set up in different districts to serve as the headquarters of the salt Satyagraha, and bands of Satyagrahis marched through villages on their way to the coastal centres to defy the law. On their return journeys, they again toured through another set of villages. The Government's failure to arrest Gandhiji for breaking the salt law was used by the local level leaders to impress upon the people that `the Government is afraid of persons like ourselves,' and that since the starting of the salt Sa1yagrah the Government `has disappeared and hidden itself somewhere, and that Gandhi Government has already been established.'9 Jawaharlal Nehru's arrest on 14 April, for defiance of the salt law, was answered with huge demonstrations and clashes with the police in the cities of Madras, Calcutta and Karachi. 

On 23 April, the arrest of Congress leaders in the North West Frontier Province led to a mass demonstration of unprecedented magnitude in Peshawar. Khan Abdul Gaffar Khan had been active for several years in the area, and it was his mass work which lay behind the formation of the band of non-violent revolutionaries, the Khudai Khidmatgars, popularly known as the Red Shirts — who were to play an extremely active role in the Civil Disobedience Movement. The atmosphere created by their political work contributed to the mass upsurge in Peshawar during which the city was virtually in the hands of the crowd for more than a week. The Peshawar demonstrations are significant because it was here that the soldiers of the Garhwali regiments refused to fire on the unarmed crowd.

\begin{center}*\end{center}

\paragraph*{}
It was becoming increasingly clear that the Government's gamble — that non-interference with the movement would result in its spending itself out, that Gandhiji's salt strategy would fail to take off— had not paid off. In fact, the Government had never been clear on what course it should follow, and was, as Gandhiji had predicted, `puzzled and perplexed.' The dilemma in which it found itself was a dilemma that the Gandhian strategy of non-violent civil disobedience was designed to create. The Government was placed in a classic `damned if you do, damned if you don't' fix, i.e. if it did not suppress a movement that brazenly defied its laws, its administrative authority would be seen to be undermined and its control would be shown to be weak, and if it did suppress it, it would be seen as a brutal, anti-people administration that used violence on non-violent agitators. `If we do too much, Congress will cry ``repression'' ... if we do too little. Congress will cry ``victory,'' `— this is how a Madras civilian expressed the dilemma in early 1930.'' Either way, it led to the erosion of the hegemony of the British government. 

The rapid spread of the movement left the Government with little choice but to demonstrate the force that lay behind its benevolent facade. Pressure from officials, Governors and the military establishment started building up, and, on 4 May, the Viceroy finally ordered Gandhiji's arrest. Gandhiji's announcement that he would now proceed to continue his defiance of the salt laws by leading a raid on the Dharasana Salt Works certainly forced the Government's hand, but its timing of Gandhiji's arrest was nevertheless ill-conceived. It had neither the advantage of an early strike, which would have at least prevented Gandhiji from carefully building up the momentum of the movement, nor did it allow the Government to reap the benefits of their policy of sitting it out. Coming as it did at a high point in the movement, it only acted as a further spur to activity, and caused endless trouble for the Government.' 

There was a massive wave of protest at Gandhiji's arrest. In Bombay, the crowd that spilled out into the streets was so large that the police just withdrew. Its ranks were swelled by thousands of textile and railway workers. Cloth-merchants went on a six-day hartal. There were clashes and firing in Calcutta and Delhi. But it was in Sholapur, in Maharashtra, that the response was the fiercest. The textile workers, who dominated the town went on strike from 7 May, arid along with other residents, burnt liquor shops and proceeded to attack all symbols of Government authority -- the railway station, law courts, police stations and municipal buildings. They took over the city and established a virtual parallel government which could only be dislodged with the imposition of martial law after 16 May.

\begin{center}*\end{center}

\paragraph*{}


But it was non-violent heroism that stole the show as the salt Satyagraha assumed yet another, even more potent form. On May 21, with Sarojini Naidu, the first Indian woman to become President of the Congress, and Imam Saheb, Gandhiji's comrade of the South African struggle, at the helm, and Gandhiji's son, Manual, in front ranks, a band of 2000 marched towards the police cordon that had sealed off the Dharasana salt works. As they caine close, the police rushed forward with their steel-tipped lathis and set upon the non-resisting Satyagrahis till they fell down. The injured would be carried away by their comrades on make-shift stretchers and another column would take their place, be beaten to pulp, and carried away. Column after column advanced in this way; after a while, instead of walking up to the cordon the men would sit down and wait for the police blows. Not an arm was raised in defence, and by 11 a.m., when the temperature in the shade was 116 degrees Fahrenheit, the toll was already 320 injured and two dead. Webb Miller, the American journalist, whose account of the Dharasana Satyagraha was to carry the flavour of Indian nationalism to many distant lands, and whose description of the resolute heroism of the Satyagrahis demonstrated effectively that non-violent resistance was no meek affair, summed up his impressions in these words: `In eighteen years of my reporting in twenty countries, during which I have witnessed innumerable civil disturbances, riots, street fights and rebellions, I have never witnessed such harrowing scenes as at Dharasana.' 

This new form of salt Satyagraha was eagerly adopted by the people, who soon made it a mass affair. At Wadala, a suburb of Bombay, the raids on the salt works culminated on 1 June in mass action by a crowd of 15,000 who repeatedly broke the police cordon and triumphantly carried away salt in the face of charges by the mounted police. In Karnataka, 10,000 invaded the Sanikatta salt works and faced lathis and bullets. In Madras, the defiance of salt laws led to repeated clashes with the police and to a protest meeting on 23 April on the beach which was dispersed by lathi charges and firing, leaving three dead. This incident completely divided the city on racial lines, even the most moderate of Indians condemning the incident, and rallying behind the nationalists. In Andhra bands of village women walked miles to carry away a handful of salt, and in Bengal, the old Gandhian ashrams, regenerated by the flood of volunteers from the towns, continued to sustain a powerful salt Satyagraha in Midnapore and other coastal pockets. The districts of Balasore, Pun and Cuttack in Orissa remained active centres of illegal salt manufacture.

\begin{center}*\end{center}

\paragraph*{}


But salt Satyagraha was only the catalyst, and the beginning, for a rich variety of forms of defiance that it brought in its wake. Before his arrest Gandhiji had already called for a vigorous boycott of foreign cloth and liquor shops and had especially asked the women to play a leading role in this movement. `To call woman the weaker sex is a libel: it is man's injustice to woman,'' he had said; and the women of India certainly demonstrated in 1930 that they were second to none in strength and tenacity of purpose. Women who had never stepped unescorted out of their homes, woen who had stayed in purdah, young mothers and widows and unmarried girls, became a familiar sight as they stood from morning to night outside liquor shops and opium dens and stores selling foreign cloth, quietly but firmly persuading the customers and shopkeepers to change their ways. 

Along with the women, students and youth played the most prominent part in the boycott of foreign cloth and liquor. In Bombay, for example, regular Congress sentries were posted in business districts to ensure that merchants and dealers did not flout the foreign cloth boycott. Traders' associations and commercial bodies were themselves quite active in implementing the boycott, as were the many mill owners who refused to use foreign yarn and pledged not to manufacture coarse cloth that competed with khadi. The recalcitrant among them were brought in line by fines levied by their own associations, by social boycott, by Congress black-listing, and by picketing. 

The liquor boycott brought Government revenues from excise duties crashing down; it also soon assumed a new popular form, that of cutting off the heads of toddy trees. The success of the liquor and drugs boycott was obviously connected with the popular tradition of regarding abstinence as a virtue and as a symbol of respectability. The depth of this tradition is shown by the fact that lower castes trying to move up in the caste hierarchy invariably tried to establish their upper caste status by giving up liquor and eating of meat.

\begin{center}*\end{center}

\paragraph*{}
Eastern India became the scene of a new kind of no-tax campaign — refusal to pay the chowkidara tax. Chowkidars, paid out of the tax levied specially on the villages, were guards who supplemented the small police force in the rural areas in this region. They were particularly hated because they acted as spies for the Government and often also as retainers for the local landlords. The movement against this tax and calling for the resignation of Chowkidars, and of the influential members of chowkidari panchayats who appointed the Chowkidars, first started in Bihar in May itself, as salt agitation had not much scope due to the land-locked nature of the province. In the Monghyr, Saran and Bhagalpur districts, for example, the tax was refused, Chowkidars induced to resign, and social boycott used against those who resisted. The Government retaliated by confiscation of property worth hundreds and thousands in lieu of a few rupees of tax, and by beatings and torture. Matters came to a head in Bihpur in Bhagalpur on May 31 when the police, desperate to assert its fast-eroding authority, occupied the Congress ashram which was the headquarters of nationalist activity in the area. The occupation triggered off daily demonstrations outside the ashram, and a visit by Rajendra Prasad and Abdul Ban from Patna became the occasion for, a huge mass rally, which was broken up by a lathi charge in which Rajendra Prasad was injured. As elsewhere, repression further increased the nationalists' strength, and the police just could not enter the rural areas. 

In Bengal, the onset of the monsoon, which made it difficult to make salt, brought about a shift to anti-chowkidara and anti-Union Board agitation. Here too, villagers withstood severe repression, losing thousands of rupees worth of property through confiscation and destruction, and having to hide for days in forests to escape the wrath of the police. 

In Gujarat, in Kheda district, in Bardoli taluqa in Surat district, and in Jambusar in Broach, a determined no-tax movement was in progress — the tax refused here was the land revenue. Villagers in their thousands, with family, cattle and household goods, crossed the border from British India into the neighbouring princely states such as Baroda and camped for months together in the open fields. Their houses were broken into, their belongings destroyed, their lands confiscated. The police did not even spare Vallabhbhai Patel's eighty-year-old mother, who sat cooking in her village house in Karamsad; her cooking utensils were kicked about and filled with kerosene and stone. Vallabhbhai, on his brief sojourns out of jail throughout 1930, continued to provide encouragement and solace to the hard-pressed peasants of his native land. Though their meagre resources were soon exhausted, and weariness set in, they stuck it out in the wilderness till the truce in March 1931 made it possible for them to return to their homes. 

Defiance of forest Jaws assumed a mass character in Maharashtra, Karnataka and the Central Provinces, especially in areas with large tribal populations who had been the most seriously affected by the colonial Government's restrictions on the use of the forest. At some places the size of the crowd that broke the forest laws swelled to 70,000 and above. In Assam, a powerful agitation led by students was launched against the infamous `Cunningham circular' which forced students and their guardians to furnish assurances of good behaviour. 

The people seemed to have taken to heart Jawaharlal Nehru's message when he unfurled the national flag at Lahore in December 1929: `Remember once again, now that this flag is unfurled, it must not be lowered as long as a single Indian, man, woman, or child lives in India.'' Attempts to defend the honour of the national flag in the face of severe brutalities often turned into heroism of the most spectacular variety. At Bundur, on the Andhra Coast, Tota Narasaiah Naidu preferred to be beaten unconscious by a fifteen-member police force rather than give up the .national flag. In Calicut, P. Krishna Pillai, who later became a major Communist leader, suffered lathi blows with the same determination. In Surat, a group of children used their ingenuity to defy the police. Frustrated by the repeated snatching of the national flag from their hands, they came up with the idea of stitching khadi dresses in the three colours of the national flag, and thereafter these little, `living flags' triumphantly paraded the streets and defied the police to take away the national flag!'6 The national flag, the symbol of the new spirit, now became a common sight even in remote villages. U.P. was the setting of another kind of movement — a no- revenue, no-rent campaign. The no-revenue part was a call to the zamindars to refuse to pay revenue to the Government, the no- rent a call to the tenants not to pay rent to the zamindars. In effect, since the zamindars were largely loyal to the Government, this became a no-rent struggle. The civil Disobedience Movement had taken a firm hold in the province iii the initial months, but repression had led to a relative quiet, and though no- rent was in the air, it was only in October that activity picked up again when Jawaharlal Nehru, out of jail for a brief period, got the U.P. Congress Committee to sanction the no-rent campaign. Two months of preparation and intensive propaganda led to the launching of the campaign in December; by January, severe repression had forced many peasants to flee the villages. Among the important centres of this campaign were the districts of Agra and Rae Bareli. 

The movement also popularized a variety of forms of mobilization. Prabhatpheris, in which bands of men, women and children went around at dawn singing nationalist songs, became the rule in villages and towns. Patrikas, or illegal news-sheets, sometimes written by hand and sometimes cyclostyled, were part of the strategy to defy the hated Press Act, and they flooded the country. Magic lanterns were used to take the nationalist message to the villages. And, as before, incessant tours by individual leaders and workers, and by groups of men and women, and the holding of public meetings, big and small, remained the staple of the movement. Children were organized into vanar senas or monkey armies and at least at one place the girls decided they wanted their own separate manjari sena or cat army!

\begin{center}*\end{center}

\paragraph*{}


The Government's attitude throughout 1930 was marked by ambivalence. Gandhiji's arrest itself had come after much vacillation. After that, ordinances curbing the civil liberties of the people were freely issued and provincial governments were given the freedom to ban civil disobedience organizations. But the Congress Working Committee was not declared unlawful till the end of June and Motilal Nehru, who was functioning as the Congress President, also remained free till that date. Many local Congress Committees were not banned till August. Meanwhile, the publication of the report of the Simon Commission, which contained no mention of Dominion Status and was in other ways also a regressive document, combined with the repressive policy, further upset even moderate political opinion. Madan Mohan Malaviya and M.S. Aney courted arrest. In a conciliatory gesture, the Viceroy on 9 July suggested a Round Table Conference and reiterated the goal of Dominion Status. He also accepted the suggestion, made by forty members of the Central Legislature, that Tej Bahadur Sapru and M.R. Jayakar be allowed to explore the possibilities of peace between the Congress and the Government. In pursuance of this, the Nehrus, father and son, were taken in August to Yeravada jail to meet Gandhiji and discuss the possibilities of a settlement. Nothing came of the talks, but the gesture did ensure that some sections of political opinion would attend the Round Table Conference in London in November. The proceedings in London, the first ever conducted between the British and Indians as equals, at which virtually every delegate reiterated that a constitutional discussion to which the Congress was not a party was a meaningless exercise, made it clear that if the Government's strategy of survival was to be based on constitutional advance, then an olive branch to the Congress was imperative. The 1ritish Prime Minister hinted this possibility in his statement at the conclusion of the Round Table Conference. He also expressed the hope that the Congress would participate in the next round of deliberations to be held later in the year. On 25 January, the Viceroy announced the unconditional release of Gandhiji and all the other members of the Congress Working Committee, so that might be to respond to the Prime Minister's statement `freely and fearlessly.' 

After deliberating amongst itself for close to three weeks, and after long discussions with delegates who had returned from London, and with other leaders representing a cross-section of political opinion, the Congress Working Committee authorized Gandhiji to initiate discussions with the Viceroy. The fortnight- long discussions culminated on 5 March 1931 in the Gandhi- Irwin Pact, which was variously described as a `truce' and a `provisional settlement.' 

The Pact was signed by Gandhiji on behalf of the Congress and by Lord Irwin on behalf of the Government, a procedure that was hardly popular with officialdom as it placed the Congress on an equal footing with the Government. The terms of the agreement included the immediate release of all political prisoners not convicted for violence, the remission of all fines not yet collected, the return of confiscated lands not yet sold to third parties, and lenient treatment for those government employees who had resigned. The Government also conceded the right to make salt for consumption to villages along the coast, as also the right to peaceful and non-aggressive picketing. The Congress demand for a public inquiry into police excesses was not accepted, but Gandhiji's insistent request for an inquiry was recorded in the agreement. The Congress, on its part, agreed to discontinue the Civil Disobedience Movement. It was also understood that the Congress would participate in the next Round Table Conference.

\begin{center}*\end{center}

\paragraph*{}


The terms on which the Pact was signed, its timing, the motives of Gandhiji in signing the Pact, his refusal to make the Pact conditional on the commutation of the death-sentences of Bhagat Singh and his comrades, (even though he had tried his best to persuade the Viceroy to do so), have generated considerable controversy and debate among contemporaries and historians alike. The Pact has been variously seen as a betrayal, as proof of the vacillating nature of the Indian bourgeoisie and of Gandhiji succumbing to bourgeois pressure. It has been cited as evidence of Gandhiji's and the Indian bourgeoisie's fear of the mass movement taking a radical turn; a betrayal of peasants' interests because it did not immediately restore confiscated land, already sold to a third party, and so on. 

However, as with arguments relating to the withdrawal of the Non Cooperation Movement in 1922 after Chauri Chaura, these perceptions are based on an understanding which fails to grasp the basic strategy and character of the Indian national movement. For one, this understanding ignores the fact which has been stressed earlier — that mass movements are necessarily short-lived they cannot go on for ever, the people's capacity to sacrifice, unlike that of the activists', is not endless. And signs of exhaustion there certainly were, in large and important sectors of the movement. In the towns, while the students and other young people still had energy to spare, shopkeepers and merchants were finding it difficult to bear any more losses and the support from these sections, so crucial in making the boycott a success, had begun to decline by September of 1930. In rural India as well, those areas that had begun their resistance early in the year were fairly quiet in the second half. Through sporadic incidents of resistance and attacks on and clashes with police continued, this was as true of Bengal and Bihar as it was of Andhra and Gujarat. Those areas like U.P., which began their no-rent campaigns only at the end of 1930, still had more fight left in them, but the few instances of militant resistance that carried on and the ability of one or two regions to sustain activity can hardly be cited as proof of the existence of vast reserves of energy all over the country. And what was the guarantee that when those reserves were exhausted, as they were bound to be sooner rather than later, the Government would still be willing to talk? 1931 was not 1946; and as 1932 was to show, the Government could change tack and suppress with a ferocity that could effectively crush the movement. No doubt the youth were disappointed, for they would have preferred their world to end with a bang' rather than with a whimper' and surely the peasants of Gujarat were not happy that some of their lands did not come back to them immediately (they were returned after the Congress Ministry assumed office in Bombay in 1937). But the vast mass of the people were undoubtedly impressed that the mighty British Government had had to treat their movement and their leader as an equal and sign a pact with him. They saw this as a recognition of their own strength, and as their victory over the Govemment.ihe thousands who flocked out of the jails as a result of the pact were treated as soldiers returning from a victorious battle and not as prisoners of war returning from a humiliating defeat. They knew that a truce was not a surrender, and that the battle could be joined again, if the enemy so wanted. Meanwhile, their soldiers could rest and they could all prepare for the next round: they retained their faith in their General, and in themselves.

\begin{center}*\end{center}

\paragraph*{}


The Civil Disobedience Movement of 1930-31, then, marked a critically important stage in the progress of the anti-imperialist struggle. The number of people who went to jail was estimated at over 90,000 — more than three times the figure for the Non- Cooperation Movement of 1920- 22. Imports of cloth from Britain had fallen by half; other imports like cigarettes had suffered a similar fate. Government income from liquor excise and land revenue had been affected. Elections to the Legislative Assembly had been effectively boycotted. A vast variety of social groups had been politicized on the side of Indian nationalism — if urban elements like merchants and shopkeepers and students were more active in Tamil Nadu and Punjab, and in cities in general, peasants had come to the forefront in Gujarat, U.P., Bengal, Andhra, and Bihar, and tribals in the Central Provinces, Maharashtra, Karnataka and Bengal. Workers had not been missing from the battle either — they joined numerous mass demonstrations in Bombay, Calcutta, and Madras and were in the forefront in Sholapur. 

The participation of Muslims in the Civil Disobedience Movement was certainly nowhere near that in 1920-22. The appeals of communal leaders to stay away, combined with active Government encouragement of communal dissension to counter the forces of nationalism, had their effect. Still, the participation of Muslims was not insignificant, either. Their participation in the North-West Frontier Province was, as is well known, overwhelming. In Bengal, middle class Muslim participation was quite important in Senhatta, Tripura, Gaibandha, Bagura and Noakhali, and. in Dacca, Muslim students and shopkeepers as well as people belonging to the lower classes extended support to the movement. Middle and upper class Muslim women were also active.' The Muslim weaving community in Bihar and in Delhi and Lucknow the lower classes of Muslims were effectively mobilized as were many others in different parts of the country. 

The support that the movement had garnered from the poor and the illiterate, both in the town and in the country, was remarkable indeed. Their participation was reflected even in the government statistics of jail goers — and jail-going was only one of the many forms of participation. The Inspector-General of Police in Bengal, E.J. Lowman, expressed the general official bewilderment when he noted: `I had no idea that the Congress organization could enlist the sympathy and support of such ignorant and uncultivated people... For Indian women, the movement was the most liberating experience to date and can truly be said to have marked their entry into the public space.
