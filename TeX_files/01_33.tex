\chapter{Jinnah, Golwalkar and Extreme Communalism}

Communalism remained at the second, liberal stage till 1937 when it increasingly started assuming a virulent, extremist or fascist form. The liberal communalist argued that India consisted of distinct religion-based communities which had their own separate and special interests which often came into mutual conflict. But he also accepted that the ultimate destiny of Indian politics was the merger of the different communities into a single nation: Thus, the liberal communalist demanded separate communal rights, safeguards, reservations, etc., within the broad concept of one Indian nation-in-the-making. He accepted national unity as the ultimate goal as also the concept of the ultimate common interests of Hindus, Muslims, Sikhs and Christians. Liberal communalism had also a rather narrow social base. Politically, it was based mainly on the upper and middle classes. 

Extreme communalism was based on the politics of hatred, fear psychosis and irrationality. The motifs of domination and suppression, always present in communal propaganda as we have shown earlier, increasingly became the dominant theme of communal propaganda. A campaign of hatred against the followers of other religions was unleashed. The interests of Hindus and Muslims were now declared to be permanently in conflict. The communalists attacked the other `communities' with, in W.C. Smith's words, `fervour, fear, contempt and bitter hatred,' in the extremist or fascist phase of communalism after 1937. Phrases like oppression, suppression, domination, being crushed, even physical extermination and extinction were used. The communalists increasingly operated on the principle: the bigger the lie the better. They poured venom on the National Congress and Gandhiji, and, in particular, they viciously attacked their co-religionists among the nationalists. 

Communalism also now, after 1937, increasingly acquired a popular base, and began to mobilize popular mass opinion. It was now sought to be organized as a mass movement around aggressive, extremist communal politics among the urban lower middle classes. This also required an issue or a slogan which could arouse mass emotion. Because of the reactionary, upper class base of communalism, an appeal to radical social issues could not be made. In other words, communalism could not base itself on a radical socio-economic, or political or ideological programme. Hence, inevitably, an appeal was made to religion and to irrational sentiments of fear and hatred. 

Liberal communalism was transformed into extremist communalism for several reasons. As a consequence of the growth of nationalism and in particular, of the Civil Disobedience Movement of 1930-34, the Congress emerged as the dominant political force in the elections of 1937. Various political parties of landlords and other vested interests suffered a drastic decline. Moreover, as we have seen, the youth as also the workers and peasants were increasingly turning to the Left, and the national movement as a whole was getting increasingly radicalized in its economic and political programme and policies. The zamindars and landlords — the jagirdari elements — finding that open defence of landlords' interests was no longer feasible, now, by and large, switched over to communalism for their class defence. This was not only true in U.P. and Bihar but also in Punjab and Bengal. In Punjab, for example, the big landlords of West Punjab and the Muslim bureaucratic elite had supported the semi- communal, semi-casteist and loyalist Unionist Party. But they increasingly felt that the Unionist Party, being a provincial party, could no longer protect them from Congress radicalism, and so, during the years 1937-45, they gradually shifted their support to the Mus1im League which eagerly promised to protect their interests. Very similar was the case of Muslim zamindars and jotedars in Bengal. Hindu zamindars and landlords and merchants and moneylenders in northern and western India too began to shift towards Hindu communal parties and groups. To attract them, V.D. Savarkar, the Hindu Mahasabha President, began to condemn the `selfish' class tussle between landlords and tenants. Similarly, in Punjab, the Hindu communalists became even more active than before in defending money lending and trading interests. 

Communalism also became, after 1937, the only political recourse of colonial authorities and their policy of divide and rule. This was because by this time, nearly all the other divisions, antagonisms and divisive devices promoted and fostered earlier by the colonial authorities had been overcome by the national movement, and had become politically non-viable from the colonial point of view. The Non-Brahmin challenge in Maharashtra and South India had fizzled out. The Scheduled Castes and other backward castes could no longer be mobilized against the Congress except in stray pockets. The Right and Left wings of the Congress also refused to split. Inter-provincial and inter-lingual rivalries had exhausted themselves much earlier, after the Congress accepted the validity of linguistic states and the cultural diversity of the Indian people. The effort to pit the zamindars and landlords against the national movement had also completely failed. The elections of 1937 showed that nearly all the major social and political groups of colonialism lay shattered. The communal card alone was available for playing against the national movement and the rulers decided to use it to the limit, to stake all on it. They threw all the weight of the colonial state behind Muslim communalism, even though it was headed by a man — M.A. Jinnah — whom they disliked and feared for his sturdy independence and outspoken anti-colonialism. 

The outbreak of World War II, on 1 September, 1939 further strengthened the reliance on the communal card. The Congress withdrew its ministries and demanded that the British make a declaration that India would get complete freedom after the War and transfer of effective Government power immediately. For countering the nationalist demand and dividing Indian opinion, reliance was placed on the Muslim League whose politics and demands were counterposed to the nationalist politics and demands. The League was recognized as the sole spokesperson for Muslims and given the power to veto any political settlement. India could not be given freedom, it was said, so long as Hindus and Muslims did not unite. But such unity was made impossible by the wholesale official backing of Muslim communalism. The Muslim League, in turn, agreed to collaborate with the colonial authorities and serve as their political instrument of its own reasons. The Hindu Mahasabha and other Hindu and Sikh communal organizations also offered to support the colonial Government during the War. But the colonial authorities, while accepting their support, could no longer divide their loyalties; their commitment to Muslim communalism was to remain total during the course of the Wax, and even after. Both the Muslim League and the Hindu Mahasabha had run the election campaign of 1937 on liberal communal lines — they had incorporated much of the nationalist programme and many of the Congress policies, except those relating to agrarian issues, in their election manifestoes. But they had fared poorly in the elections. The Muslim League, for example, won only 109 out of the 482 seats allotted to Muslims under separate electorates, securing only 4.8 per cent of the total Muslim votes. The Hindu Mahasabha fared even worse. 

The communalists now realized that they would gradually wither away if they did not take to militant, mass-based politics. Hitherto, organized mass movements and cadre-based politics had been built by radical, anti-status quo nationalists. The conservatives had shied away from mass movements. In the 1930s, a successful right-wing model of mass politics, which would not frighten away the vested interests, became available in the form of the fascist movement. Both Hindu and Muslim communalists decided to follow this model. Moreover, the Congress had not yet acquired firm roots among all the masses, especially among the Muslim masses; now was the time to take advantage of their political immaturity, before it was too late. Urgency was added to the need to shift to extreme Muslim communalism because the Congress decided to initiate, under Jawaharlal Nehru's guidance, a massive campaign to work among the Muslim masses, known as the Muslim Mass Contact Programme. 

The logic of communalism also inexorably led to extreme communalism. The Congress had gone quite far in the late 1920s in accepting Muslim communal demands. In 1932, the Communal Award and then the Government of India Act of 1935 accepted nearly all the liberal communal demands. Nor did the National Congress oppose these concessions to the communalists. But such concessions would have no cast iron guarantee behind them once the foreign rulers disappeared from the scene and the country came to be ruled democratically. Moreover, what would the communalists do next? Since their demands had been accepted, they had either to dissolve their political organizations, give up communalism and commit political harakiri or discover new demands, new threats to their communities, and inexorably and without necessarily, a conscious design turn towards extreme communalism. Similarly, the Hindu communalists had failed to grow. Further, till 1937, the Congress had permitted both Hindu and Muslim liberal communalists to work within the Congress organization. Under Jawaharlal Nehru's and the Left's pressure the Congress was frontally attacking the communalists. Not only did it not accommodate them in the elections of 1934 and 1937, it moved towards expelling them from the Congress, and finally did so in 1938. The Hindu communalists were facing political extinction. They also had to find a new basis and a new programme for their survival and growth. 

The proposition that communalism has a logic of its own and, if not checked in its early stages, inevitably develops into its `higher' stages is illustrated by the life history of Mohammed All Jinnah. His case shows how communalism is an inclined plane on which a constant slide down becomes inevitable unless counter steps are taken. Once the basic digits of communal ideology are accepted, the ideology takes over a person bit by bit, independent of the subjective desires of the person. This is how a person who started as the `Ambassador of Hindu-Muslim Unity' ended up by demanding Pakistan. 

M.A. Jinnah came back to India after becoming a Barrister in 1906 as a secular, liberal nationalist, a follower of Dadabhai Naoroji. On his return, he immediately joined the Congress and acted as Dadabhai's secretary at the Calcutta session of the Congress in 1906 He was an opponent of the Muslim League then being founded. The Aga Khan, the first president of the League, was to write later that Jinnah was `our toughest opponent in 1906' and that he `came out in bitter hostility toward all that I and my friends had done and were trying to do.. . He said that our principle of separate electorates was dividing the nation against itself.'' From 1906 onwards, Jinnah propagated the theme of national unity in the meetings that he addressed, earning from Sarojini Naidu the title `Ambassador of Hindu- Muslim Unity.' 

The first step towards communalism was taken without any desire of his own and perhaps against his own wishes when he entered the Central Legislative Council from Bombay as a Muslim member under the system of separate electorates. The real slide down began when from a nationalist mire and simple he became a communal nationalist in 1913 when he joined the Muslim League. This, of course, meant that he was still basically a nationalist. He remained in the Congress ad still opposed separate electorates arguing that it would divide India into `two watertight compartments.' But he also started assuming the role of a spokesperson of the Muslim `community' as a whole. These dual roles reached the height of their effectiveness in the Lucknow Congress-League Pact of which he and Tilak were the joint authors. Acting as the spokesperson of Muslim communalism, he got the Congress to accept separate electorates and the system of communal reservations. But he still remained fully committed to nationalism and secular politics. He resigned from the Legislative Council as a protest against the passing of the Rowlatt Bill. He refused the communal assumption that self- government in India would lead to Hindu rule; and argued that the real political issue in India was Home Rule or `transfer of power from bureaucracy to democracy.' 

In 1919-20, the Congress took a turn towards mass politics based on the peaceful breaking of existing laws. Jinnah disagreed and did not find it possible to go along with Gandhi. Along with many other liberals, who thought like him — persons such as Surendranath Banerjea, Bipin Chandra Pal, Tej Bahadur Sapru, C. Sankaran Nair, and many more — Jinnah left the Congress. But he could also see that mere liberal politics had no future. And he was not willing to go into political oblivion. Unlike most of the other liberals, he turned to communal politics. He became a liberal communalist. The logic of communalism had asserted itself and transformed him first from a nationalist into communal nationalist and then into a liberal communalist. 

During the 1920s, Jinnah's nationalism was not fully swallowed by communalism. He revived the down-and-out Muslim League in 1924 and started building it upon and around the demand for safeguarding `the interests and rights of the Muslims.' His politics were now based on the basic communal idea that `Muslims should organize themselves, stand united and should press every reasonable point for the protection of their community.' At the same time, he still pleaded for Hindu-Muslim unity on the basis of a fresh Lucknow Pact so as to fight the British together, and he cooperated with the Swarajists in opposing Government policies and measures in the Central Legislative Assembly. As late as 1925, he told a young Muslim, who said that he was a Muslim first: `My boy, no, you are an Indian first and then a Muslim.' In 1927-28, he supported the boycott of the Simon Commission, though he would not join in the mass demonstrations against it. But by now his entire social base comprised communal- minded persons. He could not give up communalism without losing all political influence. This became apparent in 1928-29 during the discussions on the Nehru Report. Step by step he surrendered to the more reactionary communalists, led by the Aga Khan and M. Shafi, and in the end became the leader of Muslim communalism as a whole, losing in the bargain the support of nationalist leaders like MA. Ansari, T.A.K. Sherwani, Syed Mahmud and his own erstwhile lieutenants like M.C. Chagla. His slide down was symbolized by his becoming the author of the famous 14 demands incorporating the demands of the most reactionary and virulent sections of Muslim communalism. 

Jinnah was further alienated from the main currents of nationalism as the Congress organized the massive mass movement of 1930 and started moving towards a more radical socio-economic programme. Moreover, the Muslim masses especially the younger generation were increasingly shifting to nationalist and left-wing politics and ideologies. Jinnah was faced with a dilemma. He saw little light; and decided to stay mostly in Britain. 

But Jinnah was too much of a man of action and of politics to stay there. He returned to India in 1936 to once again revive the Muslim League. He initially wanted to do so on the basis of liberal communalism. Throughout 1936, he stressed his nationalism and desire for freedom and spoke for Hindu-Muslim cooperation. For example, he said at Lahore in March 1936: `Whatever I have done. let me assure you there has been no change in me, not the slightest, since the day when I joined the 

Indian National Congress. It may be I have been wrong on some occasions. But it has never been done in a partisan spirit. My sole and only object has been the welfare of my country. I assure you that India's interest is and will be sacred to me and nothing will make me budge an inch from that position.' On the one hand, he asked Muslims to organize separately, on the other hand, he asked them to `prove that their patriotism is unsullied and that their love of India and her progress is no less than that of any other community in the country.' 

Jinnah's plan perhaps was to use the Muslim League to win enough seats to force another Lucknow Pact on the Congress. He also assumed that by participating in the 1937 elections the Congress was reverting to pre-Gandhian constitutional politics. Partially because of these assumptions and partially because the bag of communal demands was empty — nearly all the communal demands having been accepted by the Communal Award .Jinnah and the League fought elections on a semi-nationahst Congress-type of programme, the only `Muslim' demands being protection and promotion of the Urdu language and script, and adoption of measures for the amelioration of the general conditions of Muslims. 

But the poor election results showed that none of Jinnah's assumptions were correct. Jinnah had now to decide what to do: to stick to his semi- nationalist, liberal communal politics which seemed to have exhausted its potentialities or to abandon communal politics. Both would mean going into political wilderness. The third alternative was to take to mass politics which in view of the semi-feudal and semi-loyalist social base of the League and his own socially, economically, and politically conservative views could only be based on the cries of Islam in danger and the danger of a Hindu raj. Jinnah decided in 1937-38 to opt for his last option. And once he took this decision he went all the way towards extreme communalism putting all the force arid brilliance of his personality behind the new politics based on themes of hate and fear. From now on, the entire political campaign among Muslims of this tallest of communal leaders would be geared to appeal to his co-religionists' fear and insecurity and to drive home the theme that the Congress wanted not independence from British imperialism but a Hindu raj in cooperation with the British and domination over Muslims and even their extermination as also the destruction of Islam in India. 

Let us take a few examples. In his presidential address to the League in 1938, Jinnah said: `The High Command of the Congress is determined, absolutely determined to crush all other communities and cultures in this country and establish Hindu raj in this country.' In March 1940, he told the students at Aligarh: `Mr Gandhi's hope is to subjugate and vassalize the Muslims under a Hindu raj.'' Again at Aligarh in March 1941: `Pakistan is not only a practicable goal but the only goal if you wan to save Islam from complete annihilation in this country.'' In his presidential address on April 1941, Jinnah declared that in a united India `the Muslims will be absolutely wiped out of existence.''° Regarding the interim government in 1946, on 18 August, Jinnah referred to `the caste Hindu Fascist Congress,' which wanted to `dominate and rule over Mussalmans and other minor communities of India with the aid of British bayonets.' In 1946, asking Muslims to vote for the League he said: `If we fail to realize our duty today you will be reduced to the status of Sudras and Islam will be vanquished from India.'' 

If a leader of the stature of Jinnah could take up politics and agitation at this low level, it was inevitable that the average communal propagandist would be often even worse. Men like Z.A. Suleri and F.M. Durrani surpassed themselves in Goebbelsian demagogy.' Even Fazl-ul-Huq, holding a responsible position as the Premier of Bengal, told the 1938 session of the League: `In Congress provinces, riots had laid the countryside waste. Muslim life, limb and property have been lost and blood had freely flowed... There the Muslims are leading their lives in constant terror, overawed and oppressed by Hindus.. . There mosques are being defiled and the culprit never found nor is the Muslim worshipper unmolested.'' M.H. Gazdar, a prominent League leader of Sind, told a League meeting in Karachi in March 1941: `The Hindus will have to be eradicated like the Jews in Germany if they did not behave properly.'' Jinnah was however in no position to pull up such people, for his own speeches often skirted the same territory. 

The Muslim communalists now launched a vicious campaign against nationalist Muslims. Maulana Abul Kalam 

Azad and other nationalist Muslims were branded as `show boys' of the Congress, traitors to Islam and mercenary agents of the Hindus. They were submitted, during 1945- 47, to social terror through appeals to religious fanaticism and even to physical attacks. Jinnah himself in his presidential address to the League in April 1943 described Khan Abdul Ghaffar Khan as being `in- charge of the Hinduizing influences and emasculation of the martial Pathans.'' 

Religion was also now brought into the forefront of propaganda. In 1946, Muslims were asked to vote for the League because `a vote for the League and Pakistan was a vote for Islam.' League meetings were often held in the mosques after Friday prayers. Pakistan, it was promised, would be ruled under the Sharia. Muslims were asked to choose between a mosque and a temple. The Quran was widely used as the League's symbol; and the League's fight with the Congress was portrayed as a fight between Islam and Kufr (infidelity). 

Hindu communalism did not lag behind. Its political trajectory was of course different. The two main liberal communal leaders during the 1920s were Lajpat Rai and Madan Mohan Malaviya. Lajpat Rai died in 1928 and Malaviya, finding himself in 1937 in the sort of situation in which Jinnah found himself in the same year, decided to retire from active politics, partly on grounds of health. But Hindu communalism would also not commit suicide; it too advanced to the extremist or the fascist phase. The logic of communalism brought other communal leaders to the fore. The Hindu Mahasabha made a sharp turn in the fascist direction under V.D. Savarkar's leadership. The RSS (Rashtriya Swayamsevak Sangh) had been from the very beginning organized on fascist lines; it now began to branch out beyond Maharashtra. 

Year after year, V.D. Savarkar warned Hindus of the dangers of being dominated by Muslims. He said in 1937 that Muslims `want to brand the forehead of Hindudom and other non-Muslim sections in Hindustan with a stamp of self- humiliation and Muslim domination' and `to reduce the Hindus to the position of helots in their own lands.'' In 1938, he said that `we Hindus are (already) reduced to be veritable helots throughout our land.' 

It was, however, the RSS which became the chief ideologue and propagator of extreme communalism. The head of the RSS, 

M.S. Golwalkar, codified the RSS doctrines in his booklet, We. In 1939, he declared that if the minority demands were accepted, `Hindu National life runs the risk of being shattered.''9 Above all, the RSS attacked Muslims and the Congress leaders. Golwalkar attacked the nationalists for `hugging to our bosom our most inveterate enemies (Muslims) and thus endangering our very existence.'20 Condemning the nationalists for spreading the view by which Hindus `began to class ourselves with our old invaders and foes under the outlandish name — Indian,' he wrote: `We have allowed ourselves to be duped into believing our foes to be our friends ... That is the real danger of the day, our self- forgetfulness, our believing our old and bitter enemies to be our friends.' To Muslims and other religious minorities, Golwalkar gave the following advice: `The non-Hindu peoples in Hindustan must either adopt the Hindu culture and language, must learn o respect and hold in reverence Hindu religion, must entertain no ideas but those of glorification of the Hindu race and culture, i.e., they must not only give up their attitude of intolerance and ungratefulness towards this land and its age long traditions but must also cultivate the positive attitude of love and devotion instead — in one word, they must cease to be foreigners, or may stay in the country, wholly subordinated to the Hindu nation, claiming nothing, deserving no privileges, far less any preferential treatment — not even citizen's rights.' Going further, he wrote: `We Hindus are at war at once with the Muslims on the one hand and British on the other.' He said that Italy and Germany were two countries where `the ancient Race spirit' had `re-risen.' `Even so with us: our Race spirit has once again roused itself,' thus giving Hindus the right of excommunicating Muslims. The RSS launched an even more vicious attack on the Congress leaders during 1946-47. Provocatively accusing the Congress leaders in the true fascist style of asking Hindus to `submit meekly to the vandalism and atrocities of the Muslims' and of telling the Hindu `that he was imbecile, that he had no spirit, no stamina to stand on his own legs and fight for the independence of his motherland and that all this had to be injected into him in the form of Muslim blood', he said in 1947, pointing his finger at Gandhiji: `Those who declared ``No Swaraj without Hindu-Muslim unity'' have thus perpetrated the greatest treason on our society. They have committed the most heinous sin of killing the life-spirit of a great and ancient people.' He accused Gandhiji of having declared: ```There is no Swaraj without Hindu-Muslim unity and the simplest way in which this unity can be achieved is for all the Hindus to become Muslims.'' 

The Hindu communalists also tried to raise the cries of `Hinduism in danger,' `Hindu faith in danger,' and `Hindu culture or sanskriti in danger.' 

The bitter harvest of this campaign of fear and hatred carried on by the Hindu and Muslim communalists since the end of the 19th century, and in particular after 1937, was reaped by the people in the Calcutta killings of August 1946 in which over 5,000 lost their lives within five days, in the butchery of Hindus at Noakhali in Bengal and of Muslims in Bihar, the carnage of the partition riots and the assassination of Gandhiji by a communal fanatic. 

But, perhaps, the heaviest cost was paid by Muslims who remained in or migrated to Pakistan. Once Pakistan was formed, Jinnah hoped to go back to liberal communalism or even secularism. Addressing the people of Pakistan, Jinnah said in his Presidential address to the Constituent Assembly of Pakistan on 11 August 1947: `You may belong to any religion or caste or creed — that has nothing to do with the business of the State... We are starting with this fundamental principle that we are all citizens and equal citizens of one State... Now, I think we should keep that in front of us as our ideal, and you will find that in course of time Hindus would cease to be Hindus and Muslims would cease to be Muslims, not in the religious sense, because that is the personal faith of each individual, but in the political sense as citizens of the State.' But it was all too late. Jinnah had cynically spawned a monster which not only divided India, but would, in time, eat up his own concept of Pakistan and do more harm to Muslims of Pakistan than the most secular of persons could have predicted or even imagined. On the other hand, despite the formation of Pakistan and the bloody communal riots of 1947, nationalist India did succeed in framing a secular constitution and building a basically secular polity, whatever its weaknesses in this respect may be. In other words, ideologies have consequences. 

Two major controversies have arisen in the last thirty years or so around the communal problem. One is the view that the communal problem would have disappeared or been solved if Jinnah had been conciliated during 1937-39 and, in particular, if a coalition government with the Muslim League had been formed in U.P. in 1937. The rebuff to Jinnah's political ambitions, it is said, embittered him and made him turn to separatism. Let us first look at the general argument. It entirely ignores the fact that before he was `rebuffed' Jinnah was already a full- fledged liberal communal 1st. Second, every effort was made by the Congress leaders from 1937 to 1939 to negotiate with Jinnah and to conciliate him. But Jinnah was caught in the logic of communalism. He was left without any negotiable demands which could be rationally put forward and argued. Consequently, and it is very important to remember this historical fact, he refused to tell the Congress leaders what the demands were whose acceptance would satisfy him and lead him to join the Congress in facing imperialism. The impossible condition he laid down to even start negotiations was that the Congress leadership should first renounce its secular character and declare itself a Hindu communal body and accept the Muslim League as the sole representative of the Muslims. The Congress could not have accepted this demand. As Rajendra Prasad put it, for the Congress to accept that it was a Hindu body `would be denying its own past, falsifying its history, and betraying its future' — in fact, it would be betraying the Indian people and their future. If the Congress had accepted Jinnah's demand and `conciliated' him, we might well have been living under a Hindu replica of Pakistan or a Hindu fascist state. So no serious negotiations could even begin. 

Jinnah, too, all the while, was following the logic of his ideology and politics. But this posture could also not be maintained for long. The motive towards Pakistan was then inevitable, for separatism was the only part of the communal ideological programme left unfulfilled. The alternative was to abandon communal politics. And so Jinnah and the Muslim League took the ultimate step in early 1940 and, basing themselves on the theory that Hindus and Muslims were two separate nations which must have separate homelands, put forward the demand for Pakistan. Hindu communalism too had moved in the same direction. Its separatism could not take the form of demanding a part of India as Hindustan — that would be playing into the hands of Muslim communalism. It, therefore, increasingly asserted that Hindus were the only nation living in India and Muslims should either be expelled from India or live in it as second-class citizens. 

Something similar was involved in the U.P. decision of 1937. Jinnah and the League were firmly opposed to mass politics. To have joined hands with them would have meant retreating to constitutional politics in which people had little role to play. Much before the ministerial negotiations occurred or broke down, Jinnah had declared Muslims to be a distinct third party in India, as distinguished from the British and Indian nationalism represented by the Congress. As S. Gopal has put it: `Any coalition with the League implied the Congress accepting a Hindu orientation and renouncing the right to speak for all Indians.'28 It would have also meant the betrayal of nationalist Muslims, who had firmly taken their stand on the terrain of secular nationalism. Furthermore, it would have meant abandonment of the radical agrarian programme adopted at Faizpur in 1936 to which the Congress Ministry was fully committed, for the League was equally committed to the landlords' interests. With their representatives in the Government, no pro-peasant legislation could possibly have been passed. In fact, it was the Congress Socialists and the Communists, quite important in the U.P. Congress at the time, who put pressure on Nehru to reject any coalition with the League and threatened to launch a public campaign on the issue if their demand was rejected. Interestingly, even before negotiations for the formation of a Congress Ministry in U.P. had begun, the Muslim League had raised the cry of `Islam in Danger' in its campaign against Congress candidates in the by-elections to U.P. assembly during May 1937. Jinnah himself had issued appeals to voters in the name of Allah and the Quran. 

In any case, if a leader could turn into a vicious communalist and separatist because his party was not given two seats in a provincial ministry, then how long could he have remained conciliated? To argue in this fashion is, perhaps, to treat history and politics as a joke or as the play of individual whims. The fact is that communalism is basically an ideology which could not have been, and cannot be, appeased; it had to be confronted and opposed, as we have brought out earlier. The failure to do so was the real weakness of the Congress and the national movement. Interestingly, the Communists did try to appease the Muslim League from 1942 to 1946, hoping to wean away its better elements. They not only failed but in the bargain lost some of their best cadres to Muslim communalism. The effort to have a coalition with it turned out to be a one way street from which the Communists had the wisdom to withdraw in 1946. In fact, the negotiations by the Congress leaders as also the Left were based on the false assumption that liberal communalists could be conciliated and then persuaded to fight extreme communalism which was anti-national. After 1937 it was only the nationalist Hindus and Muslims who firmly opposed communalism. Liberal communalists like Malaviya, Shyama Prasad Mukherji and N.C. Chatterjea failed to oppose Savarkar or the RSS. Similarly, the liberal Iqbal or other liberal communal Muslims did not have the courage to oppose the campaign of hatred that Jinnah, Suleri, Fazl-ul-Huq and others unleashed after 1937. At the most, they kept quiet where they did not join it. 

It is also not true that the Congress failure regarding communalism occurred in 1947 when it accepted the partition of the country. Perhaps, there was no other option at the time. Communalism had already advanced too far. There was, it can be argued, no other solution to the communal problem left, unless the national leadership was willing to see the nation plunged in a civil war when the armed forces and the po1ice were under the control of the foreign rulers and were themselves ready to join the civil war. 

The fact is that not all historical situations have an instant solution. Certainly, no such solution existed in 1947. There is never an instant solution to a socio-political problem like communalism. Conditions and forces for a solution have to be prepared over a number of years and even decades. This the Congress and the national movement failed to do. Despite their commitment to secularism, despite Gandhiji's constant emphasis on Hindu-Muslim unity and his willingness to stake his life for its promotion, and despite Nehru's brilliant analysis of the socio-economic roots of communalism, the Indian nationalists failed to wage a mass ideological-political struggle against all forms of communalism on the basis of patient and scientific exposure of its ideological content, socio-economic roots, and political consequences. In fact, the Congress relied too heavily on negotiations with the communal leaders and failed to evolve a viable and effective long-term strategy to combat communalism at the political, ideological and cultural levels. The Congress and its leadership have to be faulted on this count.
