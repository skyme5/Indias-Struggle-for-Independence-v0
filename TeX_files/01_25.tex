\chapter{The Strategic Debate 1935--37}\label{chapter:CH25}

A major debate on strategy occurred among the nationalists in the period following the withdrawal of the Civil Disobedience Movement. In the first stage of the debate, during 1934--35, the issue was what course the national movement should take in the immediate future, that is, during its phase of non-mass struggle. How was the political paralysis that it had sunk into to be overcome? There were two traditional responses. Gandhiji emphasized constructive work in the villages, especially the revival of village crafts. Constructive work, said Gandhiji, would lead to the consolidation of people's power, and open the way to the mobilization of millions in the next phase of mass struggle.' 

Another section of Congressmen advocated the revival of the constitutional method of struggle and participation in the elections to the Central Legislative Assembly to be held in 1934. Led this time by Dr. M.A. Ansari, Asaf Ali, Satyamurthy, Bhulabhai Desai and B.C. Roy, the new Swarajists argued that in a period of political apathy and depression, when the Congress was no longer in a position to sustain a mass movement, it was necessary to utilize elections and work in the legislative councils to keep up the political interest and morale of the people. This did not amount, they said, to having faith in the capacity of constitutional politics to achieve freedom. It only meant opening up another political front which would help build up the Congress, organizationally extend its influence, and prepare the people for the next mass struggle. C. Rajagopalachari, an erstwhile no-changer, recommended the Swarajist approach to Gandhiji with the additional proviso that the Congress should itself, directly, undertake parliamentary work. A properly organized parliamentary party, he said, would enable the Congress to develop a certain amount of prestige and confidence among the masses even as (happened) during the short period when the Gandhi-Irwin Pact was in force. Since the Government was opposed to a similar pact, a strong Congress presence in the legislatures would serve the movement as `its equivalent.']

\begin{center}*\end{center}

\paragraph*{}


But unlike in the 1920s, a third tactical perspective, based on an alternative strategy, made its appearance at this time. The strong Left trend that had developed in the early l930s was critical of both the council-entry programme and the suspension of civil disobedience and its replacement b the constructive programme. Both of them, the leftists said, would sidetrack direct mass action and political work among the masses and divert attention from the basic issue of struggle against colonial rule. The leftists instead favoured the continuation or resumption of the non- constitutional mass movement since they felt that the situation continued to be revolutionary because of the continuing economic crisis and the readiness of the masses to fight. 

It was Jawaharlal Nehru who represented at this time at its most cogent and coherent this New Leftist alternative to the Gandhian anti- imperialist programme and strategy. Accepting the basic analytical framework of Marxism, Nehru put forward the Left paradigm in a series of speeches, letters, articles and books and his Presidential addresses to the Lucknow and Faizpur sessions of the Congress in 1936. The basic goal before the Indian people, as also before the people of the world, he said, had to be the abolition of capitalism and the establishment of socialism. While we've already looked at the pragmatic aspect of Nehru's challenge two of its other aspects have to be understood. 

To Nehru, the withdrawal of the Civil Disobedience Movement and council-entry and the recourse to constructive programmes represented a `spiritual defeat' and a surrender of ideals, a retreat from the revolutionary to the reformist mentality, and a going back to the pre-1919 moderate phase What was worse, it seemed that the Congress was giving up all social radicalism and `expressing a tender solicitude for every vested interest.' Many Congress leaders, he said, `preferred to break some people's hearts rather than touch others' pockets. Pockets are, indeed, more valuable and more cherished than hearts and brains and bodies and human justice and dignity.'' His alienation from Gandhiji also seemed to be complete. He wrote in his jail diary in April 1934: `Our objectives are different, our ideals are different, our spiritual outlook is different and our methods are likely to be different.' 

The way out, said Nehru, lay in grasping the class basis of society and the role of class struggle and in `revising vested interests in favour of the masses.' This meant taking up or encouraging the day-to-day class, economic demands of the peasants and workers against the landlords and capitalists, organizing the former in their class organizations --- kisan sabhas and trade unions --- and permitting them to affiliate with the Congress and, thus, influence and direct its policies and activities. There could be, said Nehru, no genuine anti-imperialist struggle which did not incorporate the class struggle of the masses. 

Throughout these years, Nehru pointed to the inadequacy of the existing nationalist ideology and stressed the need to inculcate a new, socialist or Marxist ideology, which would enable the people to study their social condition scientifically. Several chapters of his Autobiography, published in 1935, were an ideological polemic against Gandhiji even though conducted in a friendly tone. 

Jawaharlal also challenged the basic Gandhian strategy of struggle.4 Under the Gandhian strategy. which may be described as Struggle --- Truce --- Struggle (S-T-S'), phases of a vigorous extra-legal mass movement and confrontation with colonial authority alternate with phases, during which direct confrontation is withdrawn, political concessions or reforms, if any, wrested from the colonial regime, are willy-nilly worked and silent political work carried on among the masses within the existing legal framework, which, in turn, provides scope for such work. Both phases of the movement are to he utilized, each in its own way, to undermine the twin ideological notions on which the colonial regime rested --- that British rule benefits Indians and that it is too powerful to be challenged and overthrown and to recruit and train cadres and to build up the people's capacity to struggle. The entire political process of S-T-S' was an upward spiralling one, which also assumed that the freedom struggle would pass through several stages, ending with the transfer of power by the colonial regime itself. 

Nehru did not subscribe to this strategy and believed that, whatever might have been the case in the past, the Indian national movement had now reached a stage where there should be a permanent confrontation and conflict with imperialism till it was overthrown. He accepted that the struggle had to go through setbacks and phases of upswing and downswing; but these should not lead to a passive phase or a stage of compromise or `cooperation' with the colonial framework towards which permanent hostile and non-cooperation had to be maintained. The Congress, said Nehru, must maintain `an aggressive direct action policy.' This meant that even if the mass movement was at a low ebb or remained at a symbolic plane, it should be continued. There could be no interposition of a constitutional phase when the existing constitutional framework was worked; nor could there be a diversion from political and economic class issues to the constructive programme. Furthermore, said Nehru, every moment sooner or later reached a stage when it endangered the existing order. The struggle then became perpetual and could go forward only through unconstitutional and illegal means. This also happened when the masses entered politics. No compromise or half-way house was then left. This stage had been reached in India with the Lahore Resolution for Poorna Swaraj. There was now no alternative to permanent continuation of the struggle. For this reason, Nehru attacked all moves towards the withdrawal of the Civil Disobedience Movement. This would lead, he warned, to `some form of compromise with imperialism' which `would be a betrayal of the cause.' Hence, `the only way out is to struggle for freedom without compromise or going back or faltering.' Nehru also attacked the notion of winning freedom through stages. Real power could not be won gradually `bit by bit' or by `two annas and four annas.' `The citadel' --- State power --- had to be seized, though through a non-violent mass struggle. Thus, to S-T-S' he counterposed the strategy of S-V (`V' standing for victory) or the permanent waging of mass struggle till victory was won.

\begin{center}*\end{center}

\paragraph*{}


So sharp were the differences between Nehru and the leftists on the one side and proponents of council-entry on the other that many --- the nationalists with apprehension and the British officials with hope --- expected a split sooner or later. But Gandhiji once again moved into the breach and diffused the situation. Though believing that Satyagraha alone was capable of winning freedom, he conciliated the proponents of council- entry by acceding to their basic demand that they should be permitted to enter the legislatures. He also defended them from accusations of being lesser patriots Parliamentary politics, he said, could not lead to freedom but those large number of Congressmen who could not for some reason or the other offer Satyagraha or devote themselves to constructive work should not remain unoccupied. They could give expression to their patriotic energies through council work in a period when there was no mass movement, provided they were not sucked into constitutionalism or self- serving. As he put it in a letter to Sardar Patel on 23 April 1934: `Realities cannot be wished away. At the most we can improve them a little. We may exercise control. We can do neither more nor less.' 

Consequently, under Gandhiji's guidance, the AICC meeting at Patna decided in May 1934 to set up a parliamentary board to fight elections under the aegis of the Congress itself. To the Left- wing critics of the resolution, Gandhiji replied: `I hope that the majority will always remain untouched by the glamour of council work... Swaraj will never come that way. Swaraj can only come through an all-round consciousness of the masses.' 

At the same time, he assured Nehru and the leftists that the withdrawal of the civil disobedience was dictated by the reality of the political situation. But this did not mean following a policy of drift or bowing down before political opportunists or compromising with imperialism. Only civil disobedience had been discontinued, the war continued. The new policy, he said, `is founded upon one central idea --- that of consolidating the power of the people with a view to peaceful action.' Moreover, he told Nehru in August 1934: `1 fancy that I have the knack for knowing the need of the time.' He also appeased the Left by strongly backing Nehru for the Presidentship of the Lucknow Congress despite contrary pressure from C. Rajagopalachari and other right-wing leaders. 

Gandhiji was at the same time convinced that he was out of tune with powerful trends in the Congress. He felt that a large section of the intelligentsia favoured parliamentary politics with which he was in fundamental disagreement. Another section of the intelligentsia felt estranged from the Congress because of his emphasis on the spinning wheel as `the second lung of the nation,' on Harijan work based on a moral and religious approach, and on other items of the constructive programme. Similarly, the socialist group, whose leader was Jawaharlal, was growing in influence and importance but he had fundamental differences with it. Yet the Socialists felt constrained by the weight of his personality. As he put it: `But I would not, by reason of the moral pressure I may be able to exert, suppress the spread of the ideas propounded in their literature.' Thus, vis-a-vis both groups, `for me to dominate the Congress in spite of these fundamental differences is almost a species of violence which I must refrain from.' Hence, in October 1934, he announced his resignation from the Congress `only to serve it better in thought, word and deed. 

Nehru and the Socialists responded with no less a patriotic spirit. While enemies of the Congress hoped that their radicalism would lead to their breaking away from the Congress, they had their priorities clearly worked out. The British must first be expelled before the struggle for socialism could be waged. And in the anti-imperialist struggle, national unity around the Congress, still the only anti-imperialist mass organization, was indispensable. Even from the socialist point of view, argued Nehru and other leftists, it was far better to gradually radicalize the Congress, where millions upon millions of the people were, than to get isolated from these millions in the name of political or ideological purity. Nehru, for example, wrote: `I do not see why I should walk out of the Congress leaving the field clear to social reactionaries. Therefore, I think it is up to us to remain there and try to force the pace, thereby either converting others or making them depart.'' The Right was no less accommodating. C Rajagopalachari wrote: `The British, perhaps, hope for a quarrel among Congressmen over this (socialism). But we hope to disappoint them.'' 

Elections to the Central Legislative Assembly were held in November 1934. Of the seventy-five elected seats for Indians, the Congress captured forty-five. `Singularly unfortunate; a great triumph for little Gandhi,' wailed the Viceroy, Willingdon.'

\begin{center}*\end{center}

\paragraph*{}


Even though the Government had successfully suppressed the mass movement during 1932--33, it was aware that suppression could only be a short-term tactic. it could not prevent the resurgence of another powerful movement in the years to come. For that it was necessary to permanently weaken the movement. This could be achieved if the Congress was internally divided and large segments of it co-opted or integrated into the colonial constitutional and administrative structure. The phase of naked suppression should, therefore, be followed, decided the colonial policy makers, by another phase of constitutional reforms. 

In August 1935, the British Parliament passed the Government of India Act of 1935. The Act provided for the establishment of an All-India Federation to be based on the union of the British Indian provinces and Princely States. The representatives of the States to the federal legislature were to be appointed directly by the Princes who were to be used to check and counter the nationalists. The franchise was limited to about one-sixth of the adults. Defence and foreign affairs would remain outside the control of the federal legislature, while the Viceroy would retain special control over other subjects. 

The provinces were to be governed under a new system based on provincial autonomy under which elected ministers controlled all provincial departments. Once again, the Governors, appointed by the British Government, retained special powers. They could veto legislative and administrative measures, especially those concerning minorities, the rights of civil servants, law and order and British business interests. The Governor also had the power to take over and indefinitely run the administration of a province. Thus both political and economic power remained concentrated in British hands; colonialism remained intact. As Linlithgow, Chairman of the Joint Parliamentary Committee on the Act of 1935 and the Viceroy of India from 1936, stated later, the Act had been framed `because we thought that was the best way ... of maintaining British influence in India. It is no part of our policy, I take it, to expedite in India constitutional changes for their own sake, or gratuitously to hurry the handing over of the controls to Indian hands at any pace faster than that which we regard as best calculated, on a long view, to hold India to the Empire.'' 

The long-term strategy, followed by the British Government from 1935 to 1939, had several major components. Reforms, it was hoped, would revive the political standing of the Liberals and other moderates who believed in the constitutional path, and who had lost public favour during the Civil Disobedience Movement. Simultaneously, in view of the severe repression of the movement, large sections of Congressmen would be convinced of the ineffectiveness of extra-legal means and the efficacy of constitutionalism. They would be weaned away from mass politics and guided towards constitutional politics. It was also hoped that once the Congressmen in office had tasted power and dispensed patronage they would be most reluctant to go back to the politics of sacrifice. 

Another aspect of the colonial strategy was equally complex and masterly. Reforms could be used to promote dissensions and a split within the demoralized Congress ranks on the basis of constitutionalist vs. non constitutionalist and Right vs. Left. The constitutionalists and the right- wing were to be placated through constitutional and other concessions lured into the parliamentary game, encouraged to gradually give up agitational politics and coalesce with the moderate Liberals and landlords and other loyalists in working the constitution, and enabled to increase their weight in the nationalist ranks. The Left and radical elements, it was hoped, would see all this as a compromise with imperialism and abandonment of mass politics and would, therefore, become even more strident. Then, either the leftists (radicals) would break away from the Congress or their aggressive anti-Right politics and accent on socialism would lead the right- wing to kick them out. Either way, the Congress would be split and weakened. Moreover, isolated from the right-wing and devoid of the protection that a united national movement gave them, the leftist (radical) elements could be crushed through police measures. 

It was as a part of this strategy that the Government reversed its policy, followed during 1933--34, of suppressing the anti-constitutionalists in order to weaken the opposition to constitutionalism. Once division between the Left and the Right began to grow within the Congress, the Government refrained from taking strong action against revolutionary agitation by left- wing Congressmen. This happened from 1935 onwards. Above all the Government banked on Nehru's strong attacks on the constitutionalists and the right-wing and his powerful advocacy of socialism and revolutionary overthrow of colonial rule to produce a fissure in the nationalist ranks. Officials believed that Nehru and his followers had gone so far in their radicalism that they would not retreat when defeated by the right-wing in the AICC and at the Lucknow Congress. It was for this reason that nearly all the senior officials advised the Viceroy during 1935- 36 not to arrest him. Erskine, the Governor of Madras, for example, advised: `The more speeches of this type that Nehru makes the better, as his attitude will undoubtedly cause the Congress to split. Indeed, we should keep him in cotton wool and pamper him, for he is unwittingly smashing the Congress organization from inside.'' 

Provincial autonomy, it was further hoped, would create powerful provincial leaders in the Congress who would wield administrative power in their own right, gradually learn to safeguard their administrative prerogatives, and would, therefore, gradually become autonomous centres of political power. The Congress would, thus, be provincialize; the authority of the central all-India leadership would be weakened if n destroyed. As Linlithgow wrote in 1936, `our best hope of avoiding a direct clash is in the potency of Provincial Autonomy to destroy the effectiveness of Congress as an All-India instrument of revolution.'' 

The Act of 1935 was condemned by nearly all sections of Indian opinion and was unanimously rejected by the Congress. The Congress demanded instead, the convening of a Constituent Assembly elected on the basis of adult franchise to frame a constitution for an independent India.

\begin{center}*\end{center}

\paragraph*{}


The second stage of the debate over strategy occurred among Congressmen over the question of office acceptance. `The British, after imposing the Act of 1935, decided to immediately/put into practice provincial autonomy, and announced the holding of elections to provincial legislatures in early 1937. Their strategy of co-option or absorption into the colonial constitutional framework was underway. The nationalists were faced with a new political reality. All of them agreed that the 1935 Act must be opposed root and branch; but the question was how to do so in a period when a mass movement was not yet possible. 

Very sharp differences once again emerged in the ranks of the Congress leaders. There was, of course, full agreement that the Congress should fight the coming elections on the basis of a detailed political and economic programme, thus deepening the anti-imperialist consciousness of the people. But what was to be done after the elections? If the Congress got a majority in a province, should it agree to form the Government or not? Basic question of the strategy of the national movement and divergent perceptions of the prevailing political situation were involved. Moreover, the two sides to the debate soon got identified with the emerging ideological divide along Left and Right lines. 

Jawaharlal Nehru, Subhas Bose, the Congress Socialists and the Communists were totally opposed to office acceptance and thereby working the 1935 Act. The Left case was presented effectively and passionately by Nehru, especially in his Presidential Address at Lucknow in early 1936. Firstly, to accept office, was `to negate our rejection of it (the 1935 Act) and to stand self-condemned.' It would mean assuming responsibility without power, since the basic state structure would remain the same. While the Congress would be able to do little for the people, it would be cooperating `in some measure with the repressive apparatus of imperialism, and we would become partners in this repression and in the exploitation of our people.' 

Secondly, office acceptance would take away the revolutionary character of the movement imbibed since 1919. Behind this issue, said Nehru. lay the question `whether we seek revolutionary changes in India or (whether we) are working for petty reforms under the aegis of British imperialism.' Office acceptance would mean, in practice, `a surrender' before imperialism. The Congress would get sucked into parliamentary activity within the colonial framework and would forget the main issues of freedom, economic and social justice, and removal of poverty. It would be co-opted and deradicalized. It would fall into `a pit from which it would be difficult for us to come out.'' 

The counter-strategy that Nehru and the leftists recommended was the older, Swarajist one: enter the assemblies with a view to creating deadlocks and making the working of the Act impossible. As a long term strategy, they put forward the policy of increasing reliance on workers and peasants and their class organizations, integration of these class organizations with the Congress, imparting a socialist direction to the Congress, and preparing for the resumption of a mass movement. 

Those who favoured office acceptance said that they were equally committed to combating the 1935 Act. They denied that they were constitutionalists; they also believed that `real `work lies outside the legislature' and that work in the legislatures had to be a short-term tactic, for it could not lead to freedom --- for that a mass struggle outside the legal framework was needed. But, they said, the objective political situation made it necessary to go through a constitutional phase, for the option of a mass movement was not available at the time. The Congress should, therefore, combine mass politics with work in the legislatures and ministries in order to alter an unfavourable political situation. In other words, what was involved was not a choice between principles but a choice between the two alternative strategies of S-T-S' and S-V. The case of the right-wing was put with disarming simplicity by Rajendra Prasad in a letter to Nehru in December 1935: `So far as I can judge, no one wants to accept offices for their own sake. No one wants to work the constitution as the Government would like it to be worked. The questions for us are altogether different. What are we to do with this Constitution? Are we to ignore it altogether and go our way? Is it possible to do so? Are we to capture it and use it as we would like to use it and to the extent it lends itself to be used in that way... It is not a question to be answered a priori on the basis of pre-conceived notions of a so-called pro-changer or no-changer, cooperator or obstructionist.' And he assured Nehru that `1 do not believe that anyone has gone back to pre non-cooperation mentality. I do not think that we have gone back to 1923--28. We are in 1928--29 mentality and I have no doubt that better days will soon come.' Similarly, speaking at the Lucknow Session of the Congress, J.B. Kriplani said: `Even in a revolutionary movement there may be a time of comparative depression and inactivity. At such times, whatever programmes are devised have necessarily an appearance of reformatory activity but they are a necessary part of all revolutionary strategy.''9 Nor was the issue of socialism involved in the debate. As T. Vishwanathan of Andhra put it: `To my socialist comrades, I would say, capture or rejection of office is not a matter of socialism. I would ask them to realize that it is a matter of strategy.' 

The pro-office acceptance leaders agreed that there were pitfalls involved and that Congressmen in office could give way to wrong tendencies. But the answer, they said, was to fight these wrong tendencies and not abandon offices. Moreover, the administrative field should not be left clear to pro-Government forces. Even if the Congress rejected office, there were other groups and parties who would readily form ministries and use them to weaken nationalism and encourage reactionary and communal policies and politics. Lastly, despite their limited powers, the provincial ministries could be used to promote constructive work especially in respect of village and Harijan uplift, khadi, prohibition, education and reduction of burden of debt, taxes and rent on the peasants. 

The basic question that the ministerialists posed was whether office acceptance invariably led to co-option by the colonial state or whether ministries could be used to defeat the colonial strategy. The answer, in the words of Vishwanathan was: `There is no office and there is no acceptance... Do not look upon ministries as offices, but as centres and fortresses from which British imperialism is radiated... The Councils cannot lead us to constitutionalism, for we are not babies; we will lead the Councils and use them for Revolution.' 

Though Gandhiji wrote little on the subject, it appears that in the Working Committee discussions he opposed office acceptance and posed the alternative of quiet preparation in the villages for the resumption of civil disobedience. But by the beginning of 1936 he felt that the latter was still not feasible; he was, therefore, willing to give a trial to the formation of Congress ministries, especially as the overwhelming mood of the party favoured this course.

\begin{center}*\end{center}

\paragraph*{}


The Congress decided at Lucknow in early 1936 and at Faizpur in late 1936 to fight the elections and postpone the decision on office acceptance to the post-election period. Once again, as in 1922--24 and 1934, both wings of the Congress, having mutual respect and trust in their commitment to the anti- imperialist struggle and aware of the damage to the movement that a split would cause, desisted from dividing the party. Though often out-voted, the Left fought every inch of the way for acceptance of their approach but would not go to breaking point. 

The Congress went all out to win the elections to the provincial assemblies held in February 1937. Its election manifesto reaffirmed its total rejection of the 1935 Act. It promised the restoration of civil liberties, the release of political prisoners, the removal of disabilities on grounds of sex and untouchability, the radical transformation of the agrarian system, substantial reduction in rent and revenue, scaling down of the rural debts, provision of cheap credit, the right to form trade unions and the right to strike. 

The Congress election campaign received massive response and once again aroused the political consciousness and energy of the people. Nehru's country-wide election tour was to acquire legendary proportions. He travelled nearly 80,000 kilometres in less than five months and addressed more than ten million people, familiarizing them with the basic political issues of the time. Gandhiji did not address a single election meeting though he was very much present in the minds of the voters. 

The Congress won a massive mandate at the polls despite the narrow franchise. It won 716 out of 1,161 seats it contested. It had a majority in most of the provinces. The exceptions were Bengal, Assam, the NWPF, Punjab and Sind; and in the first three, it was the largest single party. The prestige of the Congress as the alternative to the colonial state rose even higher. The election tour and election results heartened Nehru, lifted him from the slough of despondency, and made him reconcile to the dominant strategy of S-T-S'.
