\chapter[The Quit India Movement, INA \& Bharat Choro]{The Quit India Movement and The INA}
\begin{multicols}{2}

`Quite India' and `Bharat Choro'. This simple hut powerful slogan launched ``the legendary struggle which also became famous by the name of the `August Revolution' ''. In this struggle, the common people of the country demonstrated an unparalleled heroism and militancy. Moreover, the repression that they faced was the most brutal that had ever been used against the national movement. The circumstances in which the resistance was offered were also the most adverse faced by the national movement until then --- using the justification of the war effort, the Government had armed itself with draconian measures, and suppressed even basic civil liberties. Virtually any political activity, however peaceful and `legal,' was at this time an illegal and revolutionary activity.

Why had it become necessary to launch a movement in these difficult conditions, when the possibility of brutal repression was a certainty?

For one, the failure of the Cripps Mission in April 1942 made it clear that Britain was unwilling to offer an honourable settlement and a real constitutional advance during the War, and that she was determined to continue India's unwilling partnership in the War efforts. The empty gesture of the even those Congressmen like Nehru and Gandhiji, who did not want to do anything to hamper the anti fascist War effort (and who had played a major role in keeping in check those who had been spoiling for a tight since 1939), that any further silence would be tantamount to accepting the right of the British Government to decide India's fate without any reference to the wishes of her people. Gandhiji had been as clear as Nehru that he did not want to hamper the anti-fascist struggle, especially that of the Russian and Chinese people. But by the spring of 1942 he was becoming increasingly convinced of the inevitability of a struggle. A fortnight after Cripps' departure, Gandhiji drafted a resolution for the Congress Working Committee calling for Britain's withdrawal and the adoption of non-violent non-cooperation against any Japanese invasion, Congress edged towards Quit India while Britain moved towards arming herself with special powers to meet the threat. Nehru remained opposed to the idea of a struggle right till August 1942 and gave way only at the very end.'

Apart from British obduracy, there were other factors that made a struggle both inevitable and necessary. Popular discontent, a product of rising prices and war-time shortages, was gradually mounting. High-handed government actions such as the commandeering of boats in Bengal and Orissa to prevent their being used by the Japanese had led to considerable anger among the people.

The popular willingness to give expression to this discontent was enhanced by the growing feeling of an imminent British collapse. The news of Allied reverses and British withdrawals from South-East Asia and Burma and the trains bringing wounded soldiers from the Assam-Burma border confirmed this feeling.

Combined with this was the impact of the manner of the British evacuation from Malaya and Burma. It was common knowledge that the British had evacuated, the white residents and generally left the subject people to their fate. /Letters from Indians in South-East Asia to their relatives in India were full of graphic accounts of British betrayal and their being left at the mercy of the dreaded Japanese. It not only to be expected that they would repeat the performance in India, in the event of a Japanese occupation? In fact, one major reason for the leadership of the national movement thinking it necessary to launch a struggle was their feeling that the people were becoming demoralized and, that in the event of a Japanese occupation, might not resist at all, In order to build up their capacity to resist Japanese aggression, It was necessary to draw them t of this demoralized state of mind and convince them of their own power. Gandhiji, as always, was particularly clear on this aspect. The popular faith in the stability of British rule had reached such a low that there was a run on the banks and people withdrew deposits from post-office savings accounts and started hoarding gold, silver and coins. This was particularly marked in East U.P. and Bihar, but it also took place in Madras Presidency.

So convinced was Gandhiji that the time was now ripe for struggle that he said to Louis Fischer in an interview in the beginning of June: `I have become impatient ... I may not be able to convince the Congress I will go ahead nevertheless and address myself directly to the people.' He did not have to carry out this threat and, as before, the Congress accepted the Mahatma's expert advice on the timing of a mass struggle.

Though Gandhiji himself had begun to talk of the coming struggle for some time now, it was at the Working Committee meeting at Wardha on 14 July, 1942 that the Congress first accepted the idea of a struggle. The All-India Congress Committee was then to meet in Bombay in August to ratify this decision. The historic August meeting at Gowalia Tank in Bombay was unprecedented in the popular enthusiasm it generated. Huge crowds waited outside as the leaders deliberated on the issue. And the feeling of anticipation and expectation ran so high that in the open session, when the leaders made their speeches before the many thousands who had collected to hear them, there was pin-drop silence.

Gandhiji's speech's delivered in his usual quiet and unrhetorical style, recount many who were in the audience, had the most electrifying impact. He first made it clear that `the actual struggle does not commence this moment. You have only placed all your powers in my hands. I will now wait upon the Viceroy a' plead with him for the acceptance of the Congress demand. That process is likely to take two or three weeks.' But, he added: `you may take it from me that I am not going to strike a bargain with the Viceroy for ministries and the like. I am not going to be satisfied with anything short of complete freedom. Maybe, he will propose the abolition of salt tax, the drink evil, etc. But I will say: ``Nothing less than freedom.''' He followed this up with the now famous exhortation: `Do or Die.' To quote: `Here is a mantra, a short one, that I give you. You may imprint it on your hearts and let every breath of yours give expression to it. The mantra is. ``Do or Die'' We shall either free India or die in the attempt: we shall not live to see the perpetuation of our slavery.'

Gandhiji's speech also contained specific instructions for different sections of the peop1e. Government servants would not yet be asked to resign, but they should openly declare their allegiance to the Congress, soldiers were also not to leave their posts, but they were to `refuse to fire on our own people . The Princes were asked to `accept the sovereignty of your own people, instead of paying homage to a foreign power.' And the people of the Princely States were asked to declare that they `(were) part of the Indian nation and that they (would) accept the leadership of the Princes, if the latter cast their lot with the People, but not otherwise.' Students were to give up studies if they were sure they could continue to remain firm independence was achieved. On 7 August, Gandhiji had placed the instructions he had drafted before the Waking Committee, and in these he had proposed that peasants `who have the courage, and are prepared to risk their all' should refuse to pay the land revenue. Tenants were told that `the Congress holds that the land belongs to those who work on it and to no one else.' Where the zamindari system prevails ... if the zamindar makes common cause with the ryot, his portion of the revenue, which may be settled by mutual agreement, should be given to him. But if a zamindar wants to side with the Government, no tax should be paid to him.' These instructions were not actually issued because of the preventive arrests, but they do make Gandhiji's intentions clear.

The Government, however, was in no mood to either negotiate with the Congress or wait for the movement to be formally launched. In the early hours of 9 August, in a single sweep, all the top leaders of the congress were arrested and taken to unknown destinations. The Government had been preparing for the strike since the outbreak of the War itself, and since 1940 had been ready with an elaborate Revolutionary Movement Ordinance. On 8 August, 1940, the Viceroy, Linlithgow, in a personal letter to the Governors made his intentions clear: `I feel very strongly that the only possible answer to a `declaration of war' by any section of Congress in the present circumstances must be a declared determination to crush the organization as a while.' For two years, Gandhiji had avoided walking into the trap set for him by refusing to make a rash and premature strike and had carefully built up the tempo through the Individual Civil Disobedience Movement, organizational revamping and a consistent propaganda campaign. But now, the Government was unwilling to allow him any more time to pursue his strategy. In anticipation of the A ICC's passing the Quit India resolution, instructions for arrests and suppression had gone out to the provinces.

The sudden attack by the Government produced an instantaneous reaction among the people. In Bombay, as soon as the news of arrests spread lakhs of people flocked to Gowalia Tank where a mass meeting had been scheduled and there were clashes with the authorities. There were similar disturbances on 9 August in Ahmedabad and Poona. On the 10th Delhi and many towns in U.P. and Bihar, including Kanpur, Allahabad, Varanasi and Patna followed suit with hartals, public demonstrations and processions in defiance of the law. The Government responded by gagging the press. The National Herald and Harijan ceased publication for the entire duration of the struggle, others for shorter periods.

Meanwhile, provincial and local level leaders who had evaded arrest returned to their homes through devious routes and set about organizing resistance. As the news spread further in the rural areas, the villagers joined the townsmen in recording their protest. For the first six or seven weeks after 9 August, there was a tremendous' mass upsurge all over the country. People devised a variety of ways of expressing their anger. In some places, huge crowds attacked police stations, post offices, kutcheries (courts), railway stations and other symbols of Government authority. National flags were forcibly hoisted on public buildings in defiance of the police. At other places, groups of Satyagrahis offered arrest in tehsil or district headquarters. Crowds of villagers, often numbering a few hundreds or even a couple of thousand, physically removed railway tracks. Elsewhere, small groups of individuals blew up bridges and removed tracks, and cut telephone and telegraph wires. Students went on strike in schools and colleges all over the country and busied themselves taking processions, writing and distributing illegal news-sheets: hundreds of these patrikas' came our all over the country. They also became couriers for the emerging underground networks' Workers too stuck work: in Ahmedabad, the mills were closed for three and a half months, workers in Bombay stayed away from work for over a week following the 9 August arrests, in Jamshedpur there was a strike for thirteen days and workers in Ahmednagar and Poona were active for several months.

The reaction to the arrests was most intense in Bihar and Eastern U P, where the movement attained the proportions of a rebellion. From about the middle of August, the news reached the rural areas through students and other political activists who fanned out from the towns. Students of the Banaras Hindu University decided to go to the villages to spread the message of Quit India. They raised slogans of `Thana jalao' (Burn police station), `Station phoonk do' (Burn the railway stations) `Angez Bhag Gaya'(Englishmen have fled). They hijacked trains and draped them in national flags. In rural areas, the pattern was of large crowds of peasants descending on the nearest tehsil or district town and attacking all symbols of government authority. There was government fiññg and repression, but the rebellion only gathered in momentum. For two weeks, Tirhut division in Bihar was totally cut off from the rest of the country and no Government authority existed. Control was lost over Patna for two days after firing at the Secretariat. Eighty percent of the police stations were captured or temporarily evacuated in ten districts of North and Central Bihar. There were also physical attacks on Europeans. At Fatwa, near Patna, two R.A. F. officers were killed by a crowd at the railway station and their bodies paraded through the town. In Monghyr, the crews of two R.A. F. planes that crashed at Pasraha on 18 August and Rulhar on 30 August were killed by villagers. Particularly important centres of resistance in this phase were Azamgarh, Ballia and Gorakhpur in East U.P. and Gaya, Bhagalpur, Saran, Purnea, Shahabad, Muzaffarpur and Champaran in Bihar.

According to official estimates, in the first week after the arrests of the leaders, 250 railway stations were damaged or destroyed, and over 500 post offices and 150 police stations were attacked. The movement of trains in Bihar and Eastern U.P., was disrupted for many weeks. In Karnataka alone, there were 1600 incidents of cutting of telegraph lines, and twenty- six railway stations and thirty-two post offices were attacked. Unarmed crowds faced police and military firing on 538 occasions and they were also machine-gunned by low-flying aircraft. Repression also took the form of taking hostages from the villages, imposing collective fines running to a total of Rs 90 lakhs (which were often realized on the spot by looting the people's belongings), whipping of suspects and burning of entire villages whose inhabitants had run away and could not be caught. By the end of 1942, over 60,000 persons had been arrested. Twenty-six thousand people were convicted and 18,000 detained under the Defence of India Rules. Martial law had not been proclaimed, but the army, though nominally working under the orders of the civilian authorities, often did what it wanted to without any reference to the direct officers. The repression was as severe as it could have been under martial law.

The brutal and all-out repression succeeded within a period of six or seven weeks in bringing about a cessation of the mass phase of the struggle. But in the meantime, underground networks were being consolidated in with prominent members such as Achyut Patwardhan, Aruna Asaf Ali, Ram Mañohar Lohia, Sucheta Kripalani, Chootubhai Puranik, Biju Patnaik, R.P. Goenka and later, after his escape from jail, Jayaprakash Narayan had lo begun to emerge. This leadership saw the role of the underground movement as being that of keeping up popular morale by continuing to a line of command and a source of guidance and leadership to activists all over the country. They also collected and distributed money as well as material like bombs, arms, and dynamite to underground groups all over the country. They, however, did see their role as that of directing the exact pattern of activities at the local level. Here, local groups retained the initiative. Among the places in which local underground organizations were active were Bombay, Poona, Satara, Baroda and other parts of Gujarat, Karnataka, Kerala, Andhra, U P, Bihar and Delhi. Congress Socialists were generally in the lead, but also active were Gandhian ashramites, Forward Bloc members and revolutionary terrorists, as well as other Congressmen.

Those actually involved in the underground activity may have been few, but they received all manner of support from a large variety of people. Businessmen donated generously. Sumati Morarjee, who later became India's leading woman industrialist, for example, helped Achyut Patwardhan to evade detection by providing, him with a different car every day borrowed from her unsuspecting wealthy friends. Others provided hideouts for the underground leaders and activists. Students acted as couriers.

Simple villagers helped by refusing information to the police. Pilots and train drivers delivered bombs and other material across the country. Government officials, including those in the police, passed on crucial information about impending arrests. Achyut Patwardhan testifies that one member of the three-man high level official committee formed to track down the Congress underground regularly informed him of the goings on that committee.

The pattern of activity of the underground movement was generally that of organizing the disruption of communications by blowing up bridges, cutting telegraph and telephone wires and derailing trains There were also a few attacks on government and police officials and police informers. Their success in actually disrupting communications may not have been more than that of having nuisance value, but they did succeed in keeping up the spirit of the people in a situation when open mass activity was impossible because of the superior armed might of the state. Dissemination of news was a very important part of the activity, and considerable success was achieved on this score, the most dramatic being the Congress Radio operated clandestinely from different locations in Bombay city, whose broadcast could be heard a far as Madras. Ram Manohar Lohia regularly broadcast on this radio, and the radio continued till November 1942 when it was discovered and confiscated by the police.

In February 1943, a striking new development provided a new burst of political activity. Gandhiji commenced a fast on' 10 February in jail. He declared the fast would last for twenty-one days. This was his answer to die Government which had been constantly exhorting him to condemn the violence of the people in the Quit India Movement. Gandhiji not only refused to condemn the people's resort to violence but unequivocally held the Government responsible for it. It was the `leonine violence' of the state which had provoked the people, he said. And it was against this violence of the state, which included the unwarranted detention of thousands of Congressmen that Gandhiji vowed to register his protest, in the only way open to him when in jail, by fasting.

The popular response to the news of the fast was immediate and overwhelming.' All over the country, there were hartals, demonstrations and strikes. Calcutta and Ahmedabad were particularly active. Prisoners in jails and those outside went on sympathetic fasts. Groups of people secretly reached Poona to offer Satyagraha outside the Aga Khan Palace where Gandhiji was being held in detention. Public meetings demanded his release and the Government was bombarded with thousands of letters and telegrams from people from all walks of life --- students and youth, men trade and commerce, lawyers, ordinary citizens, and labour organizations. From across the seas, the demand for his release was made by newspapers such as the

Manchester Guardian, New Statesmen, Nation, News Chronicle, Chicago Sun, as well as by the British Communist Party, the citizens of London and Manchester, the Women's International League, the Australian Council of Trade Unions and the Ceylon State Council. The U.S. Government, too, brought pressure to bear.

A Leaders' Conference was held in Delhi on 19-20 February and was attended by prominent men, politicians and public figures. They all demanded Gandhiji's release. Many of those otherwise unsympathetic to the Congress felt that the Government was going too far in its obduracy. The severest blow to the prestige of the Government was the resignation of the three Indian members of the Viceroy's Executive Council, M.S. Aney,

N.R. Sarkar and H.P. Mody, who had supported the Government in its suppression of the 1942 movement, but were in no mood to be a party to Gandhiji's death.

But the Viceroy and his officials remained unmoved. Guided by Winston Churchill's statement to his Cabinet that `this our hour of triumph everywhere in the world was not the time to crawl before a miserable old man who had always been our enemy,''° they arrogantly refused to show any concern for Indian feeling. The Viceroy contemptuously dismissed the consequences of Gandhiji's possible death: `Six months unpleasantness, steadily declining in volume, little or nothing at the end of it.' He even made it sound as if he welcomed the possibility: `India would be far more reliable as a base for operations. Moreover, the prospect of a settlement will be greatly enhanced by the disappearance of Gandhi, who had for years torpedoed every attempt at a settlement.'' `While an anxious nation appealed for his life, the Government went ahead with finalizing arrangements for his funeral. Military troops were asked to stand by for any emergency. `Generous' provision was made for a plane to carry his ashes and for a public funeral and a half-day holiday in offices.' But Gandhiji, as always, got the better of his opponents, and refused to oblige by dying.

The fast had done exactly what it had been intended to do. The public morale was raised, the anti-British feeling heightened, and an opportunity for political activity provided. A symbolic gesture of resistance had sparked off widespread resistance and exposed the Government's high-handedness to the whole world.' The moral justification that the Government had been trying to provide for its brutal suppression of 1942 was denied to it and it was placed clearly in the wrong.

A significant feature of the Quit India Movement was the emergence of what came to be known as parallel governments in some parts of the country. The first one was proclaimed in Ballia, in East U P, in August 1942 under the leadership of Chittu Pande, who called himself a Gandhian. Though it succeeded in getting the Collector to hand over power and release all the arrested Congress leaders, it could not survive for long and when the soldiers marched in, a week after the parallel government was formed, they found that the leaders had fled.'

In Tamluk in the Midnapur district of Bengal, the Jatiya Sarkar came into existence on 17 December, 1942 and lasted till September 1944. Tamluk was an area where Gandhian constructive work had made considerable headway and it was also the scene of earlier mass struggles.

The Jatiya Sarkar undertook cyclone relief work, gave grants to schools and organized an armed Vidyut Vahini. It also set up arbitration courts and distributed the surplus paddy of the well- to-do to the poor. Being located in a relatively remote area, it could continue its activities with comparative ease. Satara, in Maharashtra, emerged as the base of the longest- lasting and effective parallel government. From the very beginning of the Quit India Movement, the region played an active role. In the first phase from August 1942, there were marches on local government headquarters the ones on Karad, Tasgaon and Islampur involving thousands. This was followed by sabotage, attacks on post offices, the looting of banks and the cutting of telegraph wires. Y.B. Chavan, had contacts with Achyut Patwardhan and other underground leaders, was the most important leader. But by the end of 1942, this phase came to an end with the arrest of about two thousand people. From the very beginning of 1943, the underground activists began to regroup, and by the middle of the year, succeeded in consolidating the organization. A parallel government or Prati Sarkar was set up and Nani Patil was its most important leader. This phase was marked by attacks on Government collaborators, informers and talatis or lower-level officials and Robin Hood-style robberies. Nyayadan Mandals or people's courts were set up and justice dispensed. Prohibition was enforced, and `Gandhi marriages' celebrated to which untouchables were invited and at which no ostentation was allowed. Village libraries were set up and education encouraged. The native state of Aundh, whose ruler was pro-nationalist and had got the constitution of his state drafted by Gandhiji, provided invaluable support by offering refuge and shelter to the Prati Sarkar activists. The Prati Sarkar continued to function till 1945.'

The Quit India Movement marked a new high in terms of popular participation in the national movement and sympathy with the national cause in earlier mass struggles, the youth were in the forefront of the struggle. Students from colleges and even schools were the most visible element, espeecia1ly in the early days of August (probably the average age of participants in the 1942 struggle was even lower than that in earlier movements). Women especially college an school girls, played a very important role. Aruna Asaf Ali and Sucheta Kripalani were two major women organizers of the underground, and Usha Mehta an important member of the small group that ran the Congress Radio. Workers were prominent as well, and made considerable sacrifice by enduring long strikes and braving police repression in the streets.

Peasants of all strata, well-to-do as well as poor, were the heart of the movement especially in East U.P. and Bihar,

Midnapur in Bengal, Satara in Maharashtra, but also in other parts including Andhra, Gujarat and `Kerala. Many small zamindars also participated especially in U.P. and Bihar. Even the big zamindars maintained a stance of neutrality and to assist the British in crushing the rebellion. The most spectacular was the Raja of Darbhanga, one of the biggest zamindars, who refused to let his armed retainers to be used by the Government and even instructed his managers to assist the tenants who had been arrested. A significant feature of the pattern of peasant activity was its total concentration on attacking symbols of British authority and a total lack of any incidents of anti-zamindar violence, even when, as in Bihar, East U P. Satara, and Midnapur, the breakdown of Government authority for long periods of time provided the opportunity.bb Government officials, especially those at the lower levels of the police and the administration, were generous in their assistance to the movement. They gave shelter, provided information and helped monetarily. In fact, the erosion of loyalty to the British Government of its own officers was one of the most striking aspects of Quit India struggle. Jail officials tended to be much kinder to prisoners than n earlier years, and often openly expressed their sympathy.

While it is true that Muslim mass participation in the Quit India movement was not high, yet it is also true that even Muslim League supporters not act as informers. Also, there was a total absence of any communal clashes, a sure sign that though the movement may not have aroused much support from among the majority of the Muslim masses, it did not arouse their hostility either.

The powerful attraction of the Quit India Movement and its elemental quality is also demonstrated by the fact that hundreds of Communists at the local and village levels participated in the movement despite the official position taken by the Communist Party. Though they sympathized with the strong anti-fascist sentiments of their leaders, yet they felt the irresistible pull of the movement and, for at least a few days or weeks, joined in it along with the rest of the Indian people.

The debate on the Quit India Movement has cantered particularly on two issues. First, was the movement a spontaneous outburst, or an organized rebellion. Second, how did the use of violence by the people in this struggle square with the overall Congress policy of non-violent struggle?

First, the element of spontaneity of 1942 was certainly larger than in the earlier movements, though even in 1919--22, as well as in 1930--31 and 1932, the Congress leadership allowed considerable room for an initiative and spontaneity. In fact, the whole pattern of the Gandhian mass movements was that the leadership chalked out a broad programme of action and left its implementation at the local level to the initiative of the local and grass roots level political activists and the masse. Even in the Civil Disobedience Movement of 1930, perhaps the most organized of the Gandhian mass movements, Gandhiji signalled the launching of the struggle by the Dandi March and the breaking of the salt law, the leaders and the people at the local levels decided whether they were going to stop payment of land revenue and rent, or offer Satyagrahi against forest Laws, or picket liquor shops, or follow any of the other items of the programme. Of course, in 1942, even the broad programme had not yet been spelt out clearly since the leadership was yet to formally launch the movement. But, in a way, the degree of spontaneity and popular initiative that was actually exercised had sanctioned by the leadership itself. The resolution passed by the AICC on 8 August 1942 clearly stated: `A time may come when it may not be possible to issue instruction or for instructions to reach our people, and when no Congress committees can function. When this happens, every man and woman who is participating in this movement must function for himself or herself within the four corners of the general instructions issued. Every Indian who desires freedom and strives for it must be his own guide.''

Apart from this, the Congress had been ideologically, politically and organizationally preparing for the struggle for a long time. From 1937 the onwards, the organization had been revamped to undo the damage suffered during the repression of 1932--34. In political and ideological terms as well, the Ministries had added considerably to Congress support and prestige. In East U.P. and Bihar, the areas of the most intense activity in 1942 were precisely the ones in which considerable mobilization and organizational work had been carried out from 1937 onwards.' In Gujarat, Sardar Patel had been touring Bardoli and other areas since June 1942 warning the people of an impending struggle and suggesting that no- revenue campaigns could well be part of it. Congress Socialists in Poona had been holding training camps for volunteers since June 1942) Gandhiji himself, through the Individual Civil Disobedience campaign in 1940--41, and more directly since early 1942, had prepared the people for the coming battle, which he said would be `short and swift.'

In any case, in a primarily hegemonic struggle as the Indian national movement was, preparedness for struggle cannot be measured by the volume of immediate organizational activity but by the degree of hegemonic influence the movement bas acquired over the people.

How did the use of violence in 1942 square with the Congress policy of non-violence. For one, there were many who refused to use or sanction violent means and confined themselves to the traditional weaponry of the Congress. But many of those, including many staunch Gandhians, who used `violent means' in 1942 felt that the peculiar circumstances warranted their use. Many maintained that the cutting of telegraph wires and the blowing up of bridges was all right as long as human life was not taken. Others frankly admitted that they could not square the violence they used, or connived at with their belief in non-violence, but that they did it all the same. Gandhiji refused to condemn the violence of the people because he saw it as a reaction to the much bigger violence of the state. In Francis Hutchins' view, Gandhiji's major objection to violence was that its use prevented mass participation in a movement, but that, in 1942, Gandhiji had come round to the view that mass participation would not be restricted as a result of violence.

The great significance of this historic movement was that it placed the demand for independence on the immediate agenda of the national movement. After Quit India there cou1d be no retreat. Any future negotiations with the British Government could only be on the manner of the transfer of power. Independence was no longer a matter of bargain. And this became amply clear after the War.

With Gandhiji's release on 6 May 1944, on medical grounds, political activity regained momentum. Constructive work became the main form of Congress activity, with a special emphasis on the reorganization of the Congress machinery. Congress committees were revived under different names --- Congress Workers Assemblies or Representative Assemblies of Congressmen --- rendering the ban on Congress committees ineffective. The task of training workers, membership drives and fund collection was taken up. This reorganization of the Congress under the `cover' of the constructive programme was viewed with serious misgivings by the Government which saw it as an attempt to rebuild Congress influence and organization in the villages in preparation for the next round of struggle? A strict watch was kept on these developments, but no repressive action was contemplated and the Viceroy's energies were directed towards formulating an offer (known as the Wavell Offer or the Simla Conference) which would pre-empt a struggle by effecting an agreement with the Congress before the War with Japan ended. The Congress leaders were released to participate in the Simla Conference in June 1945. That marked w end of the phase of confrontation that had existed since August 1942.

Before we end this chapter, a brief look at the Indian National Army is essential. The idea of the INA was first conceived in Malaya by Mohan Singh, an Indian officer of the British Indian Army, when he decided not to join the retreating British army and instead went to the Japanese for help. The Japanese had till then only encouraged civilian Indians to form anti-British organizations, but had no conception of forming a military wing consisting of Indians.

Indian prisoners of war were handed over by the Japanese to Mohan Singh who then tried to recruit them into an Indian National Army. The fall of Singapore was crucial, for this brought 45,000 Indian POWs into Mohan Singh's sphere of influence. By the end of 1942, forty thousand men expressed their willingness to join the INA. It was repeatedly made clear at various meetings of leaders of the Indian community and of Indian Army officers that the INA would go into action only on the invitation of the Indian National Congress and the people of India. The 1NA was also seen by many as a means of checking the misconduct of the

Japanese against Indians in South-East Asia and a bulwark against a future Japanese occupation of India.

The outbreak of the Quit India Movement gave a fillip to the [NA as well. Anti-British demonstrations were organized in Malaya. On 1 September 1942 the first division of the INA was formed with 16,300 men. The Japanese were by now more amenable to the idea of an armed Indian wing because they were contemplating an Indian invasion. But, by December 1942, serious differences emerged between the Indian army officers led by Mohan Singh and the Japanese over the role that the INA was to play. Mohan Singh and Niranjan Singh Gill, the senior-most Indian officer to join the INA, were arrested. The Japanese, it turned out, wanted only a token force of 2,000 men, while Mohan Singh wanted to raise an Indian National Army of 20,000.

The second phase of the 1NA began when Subhas Chandra Bose was brought to Singapore on 2 July 1943, by means of German and Japanese submarines. He went to Tokyo and Prime Minister Tojo declared that Japan had no territorial designs on India. Bose returned to Singapore and set up the Provisional Government of Free India on 21 October1943. The Provisional Government then declared war on Britain and the United State and was recognised by the Axis powers and their satellites. Subhas Bose set up two INA headquarters, in Rangoon and in Singapore, and began to reorganize the INA. Recruits were sought from civilians, funds were gathered, and even a women's regiment called the Rani Jhansi regiment was formed. On 6 July 1944, Subhas Bose, in a broadcast on Azad Hind Radio addressed to Gandhiji, said: `India's last war of independence has begun ... Father of our Nation! In this holy war of India's liberation, we ask for your blessing and good wishes.'

One INA battalion commanded by Shah Nawaz was allowed to accompany the Japanese Army to the Indo-Burma front and participate in the Imphal campaign. But the discriminatory treatment which Included being denied rations, arms and being made to do menial work for the Japanese units, completely demoralized the INA men. The failure of the Imphal campaign, and the steady Japanese retreat thereafter, quashed any hopes of the INA liberating the nation. The retreat which began in mid-1944 continued till mid-1945 and ended only with the final surrender to the British in South-East Asia. But, when the INA men were brought back home and threatened with serious punishment, a powerful movement was to emerge in their defence.
\end{multicols}