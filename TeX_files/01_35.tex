\chapter{Post-War National Upsurge}



The end of World War II marked a dramatic change. From then till the dawn of freedom in 1947 the political stage witnessed a wide spectrum of popular initiative. We are constrained to leave out of our purview the struggles of workers, peasants and people of the native states, which took the form of the country-wide strike wave, the Tebhaga Movement, the Warlis Revolt, the Punjab kisan morchas, the Travancore people's struggle (especially the Punnapra-Vayalar episode) and the Telengana Movement. These movements had an anti-imperialist edge — as the direct oppressors they challenged were also the vested interests that constituted the social support of the Raj — but they did not come into direct conflict with the colonial regime. We shall confine ourselves to that stream of anti- imperialist activity which directly challenged the legitimacy of British rule and was perceived to be doing so by the colonial authorities.

\begin{center}*\end{center}

\paragraph*{}


The end of the War was greeted in India with a vast sigh of relief. Its few benefits such as windfall gains and super-profits for the capitalists and employment opportunities for the middle classes were far outweighed by the ravages and miseries wrought by it. The colony reeled under the heavy yoke of the war effort. Famine, inflation, scarcity, hoarding and black-marketing plagued the land. The heroic action of a leaderless people notwithstanding, the Quit India Movement was snuffled out in eight weeks. Pockets of resistance, where the torch was kept ablaze, could not hold out for long. 

When Congress leaders emerged from jail in mid-June 1945, they expected to find a demoralized people, benumbed by the repression of 1942, bewildered by the absence of leadership and battered by the privations that the War brought. To their surprise, they found tumultuous crowds waiting for them, impatient to do something, restless and determinedly anti- British. Repression had steeled the brave and stirred the conscience of the fence-sitter. Political energies were surfacing after more than three years of repression and the expectations of the people were now heightened by the release of their leaders. The popular belief was that the release would mark the beginning of a period of rapid political progress. Crowds thronged the gates of Almora jail on hearing that Jawaharlal Nehru was to be released. They waited a long while outside Bankura jail where Maulana Azad was lodged. When the Congress Working Committee met, more than half a million people lined the streets of Bombay, braving the rain to welcome their leaders. Similar scenes were witnessed when the leaders went to Simla to attend the conference called by the Viceroy. Villagers from places far away from Simla converged and sat atop trees, waiting for hours to catch a glimpse of their leaders. 

The Labour Party, which had come to power in Britain after the War, was in a hurry to settle the Indian problem. As a result the ban on the Congress was lifted and elections declared. People were elated at the prospect of popular ministries and turned out in large numbers at election meetings — 50,000 on an average, and a lakh or so when all India leaders were expected. Nehru, a seasoned campaigner of the 1937 elections, confessed that he had not previously seen such crowds, such frenzied excitement. Except in constituencies where nationalist Muslims were put up, candidates did not really need to canvass for votes or spend money. The election results indicated that people had not only flocked to the meetings but had rallied behind the Congress at the ballot-box too. The Congress won over 90 percent of the general seats (including twenty-three of the thirty-six labour seats) in the provincial elections while the Muslim League made a similar sweep in the Muslim constituencies. But, perhaps, the most significant feature of the election campaign was that it sought to mobilize Indians against the British, not merely voters for the elections. This was evident from the two issues which were taken up and made the main plank of the election campaign — the repression in 1942 and the Indian National Army trials. 

The question of official excesses during 1942 was taken up by Congress leaders soon after release from jail. Glorification of martyrs was one side of the coin, condemnation of official action the other. Congressmen lauded the brave resistance offered by the leaderless people, martyrs' memorials were erected in many places and relief funds organized for sufferers. Stories of repression were recounted in grim detail, the officials responsible condemned, often by name, promises of enquires held out, and threats of punishment freely made. While such speeches, which the Government failed to check, had a devastating effect on the morale of the services, that was more alarming for the officials was the rising crescendo of demands for enquiries into official actions. The forthcoming elections were likely to bring the Congress ministries back to power, significantly in those provinces where repression had been most brutal. The U.P. Governor, Wylie, confessed on 19th February, 1946 that officials in U.P. in 1942 `used on occasion methods which I cannot condone and which, dragged out in the cold light of 1946, nobody could defend.'' The Viceroy concluded that only a `gentleman's agreement' with the Congress could resolve the matter. 

However, the issue which most caught the popular imagination was the fate of the members of Subhas Chandra Bose's Indian National Army (INA), who were captured by the British in the eastern theatre of War. An announcement by the Government, limiting trials of the INA personnel to those guilty of brutality or active complicity, was due to be made by the end of August, 1945. However, before this statement could be issued. Nehru raised the demand for leniency at a meeting in Srinagar on August 1945 — making the proposed statement seem a response to his call rather than an act of generosity on the part of the Government. Hailing them as patriots, albeit misguided, Nehru called for their judicious treatment by the authorities in view of the British promise that `big changes' are impending in India. Other Congress leaders soon took up the issue and the AICC at its first post-War session held in Bombay from 21 to 23 September 1945, adopted a strong resolution declaring its support for the cause. The defence of the INA prisoners was taken up by the Congress and Bhulabhaj Desai, Tej Bahadur Sapru, K.N. Katju, Nehru and Asaf All appeared in court at the historic Red Fort trials. The Congress organised an INA Relief and Enquiry Committee, which provided small sums of money and food to the men on their release, and attempted, though with marginal success, to secure employment for these men. The Congress authorized the Central INA Fund Committee, the Mayor's Fund in Bombay, the AICC and the PCC offices and Sarat Bose to collect funds. The INA question was the main issue highlighted from the Congress platform in meetings held all over the country — in fact, very often it was difficult to distinguish between an INA and an election meeting. In view of Nehru's early championing of the INA cause and the varied involvement of the Congress later, the oft made charge that the Congress jumped on to the INA bandwagon and merely used the issue as an election stunt does not appear to have any validity. 

The INA agitation was a landmark on many counts. Firstly, the high pitch or intensity at which the campaign for the release of INA prisoners was conducted was unprecedented. This was evident from the press coverage and other publicity it got, from the threats of revenge that were publicly made and also from the large number of meetings held. 

Initially, the appeals in the press were for clemency to `misguided' men, but by November 1945, when the first Red Fort trials began, there were daily editorials hailing the INA men as the most heroic patriots and criticizing the Government stand. Priority coverage was given to the INA trials and to the [NA campaign, eclipsing international news. Pamphlets, the most popular one being `Patriots Not Traitors,' were widely circulated, `Jai Hind' and `Quit India' were scrawled on walls of buildings in Ajmer. Posters threatening death to `20 English dogs' for every [NA man sentenced, were pasted all over Delhi. In Banaras, it was declared at a public gathering that `if INA men were not saved, revenge would be taken on European children.' One hundred and sixty political meetings were held in the Central Provinces and Berar alone in the first fortnight of October 1945 where the demand for clemency for INA prisoners was raised. INA Day was observed on 12 November and INA Week from 5 to II November 1945. While 50,000 people would turn out for the larger meetings, the largest meeting was the one held in Deshapriya Park, Calcutta. Organized by the INA Relief Committee, it was addressed by Sarat Bose, Nehru and Patel. Estimates of attendance ranged from to two to three lakhs to Nehru's five to seven Iakhs. 

The second significant feature of the INA campaign was its wide geographical reach and the participation of diverse social groups and political parties. This had two aspects. One was the generally extensive nature of the agitation, the other was the spread of pro-INA sentiment to social groups hitherto outside the nationalist pale. The Director of the 

Intelligence Bureau Conceded: `There has seldom been a matter which has attracted so much Indian public interest, and, it is safe to say, sympathy.' `Anxious enquiries' and `profuse sympathies' were forthcoming from the `remotest villages' from all men, `Irrespective of Caste, colour and creed.' Nehru confirmed the same: `Never before in Indian history had such unified sentiments and feelings been manifested by various divergent sections of the Indian population as it had been done with regard to the question of the Azad Hind Fauj.' While the cities of Delhi, Bombay, Calcutta and Madras and the towns of U.P. and Punjab were the nerve centres of the agitation, what was more noteworthy was the spreading of the agitation to places as distant as Coorg, Baluchistan and Assam. Participation was of many kinds — some contributed funds, others attended or organized meetings, shopkeepers downed shutters and political parties and organizations raised the demand for the release of the prisoners. Municipal Committees, Indians abroad and Gurdwara committees subscribed liberally to INA funds. The Shiromani Gurdwara Prabandhak Committee, Amritsar donated Rs 7000 and set aside another Rs 10,000 for relief. The Poona City Municipality, the Kanpur City Fund and a local district board in Madras Presidency contributed Rs 1,000 each. More newsworthy contributions were those by film stars in Bombay and Calcutta, by the Cambridge Majlis and the tongawallas of Amraoti. Students, whose role in the campaign was outstanding, held meetings and rallies and boycotted classes from Salem in the south to Rawalpindi in the north. Commercial institutions, shops and markets stopped business on the day the first trial began, 5 November 1945, on NA Day and during NA Week. Demands for release were raised at kisan Conferences in Dhamangaon and Sholapur on 16 November 1945 and at the tenth session of the All India Women's Conference in Hyderabad on 29 December 1945. `Even English intellectuals, birds of a year or two's sojourn in India, were taking a keen interest in the rights and wrongs, and the degrees of wrong, of the INA men,' according to General Tuker of the Eastern Command. Diwali was not celebrated in some areas in sympathy with the NA men. Calcutta Gurdwaras became a campaigning centre for the NA cause. The Muslim League, the Communist Party of India, the Unionist Party, the Akalis, the Justice Party, the Abrars in Rawalpindi, the Rashtriya Swayamsevak Sangh, the Hindu Mahasabha and the Sikh League supported the NA cause in varying degrees. The Viceroy noted that `all parties have taken the same line though Congress are more vociferous than the others.' 

The most notable feature of the INA agitation was the effect it had on the traditional bulwarks of the Raj. Significant sections of Government employees, loyalist sections and even men of the armed forces were submerged in the tide of pro-INA sentiment. Many officials saw in this a most disquieting trend. The Governor of Northwest Frontier Province warned that `every day that passes now brings over more and more well- disposed Indians to the anti-British camp'. The Director of the Intelligence Bureau observed that `sympathy for the INA is not the monopoly of those who are ordinarily against Government,' and that it was `usually the case that INA men belonged to families which had traditions of loyalty.' In Punjab (to which province 48.07 per cent of the INA men released till February 1946 belonged) the return of the released men to their villages' stimulated interest among groups which had hitherto remained politically unaffected. Local interest was further fuelled by virtue of many of the INA officers belonging to influential families in the region. P.K. Sehgal, one of the trios tried in the first Red Fort trial, was the son of Dewan Achhru Rain, an ex-Judge of the Punjab High Court. The gentlemen with titles who defended men accused of war time treason did not glorify' the action of INA men — they appealed to the Government to abandon the trials in the interest of good relations between India and Britain. Government officials generally sympathized privately, but there were some instances, as in the Central Provinces and Berar, where railway officials collected finds. 

The response of the armed forces was unexpectedly sympathetic, belying the official perception that loyal soldiers were very hostile to the INA `traitors'. Royal Indian Air Force (RIAF) men in Kohat attended Shah Nawaz's meetings and army men in UP and Punjab attended INA meetings, often in uniform. RIAF men in Calcutta, Kohat, Allahabad, Bamrauli and Kanpur contributed money for the INA defence, as did other service personnel in U.P. Apart from these instances of overt support, a `growing feeling of sympathy for the [NA' pervaded the Indian army, according to the Commander-in-Chief. He concluded that the `general opinion in the Army is in favour of leniency' and recommended to Whitehall that leniency be shown by the Government. Interestingly, the question of the right or wrong of the NA men's action was never debated. What was in question was the right of Britain to decide a matter concerning Indians. As Nehru often stressed, if the British were sincere in their declaration that Indo-British relations were to be transformed; they should demonstrate their good faith by leaving it to Indians to decide the INA issue. Even the appeals by liberal Indians were made in the interest of good future relations between India and Britain. The British realised this political significance of the INA issue. The Governor of North-West Frontier Province advocated that the trials be abandoned, on the ground that with each day the issue became `more and more purely Indian versus British.' 

The growing nationalist sentiment, that reached a crescendo around the INA trials, developed into violent confrontations with authority in the winter of 1945-46. There were three upsurges — one on 21 November 1945 in Calcutta over the INA trials; the second on 11 February 1946 in Calcutta to protest against the seven year sentence given to an [NA officer, Rashid Mi; and the third in Bombay of 18 February 1946 when the ratings of the Royal Indian Navy (RIN) went on strike. The upsurges followed a fairly similar pattern an initial stage when a group (such as students or ratings) defied authority and was repressed, a second stage when people in the city joined in, and finally a third stage when people in other parts of the country expressed sympathy and solidarity. 

The first stage began with the students' and ratings' challenge to authority and ended in repression. On 21 November 1945, a procession of students, consisting of Forward Bloc sympathizers and joined by Students Federation activists and Islamia College students, marched to Dalhousie Square, the seat of the Government in Calcutta, and refused to disperse. Upon a lathi-charge., the processionists retaliated with stones and brickbats which the police, in turn, met with firing and two persons died, while fifty- two were injured. On 11 February 1946, Muslim League students led the procession, Congress and Communiist student organizations joined in and this time some arrests were made on Dharamatola Street. This provoked the large body of students to defy Section 144 imposed in the Dalhousie Square area and more arrests, in addition to a lathi­ charge, ensued. 

The RIN revolt started on 18 February when 1100 naval ratings of HMIS Talwar struck work at Bombay to protest against the treatment meted out to them — flagrant racial discrimination, unpalatable food and abuses to boot. The arrest of B.C. Dutt, a rating, for scrawling `Quit India' on the HMIS Talwar, was sorely resented. The next day, ratings from Castle and Fort Barracks joined the strike and on hearing that the HUJS Talwar ratings had been fired upon (which was incorrect) left their posts and went around Bombay in lorries, holding aloft Congress flags. threatening Europeans and policemen and occasionally tweaking a shop window or two. 

The second stage of these upsurges, when people in the city joined in. was marked by a virulent anti-British mood and resulted in the virtual paralysis of the two great cities of Calcutta and Bombay. Meetings and processions to express sympathy, as also strikes and hartals, were quickly overshadowed by the barricades that came up. the pitched battles fought from housetops and by-lanes, the attacks on Europeans, and the burning of police stations, post offices, shops, tram depots, railway stations, banks, grain shops, and even a YMCA centre. This was the pattern that was visible in all the three cases. The RIN revolt and popular fbry in Bombay alone accounted for, according to official estimates, the destruction of thirty shops, ten post offices, ten police chowkis, sixty-four food grains shops and 200 street lamps. Normal life in the city was completely disrupted. The Communist call for a genera) strike brought lakhs of workers out of their factories into the streets. Hartals by shopkeepers, merchants and hotel-owners and strikes by student workers, both in industry and public transport services almost brought the whole city to a grinding halt. Forcible stopping of trains by squatting on rail-tracks, stoning and burning of police and military lorries and barricading of streets did the rest. 

The third phase was characterized by a display of solidarity by people in other parts of the county. Students boycotted classes, hartals and processions were organized to express sympathy with the students and ratings and to condemn official repression. In the RIN revolt, Karachi was a major centre, second only to Bombay. The news reached Karachi on 19 February, upon which the HMIS Hindustan along with one more ship and three shore establishments, went on a lightning strike. Sympathetic token strikes took place in military establishments in Madras. Vishakhapatnam. Calcutta, Delhi, Cochin, Jamnagar, the Andamans, Bahrain and Aden Seventy eight ships and 20 shore establishments, involving 20,000 ratings, were affected. RJAF men went on sympathetic strikes in the Marine Drive, Andheri and Sion areas of Bombay and in Poona, Calcutta, Jessore and Ambala units. Sepoys at Jabalpur went on strike while the Colaba cantonment showed ominous `restlessness.' 

What was the significance of these events? There is no doubt that these three upsurges were significant in as much as they gave expression to the militancy in the popular mind. Action, however reckless, was fearless and the crowds which faced police firing by temporarily retreating, only to return to their posts, won the Bengal Governor's grudging admiration. The RIN revolt remains a legend to this day. When it took place, it had a dramatic impact on popular Consciousness. A revolt in the armed forces, even if soon suppressed, had a great liberating effect on the minds of people. The RIN revolt was seen as an event which marked the end of British rule almost as finally as Independence Day, 1947. But reality and how men perceive that reality often proves to be different, and this was true of these dramatic moments in 1945-46. Contemporary perceptions and later radical scholarship have infused these historical events with more than a symbolic significance.' These events are imbued with an unrealized potential and a realized impact which is quite out of touch with reality. A larger than life picture is drawn of their militancy, reach and effectiveness. India is seen to be on the brink of a revolution. The argument goes that the communal unity witnessed during these events could, if built upon, have offered a way out of the communal deadlock. 

When we examine these upsurges closely we find that the form they took, that of an extreme, direct and violent conflict with authority, had certain limitations. Only the most militant sections of society could participate. There was no place for the liberal and conservative groups which had rallied to the INA cause earlier or for the men and women of small towns and villages who had formed the backbone of the mass movements in earlier decades. Besides, these upsurges were short-lived, as the tide of popular fury- surged forth, only to subside all too quickly. Interestingly, Calcutta, the scene of tremendous enthusiasm from 11 to 13 February 1946, was relatively quiet during the RIN revolt a week later. One lakh workers went on a one day strike, but the rest of the city, barring the organized working class, remained subdued, despite a seven-thy ratings strike in Calcutta which had to be broken by a siege by troops. In addition, the upheavals were confined to a few urban centres, while the general INA agitation reached the remotest villages. This urban concentration made it easy for the authorities to deploy troops and effectively suppress the upsurge. 

The communal unity witnessed was more organizational unity than unity of the people. Moreover, the organizations came together only for a specific agitation that lasted a few days, as was the case in Calcutta on the issue of Rashid Mi's trial. Calcutta, the scene of `the almost revolution' in February 1946, according to Gautam Chattopadhaya'', became the battle ground of communal frenzy only six months later, on 16 August 1946. The communal unity evident in the RIN revolt was limited, despite the Congress, League and Communist flags being jointly hoisted on the ships' masts. Muslim ratings went to the League to seek advice on future action, while the rest went to the Congress and the Socialists; Jinnah's advice to surrender was addressed to Muslim ratings alone, who duly heeded It. The view that communal unity forged in the struggles of 1945- 46 could, if taken further, have averted partition, seems to be based on wishful thinking rather than concrete historical possibility. The `unity at the barricades' did not show this promise. 

Popular perceptions differ from reality when it comes to the response these upsurges, especially the JUN revolt, evoked from the colonial authorities. It is believed that `the RAN revolt shook the mighty British Empire to its foundations.' In fact these upsurges demonstrated that despite considerable erosion of the morale of the bureaucracy and the steadfastness of the armed forces by this time, the British wherewithal to repress was intact. The soldier-Viceroy, Wavell, gave a clean chit to the army a few days after the naval strike: `On the whole, the Indian army has been most commendably steady.'' Those who believed that the British would succumb to popular pressure if only it was exerted forcefully were proved wrong. It was one thing for the British Government to question its own stand of holding the INA trials when faced with opposition from the army and the people. It was quite another matter when they faced challenges to their authority. Challenges to the peace, the British were clear, had to be repressed. 

Events in November 1945 in Calcutta had the troops standing by, but the Governor of Bengal preferred to and was able to control the situation with the police. Troops were called in on 12 February 1946 in Calcutta and thirty-six civilians were killed in the firing. Similarly, during the JUN revolt, ratings were forced to surrender in Karachi and six of them were killed in the process. Contrary to the popular belief that Indian troops in Bombay had refused to fire on their countrymen, it was a Maratha battalion in Bombay that rounded up the ratings and restored them to their barracks. In Bombay, troop subdued not only the ratings but also the people, who had earlier supported the ratings with food and sympathy and later joined them in paralyzing Bombay. The British Prime Minister, Attlee, announced in the House of Commons that Royal Navy ships were on their way to Bombay Admiral Godfrey, of the RIN gave the ratings a stem ultimatum after which troops circled the ships and bombers were flown over them The Amrita Bazar Patrika referred to the virtual steel ring around Bombay. Two hundred aid twenty eight civilians died in Bombay while 1046 were injured. 

The corollary to the above argument is the attribution of the sending of Cabinet Mission to the Impact of the RIN revolt. R.P. Dutt had yoked the two together many years ago – On February 18 the Bombay Naval strike began. On 19 February, Attlee in the House of Commons announced the decision to despatch the Cabinet mission.' This is obviously untenable. The decision to send out the mission was taken by the British Cabinet on 22 January 1946 and even as announcement on 19 February 1946 had been slated a week earlier. Others have explained the willingness of the British to make substantial political concessions at this point of time to the combined impact of the popular militant struggles. However, as we shall see in the next chapter, the British decision to transfer power was not merely a response to the immediate situation prevailing in the winter of 1945-46, but a result of their realization that their legitimacy to rule had been irrevocably eroded over the years. The relationship between these upsurges and the Congress is seen as one of opposition, or at best dissociation. These agitations are believed to have been led by the Communists, the Socialists or Forward Blocists or all of them together. The Congress role is seen as one of defusing the revolutionary situation, prompted by its fear that the situation would go out of its control or by the concern that disciplined armed forces were vital in the free India that the party would rule soon The Congress is seen to be immersed in negotiations and ministry-making and hankering for power. The belief is that if the Congress leaders had not surrendered to their desire for power, a different path to independence would have emerged. 

In our view, the three upsurges were an extension of the earlier nationalist activity with which the Congress was integrally associated. It was the strong anti-imperialist sentiment fostered by the Congress through its election campaign, its advocacy of the INA cause and its highlighting of the excesses of 1942 that found expression in the three upsurges that took place between November 1945 and February 1946. The Home Department's provincial level enquiry into the causes of these `disturbances' came to the conclusion that they were the outcome of the `inflammatory atmosphere created by the intemperate speeches of Congress leaders in the last three months.' The Viceroy had no doubt that the primary cause of the REN `mutiny' was the `speeches of Congress leaders since September last.'' In fact, the Punjab CID authorities warned the Director of the Intelligence Bureau of the `considerable danger,' while dealing with the Communists, `of putting the cart before the horse and of failing to recognize Congress as the main enemy.' 

These three upsurges were distinguishable from the activity preceding them because the form of articulation of protest was different. They took the form of a violent, flagrant challenge to authority. The earlier activity was a peaceful demonstration of nationalist solidarity. One was an explosion, the other a groundswell. 

The Congress did not give the call for these upsurges; in fact, no political organization did. People rallied in sympathy with the students and ratings as well as to voice their anger at the repression that was let loose. Individual Congressmen participated actively as did individual Communists and others. Student sympathizers of the Congress, the Congress Socialist Party, the Forward Bloc and the Communist Party of India jointly led the 21 November 1945 demonstration in Calcutta. The Congress lauded the spirit of the people and condemned the repression by the Government. It did not officially support these struggles as it felt their tactics and timing were wrong. It was evident to Congress leaders that the Government was able and determined to repress. Vallabhbhai Patel asked the ratings to surrender because he saw the British mobilization for repression in Bombay. He wrote to Nehru on 22 February 1946: `The overpowering force of both naval and military personnel gathered here is so strong that they can be exterminated altogether and they have been also threatened with such a contingency.'2° Congress leaders were not the only ones who felt the need to restore peace. Communists joined hands with Congressmen in advising the people of Calcutta in November 1945 and February 1946 to return to their homes. Communist and Congress peace vans did the rounds of Karachi during the JUN revolt. 

The contention that `fear of popular excesses made Congress leaders cling to the path of negotiations and compromise, and eventually even accept Partition as a necessary price,' has little validity. Negotiations were an integral part of Congress strategy, a possibility which had to be exhausted before a mass movement was launched. As late as 22 September 1945 this had been reiterated in a resolution on Congress policy passed by the AICC: `The method of negotiation and conciliation which is the keynote of peaceful policy can never be abandoned by the Congress, no matter how grave may be the provocation, any more than can that of non-cooperation, complete or modified. Hence the guiding maxim of the Congress must remain: negotiations and settlement when possible and non-cooperation and direct action when necessary.' 

In 1946, exploring the option of negotiation before launching a movement was seen to be crucial since the British were likely to leave India within two to five years, according to Nehru. The Secretary of State's New Year statement and the British Prime Minister's announcement of the decision to send a Cabinet Mission on 19 February 1946 spoke of Indian independence coming soon. However, pressure had to be kept up on the British to reach a settlement and to this end preparedness for a movement (built steadily through 1945 by refurbishing the organization, electioneering and spearheading the [NA agitation) was sought to be maintained. But the card of negotiation was to be, played first, that of mass movement was to be held in reserve. Gandhiji, in three statements that he published in Harm, on 3 March 1946, indicated the perils of the path that had been recently taken by the people. `It is a matter of great relief that the ratings have listened to Sardar Patel's advice to surrender. They have not surrendered their honour. So far as I can see, in resorting to mutiny they were badly advised. If it was for grievance, fancied or real, they should have waited for the guidance and intervention of political leaders of their choice. If they mutinied for the freedom of India, they were doubly wrong. They could not do so without a call from a prepared revolutionary party. They were thoughtless and igno.ant, if they believed that by their might they would deliver India from foreign domination... `Lokamanya Tilak has taught us that Home Rule or Swaraj is our birthright. That Swaraj is not to be obtained by what is going on now in Bombay, Calcutta and Karachi... `They who incited the mutineers did not know what they were doing. The latter were bound to submit ultimately... Aruna would ``rather unite Hindus and Muslims at the barricade than on the constitution front.'' Even in terms of violence, this is a misleading proposition. If the union at the barricade is honest there must be union also at the constitutional front. Fighters do not always live at the barricade. They are too wise to commit suicide. The barricade life has always to be followed by the constitutional. That front is not taboo for ever. `Gandhiji went on to outline the path that should be followed by the nation: `Emphatically it betrays want of foresight to disbelieve British declarations and precipitate a quarrel in anticipation. Is the official deputation coming to deceive a great nation? It is neither manly or womanly to think so. What would be lost by waiting? Let the official deputation prove for the last time that British declarations are unreliable. The nation will gain by trusting. The deceiver loses when there is correct response from the deceived ... The rulers have declared their intention to `quit' in favour of Indian rule. `But the nation too has to play the game. If it does, the barricade must be left aside, at least for the time being.'
