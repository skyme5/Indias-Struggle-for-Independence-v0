
\chapter{World War I and Indian Nationalism: The Ghadar}

The outbreak of the First World War in 1914 gave a new lease of life to the nationalist movement which had been dormant since the heady days of the Swadeshi Movement. Britain's difficulty was India's `opportunity.' This opportunity was seized, in different ways arid with varying success, by the Ghadar revolutionaries based in North America and by Lokamanya Tilak, Annie Besant and their Home Rule Leagues in India. The Ghadarites attempted a violent overthrow of British rule, while the Home Rule Leaguers launched a nation-wide agitation for securing Home Rule or Swaraj.

\begin{center}*\end{center}

\paragraph*{}

The West Coast of North America had, since 1904, become home to a steadily increasing number of Punjabi immigrants. Many of these were land-hungry peasants from the crowded areas of Punjab, especially the Jullundur and Hoshiarpur districts, in search of some means of survival. Some of them came straight from their villages in Punjab while others had emigrated earlier to seek employment in various places in the Far East, in the Malay States, and in Fiji. Many among them were ex- soldiers whose service in the British Indian Army had taken them to distant lands and made them aware of the opportunities to be had there. Pushed out from their homes by economic hardship and lured by the prospect of building a new and prosperous life for themselves and their kin, they pawned the belonging, mortgaged or sold their land, and set out for the promised lands.

The welcome awaited the travel-weary immigrants in Canada and the USA was, however not what they had expected. Many were refused entry, especially those who came straight from their villages and did not know Western Ways and manners those who were allowed to stay not only had to face racial Contempt but also the brunt of the hostility of the White labour force and unions who resented the competition they offered. Agitations against the entry of the Indians were launched by native American labourers and these were supported by politicians looking for the popular vote.

Meanwhile, the Secretary of State for India had his own reasons for urging restrictions on immigration. For one, he believed that the terms of close familiarity of Indians with Whites which would inevitably take place in America was not good for British prestige; it was by prestige alone that India was held and not by force. Further, he was worried that the immigrants would get contaminated by socialist ideas, and that the racial discrimination to which they were bound to be subjected would become the source of nationalist agitation in India.' The combined pressure resulted in an effective restriction on Indian immigration into Canada in 1908. Tarak Nath Das, an Indian student, and one of the first leaders of the Indian community in North America to start a paper (called Free Hindustan) realized that while the British government was keen on Indians going to Fiji to work as labourers for British planters, it did not want them to go to North America where they might be infected by ideas of liberty.

\begin{center}*\end{center}

\paragraph*{}

The discriminatory policies of the host countries soon resulted in a flurry of political activity among Indian nationalists. As early as 1907, Ramnath Purl, a political exile on the West Coast, issued a Circular-e-Azadi (Circular of Liberty) in which he also pledged support to the Swadeshi Movement; Tarak Nath Das in Vancouver started the Free Hindustan and adopted a very militant nationalist tone; G.D. Kumar set up a Swadesh Sevak Home in Vancouver on the lines of the India House in London and also began to bring out a Gurmukhi paper called Swadesh Sevak which advocated social reform and also asked Indian troops to rise in revolt against the British. In 1910, Tarak Nath Das and G.D. Kumar, by now forced out of Vancouver, set up the United India House in Seattle in the US, where every Saturday they lectured to a group of twenty-five Indian labourers. Close links also developed between the United India House group, consisting mainly of radical nationalist students, and the Khalsa Diwan Society, and in 1913 they decided to send a deputation to meet the Colonial Secretary in London and the Viceroy and other officials in India The Colonial Secretary in London could not find the time to see them even though they waited for a whole month, but in India they succeed in meeting the Viceroy and the Lieutenant Governor of the Punjab But, more important, their visit became the occasion for a series of public meetings in Lahore, Ludhiana, Ambala, Ferozepore, Jullundur, Amritsar Lyallpur, Gujranwala, Sialkot and Simla and they received enthusiastic support from the Press and the general public.

The result of this sustained agitation, both in Canada and the United States, was the creation of a nationalist consciousness and feeling of solidarity among immigrant Indians. Their inability to get the Government of India or the British Government to intercede on their behalf regarding immigration restrictions and other disabilities, such as those imposed by the Alien Land law which practically prohibited Indians from owning land in the US, led to an impatience and a mood of discontent which blossomed into a revolutionary movement.

\begin{center}*\end{center}

\paragraph*{}

The first fillip to the revolutionary movement was provided by the visit to Vancouver, in early 1913, of Bhagwan Singh, a Sikh priest who had worked in Hong Kong and the Malay States. He openly preached the gospel of violent overthrow of British rule and urged the people to adopt Bande Mataram as a revolutionary salute. Bhagwan Singh was externed from Canada after a stay of three months.

The centre of revolutionary activity soon shifted to the US, which provided a relatively free political atmosphere. The crucial role was OW played by Lala Har Dayal, a political exile from India. Har Dayal arrived in California in April 1911, taught briefly at Stanford University, and soon immersed himself in political activity. During the summer of 1912, he concentrated mainly on delivering lectures on the anarchist and syndicalist movements to various American groups of intellectuals, radicals and workers, and did not show much interest in the problems that were agitating the immigrant4ndian community. But the bomb attack on Lord Hardinge, the Viceroy of India, in Delhi on 23 December, 1912, excited his imagination and roused the dormant Indian revolutionary in him. His faith in the possibility of a revolutionary overthrow of the British regime m India was renewed, and he issued a Yugantar Circular praising the attack on the Viceroy.

Meanwhile, the Indians on the West Coast of the US had been in search of a leader and had even thought of inviting Ajit Singh, who had become famous in the agitation in Punjab in 1907. But Har Dayal was already there and, after December 1912, showed himself willing to play an active political role. Soon the Hindi Association was set up in Portland in May 1913.

At he very first meeting of the Association, held in the house of Kanshi Rain, and attended among others by Bhai Parmanand, Sohan Singh Bhakna, and Harnam Singh `Tundilat,' Har Dayal set forth his plan of action: `Do not fight the Americans, but use the freedom that is available in the US to fight the British; you will never be treated as equals by the Americans until you are free in your own land, the root cause of Indian poverty and degradation is British rule and it must be overthrown, not by petitions but by aimed revolt; carry this message to the masses and to the soldiers in the Indian Anny; go to India in large numbers and enlist their support.' Har Dayal's ideas found immediate acceptance. A Working Committee was set up and the decision was taken to start a weekly paper, The Ghadar, for free circulation, and to set up a headquarters called Yugantar Ashram in San Francisco. A series of meetings held in different towns and centres and finally a representatives' meeting in Astoria confirmed and approved the decisions of the first meeting at Portland. The Ghadar Movement had begun.

\begin{center}*\end{center}

\paragraph*{}

The Ghadar militants immediately began an extensive propaganda Campaign; they toured extensively, visiting mills and farms where most of the Punjabi immigrant labour worked. The Yugantar Ashram became the home and headquarters and refuge of these political workers.

On 1 November 1913, the first issue of Ghadar, in Urdu was published and on 9 December, the Grumukhi edition. The name of the paper left no doubts as to its aim. Ghadar means Revolt. And if any doubts remained, they were to be dispelled by the captions on the masthead: `Angrezi Raj ka Dushman' or `An Enemy of British Rule.' On the front page of each issue was a feature titled Angrezi Raj Ka Kacha Chittha or `An Expose of British Rule.' This Chittha consisted of fourteen points enumerating the harmful effects of British rule, including the of wealth, the low per capita income of Indians, the high land tax, the contrast between the low expenditure on health and the high expenditure on the military, the destruction of Indian arts and industries, the recurrence of famines and plague that killed millions of Indians, the use of Indian tax payers' money for wars in Afghanistan, Burma, Egypt, Persia and China the British policy of promoting discord in the Indian States to extend their own influence, the discriminatory lenient treatment given to Englishmen who were guilty of killing Indians or dishonouring Indian women the policy of helping Christian missionaries with money raised from Hindus and Muslims, the effort to foment discord between Hindus and Muslims: in sum, the entire critique of British rule that had been formulated by the Indian national movement was summarized and presented every week to Ghadar readers. The last two points of the Chittha suggested the solution:

\begin{center}
The Indian population numbers seven crores in the Indian States and 24 crores in British India, while there are only 79,614 officers and soldiers and 38,948 volunteers who are Englishmen. Fifty-six years have lapsed since the Revolt of 1857; now there is urgent need for a second one.
\end{center}

Besides the powerful simplicity of the Chittha, the message was also conveyed by serializing Savarkar's The Indian War of independence --- 1857. The Ghadar also contained references to the contributions of Lokamanya Tilak, Sri Aurobjndo, V.D. Savarkar, Madame Cama, Shyamji Krishna Varma, Ajit Singh and Sufi Amba Prasad, as well as highlights of the daring deeds of the Anushilan Samiti, the Yugantar group and the Russian secret societies.

But, perhaps, the most powerful impact was made by the poems that appeared in The Ghadar, soon collected and published as Ghadar di Goonj and distributed free of cost. These poems were marked as much by their secular tone as by their revolutionary zeal, as the following extract demonstrates:

\multicolinterrupt{
\poemtitle{Ghadar di Goonj}
\begin{verse}
	Hindus, Sikhs, Pathans and Muslims,\\
	Pay attention ye all people in the army.\\
	Our country has been plundered by the British,\\
	We have to wage a war against them.\\
	We do not need pandits and quazis,\\
	We do not want to get our ship sunk.\\
	The time of worship is over now,\\
	It is time to take up the sword.
\end{verse}
\attrib{The Ghadar}
}

The Ghadar was circulated widely among Indians in North America, and within a few months it had reached groups settled in the Philippines, Hong Kong, China, the Malay States, Singapore, Trinidad, the Honduras, and of course, India. It evoked an unprecedented response, becoming the subject of lively discussion and debate. The poems it carried were recited at gatherings of Punjabi immigrants, and were soon popular everywhere. Unsurprisingly, The Ghadár, succeeded, in a very brief time, in changing the self-image of the Punjabi immigrant from that of a loyal soldier of the British Raj to that of a rebel whose only aim was to destroy the British hold on his motherland. The Ghadar consciously made the Punjabi aware of his loyalist past, made him feel ashamed of it, and challenged him to atone for it in the name of his earlier tradition of resistance to oppression:

\multicolinterrupt{
\poemtitle{Ghadar di Goonj}
\begin{verse}
	Why do you disgrace the name of Singhs?\\
	How come! you have forgotten the majesty of `Lions'\\
	Had the like of Dip Singh been alive today\\
	How could the Singhs have been taunted?\\
	People say that the Singhs are no good\\
	Why did you turn the tides during the Delhi mutiny?\\
	Cry aloud. `Let us kill the Whites'\\
	Why do you sit quiet, shamelessly\\
	Let the earth give way so we may drown\\
	To what good were these thirty crores born.
\end{verse}
\attrib{The Ghadar}
}

The message went home, and ardent young militants began thirsting for `action.' Har Dayal himself was surprised by the intensity of the response. He had, on occasion, spoken in terms of `ten years' or `some years' when asked how long it would take to organize the revolution in India But those who read the heady exhortations of The Ghadar were too impatient, and ten years seemed a long time.

\begin{center}*\end{center}

\paragraph*{}

Fina11y, in 1914, three events influenced the course of the Ghadar movement: the arrest and escape of Har Dayal, the Komagata Maru incident, and the outbreak of the First World War.

Dayal was arrested on 25 March 1914 on the stated ground of his anarchist activities though everybody suspected that the British Government had much to do with it. Released on bail, he used the opportunity to slip out of the country. With that, his active association with the Ghadar Movement came to an abrupt end.

Meanwhile, n March 1914, the ship, Komagata Maru had begun its fateful voyage to Canada. Canada had for some rears imposed very strict restrictions on Indian immigration by means of a law that forbade entry to all, except those who made a continuous journey from India. This measure had proved effective because there were no shipping lines that offered such a route. But in November 1913, the Canadian Supreme Court allowed entry to thirty-five Indians who had not made a continuous journey. Encouraged by this judgment, 'Gurdit Singh, an Indian contractor living in Singapore, decided to charter a ship and carry to Vancouver, Indians who were living in various places in East and South-East Asia. Carrying a total of 376 Indian passengers, the ship began its journey to Vancouver. Ghadar activists visited the ship at Yokohama in Japan, gave lectures and distributed literature. The Press in Punjab warned of serious consequences if the Indians were not allowed entry into Canada. The Press in Canada took a different view and some newspapers in Vancouver alerted the people to the `Mounting Oriental Invasion.' The Government of Canada had, meanwhile, plugged the legal loopholes that had resulted in the November Supreme Court judgment. The battle lines were clearly drawn.

When the ship arrived in Vancouver, it was not allowed into the port and was cordoned off by the police. To fight for the rights of the passengers, a `Shore Committee' was set up under the leadership of Husain Rahim, Sohan Lal Pathak and Balwant Singh, funds were raised, and protest meetings organized. Rebellion against the British in India was threatened. In the United States, under the leadership of Bhagwan Singh, Baikatullah, Ram Chandra and Sohan Singh Bhakna, a powerful campaign was organized and the people were advised to prepare for rebellion.

Soon the Komagata Maru was forced out of Canadian waters. Before it reached Yokohama, World War I broke out, and the British Government passed orders that no passenger be allowed to disembark anywhere on the way --- not even at the places from where they had joined the ship --- but only at Calcutta. At every port that the ship touched, it triggered off a wave of resentment and anger among the Indian community and became the occasion for anti-British mobilization. On landing at Budge Budge near Calcutta, the harassed and irate passengers, provoked by the hostile attitude of the authorities, resisted the police and this led to a clash in which eighteen passengers were killed, and 202 arrested. A few of them succeeded in escaping.

The third and most important development that made the Ghadar revolution imminent was the outbreak of the World War 1. After all, this was the opportunity they had been told to seize. True, they were not really prepared, but should they now let it just pass by? A special meeting of the leading activists of the Ghadar Movement decided that the opportunity must be seized, that it was better to die than to do nothing at all, and that their major weakness, the lack of arms, could be overcome by going to India and winning over the Indian soldiers to their cause. The Ailan-e-Jung or Proclamation of War of the Ghadar Party was issued and circulated widely. Mohammed Barkatullah, Ram Chandra and Bhagwan Singh organized and addressed a series of public meetings to exhort Indians to go back to India and organize an armed revolt. Prominent leaders were sent to persuade Indians living in Japan, the Philippines, China, Hong Kong, The Malay States, Singapore and Burma to return home and join the rebels. The more impatient among the Ghadar activists, such as Kartar Singh Sarabha, later hanged by the British in a conspiracy case, and Raghubar Dayal G1rta immediately left for India.

\begin{center}*\end{center}

\paragraph*{}

The Government of India, fully informed of the Ghadar plans, which were, in any case, hardly a secret, armed itself with the Ingress into India Ordinance and waited for the returning emigrants. On arrival, the emigrants, were scrutinized, the `safe' ones allowed to proceed home, the more `dangerous' ones arrested and the less dangerous' ones ordered not to leave their home villages. Of course, some of `the dangerous' ones escaped detection and went to Punjab to foment rebellion. Of an estimated 8000 emigrants who returned to India, 5000 were allowed to proceed unhindered. Precautionary measures were taken for roughly 1500 men. Upto February 1915, 189 had been interned and 704 restricted to their villages. Many who came via Colombo and South India succeeded in reaching Punjab without being found out. But Punjab in 1914 was very different from what the Ghadarites had been led to expect --- they found the Punjabis were in no mood to join the romantic adventure of the Ghadar. The militants from abroad tried their best, they toured the villages, addressed gatherings at melas and festivals, all to no avail. The Chief Khalsa Diwan proclaiming its loyalty to the sovereign, declared them to be `fallen' Sikhs and criminals, and helped the Government to track them down.

Frustrated and disillusioned with the attitude of the civilian population, the Ghadarites turned their attention to the army and made a number of naive attempts in November 1914 to get the army units to mutiny. But the lack of an organized leadership and central command frustrated all the Ghadar`s efforts.

Frantically, the Ghadar made an attempt to find a leader; Bengali revolutionaries were contacted and through the efforts of Sachindranath Sanyal and Vishnu Ganesh Pingley, Rash Behari Bose, the Bengali revolutionary who had become famous by his daring attack on Hardinge, the Viceroy, finally arrived in Punjab in mid-January 1915 to assume leadership of the revolt.

Bose established a semblance of an organization and sent out men to contact army units in different centres, (from Bannu in the North-West Frontier to Faizabad and Lucknow in the U.P.) and report back by 11 February 1915. The emissaries returned with optimistic reports, and the date for the mutiny was set first for 21 and then for 19 February. But the Criminal Investigation Department (CID) had succeeded in penetrating the organization, from the very highest level down, and the Government succeeded in taking effective pre-emptive measures. Most of the leaders were arrested, though Bose escaped. For all practical purposes, the Ghadar Movement was crushed. But the Government did not stop there. In what was perhaps the most repressive action experienced by the national movement this far, conspiracy trials were held in Punjab and Mandalay, forty-five revolutionaries were sentenced to death and over 200 to long terms of imprisonment. An entire generation of the nationalist leadership of Punjab was, thus, politically beheaded.

Some Indian revolutionaries who were operating from Berlin, and who had links with the Ghadar leader Ram Chandra in America, continued, with German help, to make attempts to organize a mutiny among Indian troops stationed abroad. Raja Mahendra Pratap and Barkatullah tried to enlist the help of the Amir of Afghanistan and even, hopefully, set up a Provisional Government in Kabul, but these and other attempts failed to record any significant success. It appeared that violent opposition to British rule was fated to fail.

\begin{center}*\end{center}

\paragraph*{}

Should we, therefore, conclude that the Ghadarites fought in vain? Or that, because they could not drive out the British, their movement was a failure? Both these conclusions are not necessarily correct because the success or failure of a political movement is not always to be measured in terms of its achievement of stated objectives. By that measure, all the major national struggles whether of 1920--22, 1930--34, or 1942 would have to be classified as failures, since none of them culminated in Indian independence. But if success and failure are to be measured in terms of the deepening of nationalist consciousness, the evolution and testing of new strategies and methods of struggle, the creation of tradition of resistance, of secularism, of democracy, and of egalitarianism, then, the Ghadarites certainly contributed their share to the struggle for India's freedom.

Ironic though it may seem, it was in the realm of ideology that Ghadar success was the greatest. Through the earlier papers, but most of all through The Ghadar itself, the entire nationalist critique of co1onialin, which was the most solid and abiding contribution of the moderate nationalists, was carried, in a powerful and simple form, to the mass of Indian immigrants, many of whom were poor workers and agricultural labourers. This huge propaganda effort motivated and educated an entire generation. Though a majority of the leaders of the Ghadar Movement, and most of the participants were drawn from among the Silchs, the ideology that was created and spread through The Ghadar and Ghadar di Goonj and other publications was strongly secular in tone. Concern with religion was seen as petty and narrow-minded, and unworthy of revolutionaries. That this was not mere brave talk is seen by the ease with which leaders belonging to, different religions and regions were accepted by the movement. Lala Har Dayal was a Hindu, and so were Ram Chandra and many others, Barkatullah was a Muslim and Rash Behari Bose a Hindu and a Bengali! But perhaps much more important, the Ghadarites consciously set out to create a secular consciousness among the Punjabis. A good example of this is the way in which the term Turka Shahi (Turkish rule), which in Punjabi was a synonym for oppression and high-handed behavior , was sought to be reinterpreted and the Punjabis were urged to look upon the `Turks' (read Muslims) as their brothers who fought hard for the country's freedom. Further, the nationalist salute Bande Mataram (and not any Sikh religious greeting such as Sag Sri Akal) was urged upon and adopted as the rallying cry of the Ghadar Movement. The Ghadarites sought to give a new meaning to religion as well. They urged that religion lay not in observing the outward forms such as those signified by long hair and Kirpan (sword), but in remaining true to the model of good behavior that was enjoined by all religious teachings.

The ambiguities that remained in the Ghadar ideological discourse, such as those evidenced by Har Dayal's advocacy of Khilafat as a religious cause of the Muslims, or when th.e British policy of not allowing Sikhs to carry arms was criticized, etc., were a product of the transitional stage in the evolution of a secular nationalist ideology that was spanned by the Ghadar Movement and its leaders. Also, the defence of religious interests has to be seen as part of the whole aspect of cultural defence against colonialism, and not necessarily as an aspect of communalism or communal ideology and consciousness.

Nor did the Ghadarites betray any narrow regional loyalties. Lokamanya Tilak, Aurobindo Ghose, Khudi Ram Bose, Kanhia Lal Dutt, Savarkar were all the heroes of the Ghadars. Rash Behari Bose was importuned and accepted as the leader of the abortive Ghadar revolt in 1915. Far from dwelling on the greatness of the Sikhs or the Punjabis, the Ghadars constantly criticized the loyalist role played by the Punjabis during 1857. Similarly, the large Sikh presence in the British Indian Army was not hailed as proof of the so-called `martial' traditions of the Sikhs, as became common later, but was seen as a matter of shame and Sikh soldiers were asked to revolt against the British. The self-image of the Punjabi, and especially of the Punjabi Sikh, that was created by the Ghadar Movement was that of an Indian who had betrayed his motherland in 1857 by siding with the foreigner and who had, therefore, to make amends to Bharat Mata, by fighting for her honor. In the words of Sohan Singh Bhakna, who later became a major peasant and Communist leader: `We were not Sikhs or Punjabis. Our religion was patriotism.'

Another marked feature of Ghadar ideology was its democratic and egalitarian content. It was clearly stated by the Ghadarites that their objective was the establishment of an independent republic of India. Also, deeply influenced as he was by anarchist and syndicalist movements, and even by socialist ideas, Har Dayal imparted to the movement an egalitarian ideology. Perhaps this facilitated the transformation of many Ghadarites into peasant leaders and Communist in the `20s and `30s.

Har Dayal's other major contribution was the creation of a truly internationalist outlook among the Ghadar revolutionaries. His lectures and articles were full of references to Irish, Mexican, and Russian revolutionaries. For example, he referred to Mexican revolutionaries as `Mexican Ghadarites.''Ghadar militants were thus distinguished by their secular, egalitarian, democratic and non-chauvinistic internationalist outlook.

This does not, however, mean that the Ghadar Movement did not suffer from any weaknesses. The major weakness of the Ghadar leaders was that they completely under-estimated the extent and amount of preparation at every level --- organizational, ideological, strategic, tactical, financial --- that was necessary before an attempt at an armed revolt could be organized. Taken by surprise by the outbreak of the war and roused to a fever- pitch by the Komagata Maru episode, they sounded the bugles of war without examining the state of their army. They forgot that to mobilize a few thousand discontented immigrant Indians, who were already in a highly charged emotional state because of the racial discrimination they suffered at me hands of white foreigners, was very different from the stupendous task of mobilizing and motivating lakhs of peasants and soldiers in India. They underestimated the strength of the British in India, both their aimed and organizational might as well as the ideological foundations of their rule and led themselves to imagine that all that the masses of India lacked was a call to revolt, which, once given, would strike a fatal blow to the tottering structure of British rule.

The Ghadar Movement also failed to generate an effective and sustained leadership that was capable of integrating the various aspects of the movement. Har Dayal himself was temperamentally totally unsuited to the role of an organizer; he was a propagandist, an inspirer, an ideologue. Even his ideas did not form a structured whole but remained a shifting amalgam of various theories that attracted him from time to time- Further, his departure from the U.S. at a critical stage left his compatriots floundering.

Another major weakness of the movement was its almost none existent organizational structure; the Ghadar Movement was sustained, more by the enthusiasm of the militants than by their effective organization.

These weaknesses of understanding, of leadership, of organization, all resulted in what one can only call a tremendous waste of valuable human resources. If we recall that forty' Ghadarities were sentenced to be hanged and over 200 given long terms of imprisonment, we can well realize that the particular romantic adventure of 1914--15 resulted in the beheading of an entire generation of secular nationalist leadership, who could perhaps have, if they had remained politically effective, given an entirely different political complexion to Punjab in the following years. They would certainly have given their strong secular moorings, acted as a bulwark against the growth of communal tendencies that were to raise their heads in later years. That this is not just wild speculation is seen from the fact that, in the late `20s, and `30s, the few surviving Ghadarites helped lay the foundations of a secular national and peasant movement in Punjab.
