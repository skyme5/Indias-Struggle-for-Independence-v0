\cleardoublepage
\chapter{Twenty-Eight Months of Congress Rule}



After a few months' tussle with the Government, the Congress Working Committee decided to accept office under the Act of 1935. During July, it formed Ministries in six provinces: Madras, Bombay, Central Provinces, Orissa, Bihar and U.P. Later, Congress Ministries were also formed in the North-West Frontier Province and Assam. To guide and coordinate their activities and to ensure that the British hopes of the provincialization of the Congress did not materialize, a central control board known as the Parliamentary Sub-Committee was formed, with Sardar Patel, Maulana Abul Kalam Azad and Rajendra Prasad as members. Thus began a novel experiment — a party which was committed to liquidate British rule took charge of administration under a constitution which was framed by the British and which yielded only partial state power to the Indians; this power could moreover be taken away from the Indians whenever the imperial power so desired. The Congress was now to function both as a government in the provinces and as the opposition vis-a-vis the Central Government where effective state power lay. It was to bring about social reforms through the legislature and administration in the provinces and at the same time carry on the struggle for independence and prepare the people for the next phase of mass struggle. Thus the Congress had to implement its strategy of Struggle-Truce-Struggle (S-T-S') in a historically unique situation.' 

As Gandhiji wrote on the meaning of office acceptance in Harijan on 7 August 1937: `These offices have to be held lightly, not tightly. They are or should be crowns of thorns, never of renown. Offices have been taken in order to see if they enable us to quicken the pace at which we are moving towards our goal.' Earlier he had advised Congressmen to use the Act of 1935 `in a manner not expected by them (the British) and by refraining from using it in the way intended by them.'

\begin{center}*\end{center}

\paragraph*{}


The formation of the Ministries by the Congress changed the entire psychological atmosphere in the country. People felt as if they were breathing the very air of victory and people's power, for was it not a great achievement that khadi clad men and women who had been in prison until just the other day were now ruling in the secretariat and the officials who were used to putting Congressmen in jail would now be taking orders from them? The exhilarating atmosphere of the times is, perhaps, best brought out by the following passage from Jawaharlal Nehru's Discovery of India: `There was a sense of immense relief as of the lifting of a weight which had been oppressing the people; there was a release of long- suppressed mass energy which was evident everywhere ... At the headquarters of the Provincial Governments, in the very citadels of the old bureaucracy, many a symbolic scene was witnessed... Now, suddenly, hordes of people, from the city and the village, entered these sacred precincts and roamed about almost at will. They were interested in anything; they went into the Assembly Chamber, where the sessions used to be held; they even peeped into the Ministers' rooms. It was difficult to stop them for they no longer felt as outsiders; they had a sense of ownership in all this ... The policemen and the orderlies with shining daggers were paralyzed; the old standards had fallen; European dress, symbol of position and authority, no longer counted. It was difficult to distinguish between members of the Legislatures and the peasants and townsmen who came in such large numbers.' 

There was an immense increase in the prestige of the Congress as an alternative power that would look after the interests of the masses, especially of the peasants. At the same time, the Congress had got an opportunity to demonstrate that it could not only lead the people in mass struggles but also use state power for their benefit. 

The responsibility was, of course, tremendous. However, there were limitations on the Congress Ministries' power and financial resources. They could obviously not change the basically imperialist character of the administration; they could not introduce a radical era. But, within the narrow limits of their powers, and the time available to them (their tenure lasted only two years and four months), they did try to introduce some reforms, take some ameliorative measures, and make some improvement in the condition of the people — to give the people a glimpse of the future Swaraj. 

The Congress Ministers set an example in plain living. They reduced their own salaries drastically from Rs. 2000 to Rs. 500 per month. They were easily accessible to the common people. And in a very short time, they did pass a very large amount of ameliorative legislation, trying to fulfil many of the promises made in the Congress election manifesto.

\begin{center}*\end{center}

\paragraph*{}


The commitment of the Congress to the defence and extension of civil liberties was as old as the Congress itself, and it is hardly surprising, therefore, that the Congress Ministries registered major achievements in this sphere. All emergency powers acquired by the provincial governments during 1932, through Public Safety Acts and the like, were repealed; bans on illegal political organizations such as the Hindustan Seva Dal and Youth Leagues and on political books and journals were lifted. Though the ban on the Communist Party remained, since it was imposed by the Central Government and could only be lifted on its orders, the Communists could in effect now function freely and openly in the Congress provinces. All restrictions on the press were removed. Securities taken from newspapers and presses were refunded and pending prosecutions were withdrawn. The blacklisting of newspapers for purposes of government advertising was given up. Confiscated arms were returned and forfeited arms licenses were restored. 

Of all the British functionaries, the ones the people were most afraid of, as also hated, were the police. On 21 August 1937, after the formation of the Ministries, Gandhiji wrote, `Indeed, the triumph of the Congress will be measured by the success it achieves in rendering the police and military practically idle... The best and the only effective way to wreck the existing Constitution is for the Congress to prove conclusively that it can rule without the aid of military and with the least possible assistance of the police ...` In the Congress provinces, police powers were curbed and the reporting of public speeches and the shadowing of political workers by CID (Central Investigation Department) agents stopped. 

One of the first acts of the Congress Government was to release thousands of political prisoners and detenus and to cancel internment and deportation orders on political workers. Many of the revolutionaries involved in the Kakori and other conspiracy cases were released. But problems remained in U.P. and Bihar where several revolutionaries convicted of crimes involving violence remained in jails. Most of these prisoners had earlier been sent to kala pani (Cellular Jail in Andamans) from where they had been transferred to their respective provinces after they had gone on a prolonged hunger strike during July 1937. In February 1938, there were fifteen such prisoners in U.P. and twenty-three in Bihar. Their release required consent by the Governors which was refused. But the Congress Ministries were determined to release them. The Ministries of U.P. and Bihar resigned on this issue on 15 February. The problem was finally resolved through negotiations. All the prisoners in both provinces were released by the end of March. The difference between the Congress provinces and the non- Congress provinces of Bengal and Punjab was most apparent in this realm. In the latter, especially in Bengal, civil liberties continued to be curbed and revolutionary prisoners and detenus, kept for years in prison without trial, were not released despite repeated hunger strikes by the prisoners and popular movements demanding their release. 

In Bombay, the Government also took steps to restore to the original owners lands which had been confiscated by the Government as a result of the no-tax campaign during the Civil Disobedience Movement in 1930. It, too, had to threaten resignation before it could persuade the Governor to agree. The pensions of officials dismissed during 1930 and 1932 for sympathizing with the movement were also restored. There were, however, certain blemishes on the Congress ministerial record in this respect. In July 1937, Yusuf Meherally, a Socialist leader, was prosecuted by the Madras Government for making an inflammatory speech in Malabar, though he was soon let off. In October 1937, the Madras Government prosecuted S.S. Batliwala, another Congress Social leader, for making a seditious speech and sentenced him to six months' imprisonment. There was a furore in the Congress ranks led by Jawaharlal Nehru, for this action went against the well-known Congress position that nobody should be prosecuted for making a speech and least of all for a speech against colonial rule. During the discussion on the subject in the Congress Working Committee, Nehru, reportedly, asked C. Rajagopalachari, the Premier of Madras (the head of the Provincial ministry was then known as Premier and not Chief Minister as now is the case): `Do you mean to say that if I come to Madras and make a similar speech you would arrest me?' `I would,' the latter is said to have replied. In the end Batliwala was released and went around Madras Presidency making similar speeches. The affair proved to be an exception; but it bred a certain suspicion regarding the future attitude of the Congress Right wing. `You have already become a police officer.'7 The Madras Government, too, used the police to shadow radical 

Much worse was the mentality of a few of the right-wing Congress ministers. For instance, K.M. Munshi, the Home Minister of Bombay, and a light-weight within the Congress leadership, used the C1D to watch the Communists and other left-wing Congressmen, earning a rebuke from Jawaharlal Nehru: Congressmen. These blemishes have, however, to be seen in the larger context of the vast expansion of civil liberties even in Bombay and Madras. Moreover, the mass of Congressmen were vigilant on this question. Led by the left-wing, they exerted intense pressure on the right-wing Congress ministers to avoid tampering with civil liberties.

\begin{center}*\end{center}

\paragraph*{}


The Congress Ministries tried to give economic relief to the peasants and the workers as quickly as possible. The Congress had succeeded, in the past, in acquiring massive support among them by exposing the roots of their poverty in colonial structure and policy, appealing to their nationalism, leading them in anti- imperialist struggles, and organizing and supporting their struggles around their economic demands. Now that the Congress had acquired some elements of state and administrative power, it was necessary to use these powers to improve their economic condition, and, thus, consolidate Congress support. 

The strategy of Congress agrarian legislation was worked out within certain broad parameters. First, the Congress was committed by its election manifesto and the election campaign to a policy of agrarian reform through reform of the system of land tenures and the reduction of rent, land revenue and the burden of debt. The Congress had asked rural voters to vote for its candidates by making large promises in this respect. The voters had taken them seriously; for example, according to government reports from Pratapgarh in U.P., on election day `a very large number of voters had brought with them pieces of dried cow dung to the various polling stations where these were lighted and, according to the tenants, ``bedakhlis'', i.e., ejectment orders, were burnt once for all. 

The Congress could not attempt a complete overhaul of the agrarian structure by completely eliminating the zamindari system. This, for two reasons, According to the constitutional structure of the 1935 Act, the provincial Ministries did not have enough powers to do so. They also suffered from an extreme lack of financial resources, for the lion's share of India's revenues was appropriated by the Government of India. The Congress Ministries could also not touch the existing administrative structure, whose sanctity was guarded by the Viceroy's and Governor's powers. What is more important, the strategy of class adjustment also forbade it. A multi-class movement could develop only by balancing or adjusting various, mutually clashing class interests. To unite all the Indian people in their struggle against colonialism, the main enemy of the time, it was necessary to make such an adjustment. The policy had to be that of winning over or at least neutralizing as large a part of the landlord classes as possible so as to isolate the enemy and deprive him of all social support within India. This was even more necessary because, in large parts of the country, the smaller landlords were active participants in the national movement. This was recognized by most of the leaders of the time. Swami Sahajanand, the militant peasant leader of Bihar, for example, wrote in his memoirs: `As a national organization, the Congress is the forum of all classes. All the classes are a part of the Congress. It represents all sections and classes. This is the claim of the Congress and this is desirable also ... The major function of the Congress is to maintain harmony between different classes and to further its struggle while doing so. 

There was also the constraint of time. The Congress leadership knew that their Ministries would not last long and would have to quit soon as the logic of their politics was to confront imperialism and not cooperate with it. As Nehru put it later in his Discovery of India, a `sense of impending crisis was always present; it was inherent in the situation.' Even when the Congress had accepted office, the usual figure given for longevity of the policy was two years. The time constraint became even more apparent as war clouds gathered in Europe from 1938 onwards. The Congress Ministries had, therefore, to act rapidly and achieve as much as possible in the short time available to them. 

Further, nearly all the Congress-run states (that is, U.P., Bihar, Bombay, Madras and Assam) had reactionary second chambers in the form of legislative councils, which were elected on a very narrow franchise — while the number of voters for the assemblies in these states was over 17.5 million, it was less than 70 thousand for the second chambers. These were, therefore, dominated by landlords, capitalists and moneylenders, with the Congress forming a small minority. As a majority in the lower house was not enough, in order to get any legislation passed through the second chamber, the Congress had to simultaneously pressure their upper class elements and conciliate them. Thus the Bihar Government negotiated a compromise with the zamindars on its tenancy bills while the 

U.P. Government conciliated the moneylender and merchant members of its upper house by going slow on debt legislation so that their support could be secured for tenancy legislation. 

Finally, the agrarian structure of various parts of India had developed over the centuries and was extremely complex and complicated.. There was not even enough information about its various components — land rights, for instance. The problem of debt and money lending was also integrated with peasant production and livelihood in too complex a manner to be tackled by an easy one-shot solution. Consequently, any effort at structural reform was bound to be an extremely formidable and time-consuming operation, as was to be revealed later after independence when the Congress and the Communists attempted to transform the agrarian structure in different states of the Indian union. 

Within these constraints, the agrarian policy of the Congress Ministries went a long way towards promoting the interests of the peasantry. Agrarian legislation by these Ministries differed from province to province depending on differing agrarian relations, the mass base of the Congress, the class composition and the outlook of the provincial Congress organization and leadership and the nature and extent of peasant mobilization. In general, it dealt with questions of tenancy rights, security of tenure and rents of the tenants and the problem of rural indebtedness. 

To enumerate the achievements of the Ministries, in this regard, briefly: In U.P. a tenancy act was passed in October 1939 which gave all statutory tenants both in Agra and Oudh full hereditary rights in their holdings while taking away the landlord's right to prevent the growth of occupancy. The rents of hereditary tenants could be changed only after ten years, while restrictions were placed on the rights of landlords to enhance rents even after this period. A tenant could no longer be arrested or imprisoned for non-payment of rent. All illegal exactions such as nazrana (forced gifts) and begar (forced unpaid labour) were abolished. In Bihar, the new tenancy legislation was passed mainly in 1937 and 1938, that is, more quickly than in U.P. More radical than that of U.P. in most respects, its main provisions were: All increases in rent made since 1911 were abolished; this was estimated to mean a reduction of about twenty-five per cent in rent. The rent was also reduced if the prices had fallen, during the currency of the existing rent, the deterioration of soil and the neglect of irrigation by the landlord. Occupancy ryots were given the absolute right to transfer their holding on the payment of a nominal amount of two per cent of rent to the landlord. A point of radical departure was the grant to under-ryots of occupancy rights if they had cultivated the land for twelve years. Existing arrears of rent were substantially reduced and the rate of interest on arrears was reduced from 12.5 to 6.25 per cent. The landlord's share in case of share-cropping was not to exceed 9/20 of the produce. Lands which had been sold in the execution of decrees for the payment of arrears between 1929 and 1937 (bakasht land) were to be restored to previous tenants on payment of half the amount of arrears. The landlord's power to realize rent was greatly reduced — the tenant could no longer be arrested or imprisoned on this account, nor could his immovable property be sold without his consent. Landlords were forbidden from charging illegal dues; any violation would lead to six months imprisonment. Occupancy tenants could no longer be ejected from their holdings for non-payment of rent. In fact, the only right that the landlord retained was the right to get his rent which was reduced significantly. 

In Orissa, a tenancy bill was passed in May 1938 granting the right of free transfer of occupancy holdings, reducing the interest on arrears of rent from 12.5 to 6 per cent and abolishing all illegal levies on tenants. Another bill passed in February 1938 reduced all rents in the zamindari areas, transferred in the recent past from Madras presidency to Orissa, to the rate of land revenue payable for similar lands in the nearest ryotwari areas plus 12.5 per cent as compensation to the zamindars. The Governor refused to give assent to the bill as it would have reduced the zamindars' incomes by fifty to sixty per cent. In Madras, a committee under the chairmanship of T. Prakasam (1872-1957), the Revenue Minister, recommended that in the areas under Permanent Zamindari Settlement the ryot and not the zamindar was the owner of the soil and that therefore the level of rents prevailing when the Settlement was made in 1802 should be restored. This would have reduced the rents by about two-thirds and would have meant virtual liquidation of the zamindari system. The Premier, C. Rajagopalachari, gave full support to the report. He also rejected the idea of compensating the zamindars. The Legislative Assembly passed, in January 1939, a resolution accepting the recommendations, but before a bill could be drafted, the Ministry resigned. 

Measures of tenancy reform, usually extending security of tenure to tenants in landlord areas, were also carried in the legislatures of Bombay, the Central Provinces and the North-West Frontier Province. The agrarian legislation of the Congress Ministries thus improved and secured the status of millions of tenants in zamindari areas. The basic system of landlordism was, of course, not affected. Furthermore, it was, in the main, statutory and occupancy tenants who benefited. The interests of the sub-tenants of the occupancy tenants were overlooked. Agricultural labourers were also not affected. This was partially because these two sections had not yet been mobilized by the kisan sabhas, nor had they become voters because of the restricted franchise under the Act of 1935. Consequently, they could not exert pressure on the Ministries through either elections or the peasant movement. 

Except for U.P. and Assam, the Congress Government passed a series of stringent debtors' relief acts which provided for the regulation of the moneylenders' business -— provisions of the acts included measures such as the cancellation or drastic reduction of accumulated interest ranging from 6.25 per cent in Madras to 9 per cent in Bombay and Bihar. These Governments also undertook various rather modest rural reconstruction programmes. In Bombay 40,000 dublas or tied serfs were liberated. Grazing fees in the forests were abolished in Bombay and reduced in Madras. While the tenancy bills were strongly opposed by the landlords, the debtors' relief bills were opposed not only by the moneylenders but also by lawyers, otherwise supporters of the Congress, because they derived a large part of their income from debt litigation.

\begin{center}*\end{center}

\paragraph*{}


The Congress Ministries adopted, in general. a pro-labour stance. Their basic approach was to advance workers' interests while promoting industrial peace, reducing the resort to strikes as far as possible, establishing conciliation machinery, advocating compulsory arbitration before resorting to strikes, and creating goodwill between labour and capital with the Congress and its ministers assuming the role of intermediaries, while, at the same time, striving to improve the conditions of the workers and secure wage increases. This attitude alarmed the Indian capitalist class which now felt the need to organize itself to press the `provincial governments to hasten slowly' on such matters.' 

Immediately after assuming office, the Bombay Ministry appointed a Textile Enquiry Committee which recommended, among other improvements, the increase of wages amounting to a crore of rupees. Despite mill owners protesting against the recommendations, they were implemented. In November 1938, the Governments passed the Industrial Disputes Act which was based on the philosophy of `class collaboration and not class conflict,' as the Premier B.G. Kher put it. The emphasis in the Act was on conciliation, arbitration and negotiations in place of direct action. The Act was also designed to prevent lightning strikes and lockouts. The Act empowered the Government to refer an industrial dispute to the Court of Industrial Arbitration. No strike or lock-out could occur for an interim period of four months during which the Court would give its award. The Act was strongly opposed by Left Congressmen, including Communists and Congress Socialists, for restricting the freedom to strike and for laying down a new complicated procedure for registration of trade unions, which, they said, would encourage unions promoted by employers in Madras, too, the Government promoted the policy of `internal settlement' of labour disputes through government sponsored conciliation and arbitration proceedings. In U.P., Kanpur was the seat of serious labour unrest as the workers expected active support from the popularly elected Government. A major strike occurred in May 1938. The Government set up a Labour Enquiry Committee, headed by Rajendra Prasad. The Committee's recommendations included an increase in workers' wages with a minimum wage of Rs. 15 per month, formation of an arbitration board, recruitment of labour for all mills by an independent board, maternity benefits to women workers, and recognition of the Left- dominated Mazdur Sabha by the employers. But the employers, who had refused to cooperate with the Committee, rejected the report. They did, however, in the end, because of a great deal of pressure from the Government, adopt its principal recommendations. A similar Bihar Labour Enquiry Committee headed by Rajendra Prasad was set up in 1938. It too recommended the strengthening of trade union rights, an improvement in labour conditions, and compulsory conciliation and arbitration to be tried before a strike was declared.

\begin{center}*\end{center}

\paragraph*{}


The Congress Governments undertook certain other measures of social reform and welfare. Prohibition was introduced in selected areas in different states. Measures for the advancement of untouchables or Harijans (children of God), as Gandhiji called them, including the passing of laws enabled Harijans to enter temples. and to get free access to public office, public sources of water such as wells and ponds, roads, means of transport, hospitals, educational and other similar institutions maintained out of public funds, and restaurants and hotels. Moreover, no court or public authority was to recognize any custom or usage which imposed any civil disability on Harijans. The number of scholarships and freeships for Harijan students was increased. Efforts were made to increase the number of Harijans in police and other government services. 

The Congress Ministries paid a lot of attention to primary, technical and higher education and public health and sanitation. Education for girls and Harijans was expanded. In particular, the Ministries introduced basic education with an emphasis on manual and productive work. Mass literacy campaigns among adults were organized. Support and subsidies were given to khadi, spinning and village industries. Schemes of prison reforms were taken up. The Congress Governments removed impediments in the path of indigenous industrial expansion and, in fact, actively attempted to promote several modern industrial ventures such as automobile manufacture. 

The Congress Governments also joined the effort to develop planning through the National Planning Committee appointed in 1938 by the Congress President Subhas Bose.

\begin{center}*\end{center}

\paragraph*{}


It was a basic aspect of the Congress strategy that in the non-mass struggle phases of the national movement, mass political activity and popular mobilization were to continue, though within the four-walls of legality, in fact, it was a part of the office-acceptance strategy that offices would be used to promote mass political activity. Jawaharlal Nehru, as the president of the Congress, for example, sent a circular to all Congressmen on 10 July, 1937 emphasizing that organizational and other work outside the legislature was to remain the major occupation of the Congress for `without it legislative activity would have little value' and that `the two forms of activity must be coordinated together and the masses should be kept in touch with whatever we do and consulted about it. The initiative must come from the masses.'' 

The question was the forms this mass political activity should take, and how work in administration and legislature was to be coordinated with political work outside and, equally important, what attitude the popularly elected government should adopt towards popular agitations, especially those which stepped outside the bounds of existing legality? There were no historical precedents to learn from or to follow. Different answers were found in different provinces. Unfortunately, the subject has not been studied in any depth by historians, except in a case study of U.P. by Visalakshi Menon.' According to Menon, the coordination of legislative and administrative activities and extra- parliamentary struggles was quite successful in U.P. There was widespread mass mobilization which took diverse forms, from the organization of Congress committees in villages to the setting up of popular organs of authority in the form of Congress police stations and panchayats dispensing justice under the leadership of local Congress committees, from organizing of mass petitions to officials and Ministers to setting up of Congress grievance committees in the districts to hear local grievances and reporting them to MLAs and Ministers, from mass literacy campaigns to explain to the people the working of the Ministries, and from organization of local, district and provincial camps and conferences to celebration of various days and weeks. Local Congress committees, members of Legislative Assembly, provincial and all-India level leaders and even ministers were involved in many of these extra-parliamentary mass mobilization programmes. More detailed research is likely to show that not all Congress Governments were able to coordinate administration with popular mobilization, especially where the right-wing dominated the `provincial Congress and the Government. Moreover, even in U.P., mass mobilization was losing steam by 1939. 

However, the dilemma also arose in another manner. Political work outside the legislatures would involve organizing popular protest. How far could a movement go in organizing protests and agitations against itself? 

Could a party which ran a government be simultaneously the organizer of popular movements and enforcer of law and order? And what if some of the protests took a violent or extra-legal form? Could civil liberties have their excess? How should the governmental wing of the movement then respond, since it is one of the functions of any government — colonial or nationalist, leftist or rightist or centrist --- to see that the existing laws are observed, in fact, the issue looks at the very question of the role of the state in modern society, whether capitalist or socialist. Moreover, part of the strategy of increasing Congress influence or rather hegemony among the people w as dependent on the demonstration, by the party leading the national movement, of its ability to govern and the capacity to rule. At the same time, existing laws were colonial laws. How far could a regime committed to their over-throw go in enforcing them? Furthermore, it was inevitable that, on the one hand, the long suppressed masses would try to bring pressure on the Ministries to get their demands fulfilled as early as possible, especially as they looked upon the Congress Ministries with `a sense of ownership' while, on the other, the satisfaction of these demands by the Ministries would be slow because of the constraints inherent in working through constitutional processes. The issue was, perhaps, posed as an easily solvable problem as far as Congressmen committed to non violence were concerned, but there were many other Congressmen for example, Communists, Socialists, Royists and Revolutionary Terrorists — and non- Congressmen who were not so committed, who tell that expanded civil liberties should be used to turn the masses towards more militant or even violent forms of agitation, and who tried to prove through such agitations and inadequacy of non-violence, the Congress strategy of S-TS 'and the policy of the working of reforms. Could governance and tolerance, it' not promotion, of violent forms of protest coexist? 

There was one other problem. While many Congressmen agitated within the perspective of accepting the Congress Ministries as their own and their role as one of strengthening them and the Congress through popular agitations and refrained from creating situations in which punitive action by the Government would become necessary, mans' others were out to expose the `breaches of faith and promises' by these Ministries and show tip the true' character of the Congress as the political organ of the upper classes and one which was, perhaps, no different from the imperialist authorities so far as the masses and their agitations were concerned, in their turn, many of the Congressmen looked upon all hostile critics and militants as forces of disorder and all situations in which people expressed their feelings in an angry manner as `getting out of hand.' Moreover, Congressmen like C. Rajagopalachari and K.M. Munshi did not hesitate to use their respective state apparatuses in a politically repressive manner. Unfortunately, the lull dimensions of this dilemma have not been adequately explored by historians so far. Today they can, perhaps, be usefully analyzed in a comparative framework vis-a-vis the functioning of the Communists and other radical parties as ruling parties in several states of the Indian Union after 1947, or as parts of ruling groups as seen in France or Portugal, or as rulers in socialist countries. 

The formation of Congress Ministries and the vast extension of civil liberties unleashed popular energies everywhere. Kisan sabhas sprang up in every part of the country and there was an immense growth in trade union activity and membership. Student and youth movements revived and burgeoned. A powerful fillip was given to the state peoples' movement. Left parties were able to expand manifold. Even though it was under a Central Government ban, the Communist Party was able to bring out its weekly organ, The National Front, from Bombay. The CSP brought out The Congress Socialist and several other journals in Indian languages. Of particular interest is the example of Kirti Lehar which the Kirti Communists of Punjab brought out from Meerut, U.P., because they could not do so in Unionist-ruled Punjab. 

Inevitably, many of the popular movements clashed with the Congress Governments. Even though peasant agitations usually took the form of massive demonstrations and spectacular peasant marches, in Bihar, the kisan movement often came in frontal confrontation with the Ministry, especially when the Kisan Sabha asked the peasants not to pay rent or to forcibly occupy landlords' lands. There were also cases of physical attacks upon landlords, big and small, and the looting of crops. Kisan Sabha workers popularized Sahajanand's militant slogans: Logan Lenge Kaise, Danda Hamara Zindabad (How will you collect rent, long live our lathis or sticks) and Lathi Men Sathi (Lathi is my companion). Consequently, there was a breach in relations between the Bihar kisan Sabha and the provincial Congress leadership. 

In Bombay, the AITUC, the Communists, and the followers of Dr. BR. Ambedkar organized a strike on 7 November 1938, in seventeen out of seventy-seven textile mills against the passage of the Industrial Disputes Act. There was some `disorder' and large- scale stone throwing at two mills and some policemen were injured. The police opened fire, killing two and injuring over seventy. The Madras Government (as also the Provincial Congress Committee) too adopted a strong policy towards strikes, which sometimes took a violent turn. Kanpur workers struck repeatedly, sometimes acting violently and attacking the police. But they tended to get Congress support. 

Congress Ministries did not know how to deal with situations where their own mass base was disaffected. They tried to play a mediatory role which was successful in U.P. and Bihar and to a certain extent in Madras, but not in Bombay. But, in general, they were not able to satisfy the Left- wing critics. Quite often they treated all militant protests, especially trade union struggles, as a law and order problem. They took recourse to Section 144 of the Criminal Code against agitating workers and arrested peasant and trade union leaders, even in Kanpur. 

Jawaharlal Nehru was privately unhappy with the Ministries' response to popular protest but his public stance was different. Then his answer was: `We cannot agitate against ourselves.' He tended `to stand up loyally for the ministers in public and protect them from petty and petulant criticism.'' To put a check on the growing agitations against Congress Ministries, the All India Congress Committee passed a resolution in September 1938, condemning those, `including a few Congressmen,' who `have been found in the name of civil liberty to advocate murder, arson, looting and class war by violent means.' `The Congress,' the resolution went on, `will, consistently with its tradition, support measures that may be undertaken by Congress governments for the defence of life and property.'' 

The Left was highly critical of the Congress Governments' handling of popular protest; it accused them of trying to suppress peasants' and workers' organizations. The Communist critique of the Congress Ministries was later summed up by R. Palme Dutt: `The experience of the two years of Congress Ministries demonstrated with growing acuteness the dangers implicit in entanglement in imperialist administration under a leadership already inclined to compromise. The dominant moderate leadership in effective control of the Congress machinery and of the Ministries was in practice developing an increasing cooperation with imperialism, was acting more and more openly in the interests of the upper-class landlords and industrialists, and was showing an increasingly marked hostility to all militant expression and forms of mass struggle ... Hence a new crisis of the national movement began to develop.'' 

Gandhiji too thought that the policy of ministry formation was leading to a crisis. But his angle of vision was very different from that of the Communists. To start with, he opposed militant agitations because he felt that their overt to covert violent character threatened his basic strategy based on non-violence. At the beginning of office acceptance, as pointed out earlier, he had advised the Congress Ministries to rule without the police and the army. Later he began to argue that `violent speech or writing does not come under the protection of civil liberty.'' But even while bemoaning the militancy and violence of the popular protest agitations and justifying the use of existing legal machinery against them, Gandhiji objected to the frequent recourse to colonial laws and law and order machinery to deal with popular agitations. He wanted reliance to be placed on the political education of the masses against the use of violence. He questioned, for example, the Madras Government's resort to the Criminal Law Amendment Act, especially to its `obnoxious clauses.' While criticizing Left-wing incitement to class violence, he constantly sought to curb Right-wing confrontation with the Left. He also defended the right of the Socialists and the Communists to preach and practise their politics in so far as they abided by Congress methods. Gandhiji was able to see the immense harm that the Congress would suffer in terms of erosion of popular support, especially of the workers and peasants, because of the repeated use of law and order machinery to deal with their agitations. This would make it difficult to organize the next wave of extra-legal mass movement against colonial rule. He thus perceived the inherent dilemma in the situation and dealt with it in a large number of articles in Haryana during 1938-39. This was one major reason why he began to question the efficacy of continuing with the policy of office acceptance.'8 He wrote in December 1938 that if the Congress Ministries `find that they cannot run the State without the use of the police and the military, it is the clearest possible sign, in terms of non-violence, that the Congress should give up office and again wander in the wilderness in search of the Holy Grail.'

\begin{center}*\end{center}

\paragraph*{}


The period of the Congress Ministries witnessed the emergence of serious weaknesses in the Congress. There was a great deal of factional strife and bickering both on ideological and personal bases, a good example of which was the factional squabbles within the Congress Ministry and the Assembly party in the Central provinces which led to the resignation of Dr. N.B. Khare as premier. The practice of bogus membership made its appearance and began to grow. There was a scramble for jobs and positions of personal advantage. Indiscipline among Congressmen was on the increase everywhere. Opportunists, self- seekers and careerists, drawn by the lure of associating with a party in power, began to enter the ranks of the Congress at various levels. This was easy because the Congress was an open party which anybody could join. Many Congressmen began to give way to casteism in their search for power. 

Gandhiji began to feel that `We seem to be weakening from within.' Full of despondency, Gandhiji repeatedly lashed out in the columns of Haryana against the growing misuse of office and creeping corruption in Congress ranks. `I would go to the length of giving the whole Congress organization a decent burial, rather than put up with the corruption that is rampant,' he told the Gandhi Seva Sangh workers in May 1939.20 Earlier, in November 1938, he had written in Haryana: `If the Congress is not purged of illegalities and irregularities, it will cease to be the power it is today and will fail to fulfil expectations when the real struggle faces the country.' Gandhiji, of course, saw that this slackening of the movement and weakening of the moral fibre of Congressmen was in part inevitable in a phase of non-mass struggle. He, therefore, advised giving up of offices and starting preparations for another phase of Satyagraha. 

Jawaharlal too had been feeling for some time that the positive role of the Ministries was getting exhausted. He wrote to Gandhiji on 28 April 1938: `1 feel strongly that the Congress ministries are working inefficiently and not doing much that they could do. They are adapting themselves far too much to the old order and trying to justify it. But all this, bad as it is, might be tolerated. What is far worse is that we are losing the high position that we have built up, with so much labour, in the hearts of the people. We are sinking to the level of ordinary politicians who have no principles to stand by and whose work is governed by a day to day opportunism. .. I think there are enough men of goodwill in the Congress. But their minds are full of party conflicts and the desire to crush this individual or that group.' 

The Congress Ministries resigned in October 1939 because of the political crisis brought about by World War 11. But Gandhiji welcomed the resignations for another reason — they would help cleanse the Congress of the `rampant corruption.' He wrote to C. Rajagopalachari on 23 October 1939: `1 am quite clear in my mind that what has happened is best for the cause. It is a bitter pill I know. But it was needed. It will drive away all the parasites from the body. We have been obliged to do wrong things which we shall be able to avoid.' The resignations produced another positive effect. They brought the Left and the Right in the Congress closer because of a common policy on the question of participation in the war.

\begin{center}*\end{center}

\paragraph*{}


In the balance, the legislative and administrative record of the Congress Ministries was certainly positive. As R. Coupland was to remark in 1944: `The old contention that Indian self- government was a necessity for any really radical attack on the social backwardness of India was thus confirmed.'' And Nehru, a stern critic of the Congress Ministries in 1938- 39, wrote in 1944: `Looking back, I am surprised at their achievements during a brief period of two years and a quarter, despite the innumerable difficulties that surrounded them.' Even though the Left was critical, in the long view even its expectations were fulfilled in a large measure. In 1935, Wang Ming, in his report on the revolutionary movements in colonial countries at the 6th Congress of the Communist International, said in the section on India: `Our Indian comrades in attempting to establish a united anti-imperialist front with the National Congress in December last year put before the latter such demands as ``the establishment of an Indian workers' and peasants' soviet republic,'' ``confiscation of all lands belonging to the zamindars without compensation,'' ``a general strike as the only effective programme of action,'' etc. Such demands on the part of our Indian comrades can serve as an example of how not to carry on the tactics of the anti-imperialist united front. . The Indian communists must formulate a programme of popular demands which could serve as a platform for a broad anti-imperialist united front ... this programme for struggle in the immediate future should include approximately the following demands: 1) against the slavish constitution, 2) for the immediate liberation of all political prisoners, 3) for the abolition of all extraordinary laws etc., 4) against the lowering of wages, the lengthening of working day and discharge of workers, 5) against burdensome taxes, high land rents and against confiscation of peasants' lands for non-payment of debts and obligations, and 6) for the establishment of democratic rights.' Certainly, the Congress Ministries fulfilled this agenda more or less in entirety. 

One of the great achievements of the Congress Governments was their firm handling of the communal riots. They asked the district magistrates and police officers to take strong action to deal with a communal outbreak. 

The Congress leadership foiled the imperialist design of using constitutional reforms to weaken the national movement and, instead demonstrated how the constitutional structure could be used by a movement aiming at capture of state power to further its own aims without getting co-opted. Despite certain weaknesses, the Congress emerged stronger from the period of office acceptance. Nor was the national movement diverted from its main task of fighting for self government because of being engaged in day-to-day administration. Offices were used successfully for enhancing the national consciousness and increasing the area of nationalist influence and thus strengthening the movement's capacity to wage a mass struggle in the future. The movement's influence was now extended to the bureaucracy, especially at the lower levels. And the morale of the ICS (Indian Civil Service), one of the pillars of the British Empire, suffered a shattering blow. Many ICS officers came to believe that the British departure from India was only a matter of time. In later years, especially during the Quit India Movement, the fear that the Congress might again assume power in the future, a prospect made real by the fact that Congress Ministries had already been in power once, helped to neutralize many otherwise hostile elements, such as landlords and even bureaucrats, and ensured that many of them at least sat on the fence. 

One may quote in this respect Visalakshi Menon's judgement: `From the instance of the United Provinces, it is obvious that there was no popular disillusionment with the Congress during the period of the Ministry. Rather, the people were able to perceive, in more concrete terms, the shape of things to come, if independence were won.' 

There was also no growth of provincialism or lessening of the sense of Indian unity, as the framers of the Act of 1935 and of its provision for Provincial Autonomy had hoped. The Ministries succeeded in evolving a common front before the Government of India. Despite factionalism, the Congress organization as a whole remained disciplined. Factionalism, particularly at the top, was kept within bounds with a strong hand by the central leadership. When it came to the crunch, there was also no sticking so office. Acceptance of office thus did prove to be just one phase in the freedom struggle. When an all- India political crisis occurred and the central Congress leadership wanted it, the Ministries promptly resigned. And the opportunists started leaving. As the Congress General Secretary said at the time: `The resignations of the ministries demonstrated to all thou who had any doubts that Congress was not out for power and office but for the emancipation of the people of India from the foreign yoke.' The Congress also avoided a split between its Left and Right wings — a split which the British were trying to actively promote since 1934. Despite strong critiques of each other by the two wings, they not only remained united but tended to come closer to each other, as the crisis at Tripuri showed. 

Above all, the Congress gained by influencing all sections of the people. The process of the growth of Congress and nationalist hegemony in Indian society was advanced. If mass struggles destroyed one crucial element of the hegemonic ideology of British colonialism by demonstrating that British power was not invincible then the sight of Indians exercising power shattered another myth by which the British had held Indians in subjection: that Indians were not fit to rule.
