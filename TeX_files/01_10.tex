\chapter{The Swadeshi Movement --- 1903--08}\label{chapter:CH10}
\begin{multicols}{2}

With the start of the Swadeshi Movement at the turn of the century, the Indian national movement took a major leap forward. Women, students and a large section of the urban and rural population of Bengal and other parts of India became actively involved in politics for the first time. The next half a decade saw the emergence of almost all the major political trends of the Indian national movement. From conservative moderation to political extremism, from terrorism to incipient socialism, from petitioning and public speeches to passive resistance and boycott, all had their origins in the movement. The richness of the movement was not confined to politics alone. The period saw a breakthrough in Indian's literature, music, science and industry. Indian society, as a `whole, was experimenting and the creativity of the people expanded in every direction'.

\begin{center}*\end{center}

\paragraph{The Swadeshi Movement} had its genesis in the anti-partition movement which was started to oppose the British decision to partition Bengal. There was no questioning the fact that Bengal with a population of 78 million (about a quarter of the population of British India) had indeed become administratively unwieldy. Equally there was no escaping the fact that the real motive or partitioning Bengal was political. Indian nationalism was gaining in strength and partition expected to weaken what was perceived as the nerve centre of Indian nationalism at that time. The attempt, at that time in the words of Lord Curzon, the Viceroy (1899--1905) was to `dethrone Calcutta' from its position as the `centre from which the Congress Party is manipulated throughout Bengal, and indeed which the Congress Party centre of successful intrigue' and `divide, the Bengali speaking population.' Risley, the Home Secretary to the Government of India, was more blunt. He said on 6 December 1904: `Bengal united, is power, Bengal divided, will pull several different ways.

``That is what the Congress leaders feel: their apprehensions are perfectly correct and they form one of the great merits of the scheme ... in this scheme ... one of our main objects is to split up and thereby weaken a solid body of opponents to our rule.''

Curzon reacted sharply to the almost instant furore that was raised in Bengal over the partition proposals and wrote to the Secretary of State. `If we are weak enough to yield to their clamour now, we shall not be able to dismember or reduce Bengal again: and you will be cementing and solidifying a force already formidable and certain to be a source of increasing trouble in the future'. The partition of the state intended to curb Bengali influence by not only placing Bengalis under two admininistrations but by reducing them to a minority in Bengal itself as in the new proposal Bengal proper was to have seventeen million Bengali and thirty-seven million Oriya and Hindi speaking people! Also, the partition was meant to foster another kind of division --- this time on the basis of religion. The policy of propping up Muslim communalists as a counter to the Congress and the national movement, which was getting increasingly crystallized in the last quarter of the 19th century. was to be implemented once again. Curzon's speech at Dacca, betrayed his attempt to `woo the Muslims' to support partition. With partition, he argued, Dacca could become the capital of the new Muslim majority province (with eighteen million Muslims and twelve million Hindus) `which would Invest the Mohammedans in Eastern Bengal with a unity which they have not enjoyed since the days of the old Mussulman Viceroys and Kings.' The Muslims would thus get a `better deal' and the eastern districts would be freed of the `pernicious influence of Calcutta.'

And even Lord Minto, Curzon's successor was critical of the way in which partition was imposed disregarding public opinion saw that it was good political strategy; Minto argued that `from a political point of View alone, putting aside the administrative difficulties of the old province, I believe partition to have been very necessary.'

The Indian nationalists clearly saw the design behind the partition and condemned it unanimously. The anti-partition and Swadeshi Movement had begun.

\begin{center}*\end{center}

\paragraph*{}

In December 1903, the partition proposals became publicly known, immediate and spontaneous protest followed. The strength of this protest can be gauged from the fact that in the first two months following the announcement 500 protest meetings were held in East Bengal alone, especially in Dacca, Mymensingh and Chittagong. Nearly fifty thousand copies of pamphlets giving a detailed critique of the partition proposals were distributed all over Bengal. Surendranath Banerjee, Krishna Kumar Mitra, Prithwi Chandra Ray and other leaders launched a powerful press campaign against the partition proposals through journals and newspapers like the Bengalee, Hitabadi and Sanjibani. Vast protest meetings were held in the town hail of Calcutta in March 1904 and January 1905, and numerous petitions (sixty-nine memoranda from the Dacca division alone), some of them signed by as many as 70,000 people --- a very large number keeping in view the level of politicization in those days --- were sent to the Government of India and the Secretary of State. Even, the big zamindars who had hitherto been loyal to the Raj, joined forces with the Congress leaders who were mostly intellectuals and political workers drawn from journalism, law and other liberal professions.

This was the phase, 1903 to mid-1905 when moderate techniques of petitions, memoranda, speeches, public meetings and press campaigns held full sway. The objective was to turn to public opinion in India and England against the partition proposals by preparing a foolproof case against them. The hope was that this would yield sufficient pressure to prevent this injustice from occurring.

\begin{center}*\end{center}

\paragraph*{}

The Government of India however remained unmoved. Despite the widespread protest, voiced against the partition proposals, the decision to partition Bengal was announced on 1905-07-19. It was obvious to the nationalists that their moderate methods were not working and that a different kind of strategy as needed. Within days of the government announcement numerous spontaneous protest meetings were held in mofussil towns such as Dinajpur, Pabna, Faridpur, Tangail, Jessore, Dacca, Birbhum, and Barisal. It was in these meetings that the pledge to boycott foreign goods was first taken In Calcutta; students organized a number of meetings against partition and for Swadeshi.

The formal proclamation of the Swadeshi Movement was, made on the 1905-08-07, in meeting held at the Calcutta to hall. The movement; hitherto sporadic and spontaneous, now had a focus and a leadership that was coming together. At the 7 August meeting, the famous Boycott Resolution was passed. Even Moderate leaders like Surendranath Banerjee toured the country urging the boycott of Manchester cloth and Liverpool salt. On September 1, the Government announced that partition was to be effected on 6 October, 1905. The following weeks saw protest meetings being held almost everyday all over Bengal; some of these meetings, like the one in Barisal, drew crowds of ten to twelve thousand. That the message of boycott went home is evident from the fact that the value of British cloth sold in some of the mofussil districts fell by five to fifteen times between September 1904 and September 1905. The day partition took effect --- 1905-11-16 --- was declared a day of mourning throughout Bengal. People fasted and no fires were lit at the cooking hearth. In Calcutta a hartal was declared. People took out processions and band after band walked barefoot, bathed in the Ganges in morning and then paraded the streets singing Bande Mataram which, almost spontaneously, became the theme song of the movement. People tied rakhis on each other's hands as a symbol of the unity of the two halves of Bengal. Later in the day Anandamohan Bose and Surendranath Banerjee addressed two huge mass meetings which drew crowds of 50,000 to 75,000 people. These were, perhaps, the largest mass meetings ever to be held under the nationalist banner this far. Within a few hours of the meetings, a sum of Rs. 50,000 was raised for the movement.

It was apparent that the character of the movement in terms both its goals and social base had begun to expand rapidly. As Abdul Rasul, President of Barisal Conference, April 1906, put it: `What we could not have accomplished in 50 or 100 years, the great disaster, the partition of Bengal, has done for us in six months. Its fruits have been the great national movement known as the Swadeshi movement.'

The message of Swadeshi and the boycott of foreign goods soon spread to the rest of the country: Lokamanya Tilak took the movement to different parts of India, especially Poona and Bombay; Ajit Singh and Lala Lajpat Rai spread the Swadeshi message in Punjab and other parts of northern India. Syed Haidar Raza led the movement in Delhi; Rawalpindi, Kangra, Jammu, Multan and Haridwar witnessed active participation in the Swadeshi Movement; Chidambaram Pillai took the movement to the Madras presidency, which was also galvanized by Bipin Chandra Pal's extensive lecture tour.

The Indian National Congress took up the Swadeshi call and the Banaras Session, 1905, presided over by G.K. Gokhale, supporter the Swadeshi and Boycott Movement for Bengal. The militant nationalists led by Tilak, Bipin Chandra Pal, Lajpat Rai and Aurobindo Ghosh were, however, in favour of extending the movement to the rest of India and carrying it beyond the programme of just Swadeshi and boycott to a full fledged political mass struggle The aim was now Swaraj and the abrogation of partition had become the `pettiest and narrowest of all political objects'. The Moderates, by and large, were not as yet willing to go that far. In 1906, however, the Indian National Congress at its Calcutta Session, presided over by Dadabhai Naoroji, took a major step forward. Naoroji in his presidential address declared that the goal of the Indian National Congress was `self-government or Swaraj like that of the United Kingdom or the Colonies.' The differences between the Moderates and the Extremists, especially regarding the pace of the movement and the techniques of struggle to be adopted, came to a head in the 1907 Surat Session of the Congress where the party split with serious consequences for the Swadeshi Movement.

\begin{center}*\end{center}

\paragraph*{}

In Bengal, however, after 1905, the Extremists acquired a dominant influence over the Swadeshi Movement. Several new forms of mobilization and techniques of struggle now began to emerge at the popular level. The trend of `mendicancy', petitioning and memorials was on the retreat. The militant nationalists put forward several fresh ideas at the theoretical, propagandistic and programmatic plane. Political independence was to be achieved by converting the movement into a mass movement through the extension of boycott into a full-scale movement of non-cooperation and passive resistance. The technique of extended boycott was to include, apart from boycott of foreign goods, boycott of government schools and colleges courts, titles and government services and even the organization of strikes. The aim was to `make the administration under present conditions impossible by an organized refusal to do anything which shall help either the British Commerce in the exploitation of the country or British officialdom in the administration of it.' While some, with remarkable foresight, saw the tremendous potential of large scale peaceful resistance --- `... the Chowkidar, the constable'; the deputy and the munsif and the clerk, not to speak of the sepoy all resign their respective functions, feringhee rule in the country may come to an end in a moment. No powder and shot will be needed, no sepoys will have to be trained. Others like Aurobindo Ghosh (with his growing links with revolutionary terrorists) kept open the option of violent resistance if British repression was stepped up.

Among the several forms of struggle thrown up by the movement, it was the boycott of foreign goods which met with the greatest visible success at the practical and popular level. Boycott and public burning of foreign cloth, picketing of shops selling foreign goods, all became common in remote corners of Bengal as well as in many important towns and cities throughout the country. Women refused to wear foreign bangles and use foreign utensils, washermen refused to wash foreign clothes and even priests declined offerings which contained foreign sugar.

The movement also innovated with considerable success different forms of mass mobilization. Public meetings and processions emerged as major methods of mass mobilization and simultaneously as forms of popular expression. Numerous meetings and processions organized at the district, taluqa and village levels, in cities and towns, both testified to the depth of Swadeshi sentiment and acted as vehicles for its further spread. These forms were to retain their pre-eminence in later phases of the national movement.

Corps of volunteers (or samitis as they were called) were another major form of mass mobilization widely used by the Swadeshi Movement. The Swadesh Bandhab Samiti set up by Ashwini Kumar Dutt, a school teacher, in Barisal was the most well known volunteer organization of them all. Through the activities of this Samiti, whose 159 branches reached out to the remotest corners of the district, Dutt was able to generate an unparalleled mass following among the predominantly Muslim Peasantry of the region. The samitis took the Swadeshi message to the villages through magic lantern lectures and Swadeshi songs, gave physical and moral training to the members, did social work during famines and epidemics, organized schools, training in Swadeshi craft and arbitrary courts. By August 1906 the Barisal Samiti reportedly settled 523 disputes through eighty-nine arbitration committees. Though the samitis stuck their deepest roots in Barisal, they had expanded to other parts of Bengal as well. British officialdom was genuinely alarmed by their activities, their growing popularity with the rural masses.

The Swadeshi period also saw the creative use of traditional popular festivals and melas as a means of reaching out to the masses. The Ganapati arid Shivaji festivals, popularized by Tilak, became a medium for Swadeshi propaganda not only in Western India but also in Bengal. Traditional folk theatre forms such as jatras i.e. extensively used in disseminating the Swadeshi message in an intelligible form to vast sections of the people, many of whom were being introduced to modern political ideas for the first time.

Another important aspect of the Swadeshi Movement was the great emphasis given to self-reliance or `Atmasakti' as a necessary part of the struggle against the Government. Self reliance in various fields meant the re-asserting of national dignity, honor and confidence. Further, self-help and constructive work at the village level was envisaged as a means of bringing about the social and economic regeneration of the villages and of reaching the rural masses. In actual terms this meant social reform and campaigns against evils such as caste oppression, early marriage, the dowry system, consumption of alcohol, etc. One of the major planks of the programme of self-reliance was Swadeshi or national education. Taking a cue from Tagore's Shantiniketan, the Bengal National College was founded, with Aurobindo as the principal. Scores of national schools sprang up all over the country within a short period. In August 1906, the National Council of Education was established. The Council, consisting of virtually all the distinguished persons of the country at the time, defined its objectives in this way ... `to organize a system of Education Literary; Scientific and Technical --- on National lines and under National control from the primary to the university level'. The chief medium of instruction was to be the vernacular to enable the widest possible reach. For technical education, the Bengal Technical institute was set and funds were raise to send students to Japan for advanced learning.

Self-reliance also meant an effort to set up Swadeshi or indigenous enterprises. The period saw a mushrooming of Swadeshi textile mills, soap and match factories; - tanneries, banks, insurance companies, shops, etc. While many of these enterprises, whose promoters were more endowed with patriotic zeal than with business acumen were unable to survive for long, some others such as Acharya P.C. Ray's Bengal Chemicals Factory, became successful and famous.

It was, perhaps, in the cultural sphere that the impact of the Swadeshi Movement was most marked. The songs composed at that time by Rabindranath Tagore, Rajani Kanta Sen, Dwijendralal Ray, Mukunda Das, Syed Abu Mohammed, and others later became the moving spirit for nationalists of all hues, `terrorists, Gandhian or Communists' and are still popular. Rabindranath's Amar Sonar Bangla, written at that time, was to later inspire the liberation struggle of Bangladesh and was adopted as the national anthem of the country in 1971. The Swadeshi influence could be seen in Bengali folk music popular among Hindu and Muslim villagers (Palligeet and Jan Gàn) and it evoked collections of India fairy tales such as, Thakurmar Jhuli(Grandmother's tales) written by Daksinaranjan Mitra Majumdar which delights Bengai children to this day. In art, this was the period when Abanindranath Tagore broke the domination of Victorian naturalism over Indian art and sought inspiration from the rich indigenous traditions of Mughal, Rajput and Ajanta paintings. Nandalal Bose, who left a major imprint on Indian art, was the first recipient of a scholarship offered by the Indian Society of Oriental Art founded in 1907. In science, Jagdish Chandra Bose, Prafulla Chandra Ray, and others pioneered original research that was praised the world over.

\begin{center}*\end{center}

\paragraph*{}

In sum, the Swadeshi Movement with its multi-faceted programme and activity was able to draw for the first time large sections of society into active participation in modern nationalist into the ambit of modern political ideas.

The social base of the national movements now extended to include a certain zamindari section, the lower middle class in the cities and small towns and school and college students on a massive scale. Women came out of their homes for the first time and joined processions and picketing. This period saw, again for the first time, an attempt being made to give a political direction to the economic grievances of the working class. Efforts were Swadeshi leaders, some of whom were influenced by International socialist currents such as those in Germany and Russia, to organize strikes in foreign managed concerns such as Eastern India Railway and Clive Jute Mills, etc.

While it is argued that the movement was unable to make much headway in mobilizing the peasantry especially its lower rungs except in certain areas, such as the district of Barisal, there can be no gainsaying the fact that even if the movement was able to mobilize the peasantry only in a limited area that alone would count for a lot. This is so peasant participation in the Swadeshi Movement marked the very beginnings of modem mass politics in India. After all, even in the later, post-Swadeshi movements, intense political mobilization and activity among the peasantry largely remained concentrated in specific pockets. Also, while it is true that during the Swadeshi phase the peasantry was not organized around peasant demands, and that the peasants in most parts did not actively join in certain forms of struggle such as, boycott or passive resistance, large sections of the peasants, through meetings, jatras, constructive work, and so on were exposed for the first time to modem nationalist ideas and politics.

The main drawback of the Swadeshi Movement was that it was not able to gamer the support of the mass of Muslims and especially of the Muslim peasantry. The British policy of consciously attempting to use communalism to turn the Muslims against the Swadeshi Movement was to a large extent responsible for this. The Government was helped in its designs by the peculiar situation obtaining in large pasts of Bengal where Hindus and Muslims were divided along class lines with the former being the landlords and the latter constituting the peasantry. This was the period when the All India Muslim League was set up with the active guidance and support of the Government. More specifically, in Bengal, people like Nawab Salimullah of Dacca were propped up so centres of opposition to the Swadeshi Movement. Mullahs and maulvis were pressed into service and, unsurprisingly, at the height of the Swadeshi Movement communal riots broke out in Bengal.

Given this background, some of the forms of mobilization adopted by the Swadeshi Movement had certain unintended negative consequences. The use of traditional popular customs, festivals and institutions for mobilizing the masses --- a technique used widely in most parts of world to generate mass movements, especially in the initial stages --- was misinterpreted and distorted by communalists backed by the state. The communal forces saw narrow religious identities in the traditional forms utilized by the Swadeshi movements whereas in fact these forms generally reflected common popular cultural traditions which had evolved as a synthesis of different religious prevalent among the people.

\begin{center}*\end{center}

\paragraph*{}

By mid-1908, the open movement with its popular mass character had all but spent itself. This was due to several reasons. First, the government, seeing the revolutionary potential of the movement, came down with a heavy hand. Repression took the form of controls and bans on public meetings, processions and the press. Student participants were expelled from Government schools and colleges, debarred from Government service, fined and at times beaten up by the police. The case of the 1906, Barisal Conference, where the police forcibly dispersed the conference and brutally beat up a large number of the participants, is a telling example of the government's attitude and policy.

Second, the internal squabbles, and especially, the split, in 1907 in the Congress, the apex all-India organization, weakened the movement. Also, though the Swadeshi Movement had spread outside Bengal, the rest of the country was not as yet fully prepared to adopt the new style and stage of politics. Both these factors strengthened the hands of the government. Between 1907 and 1908, nine major leaders in Bengal including Ashwini Kumar Dutt and Krishna Kumar Mitra were deported, Tilak was given a sentence of six years imprisonment, Ajit Singh and Lala Lajpat Rai of Punjab were deported and Chidambaram Pillai and Harisarvottam Rao from Madras and Andhra were arrested. Bipin Chandra Pal and Aurobindo Ghosh retired from active politics, a decision not unconnected with the repressive measures of the Government Almost with one stroke the entire movement was rendered leaderless.

Third, the Swadeshi Movement lacked an effective organization and party structure. The movement had thrown up programmatically the entire gamut of Gandhian techniques such as passive resistance, non-violent non-cooperation, the call to fill the British jails, social reform, constructive work, etc. It was, however, unable to give these techniques a centralized, disciplined focus, carry the bulk of political - India, and convert these techniques into actual, practical political practice, as Gandhiji was able to do later. Lastly, the movement declined partially because of the very logic of mass movements itself --- they cannot be sustained endlessly at the same pitch of militancy and self-sacrifice, especially when faced with severe repression, but need to pause, to consolidate its forces for yet another struggle.

\begin{center}*\end{center}

\paragraph*{}

However, the decline of the open movement by mid-1908 engendered yet another trend in the Swadeshi phase i.e., the rise of revolutionary terrorism. The youth of the county, who had been part of the mass movement, now found themselves unable to disappear tamely into the background once the movement itself grew moribund and Government repression was stepped up. Frustrated, some among them opted for `individual heroism' as distinct from the earlier attempts at mass action.

With the subsiding of the mass movement, one era in the Indian freedom struggle was over. It would be wrong, however, to see the Swadeshi Movement as a failure. The movement made a major contribution in taking the idea of nationalism, in a truly creative fashion, to many sections of the people, hitherto untouched by it. By doing so, it further eroded the hegemony of colonial ideas and institutions. Swadeshi influence in the realm of culture and ideas was crucial in this regard and has remained unparalleled in Indian history, except, perhaps, for the cultural upsurge of the I93Os this time under the influence of the Left.

Further, the movement evolved several new methods and techniques of mass mobilization and mass action though it was not able to put them all into practice successfully. Just as the Moderates achievement in the realm of developing an economic critique of colonialism is not minimized by the fact that they could not themselves carry this critique to large masses of people, similarly the achievement of the Extremists and the Swadeshi Movement in evolving new methods of mass mobilization and action is not diminished by the fact that they could not themselves fully utilize these methods. The legacy they bequeathed was one on which the later national movement was to draw heavily.

Swadeshi Movement was only the first round in the national popular struggle against colonialism. It was to borrow this imagery used by Antonio Gramsci an `important battle' in the long drawn out and complex `war of position' for Indian independence.
\end{multicols}